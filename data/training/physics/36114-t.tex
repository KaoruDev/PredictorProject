% %%%%%%%%%%%%%%%%%%%%%%%%%%%%%%%%%%%%%%%%%%%%%%%%%%%%%%%%%%%%%%%%%%%%%%% %
%                                                                         %
% The Project Gutenberg EBook of Relativity: The Special and the General  %
% Theory, by Albert Einstein                                              %
%                                                                         %
% This eBook is for the use of anyone anywhere at no cost and with        %
% almost no restrictions whatsoever.  You may copy it, give it away or    %
% re-use it under the terms of the Project Gutenberg License included     %
% with this eBook or online at www.gutenberg.org                          %
%                                                                         %
%                                                                         %
% Title: Relativity: The Special and the General Theory                   %
%        A Popular Exposition, 3rd ed.                                    %
%                                                                         %
% Author: Albert Einstein                                                 %
%                                                                         %
% Translator: Robert W. Lawson                                            %
%                                                                         %
% Release Date: May 15, 2011 [EBook #36114]                               %
%                                                                         %
% Language: English                                                       %
%                                                                         %
% Character set encoding: ISO-8859-1                                      %
%                                                                         %
% *** START OF THIS PROJECT GUTENBERG EBOOK RELATIVITY ***                %
%                                                                         %
% %%%%%%%%%%%%%%%%%%%%%%%%%%%%%%%%%%%%%%%%%%%%%%%%%%%%%%%%%%%%%%%%%%%%%%% %

\def\ebook{36114}
%%%%%%%%%%%%%%%%%%%%%%%%%%%%%%%%%%%%%%%%%%%%%%%%%%%%%%%%%%%%%%%%%%%%%%
%%                                                                  %%
%% Packages and substitutions:                                      %%
%%                                                                  %%
%% book:     Required.                                              %%
%% inputenc: Latin-1 text encoding. Required.                       %%
%%                                                                  %%
%% ifthen:   Logical conditionals. Required.                        %%
%%                                                                  %%
%% amsmath:  AMS mathematics enhancements. Required.                %%
%% amssymb:  Additional mathematical symbols. Required.             %%
%%                                                                  %%
%% alltt:    Fixed-width font environment. Required.                %%
%% array:    Enhanced tabular features. Required.                   %%
%%                                                                  %%
%% perpage:  Start footnote numbering on each page. Required.       %%
%%                                                                  %%
%% multicol: Twocolumn environment for index. Required.             %%
%% makeidx:  Indexing. Required.                                    %%
%%                                                                  %%
%% caption:  Caption customization. Required.                       %%
%% graphicx: Standard interface for graphics inclusion. Required.   %%
%% wrapfig:  Illustrations surrounded by text. Required.            %%
%%                                                                  %%
%% calc:     Length calculations. Required.                         %%
%%                                                                  %%
%% fancyhdr: Enhanced running headers and footers. Required.        %%
%%                                                                  %%
%% geometry: Enhanced page layout package. Required.                %%
%% hyperref: Hypertext embellishments for pdf output. Required.     %%
%%                                                                  %%
%%                                                                  %%
%% Producer's Comments:                                             %%
%%                                                                  %%
%%   OCR text for this ebook was obtained on May 7, 2011, from      %%
%%   http://www.archive.org/details/relativitythespe00einsuoft.     %%
%%                                                                  %%
%%   The Methuen book catalogue from the original has been omitted. %%
%%                                                                  %%
%%   Minor changes to the original are noted in this file in three  %%
%%   ways:                                                          %%
%%     1. \Change{}{} for typographical corrections, showing        %%
%%        original and replacement text side-by-side.               %%
%%     2. \Add{} for inconsistent/missing punctuation.              %%
%%     3. [** TN: Note]s for lengthier or stylistic comments.       %%
%%   \Add is implemented in terms of \Change, so redefining \Change %%
%%   will "restore" typographical errors in the original.           %%
%%                                                                  %%
%%                                                                  %%
%% Compilation Flags:                                               %%
%%                                                                  %%
%%   The following behavior may be controlled by boolean flags.     %%
%%                                                                  %%
%%   ForPrinting (false by default):                                %%
%%   If true, compile a print-optimized PDF file: Taller text block,%%
%%   two-sided layout, US Letter paper, black hyperlinks. Default:  %%
%%   screen optimized file (one-sided layout, blue hyperlinks).     %%
%%                                                                  %%
%%                                                                  %%
%% Things to Check:                                                 %%
%%                                                                  %%
%%                                                                  %%
%% Spellcheck: .................................. OK                %%
%%                                                                  %%
%% lacheck: ..................................... OK                %%
%%   Numerous false positives from commented code                   %%
%%                                                                  %%
%% PDF pages: 154 (if ForPrinting set to false)                     %%
%% PDF page size: 4.75 x 7"                                         %%
%% PDF bookmarks: created, point to ToC entries                     %%
%% PDF document info: filled in                                     %%
%% Images: 5 pdf diagrams                                           %%
%%                                                                  %%
%% Summary of log file:                                             %%
%% * No over- or under-full boxes.                                  %%
%%                                                                  %%
%% Compile History:                                                 %%
%%                                                                  %%
%% May, 2011: adhere (Andrew D. Hwang)                              %%
%%            texlive2007, GNU/Linux                                %%
%%                                                                  %%
%% Command block:                                                   %%
%%                                                                  %%
%%     pdflatex x3                                                  %%
%%     makeindex                                                    %%
%%     pdflatex x3                                                  %%
%%                                                                  %%
%%                                                                  %%
%% May 2011: pglatex.                                               %%
%%   Compile this project with:                                     %%
%%   pdflatex 36114-t.tex ..... THREE times                         %%
%%   makeindex 36114-t.idx                                          %%
%%   pdflatex 36114-t.tex ..... THREE times                         %%
%%                                                                  %%
%%   pdfTeXk, Version 3.141592-1.40.3 (Web2C 7.5.6)                 %%
%%                                                                  %%
%%%%%%%%%%%%%%%%%%%%%%%%%%%%%%%%%%%%%%%%%%%%%%%%%%%%%%%%%%%%%%%%%%%%%%
\listfiles
\documentclass[12pt]{book}[2005/09/16]

%%%%%%%%%%%%%%%%%%%%%%%%%%%%% PACKAGES %%%%%%%%%%%%%%%%%%%%%%%%%%%%%%%
\usepackage[latin1]{inputenc}[2006/05/05]

\usepackage{ifthen}[2001/05/26]  %% Logical conditionals

\usepackage{amsmath}[2000/07/18] %% Displayed equations
\usepackage{amssymb}[2002/01/22] %% and additional symbols

\usepackage{alltt}[1997/06/16]   %% boilerplate, credits, license
\usepackage{array}[2005/08/23]   %% extended array/tabular features

\usepackage{perpage}[2006/07/15]

\usepackage{multicol}[2006/05/18]
\usepackage{makeidx}[2000/03/29]

\usepackage[font=footnotesize,labelformat=empty]{caption}[2007/01/07]
\usepackage{graphicx}[1999/02/16]%% For diagrams
\usepackage{wrapfig}[2003/01/31] %% and wrapping text around them

\usepackage{calc}[2005/08/06]

% for running heads
\usepackage{fancyhdr}

%%%%%%%%%%%%%%%%%%%%%%%%%%%%%%%%%%%%%%%%%%%%%%%%%%%%%%%%%%%%%%%%%
%%%% Interlude:  Set up PRINTING (default) or SCREEN VIEWING %%%%
%%%%%%%%%%%%%%%%%%%%%%%%%%%%%%%%%%%%%%%%%%%%%%%%%%%%%%%%%%%%%%%%%

% ForPrinting=true                     false (default)
% Asymmetric margins                   Symmetric margins
% 1 : 1.62 text block aspect ratio     3 : 4 text block aspect ratio
% Black hyperlinks                     Blue hyperlinks
% Start major marker pages recto       No blank verso pages
%
% Chapter-like ``Sections'' start both recto and verso in the scanned
% book. This behavior has been retained.
\newboolean{ForPrinting}

%% UNCOMMENT the next line for a PRINT-OPTIMIZED VERSION of the text %%
%\setboolean{ForPrinting}{true}

%% Initialize values to ForPrinting=false
\newcommand{\Margins}{hmarginratio=1:1}     % Symmetric margins
\newcommand{\HLinkColor}{blue}              % Hyperlink color
\newcommand{\PDFPageLayout}{SinglePage}
\newcommand{\TransNote}{Transcriber's Note}
\newcommand{\TransNoteCommon}{%
  The camera-quality files for this public-domain ebook may be
  downloaded \textit{gratis} at
  \begin{center}
    \texttt{www.gutenberg.org/ebooks/\ebook}.
  \end{center}

  This ebook was produced using OCR text provided by the University of
  Toronto Robarts Library through the Internet Archive.
  \bigskip

  Minor typographical corrections and presentational changes have been
  made without comment.
  \bigskip
}

\newcommand{\TransNoteText}{%
  \TransNoteCommon

  This PDF file is optimized for screen viewing, but may easily be
  recompiled for printing. Please consult the preamble of the \LaTeX\
  source file for instructions and other particulars.
}
%% Re-set if ForPrinting=true
\ifthenelse{\boolean{ForPrinting}}{%
  \renewcommand{\Margins}{hmarginratio=2:3} % Asymmetric margins
  \renewcommand{\HLinkColor}{black}         % Hyperlink color
  \renewcommand{\PDFPageLayout}{TwoPageRight}
  \renewcommand{\TransNote}{Transcriber's Note}
  \renewcommand{\TransNoteText}{%
    \TransNoteCommon

    This PDF file is optimized for printing, but may easily be
    recompiled for screen viewing. Please consult the preamble
    of the \LaTeX\ source file for instructions and other particulars.
  }
}{% If ForPrinting=false, don't skip to recto
  \renewcommand{\cleardoublepage}{\clearpage}
}
%%%%%%%%%%%%%%%%%%%%%%%%%%%%%%%%%%%%%%%%%%%%%%%%%%%%%%%%%%%%%%%%%
%%%%  End of PRINTING/SCREEN VIEWING code; back to packages  %%%%
%%%%%%%%%%%%%%%%%%%%%%%%%%%%%%%%%%%%%%%%%%%%%%%%%%%%%%%%%%%%%%%%%

\ifthenelse{\boolean{ForPrinting}}{%
  \setlength{\paperwidth}{8.5in}%
  \setlength{\paperheight}{11in}%
% ~1:1.62
  \usepackage[body={4.5in,7.3in},\Margins]{geometry}[2002/07/08]
}{%
  \setlength{\paperwidth}{4.75in}%
  \setlength{\paperheight}{7in}%
  \raggedbottom
% ~3:4
  \usepackage[body={4.5in,6in},\Margins,includeheadfoot]{geometry}[2002/07/08]
}

\providecommand{\ebook}{00000}    % Overridden during white-washing
\usepackage[pdftex,
  hyperref,
  hyperfootnotes=false,
  pdftitle={The Project Gutenberg eBook \#\ebook: Relativity},
  pdfauthor={Albert Einstein},
  pdfkeywords={University of Toronto, The Internet Archive, Andrew D. Hwang},
  pdfstartview=Fit,    % default value
  pdfstartpage=1,      % default value
  pdfpagemode=UseNone, % default value
  bookmarks=true,      % default value
  linktocpage=false,   % default value
  pdfpagelayout=\PDFPageLayout,
  pdfdisplaydoctitle,
  pdfpagelabels=true,
  bookmarksopen=true,
  bookmarksopenlevel=-1,
  colorlinks=true,
  linkcolor=\HLinkColor]{hyperref}[2007/02/07]


%%%% Fixed-width environment to format PG boilerplate %%%%
\newenvironment{PGtext}{%
\begin{alltt}
\fontsize{8.1}{9}\ttfamily\selectfont}%
{\end{alltt}}

%% No hrule in page header
\renewcommand{\headrulewidth}{0pt}

% Top-level footnote numbers restart on each page
\MakePerPage{footnote}

% Running heads
\newcommand{\FlushRunningHeads}{\clearpage\fancyhf{}\cleardoublepage}
\newcommand{\InitRunningHeads}{%
  \setlength{\headheight}{15pt}
  \pagestyle{fancy}
  \thispagestyle{plain}
  \ifthenelse{\boolean{ForPrinting}}
             {\fancyhead[RO,LE]{\thepage}}
             {\fancyhead[R]{\thepage}}
}

\newcommand{\SetOddHead}[1]{%
  \fancyhead[CO]{\textbf{\MakeUppercase{#1}}}
}

\newcommand{\SetEvenHead}[1]{%
  \fancyhead[CE]{\textbf{\MakeUppercase{#1}}}
}

\newcommand{\BookMark}[2]{\phantomsection\pdfbookmark[#1]{#2}{#2}}

% ToC formatting
\AtBeginDocument{\renewcommand{\contentsname}%
  {\protect\thispagestyle{plain}%
    \protect\centering\normalfont\large\textbf{CONTENTS}}}

\newcommand{\ToCFont}{\centering\normalfont\normalsize\scshape}
\newcommand{\TableofContents}{%
  \FlushRunningHeads
  \InitRunningHeads
  \SetOddHead{Contents}
  \BookMark{0}{Contents}
  \tableofcontents
}

% For internal bookkeeping
\newcommand{\ToCAnchor}{}

%\ToCLine[type]{<label>}{Title}{xref}
\newcommand{\ToCLine}[4][chapter]{%
  \label{toc:#4}%
  \ifthenelse{\not\equal{\pageref{toc:#4}}{\ToCAnchor}}{%
    \renewcommand{\ToCAnchor}{\pageref{toc:#4}}%
    \noindent\makebox[\textwidth][r]{\hfill\scriptsize PAGE}\\%
  }{}%
  \settowidth{\TmpLen}{\;\pageref{#4}}%
  \noindent\strut\parbox[b]{\textwidth-\TmpLen}{\small%
    \ifthenelse{\not\equal{#2}{}}{% Write unit number at start of line
      \ifthenelse{\equal{#1}{appendix}}{%
        \settowidth{\TmpLen}{III.}% Widest appendix number
      }{%
        \settowidth{\TmpLen}{XXVIII.}% Widest chapter number
      }
      \makebox[\TmpLen][r]{#2}\hspace{0.5em}%
    }{}% Empty second argument => no unit number
    \raggedright\hangindent6em #3\dotfill}%
  \makebox[\TmpLen][r]{\pageref{#4}}%
}

% Index formatting
\makeindex
\makeatletter
\renewcommand{\@idxitem}{\par\hangindent 30\p@\global\let\idxbrk\nobreak}
\renewcommand\subitem{\idxbrk\@idxitem --- \let\idxbrk\relax}
\renewcommand\subsubitem{\idxbrk\@idxitem --- --- \let\idxbrk\relax}
\renewcommand{\indexspace}{\par\penalty-3000 \vskip 10pt plus5pt minus3pt\relax}

\renewenvironment{theindex}{%
  \setlength\columnseprule{0.5pt}\setlength\columnsep{18pt}%
  \cleardoublepage
  \phantomsection
  \label{index}
  \addtocontents{toc}{\ToCLine{}{\textsc{Index}}{index}}
  \SetOddHead{Index}
  \BookMark{0}{Index}
  \begin{multicols}{2}[\SectTitle{Index}\small]% ** N.B. font size
    \setlength\parindent{0pt}\setlength\parskip{0pt plus 0.3pt}%
    \thispagestyle{plain}\let\item\@idxitem\raggedright%
  }{%
  \end{multicols}\FlushRunningHeads
}
\makeatother

% Allows \Part to communicate with \Chapter
\newboolean{StartPart}
\setboolean{StartPart}{false}

\newcommand{\SectTitle}[2][\large]{%
  \section*{\centering#1\MakeUppercase{#2}}
}
\newcommand{\SectSubtitle}[2][\normalsize]{%
  \subsection*{\centering#1\MakeUppercase{#2}}
}

\newcommand{\Part}[3]{%
  \setboolean{StartPart}{true}
  \ifthenelse{\equal{#1}{I}}{%
    \mainmatter
    \begin{center}
      \textbf{\LARGE RELATIVITY}
    \end{center}
  }{%
    \FlushRunningHeads
  }%
  \InitRunningHeads
  \BookMark{-1}{Part #1. #2}
  \label{part:#1}
  \SetEvenHead{Relativity}
  \SetOddHead{#3}
  \addtocontents{toc}{\protect\section*{\protect\ToCFont  PART #1}}
  \addtocontents{toc}{\protect\subsection*{\protect\ToCFont #2}}
  \SectTitle[\Large]{Part #1}
  \SectSubtitle{#2}
  \bigskip
}

%\Chapter[PDF name]{Number.}{Heading title}
\newcommand{\Chapter}[3][]{%
  \ifthenelse{\boolean{StartPart}}{%
    \setboolean{StartPart}{false}%
  }{%
    \newpage
  }
  \BookMark{0}{#2. #3}
  \label{chapter:#2}
  \thispagestyle{plain}
  \addtocontents{toc}{\ToCLine{#2.}{#3}{chapter:#2}}
  \SectTitle{#2}
  \SectSubtitle{#3}
}

\newcommand{\Section}[1]{%
  \newpage
  \thispagestyle{plain}
  \SectTitle{#1}
}

\newcommand{\Subsection}[2]{%
  \subsection*{\centering\normalsize\normalfont(\Item{#1}) \textsc{#2}}
  \ifthenelse{\not\equal{#1}{}}{%
    \phantomsection
    \label{subsection:#1}
    \addtocontents{toc}{%
      \ToCLine{(\protect\Item{#1})}{#2}{subsection:#1}%
    }%
  }{}%
}

\newcommand{\Bibsection}[1]{%
  \subsection*{\centering\normalsize\normalfont\textsc{#1}}
}

\newcommand{\Preface}{%
  \FlushRunningHeads
  \InitRunningHeads
  \SetOddHead{Relativity}
  \SetEvenHead{Relativity}
  \BookMark{0}{Preface}
  \SectTitle{Preface}%
}

\newcommand{\Appendix}[3]{%
  \clearpage
  \BookMark{0}{Appendix #1. #2}
  \label{appendix:#1}
  \thispagestyle{plain}
  \SetOddHead{Appendix #1}%
  \ifthenelse{\equal{#1}{I}}{%
    \addtocontents{toc}{\protect\section*{\protect\ToCFont  APPENDICES}}
  }{}
  \addtocontents{toc}{\ToCLine[appendix]{#1.}{#2 #3}{appendix:#1}}
  \SectTitle{Appendix #1}
  \subsection*{\centering\normalsize\normalfont%
    \MakeUppercase{#2} \small\textsc{#3}}
}

\newcommand{\Bibliography}[1]{%
  \cleardoublepage
  \phantomsection
  \label{biblio}
  \addtocontents{toc}{\ToCLine{}{\textsc{Bibliography}}{biblio}}
  \thispagestyle{plain}
  \SetOddHead{Bibliography}
  \BookMark{0}{Bibliography}
  \SectTitle{Bibliography}%
  \SectSubtitle{#1}%
}

\renewenvironment{itemize}{%
  \begin{list}{}{\setlength{\topsep}{4pt plus 8pt}%
      \setlength{\itemsep}{0pt plus 2pt}%
      \setlength{\parsep}{4pt plus 2pt}%
      \setlength{\leftmargin}{4em}}}{\end{list}}

\newenvironment{CenterPage}{%
  \thispagestyle{empty}%
  \null\vfill%
  \begin{center}
  }{%
  \end{center}
  \vfill%
}

\newenvironment{PubInfo}{%
  \newpage
  \begin{CenterPage}
    \footnotesize
    \settowidth{\TmpLen}{\textit{This Translation was first Published}\qquad}
    \begin{tabular}{p{\TmpLen}@{\,}c}%
    }{%
    \end{tabular}
  \end{CenterPage}
}

\newcommand{\PubRow}[2]{%
  \textit{#1}\dotfill & \textit{#2} \\
}

\newcommand{\Signature}[2][]{%
  \setlength{\TmpLen}{\textwidth-2\parindent}%
  \bigskip%
  \parbox{\TmpLen}{\centering\small#1\hfill#2}%
}

\newcommand{\Bibitem}[2]{%
\par\noindent\hangindent2\parindent\textit{#1}: #2\medskip%
}

\newcommand{\ColHead}[3]{%
\multicolumn{#1}{c}{\settowidth{\TmpLen}{#2}%
  \parbox[c]{\TmpLen}{\centering#3\medskip}}%
}

\newcommand{\Input}[2][]
  {\ifthenelse{\equal{#1}{}}
  {\includegraphics{./images/#2.pdf}}
  {\includegraphics[width=#1]{./images/#2.pdf}}%
}

\newcounter{figno}
\newcommand{\Figure}[2][0.8\textwidth]{%
\begin{figure}[hbt!]
  \refstepcounter{figno}
  \centering
  \Input[#1]{#2}
  \caption{\textsc{Fig}.~\thefigno.}
  \label{fig:\thefigno}
\end{figure}
}

\newcommand{\WFigure}[2]{%
\begin{wrapfigure}{o}{#1}
  \refstepcounter{figno}
  \centering
  \Input[#1]{#2}
  \caption{\textsc{Fig}.~\thefigno.}
  \label{fig:\thefigno}
\end{wrapfigure}
}

\newcommand{\First}[1]{\textsc{\large #1}}

% For corrections.
\newcommand{\Change}[2]{#2}
\newcommand{\Add}[1]{\Change{}{#1}}

\newcommand{\PageSep}[1]{\ignorespaces}
\setlength{\emergencystretch}{1em}

\newlength{\TmpLen}

\DeclareInputText{176}{\ifmmode{{}^\circ}\else\textdegree\fi}
\DeclareInputText{183}{\ifmmode\cdot\else\textperiodcentered\fi}

\newcommand{\Tag}[1]{%
  \phantomsection
  \label{eqn:#1}
  \tag*{\ensuremath{#1}}
}

% and links
\newcommand{\Eqref}[1]{\hyperref[eqn:#1]{\ensuremath{#1}}}
\newcommand{\Figref}[1]{\hyperref[fig:#1]{Fig.~#1}}
\newcommand{\Partref}[1]{\hyperref[part:#1]{Part~#1}}
\newcommand{\Sectionref}[1]{\hyperref[chapter:#1]{Section~#1}}
\newcommand{\Srefno}[1]{\hyperref[chapter:#1]{#1}}
\newcommand{\Appendixref}[1]{\hyperref[appendix:#1]{Appendix~#1}}

\newcommand{\ie}{\textit{i.e.}}
\newcommand{\eg}{\textit{e.g.}}
\newcommand{\NB}{\textit{N.B.}}
\newcommand{\Item}[1]{\textit{#1}}

\newcommand{\itema}{(\Item{a})}
\newcommand{\itemb}{(\Item{b})}
\newcommand{\itemc}{(\Item{c})}

\newcommand{\Z}{\phantom{0}}

%%%%%%%%%%%%%%%%%%%%%%%% START OF DOCUMENT %%%%%%%%%%%%%%%%%%%%%%%%%%
\begin{document}
\pagenumbering{Alph}
\pagestyle{empty}
\BookMark{-1}{Front Matter}
%%%% PG BOILERPLATE %%%%
\BookMark{0}{PG Boilerplate}
\begin{center}
\begin{minipage}{\textwidth}
\small
\begin{PGtext}
The Project Gutenberg EBook of Relativity: The Special and the General
Theory, by Albert Einstein

This eBook is for the use of anyone anywhere at no cost and with
almost no restrictions whatsoever.  You may copy it, give it away or
re-use it under the terms of the Project Gutenberg License included
with this eBook or online at www.gutenberg.org


Title: Relativity: The Special and the General Theory
       A Popular Exposition, 3rd ed.

Author: Albert Einstein

Translator: Robert W. Lawson

Release Date: May 15, 2011 [EBook #36114]

Language: English

Character set encoding: ISO-8859-1

*** START OF THIS PROJECT GUTENBERG EBOOK RELATIVITY ***
\end{PGtext}
\end{minipage}
\end{center}
\newpage
%%%% Credits and transcriber's note %%%%
\begin{center}
\begin{minipage}{\textwidth}
\begin{PGtext}
Produced by Andrew D. Hwang. (This ebook was produced using
OCR text generously provided by the University of Toronto
Robarts Library through the Internet Archive.)
\end{PGtext}
\end{minipage}
\end{center}
\vfill

\begin{minipage}{0.85\textwidth}
\small
\BookMark{0}{Transcriber's Note}
\subsection*{\centering\normalfont\scshape%
\normalsize\MakeLowercase{\TransNote}}%

\raggedright
\TransNoteText
\end{minipage}
%%%%%%%%%%%%%%%%%%%%%%%%%%% FRONT MATTER %%%%%%%%%%%%%%%%%%%%%%%%%%
\frontmatter
\pagestyle{empty}
\begin{center}
\bfseries \Huge RELATIVITY \\
\medskip
\normalsize THE SPECIAL \textit{\&} THE GENERAL THEORY \\
\medskip
\small A POPULAR EXPOSITION
\vfill

\footnotesize BY \\
\Large ALBERT EINSTEIN, Ph.D. \\
\smallskip\normalfont\scriptsize
PROFESSOR OF PHYSICS IN THE UNIVERSITY OF BERLIN
\vfill

\footnotesize AUTHORISED TRANSLATION BY \\
\normalsize \textbf{ROBERT W. LAWSON, D.Sc.} \\
\smallskip\scriptsize UNIVERSITY OF SHEFFIELD
\vfill

\footnotesize WITH FIVE DIAGRAMS \\
AND A PORTRAIT OF THE AUTHOR
\vfill\vfill

THIRD EDITION
\vfill\vfill


\normalsize\bfseries METHUEN \& CO. LTD. \\
36 ESSEX STREET W.C. \\
LONDON
\end{center}
\PageSep{iv}
\begin{PubInfo}
\PubRow{This Translation was first Published}{August 19th 1920}
\PubRow{Second Edition}{September 1920}
\PubRow{Third Edition}{1920}
\end{PubInfo}
\PageSep{v}


\Preface

\First{The} present book is intended, as far as possible,
to give an exact insight into the theory of Relativity
to those readers who, from a general
scientific and philosophical point of view, are interested
in the theory, but who are not conversant with the
mathematical apparatus\footnote
  {The mathematical fundaments of the special theory of
  relativity are to be found in the original papers of H.~A. Lorentz,
  A.~Einstein, H.~Minkowski, published under the title \textit{Das
  Relativit�tsprinzip} (The Principle of Relativity) in B.~G.
  Teubner's collection of monographs \textit{Fortschritte der mathematischen
  Wissenschaften} (Advances in the Mathematical
  Sciences), also in M.~Laue's exhaustive book \textit{Das Relativit�tsprinzip}---published
  by Friedr.\ Vieweg \&~Son, Braunschweig.
  The general theory of relativity, together with the necessary
  parts of the theory of invariants, is dealt with in the author's
  book \textit{Die Grundlagen der allgemeinen Relativit�tstheorie} (The
  Foundations of the General Theory of Relativity) Joh.\ Ambr.\
  Barth,~1916; this book assumes some familiarity with the special
  theory of relativity.}
of theoretical physics. The
work presumes a standard of education corresponding
to that of a university matriculation examination,
and, despite the shortness of the book, a fair amount
of patience and force of will on the part of the reader.
The author has spared himself no pains in his endeavour
\PageSep{vi}
to present the main ideas in the simplest and most intelligible
form, and on the whole, in the sequence and connection
in which they actually originated. In the interest
of clearness, it appeared to me inevitable that I should
repeat myself frequently, without paying the slightest
attention to the elegance of the presentation. I adhered
scrupulously to the precept of that brilliant theoretical
physicist L.~Boltzmann, according to whom matters of
elegance ought to be left to the tailor and to the cobbler.
I make no pretence of having withheld from the reader
difficulties which are inherent to the subject. On the
other hand, I have purposely treated the empirical
physical foundations of the theory in a ``step-motherly''
fashion, so that readers unfamiliar with physics may
not feel like the wanderer who was unable to see the
forest for trees. May the book bring some one a few
happy hours of suggestive thought!

\Signature[\textit{December}, 1916]{A. EINSTEIN}


\SectTitle{Note to the Third Edition}

\First{In} the present year (1918) an excellent and detailed
manual on the general theory of relativity, written
by H.~Weyl, was published by the firm Julius
Springer (Berlin). This book, entitled \textit{Raum---Zeit---Materie}
(Space---Time---Matter), may be warmly recommended
to mathematicians and physicists.
\PageSep{vii}


\Section{Biographical Note}

\First{Albert Einstein} is the son of German-Jewish
parents. He was born in~1879 in the
town of Ulm, W�rtemberg, Germany. His
schooldays were spent in Munich, where he attended
the \textit{Gymnasium} until his sixteenth year. After leaving
school at Munich, he accompanied his parents to Milan,
whence he proceeded to Switzerland six months later
to continue his studies.

From 1896 to 1900 Albert Einstein studied mathematics
and physics at the Technical High School in
Zurich, as he intended becoming a secondary school
(\textit{Gymnasium}) teacher. For some time afterwards he
was a private tutor, and having meanwhile become
naturalised, he obtained a post as engineer in the Swiss
Patent Office in~1902 which position he occupied till
1909. The main ideas involved in the most important
of Einstein's theories date back to this period. Amongst
these may be mentioned: \textit{The Special Theory of Relativity},
\textit{Inertia of Energy}, \textit{Theory of the Brownian Movement},
and the \textit{Quantum-Law of the Emission and Absorption of Light}~(1905).
These were followed some years
\PageSep{viii}
later by the \textit{Theory of the Specific Heat of Solid Bodies},
and the fundamental idea of the \textit{General Theory of
Relativity}.

During the interval 1909~to~1911 he occupied the post
of Professor \textit{Extraordinarius} at the University of Zurich,
afterwards being appointed to the University of Prague,
Bohemia, where he remained as Professor \textit{Ordinarius}
until~1912. In the latter year Professor Einstein
accepted a similar chair at the \textit{Polytechnikum}, Zurich,
and continued his activities there until~1914, when he
received a call to the Prussian Academy of Science,
Berlin, as successor to Van't~Hoff. Professor Einstein
is able to devote himself freely to his studies at the
Berlin Academy, and it was here that he succeeded in
completing his work on the \textit{General Theory of Relativity}
(1915--17). Professor Einstein also lectures on various
special branches of physics at the University of Berlin,
and, in addition, he is Director of the Institute for
Physical Research of the \textit{Kaiser Wilhelm Gesellschaft}.

Professor Einstein has been twice married. His first
wife, whom he married at Berne in~1903, was a fellow-student
from Serbia. There were two sons of this
marriage, both of whom are living in Zurich, the elder
being sixteen years of age. Recently Professor Einstein
married a widowed cousin, with whom he is now living
in Berlin.

\Signature{R. W. L.}
\PageSep{ix}

\Section{Translator's Note}

\First{In} presenting this translation to the English-reading
public, it is hardly necessary for me to
enlarge on the Author's prefatory remarks, except
to draw attention to those additions to the book which
do not appear in the original.

At my request, Professor Einstein kindly supplied
me with a portrait of himself, by one of Germany's
most celebrated artists. \Appendixref{III}, on ``The
Experimental Confirmation of the General Theory of
Relativity,'' has been written specially for this translation.
Apart from these valuable additions to the book,
I have included a biographical note on the Author,
and, at the end of the book, an Index and a list of
English references to the subject. This list, which is more
suggestive than exhaustive, is intended as a guide to those
readers who wish to pursue the subject farther.

I desire to tender my best thanks to my colleagues
Professor S.~R. Milner,~D.Sc., and Mr.~W.~E. Curtis,
A.R.C.Sc.,~F.R.A.S., also to my friend Dr.~Arthur
Holmes, A.R.C.Sc.,~F.G.S., of the Imperial College,
for their kindness in reading through the manuscript,
\PageSep{x}
for helpful criticism, and for numerous suggestions. I
owe an expression of thanks also to Messrs.\ Methuen
for their ready counsel and advice, and for the care
they have bestowed on the work during the course of
its publication.

\Signature{ROBERT W. LAWSON}

\noindent\textsc{The Physics Laboratory} \\
\hspace*{\parindent}\textsc{The University of Sheffield} \\
\hspace*{3\parindent}\textit{June} 12, 1920
\PageSep{xi}
\TableofContents % [** TN: Auto-generate the table of contents]
\iffalse %%%% Start of table of contents text %%%%
CONTENTS

PART I

THE SPECIAL THEORY OF RELATIVITY

PAGE

  I. Physical Meaning of Geometrical Propositions . 1
 II. The System of Co-ordinates . 5
III. Space and Time in Classical Mechanics . . 9
 IV. The Galileian System of Co-ordinates . .11
  V. The Principle of Relativity (in the Restricted
     Sense) . . . . . .12
 VI. The Theorem of the Addition of Velocities employed
     in Classical Mechanics . . 16
VII. The Apparent Incompatibility of the Law of
     Propagation of Light with the Principle of
     Relativity . . . . 17

VIII. On the Idea of Time in Physics . . .21
  IX. The Relativity of Simultaneity . . .25
   X. On the Relativity of the Conception of Distance 28
  XI. The Lorentz Transformation . . .30
 XII. The Behaviour of Measuring-Rods and Clocks
     in Motion . . . . 35
\PageSep{xii}
XIII. Theorem of the Addition of Velocities. The
     Experiment of Fizeau . . 3 %[** TN: Edge of page cut off]
 XIV. The Heuristic Value of the Theory of Relativity 4
  XV. General Results of the Theory . . .4,
 XVI. Experience and the Special Theory of Relativity 4
XVII. Minkowski's Four-dimensional Space . . 5;

PART II
THE GENERAL THEORY OF RELATIVITY

XVIII. Special and General Principle of Relativity . 5
  XIX. The Gravitational Field . . . .6
   XX. The Equality of Inertial and Gravitational Mass
      as an Argument for the General Postulate
      of Relativity .....
  XXI. In what Respects are the Foundations of Classical
      Mechanics and of the Special Theory
      of Relativity unsatisfactory? .
 XXII. A Few Inferences from the General Principle of
      Relativity .....
XXIII. Behaviour of Clocks and Measuring-Rods on a
      Rotating Body of Reference .
 XXIV. Euclidean and Non-Euclidean Continuum
  XXV. Gaussian Co-ordinates ....
 XXVI. The Space-time Continuum of the Special
      Theory of Relativity considered as a
      Euclidean Continuum
\PageSep{xiii}
PAGE

 XXVII. The Space-time Continuum of the General
      Theory of Relativity is not a Euclidean
      Continuum . . . . 93
XXVIII. Exact Formulation of the General Principle of
      Relativity . . . . 97
  XXIX. The Solution of the Problem of Gravitation on
      the Basis of the General Principle of
      Relativity ..... 100

PART III

CONSIDERATIONS ON THE UNIVERSE
AS A WHOLE

   XXX. Cosmological Difficulties of Newton's Theory 105
  XXXI. The Possibility of a ``Finite'' and yet ``Unbounded''
       Universe. . . . 108
 XXXII. The Structure of Space according to the
       General Theory of Relativity . . 113

APPENDICES

  I. Simple Derivation of the Lorentz Transformation . 115
 II. Minkowski's Four-dimensional Space (``World'')
     [Supplementary to Section XVII.] . . 121
III. The Experimental Confirmation of the General
     Theory of Relativity . . . .123
(a) Motion of the Perihelion of Mercury . 124
(b) Deflection of Light by a Gravitational Field 126
(c) Displacement of Spectral Lines towards the
    Red . . . . . 129

BIBLIOGRAPHY . . . . . . 133

INDEX . . . . . . .135
\fi %%%% End of table of contents text %%%%
\PageSep{xiv}
\FlushRunningHeads
\begin{CenterPage}
  \bfseries\LARGE RELATIVITY \\[8pt]
  \normalsize THE SPECIAL AND THE GENERAL THEORY
\end{CenterPage}
\PageSep{1}
\index{Manifold|see{Continuum}}%


\Part{I}{The Special Theory of Relativity}{Special Theory of Relativity}
\index{Special theory of relativity|(}%

\Chapter[Geometrical Propositions]
{I}{Physical Meaning of Geometrical
Propositions}

\First{In} your schooldays most of you who read this
\index{Euclidean geometry}%
book made acquaintance with the noble building of
Euclid's geometry, and you remember---perhaps
with more respect than love---the magnificent structure,
on the lofty staircase of which you were chased about
for uncounted hours by conscientious teachers. By
reason of your past experience, you would certainly
regard everyone with disdain who should pronounce even
the most out-of-the-way proposition of this science to
be untrue. But perhaps this feeling of proud certainty
would leave you immediately if some one were to ask
you: ``What, then, do you mean by the assertion that
these propositions are true?'' Let us proceed to give
this question a little consideration.

Geometry sets out from certain conceptions such as
\index{Geometrical ideas!truth of|(}%
``plane,'' ``point,'' and ``straight line,'' with which
\index{Plane}%
\index{Point}%
\index{Straight line|(}%
\PageSep{2}
we are able to associate more or less definite ideas, and
from certain simple propositions (axioms) which,
\index{Axioms}%
\index{Axioms!truth of}%
\index{Geometrical ideas!propositions}%
in virtue of these ideas, we are inclined to accept as
``true.'' Then, on the basis of a logical process, the
justification of which we feel ourselves compelled to
admit, all remaining propositions are shown to follow
from those axioms, \ie\ they are proven. A proposition
is then correct (``true'') when it has been derived in the
recognised manner from the axioms. The question
of the ``truth'' of the individual geometrical propositions
\index{Truth@{``Truth''}}%
is thus reduced to one of the ``truth'' of the
axioms. Now it has long been known that the last
question is not only unanswerable by the methods of
geometry, but that it is in itself entirely without meaning.
We cannot ask whether it is true that only one
straight line goes through two points. We can only
say that Euclidean geometry deals with things called
\index{Euclidean geometry}%
``straight lines,'' to each of which is ascribed the property
of being uniquely determined by two points
situated on it. The concept ``true'' does not tally with
the assertions of pure geometry, because by the word
``true'' we are eventually in the habit of designating
always the correspondence with a ``real'' object;
geometry, however, is not concerned with the relation
of the ideas involved in it to objects of experience, but
only with the logical connection of these ideas among
themselves.

It is not difficult to understand why, in spite of this,
we feel constrained to call the propositions of geometry
``true.'' Geometrical ideas correspond to more or less
\index{Geometrical ideas}%
exact objects in nature, and these last are undoubtedly
the exclusive cause of the genesis of those ideas. Geometry
ought to refrain from such a course, in order to
\PageSep{3}
give to its structure the largest possible logical unity.
The practice, for example, of seeing in a ``distance''
two marked positions on a practically rigid body is
something which is lodged deeply in our habit of thought.
We are accustomed further to regard three points as
being situated on a straight line, if their apparent
positions can be made to coincide for observation with
one eye, under suitable choice of our place of observation.

If, in pursuance of our habit of thought, we now
supplement the propositions of Euclidean geometry by
\index{Euclidean geometry!propositions of}%
the single proposition that two points on a practically
rigid body always correspond to the same distance
\index{Distance (line-interval)}%
(line-interval), independently of any changes in position
to which we may subject the body, the propositions of
Euclidean geometry then resolve themselves into propositions
on the possible relative position of practically
\index{Relative!position}%
rigid bodies.\footnote
  {It follows that a natural object is associated also with a
  straight line. Three points $A$,~$B$ and~$C$ on a rigid body thus
  lie in a straight line when, the points $A$~and~$C$ being given, $B$
  is chosen such that the sum of the distances $AB$~and~$BC$ is as
  short as possible. This incomplete suggestion will suffice for
  our present purpose.}
Geometry which has been supplemented
in this way is then to be treated as a branch of physics.
We can now legitimately ask as to the ``truth'' of
geometrical propositions interpreted in this way, since
we are justified in asking whether these propositions
are satisfied for those real things we have associated
with the geometrical ideas. In less exact terms we can
\index{Geometrical ideas}%
express this by saying that by the ``truth'' of a geometrical
proposition in this sense we understand its
validity for a construction with ruler and compasses.
\index{Straight line|)}%
\PageSep{4}

Of course the conviction of the ``truth'' of geometrical
propositions in this sense is founded exclusively
on rather incomplete experience. For the present we
shall assume the ``truth'' of the geometrical propositions,
then at a later stage (in the general theory of
relativity) we shall see that this ``truth'' is limited,
and we shall consider the extent of its limitation.
\index{Geometrical ideas!truth of|)}%
\PageSep{5}


\Chapter{II}{The System of Co-ordinates}
\index{System of co-ordinates}%

\First{On} the basis of the physical interpretation of distance
\index{Distance (line-interval)}%
\index{Distance (line-interval)!physical interpretation of}%
\index{Measuring-rod}%
\index{Reference-body}%
which has been indicated, we are also
in a position to establish the distance between
two points on a rigid body by means of measurements.
For this purpose we require a ``distance'' (rod~$S$)
which is to be used once and for all, and which we
employ as a standard measure. If, now, $A$~and~$B$ are
two points on a rigid body, we can construct the
line joining them according to the rules of geometry;
then, starting from~$A$, we can mark off the distance~$S$
time after time until we reach~$B$. The number of
these operations required is the numerical measure
of the distance~$AB$. This is the basis of all measurement
of length.\footnote
  {Here we have assumed that there is nothing left over, \ie\
  that the measurement gives a whole number. This difficulty
  is got over by the use of divided measuring-rods, the introduction
  of which does not demand any fundamentally new method.}

Every description of the scene of an event or of the
position of an object in space is based on the specification
of the point on a rigid body (body of reference)
with which that event or object coincides. This applies
not only to scientific description, but also to everyday
life. If I analyse the place specification ``Trafalgar
\index{Place specification}%
\PageSep{6}
Square, London,''\footnote
  {I have chosen this as being more familiar to the English
  reader than the ``Potsdamer Platz, Berlin,'' which is referred to
  in the original. (R.~W.~L.)}
I arrive at the following result.
The earth is the rigid body to which the specification
of place refers; ``Trafalgar Square, London,'' is a
well-defined point, to which a name has been assigned,
and with which the event coincides in space.\footnote
  {It is not necessary here to investigate further the significance
  of the expression ``coincidence in space.'' This conception is
  sufficiently obvious to ensure that differences of opinion are
  scarcely likely to arise as to its applicability in practice.}

This primitive method of place specification deals
\index{Place specification}%
only with places on the surface of rigid bodies, and is
dependent on the existence of points on this surface
which are distinguishable from each other. But we
can free ourselves from both of these limitations without
altering the nature of our specification of position.
\index{Conception of mass!position}%
If, for instance, a cloud is hovering over Trafalgar
Square, then we can determine its position relative to
the surface of the earth by erecting a pole perpendicularly
on the Square, so that it reaches the cloud. The
length of the pole measured with the standard measuring-rod,
\index{Measuring-rod}%
combined with the specification of the position of
the foot of the pole, supplies us with a complete place
specification. On the basis of this illustration, we are
able to see the manner in which a refinement of the conception
of position has been developed.

\itema~We imagine the rigid body, to which the place
specification is referred, supplemented in such a manner
that the object whose position we require is reached by
the completed rigid body.

\itemb~In locating the position of the object, we make
use of a number (here the length of the pole measured
\PageSep{7}
with the measuring-rod) instead of designated points of
reference.

\itemc~We speak of the height of the cloud even when the
pole which reaches the cloud has not been erected.
By means of optical observations of the cloud from
different positions on the ground, and taking into account
the properties of the propagation of light, we determine
the length of the pole we should have required in order
to reach the cloud.

From this consideration we see that it will be advantageous
\index{Physics}%
if, in the description of position, it should be
possible by means of numerical measures to make ourselves
independent of the existence of marked positions
(possessing names) on the rigid body of reference. In
\index{Reference-body}%
the physics of measurement this is attained by the
\index{Physics!of measurement}%
application of the Cartesian system of co-ordinates.
\index{Cartesian system of co-ordinates}%

This consists of three plane surfaces perpendicular
to each other and rigidly attached to a rigid body.
Referred to a system of co-ordinates, the scene of any
event will be determined (for the main part) by the
specification of the lengths of the three perpendiculars
or co-ordinates $(x, y, z)$ which can be dropped from the
scene of the event to those three plane surfaces. The
lengths of these three perpendiculars can be determined
by a series of manipulations with rigid measuring-rods
performed according to the rules and methods laid
down by Euclidean geometry.

In practice, the rigid surfaces which constitute the
system of co-ordinates are generally not available;
furthermore, the magnitudes of the co-ordinates are not
actually determined by constructions with rigid rods, but
by indirect means. If the results of physics and astronomy
\index{Astronomy}%
are to maintain their clearness, the physical meaning
\PageSep{8}
of specifications of position must always be sought
in accordance with the above considerations.\footnote
  {A refinement and modification of these views does not become
  necessary until we come to deal with the general theory of
  relativity, treated in the second part of this book.}

We thus obtain the following result: Every description
of events in space involves the use of a rigid body
to which such events have to be referred. The resulting
relationship takes for granted that the laws of Euclidean
\index{Distance (line-interval)}%
\index{Euclidean geometry!propositions of}%
geometry hold for ``distances,'' the ``distance'' being
represented physically by means of the convention of
two marks on a rigid body.
\PageSep{9}


\Chapter{III}{Space and Time in Classical Mechanics}
\index{Classical mechanics}%
\index{Space}%

\Change{}{``}\First{The} purpose of mechanics is to describe how
bodies change their position in space with
\index{Position}%
time.'' I should load my conscience with grave
sins against the sacred spirit of lucidity were I to
formulate the aims of mechanics in this way, without
serious reflection and detailed explanations. Let us
proceed to disclose these sins.

It is not clear what is to be understood here by
\index{Reference-body|(}%
``position'' and ``space.'' I stand at the window of a
railway carriage which is travelling uniformly, and drop
a stone on the embankment, without throwing it. Then,
disregarding the influence of the air resistance, I see the
stone descend in a straight line. A pedestrian who
\index{Straight line}%
observes the misdeed from the footpath notices that the
stone falls to earth in a parabolic curve. I now ask:
Do the ``positions'' traversed by the stone lie ``in
reality'' on a straight line or on a parabola? Moreover,
\index{Parabola}%
what is meant here by motion ``in space''? From the
considerations of the previous section the answer is
self-evident. In the first place, we entirely shun the
vague word ``space,'' of which, we must honestly
acknowledge, we cannot form the slightest conception,
and we replace it by ``motion relative to a
practically rigid body of reference.'' The positions
relative to the body of reference (railway carriage or
embankment) have already been defined in detail in the
\PageSep{10}
preceding section. If instead of ``body of reference''
we insert ``system of co-ordinates,'' which is a useful
\index{System of co-ordinates}%
idea for mathematical description, we are in a position
to say: The stone traverses a straight line relative to a
\index{Straight line}%
system of co-ordinates rigidly attached to the carriage,
but relative to a system of co-ordinates rigidly attached
to the ground (embankment) it describes a parabola.
\index{Parabola}%
With the aid of this example it is clearly seen that there
is no such thing as an independently existing trajectory
\index{Trajectory}%
(lit. ``path-curve''\footnotemark), but only a trajectory relative to a
\index{Path-curve}%
particular body of reference.
\footnotetext{That is, a curve along which the body moves.}

In order to have a \emph{complete} description of the motion,
we must specify how the body alters its position \emph{with
time}; \ie\ for every point on the trajectory it must be
stated at what time the body is situated there. These
data must be supplemented by such a definition of
time that, in virtue of this definition, these time-values
can be regarded essentially as magnitudes (results of
measurements) capable of observation. If we take our
stand on the ground of classical mechanics, we can
satisfy this requirement for our illustration in the
following manner. We imagine two clocks of identical
\index{Clocks}%
construction; the man at the railway-carriage window
is holding one of them, and the man on the footpath
the other. Each of the observers determines
the position on his own reference-body occupied by the
stone at each tick of the clock he is holding in his
hand. In this connection we have not taken account
of the inaccuracy involved by the finiteness of the
velocity of propagation of light. With this and with a
\index{Velocity of light}%
second difficulty prevailing here we shall have to deal
in detail later.
\PageSep{11}


\Chapter{IV}{The Galileian System of Co-ordinates}
\index{Galileian system of co-ordinates}%
\index{System of co-ordinates}%

\First{As} is well known, the fundamental law of the
mechanics of Galilei-Newton, which is known
\index{Galilei}%
\index{Newton}%
as the \emph{law of inertia}, can be stated thus:
\index{Law of inertia}%
A body removed sufficiently far from other bodies
continues in a state of rest or of uniform motion
in a straight line. This law not only says something
about the motion of the bodies, but it also
indicates the reference-bodies or systems of co-ordinates,
permissible in mechanics, which can be used
in mechanical description. The visible fixed stars are
\index{Fixed stars}%
bodies for which the law of inertia certainly holds to a
high degree of approximation. Now if we use a system
of co-ordinates which is rigidly attached to the earth,
then, relative to this system, every fixed star describes
a circle of immense radius in the course of an astronomical
day, a result which is opposed to the statement
\index{Astronomical day}%
of the law of inertia. So that if we adhere to this law
we must refer these motions only to systems of co-ordinates
relative to which the fixed stars do not move
in a circle. A system of co-ordinates of which the state
of motion is such that the law of inertia holds relative to
it is called a ``Galileian system of co-ordinates.'' The
laws of the mechanics of Galilei-Newton can be regarded
as valid only for a Galileian system of co-ordinates.
\index{Reference-body|)}%
\PageSep{12}


\Chapter{V}{The Principle of Relativity (In the
Restricted Sense)}

\First{In} order to attain the greatest possible clearness,
let us return to our example of the railway carriage
supposed to be travelling uniformly. We call its
motion a uniform translation (``uniform'' because
\index{Uniform translation}%
it is of constant velocity and direction, ``translation''
because although the carriage changes its position
relative to the embankment yet it does not rotate
in so doing). Let us imagine a raven flying through
the air in such a manner that its motion, as observed
from the embankment, is uniform and in a straight
line. If we were to observe the flying raven from
the moving railway carriage, we should find that the
motion of the raven would be one of different velocity
and direction, but that it would still be uniform
and in a straight line. Expressed in an abstract
manner we may say: If a mass~$m$ is moving uniformly
in a straight line with respect to a co-ordinate
system~$K$, then it will also be moving uniformly and in a
straight line relative to a second co-ordinate system~$K'$,
provided that the latter is executing a uniform
translatory motion with respect to~$K$. In accordance
with the discussion contained in the preceding section,
it follows that:
\PageSep{13}

If $K$~is a Galileian co-ordinate system, then every other
\index{Galileian system of co-ordinates}%
co-ordinate system~$K'$ is a Galileian one, when, in relation
to~$K$, it is in a condition of uniform motion of translation.
\index{Motion!of heavenly bodies}%
Relative to~$K'$ the mechanical laws of Galilei-Newton
\index{Laws of Galilei-Newton}%
hold good exactly as they do with respect to~$K$.

We advance a step farther in our generalisation when
we express the tenet thus: If, relative to~$K$, $K'$~is a
uniformly moving co-ordinate system devoid of rotation,
then natural phenomena run their course with respect to~$K'$
according to exactly the same general laws as with
respect to~$K$. This statement is called the \emph{principle
of relativity} (in the restricted sense).

As long as one was convinced that all natural phenomena
were capable of representation with the help of
classical mechanics, there was no need to doubt the
\index{Classical mechanics}%
\index{Classical mechanics!truth of}%
validity of this principle of relativity. But in view of
\index{Principle of relativity|(}%
the more recent development of electrodynamics and
\index{Electrodynamics}%
optics it became more and more evident that classical
\index{Optics}%
mechanics affords an insufficient foundation for the
physical description of all natural phenomena. At this
juncture the question of the validity of the principle of
relativity became ripe for discussion, and it did not
appear impossible that the answer to this question
might be in the negative.

Nevertheless, there are two general facts which at the
outset speak very much in favour of the validity of the
principle of relativity. Even though classical mechanics
does not supply us with a sufficiently broad basis for the
theoretical presentation of all physical phenomena,
still we must grant it a considerable measure of ``truth,''
since it supplies us with the actual motions of the
heavenly bodies with a delicacy of detail little short of
wonderful. The principle of relativity must therefore
\PageSep{14}
apply with great accuracy in the domain of \emph{mechanics}.
\index{Classical mechanics}%
But that a principle of such broad generality should
hold with such exactness in one domain of phenomena,
and yet should be invalid for another, is \textit{a~priori} not
very probable.

We now proceed to the second argument, to which,
moreover, we shall return later. If the principle of relativity
(in the restricted sense) does not hold, then the
Galileian co-ordinate systems $K$,~$K'$, $K''$,~etc., which are
\index{Galileian system of co-ordinates}%
moving uniformly relative to each other, will not be
\emph{equivalent} for the description of natural phenomena.
\index{Equivalent}%
In this case we should be constrained to believe that
natural laws are capable of being formulated in a particularly
simple manner, and of course only on condition
that, from amongst all possible Galileian co-ordinate
systems, we should have chosen \emph{one}~($K_{0}$) of a particular
state of motion as our body of reference. We should
\index{Motion}%
then be justified (because of its merits for the description
of natural phenomena) in calling this system ``absolutely
at rest,'' and all other Galileian systems~$K$ ``in motion.''
\index{Rest}%
If, for instance, our embankment were the system~$K_{0}$,
then our railway carriage would be a system~$K$,
relative to which less simple laws would hold than with
respect to~$K_{0}$. This diminished simplicity would be
due to the fact that the carriage~$K$ would be in motion
(\ie\ ``really'') with respect to~$K_{0}$. In the general laws
of nature which have been formulated with reference
to~$K$, the magnitude and direction of the velocity
of the carriage would necessarily play a part. We should
expect, for instance, that the note emitted by an organ-pipe
\index{Organ-pipe, note of}%
placed with its axis parallel to the direction of
travel would be different from that emitted if the axis
of the pipe were placed perpendicular to this direction.
\PageSep{15}
Now in virtue of its motion in an orbit round the sun,
\index{Motion!of heavenly bodies}%
our earth is comparable with a railway carriage travelling
with a velocity of about $30$~kilometres per~second.
If the principle of relativity were not valid we should
therefore expect that the direction of motion of the
earth at any moment would enter into the laws of nature,
and also that physical systems in their behaviour would
be dependent on the orientation in space with respect
to the earth. For owing to the alteration in direction
of the velocity of revolution of the earth in the course
of a year, the earth cannot be at rest relative to the
hypothetical system~$K_{0}$ throughout the whole year.
However, the most careful observations have never
revealed such anisotropic properties in terrestrial physical
\index{Terrestrial space}%
space, \ie\ a physical non-equivalence of different
directions. This is very powerful argument in favour
of the principle of relativity.
\index{Principle of relativity|)}%
\PageSep{16}


\Chapter{VI}{The Theorem of the Addition of Velocities
employed in Classical Mechanics}
\index{Addition of velocities}%
\index{Classical mechanics}%

\First{Let} us suppose our old friend the railway carriage
to be travelling along the rails with a constant
velocity~$v$, and that a man traverses the length of
the carriage in the direction of travel with a velocity~$w$.
How quickly or, in other words, with what velocity~$W$
does the man advance relative to the embankment
during the process? The only possible answer seems to
result from the following consideration: If the man were
to stand still for a second, he would advance relative to
the embankment through a distance~$v$ equal numerically
to the velocity of the carriage. As a consequence of
his walking, however, he traverses an additional distance~$w$
relative to the carriage, and hence also relative to the
embankment, in this second, the distance~$w$ being
numerically equal to the velocity with which he is
walking. Thus in total he covers the distance $W = v + w$
relative to the embankment in the second considered.
We shall see later that this result, which expresses
the theorem of the addition of velocities employed in
classical mechanics, cannot be maintained; in other
words, the law that we have just written down does not
hold in reality. For the time being, however, we shall
assume its correctness.
\PageSep{17}


\Chapter{VII}{The Apparent Incompatibility of the
Law of Propagation of Light with
the Principle of Relativity}
\index{Propagation of light}%

\First{There} is hardly a simpler law in physics than
that according to which light is propagated in
empty space. Every child at school knows, or
believes he knows, that this propagation takes place
in straight lines with a velocity $c = 300,000$~km./sec.
At all events we know with great exactness that this
velocity is the same for all colours, because if this were
not the case, the minimum of emission would not be
observed simultaneously for different colours during
the eclipse of a fixed star by its dark neighbour. By
\index{DeSitter@{De Sitter}}%
\index{Eclipse of star}%
means of similar considerations based on observations
of double stars, the Dutch astronomer De~Sitter
\index{Double stars}%
was also able to show that the velocity of propagation
of light cannot depend on the velocity of motion
of the body emitting the light. The assumption that
this velocity of propagation is dependent on the direction
``in space'' is in itself improbable.

In short, let us assume that the simple law of the
constancy of the velocity of light~$c$ (in vacuum) is
\index{Velocity of light}%
justifiably believed by the child at school. Who would
imagine that this simple law has plunged the conscientiously
thoughtful physicist into the greatest
\PageSep{18}
intellectual difficulties? Let us consider how these
difficulties arise.

Of course we must refer the process of the propagation
of light (and indeed every other process) to a rigid
reference-body (co-ordinate system). As such a system
\index{Reference-body}%
let us again choose our embankment. We shall imagine
the air above it to have been removed. If a ray of
light be sent along the embankment, we see from the
above that the tip of the ray will be transmitted with
the velocity~$c$ relative to the embankment. Now let
us suppose that our railway carriage is again travelling
along the railway lines with the velocity~$v$, and that
its direction is the same as that of the ray of light, but
its velocity of course much less. Let us inquire about
the velocity of propagation of the ray of light relative
to the carriage. It is obvious that we can here apply the
consideration of the previous section, since the ray of
light plays the part of the man walking along relatively
to the carriage. The velocity~$W$ of the man relative
to the embankment is here replaced by the velocity
of light relative to the embankment. $w$~is the required
velocity of light with respect to the carriage, and we
\index{Velocity of light}%
have
\[
w = c - v.
\]
The velocity of propagation of a ray of light relative to
the carriage thus comes out smaller than~$c$.

But this result comes into conflict with the principle
of relativity set forth in \Sectionref{V}. For, like every
other general law of nature, the law of the transmission
of light \textit{in~vacuo} must, according to the principle of
relativity, be the same for the railway carriage as
reference-body as when the rails are the body of reference.
\PageSep{19}
But, from our above consideration, this would
appear to be impossible. If every ray of light is propagated
relative to the embankment with the velocity~$c$,
then for this reason it would appear that another law
of propagation of light must necessarily hold with respect
\index{Propagation of light}%
to the carriage---a result contradictory to the principle
of relativity.

In view of this dilemma there appears to be nothing
else for it than to abandon either the principle of relativity
\index{Principle of relativity}%
or the simple law of the propagation of light \textit{in~vacuo}.
Those of you who have carefully followed the
preceding discussion are almost sure to expect that
we should retain the principle of relativity, which
appeals so convincingly to the intellect because it is so
natural and simple. The law of the propagation of
light \textit{in~vacuo} would then have to be replaced by a
more complicated law conformable to the principle of
relativity. The development of theoretical physics
shows, however, that we cannot pursue this course.
The epoch-making theoretical investigations of H.~A.
Lorentz on the electrodynamical and optical phenomena
\index{Electrodynamics}%
\index{Optics}%
\index{Lorentz, H. A.}%
connected with moving bodies show that experience
in this domain leads conclusively to a theory of electromagnetic
phenomena, of which the law of the constancy
of the velocity of light \textit{in~vacuo} is a necessary consequence.
Prominent theoretical physicists were therefore
more inclined to reject the principle of relativity,
in spite of the fact that no empirical data had been
found which were contradictory to this principle.

At this juncture the theory of relativity entered the
arena. As a result of an analysis of the physical conceptions
of time and space, it became evident that \emph{in
\index{Space!conception of}%
\index{Time!conception of}%
reality there is not the least incompatibility between the
\PageSep{20}
principle of relativity and the law of propagation of light},
\index{Principle of relativity}%
\index{Propagation of light}%
and that by systematically holding fast to both these
laws a logically rigid theory could be arrived at. This
theory has been called the \emph{special theory of relativity}
\index{Special theory of relativity}%
to distinguish it from the extended theory, with which
we shall deal later. In the following pages we shall
present the fundamental ideas of the special theory of
relativity.
\PageSep{21}


\Chapter{VIII}{On the Idea of Time in Physics}
\index{Time!in Physics}%

\First{Lightning} has struck the rails on our railway
embankment at two places $A$~and~$B$ far distant
from each other. I make the additional assertion
that these two lightning flashes occurred simultaneously.
If I ask you whether there is sense in this statement,
you will answer my question with a decided
``Yes.'' But if I now approach you with the request
to explain to me the sense of the statement more
precisely, you find after some consideration that the
answer to this question is not so easy as it appears at
first sight.

After some time perhaps the following answer would
occur to you: ``The significance of the statement is
clear in itself and needs no further explanation; of
course it would require some consideration if I were to
be commissioned to determine by observations whether
in the actual case the two events took place simultaneously
or not.'' I cannot be satisfied with this answer
for the following reason. Supposing that as a result
of ingenious considerations an able meteorologist were
to discover that the lightning must always strike the
places $A$~and~$B$ simultaneously, then we should be faced
with the task of testing whether or not this theoretical
result is in accordance with the reality. We encounter
\PageSep{22}
the same difficulty with all physical statements in which
the conception ``simultaneous'' plays a part. The
concept does not exist for the physicist until he has the
possibility of discovering whether or not it is fulfilled
in an actual case. We thus require a definition of
simultaneity such that this definition supplies us with
\index{Simultaneity}%
the method by means of which, in the present case, he
can decide by experiment whether or not both the
lightning strokes occurred simultaneously. As long
as this requirement is not satisfied, I allow myself to be
deceived as a physicist (and of course the same applies
if I am not a physicist), when I imagine that I am able
to attach a meaning to the statement of simultaneity.
(I would ask the reader not to proceed farther until he
is fully convinced on this point.)

After thinking the matter over for some time you
then offer the following suggestion with which to test
simultaneity. By measuring along the rails, the
connecting line~$AB$ should be measured up and an
observer placed at the mid-point~$M$ of the distance~$AB$.
This observer should be supplied with an arrangement
(\eg\ two mirrors inclined at~$90�$) which allows him
visually to observe both places $A$~and~$B$ at the same
time. If the observer perceives the two flashes of
lightning at the same time, then they are simultaneous.

I am very pleased with this suggestion, but for all
that I cannot regard the matter as quite settled, because
I feel constrained to raise the following objection:
``Your definition would certainly be right, if I only
knew that the light by means of which the observer
at~$M$ perceives the lightning flashes travels along the
length $A\longrightarrow M$ with the same velocity as along the
length $B\longrightarrow M$. But an examination of this supposition
\PageSep{23}
would only be possible if we already had at our
disposal the means of measuring time. It would thus
appear as though we were moving here in a logical circle.''

After further consideration you cast a somewhat
disdainful glance at me---and rightly so---and you
declare: ``I maintain my previous definition nevertheless,
because in reality it assumes absolutely nothing
about light. There is only \emph{one} demand to be made of
the definition of simultaneity, namely, that in every
real case it must supply us with an empirical decision
as to whether or not the conception that has to
be defined is fulfilled. That my definition satisfies
this demand is indisputable. That light requires the
same time to traverse the path $A\longrightarrow M$ as for the path
$B\longrightarrow M$ is in reality neither a \emph{supposition nor a hypothesis}
about the physical nature of light, but a \emph{stipulation}
which I can make of my own \Change{freewill}{free will} in order to arrive
at a definition of simultaneity.''

It is clear that this definition can be used to give an
exact meaning not only to \emph{two} events, but to as many
events as we care to choose, and independently of the
positions of the scenes of the events with respect to the
\index{Reference-body}%
body of reference\footnote
  {We suppose further, that, when three events $A$,~$B$ and~$C$
  occur in different places in such a manner that $A$~is simultaneous
  with~$,$ and $B$~is simultaneous with~$C$ (simultaneous
  in the sense of the above definition), then the criterion for the
  simultaneity of the pair of events $A$,~$C$ is also satisfied. This
  assumption is a physical hypothesis about the law of propagation
  of light; it must certainly be fulfilled if we are to maintain the
  law of the constancy of the velocity of light \textit{in~vacuo}.}
(here the railway embankment).
We are thus led also to a definition of ``time'' in physics.
For this purpose we suppose that clocks of identical
\index{Clocks}%
construction are placed at the points $A$,~$B$ and~$C$ of
\PageSep{24}
\index{Simultaneity|(}%
the railway line (co-ordinate system), and that they
are set in such a manner that the positions of their
pointers are simultaneously (in the above sense) the
same. Under these conditions we understand by the
``time'' of an event the reading (position of the hands)
\index{Time!of an event}%
of that one of these clocks which is in the immediate
vicinity (in space) of the event. In this manner a
time-value is associated with every event which is
essentially capable of observation.

This stipulation contains a further physical hypothesis,
the validity of which will hardly be doubted without
empirical evidence to the contrary. It has been assumed
that all these clocks go \emph{at the same rate} if they are of
identical construction. Stated more exactly: When
two clocks arranged at rest in different places of a
reference-body are set in such a manner that a \emph{particular}
position of the pointers of the one clock is \emph{simultaneous}
(in the above sense) with the \emph{same} position of the
pointers of the other clock, then identical ``settings''
are always simultaneous (in the sense of the above
definition).
\PageSep{25}


\Chapter{IX}{The Relativity of Simultaneity}

\First{Up} to now our considerations have been referred
\index{Reference-body}%
to a particular body of reference, which we
have styled a ``railway embankment.'' We
suppose a very long train travelling along the rails
with the constant velocity~$v$ and in the direction indicated
in \Figref{1}. People travelling in this train will
with advantage use the train as a rigid reference-body
(co-ordinate system); they regard all events in
%[Illustration: Fig. 1.]
\Figure{025}
reference to the train. Then every event which takes
place along the line also takes place at a particular
point of the train. Also the definition of simultaneity
can be given relative to the train in exactly the same
way as with respect to the embankment. As a natural
consequence, however, the following question arises:

Are two events (\eg\ the two strokes of lightning $A$
and~$B$) which are simultaneous \emph{with reference to the
railway embankment} also simultaneous \emph{relatively to the
train}? We shall show directly that the answer must
be in the negative.

When we say that the lightning strokes $A$~and~$B$ are
\PageSep{26}
simultaneous with respect to the embankment, we
mean: the rays of light emitted at the places $A$~and~$B$,
where the lightning occurs, meet each other at the
mid-point~$M$ of the length $A\longrightarrow B$ of the embankment.
But the events $A$~and~$B$ also correspond to positions $A$~and~$B$
\index{Time!of an event}%
on the train. Let $M'$~be the mid-point of the
distance $A\longrightarrow B$ on the travelling train. Just when
the flashes\footnote
  {As judged from the embankment.}
of lightning occur, this point~$M'$ naturally
coincides with the point~$M$, but it moves towards the
right in the diagram with the velocity~$v$ of the train. If
an observer sitting in the position~$M'$ in the train did
not possess this velocity, then he would remain permanently
at~$M$, and the light rays emitted by the
flashes of lightning $A$~and~$B$ would reach him simultaneously,
\ie\ they would meet just where he is situated.
Now in reality (considered with reference to the railway
embankment) he is hastening towards the beam of light
coming from~$B$, whilst he is riding on ahead of the beam
of light coming from~$A$. Hence the observer will see
the beam of light emitted from~$B$ earlier than he will
see that emitted from~$A$. Observers who take the railway
train as their reference-body must therefore come
\index{Reference-body}%
to the conclusion that the lightning flash~$B$ took place
earlier than the lightning flash~$A$. We thus arrive at
the important result:

Events which are simultaneous with reference to the
embankment are not simultaneous with respect to the
train, and \textit{vice versa} (relativity of simultaneity). Every
\index{Simultaneity|)}%
\index{Simultaneity!relativity of}%
reference-body (co-ordinate system) has its own particular
time; unless we are told the reference-body to which
the statement of time refers, there is no meaning in a
statement of the time of an event.
\PageSep{27}

Now before the advent of the theory of relativity
it had always tacitly been assumed in physics that the
statement of time had an absolute significance, \ie\
that it is independent of the state of motion of the body
of reference. But we have just seen that this assumption
is incompatible with the most natural definition
of simultaneity; if we discard this assumption, then
the conflict between the law of the propagation of
light \textit{in~vacuo} and the principle of relativity (developed
in \Sectionref{VII}) disappears.

We were led to that conflict by the considerations
of \Sectionref{VI}, which are now no longer tenable. In
that section we concluded that the man in the carriage,
who traverses the distance~$w$ \emph{per~second} relative to the
carriage, traverses the same distance also with respect to
the embankment \emph{in each second} of time. But, according
to the foregoing considerations, the time required by a
particular occurrence with respect to the carriage must
not be considered equal to the duration of the same
occurrence as judged from the embankment (as reference-body).
Hence it cannot be contended that the
man in walking travels the distance~$w$ relative to the
railway line in a time which is equal to one second as
judged from the embankment.

Moreover, the considerations of \Sectionref{VI} are based
on yet a second assumption, which, in the light of a
strict consideration, appears to be arbitrary, although
it was always tacitly made even before the introduction
of the theory of relativity.
\PageSep{28}


\Chapter{X}{On the Relativity of the Conception
of Distance}
\index{Distance (line-interval)}%
\index{Distance (line-interval)!relativity of}%

\First{Let} us consider two particular points on the train\footnote
  {\eg\ the middle of the first and of the hundredth carriage.}
travelling along the embankment with the
velocity~$v$, and inquire as to their distance apart.
We already know that it is necessary to have a body of
reference for the measurement of a distance, with respect
to which body the distance can be measured up. It is
the simplest plan to use the train itself as reference-body
(co-ordinate system). An observer in the train
measures the interval by marking off his measuring-rod
\index{Measuring-rod}%
in a straight line (\eg\ along the floor of the carriage)
as many times as is necessary to take him from the one
marked point to the other. Then the number which
tells us how often the rod has to be laid down is the
required distance.

It is a different matter when the distance has to be
judged from the railway line. Here the following
method suggests itself. If we call $A'$~and~$B'$ the two
points on the train whose distance apart is required,
then both of these points are moving with the velocity~$v$
along the embankment. In the first place we require to
determine the points $A$~and~$B$ of the embankment which
are just being passed by the two points $A'$~and~$B'$ at a
\PageSep{29}
particular time~$t$---judged from the embankment.
These points $A$~and~$B$ of the embankment can be determined
by applying the definition of time given in
\Sectionref{VIII}. The distance between these points $A$~and~$B$
\index{Distance (line-interval)}%
is then measured by repeated application of the
measuring-rod along the embankment.

\textit{A~priori} it is by no means certain that this last
measurement will supply us with the same result as
the first. Thus the length of the train as measured
from the embankment may be different from that
obtained by measuring in the train itself. This
circumstance leads us to a second objection which must
be raised against the apparently obvious consideration
of \Sectionref{VI}. Namely, if the man in the carriage
covers the distance~$w$ in a unit of time---\emph{measured from
the train},---then this distance---\emph{as measured from the
embankment}---is not necessarily also equal to~$w$.
\PageSep{30}


\Chapter{XI}{The Lorentz Transformation}

\First{The} results of the last three sections show
that the apparent incompatibility of the law
of propagation of light with the principle of
relativity (\Sectionref{VII}) has been derived by means of
a consideration which borrowed two unjustifiable
hypotheses from classical mechanics; these are as
\index{Classical mechanics}%
follows:
\begin{itemize}
\item[(1)] The time-interval (time) between two events is
\index{Time-interval}%
  independent of the condition of motion of the
  body of reference.

\item[(2)] The space-interval (distance) between two points
\index{Space!interval@{-interval}}%
  of a rigid body is independent of the condition
  of motion of the body of reference.
\end{itemize}

If we drop these hypotheses, then the dilemma of
\Sectionref{VII} disappears, because the theorem of the addition
of velocities derived in \Sectionref{VI} becomes invalid.
The possibility presents itself that the law of the propagation
of light \textit{in~vacuo} may be compatible with the
principle of relativity, and the question arises: How
have we to modify the considerations of \Sectionref{VI}
in order to remove the apparent disagreement between
these two fundamental results of experience? This
question leads to a general one. In the discussion of
\PageSep{31}
\Sectionref{VI} we have to do with places and times relative
both to the train and to the embankment. How are
we to find the place and time of an event in relation to
the train, when we know the place and time of the
event with respect to the railway embankment? Is
there a thinkable answer to this question of such a
nature that the law of transmission of light \textit{in~vacuo}
does not contradict the principle of relativity? In
other words: Can we conceive of a relation between
place and time of the individual events relative to both
reference-bodies, such that every ray of light possesses
the velocity of transmission~$c$ relative to the embankment
and relative to the train? This question leads to
a quite definite positive answer, and to a perfectly definite
transformation law for the space-time magnitudes of
an event when changing over from one body of reference
to another.

Before we deal with this, we shall introduce the
following incidental consideration. Up to the present
we have only considered events taking place along the
embankment, which had mathematically to assume the
function of a straight line. In the manner indicated
in \Sectionref{II} we can imagine this reference-body supplemented
laterally and in a vertical direction by means of
a framework of rods, so that an event which takes place
anywhere can be localised with reference to this framework.
Similarly, we can imagine the train travelling
with the velocity~$v$ to be continued across the whole of
space, so that every event, no matter how far off it
may be, could also be localised with respect to the second
framework. Without committing any fundamental error,
we can disregard the fact that in reality these frameworks
would continually interfere with each other, owing
\PageSep{32}
\index{Propagation of light}%
to the impenetrability of solid bodies. In every such
framework we imagine three surfaces perpendicular to
each other marked out, and designated as ``co-ordinate
\index{Coordinate@{Co-ordinate}!planes}%
planes'' (``co-ordinate system''). A co-ordinate
system~$K$ then corresponds to the embankment, and a
co-ordinate system~$K'$ to the train. An event, wherever
it may have taken place, would be fixed in space with
respect to~$K$ by the three perpendiculars $x$,~$y$,~$z$ on the
co-ordinate planes, and with regard to time by a time-value~$t$.
Relative to~$K'$, \emph{the
same event} would be fixed
in respect of space and time
by corresponding values $x'$,~$y'$,
$z'$,~$t'$, which of course are
not identical with $x$,~$y$, $z$,~$t$.
It has already been set
forth in detail how these
magnitudes are to be regarded
as results of physical measurements.
%[Illustration: Fig. 2.]
\Figure[2in]{032}

Obviously our problem can be exactly formulated in
the following manner. What are the values $x'$,~$y'$, $z'$,~$t'$,
of an event with respect to~$K'$, when the magnitudes
$x$,~$y$, $z$,~$t$, of the same event with respect to~$K$ are given?
The relations must be so chosen that the law of the
transmission of light \textit{in~vacuo} is satisfied for one and the
same ray of light (and of course for every ray) with
respect to $K$ and~$K'$. For the relative orientation in
space of the co-ordinate systems indicated in the diagram
(\Figref{2}), this problem is solved by means of the
equations:
\begin{align*}
x' &= \frac{x - vt}{\sqrt{1 - \dfrac{v^{2}}{c^{2}}}}\Add{,}\displaybreak[1] \\
\PageSep{33}
y' &= y\Add{,} \\
z' &= z\Add{,} \\
t' &= \frac{t - \dfrac{v}{c^{2}}�x}{\sqrt{1 - \dfrac{v^{2}}{c^{2}}}}\Change{}{.}
\end{align*}
This system of equations is known as the ``Lorentz
\index{Lorentz, H. A.!transformation}%
transformation.''\footnote
  {A simple derivation of the Lorentz transformation is given
  in \Appendixref{I}.}

If in place of the law of transmission of light we had
taken as our basis the tacit assumptions of the older
mechanics as to the absolute character of times and
lengths, then instead of the above we should have
obtained the following equations:
\begin{align*}
x' &= x - vt\Add{,} \\
y' &= y\Add{,} \\
z' &= z\Add{,} \\
t' &= t.
\end{align*}
This system of equations is often termed the ``Galilei
\index{Galilei!transformation}%
transformation.'' The Galilei transformation can be
obtained from the Lorentz transformation by substituting
an infinitely large value for the velocity of
light~$c$ in the latter transformation.

Aided by the following illustration, we can readily
see that, in accordance with the Lorentz transformation,
the law of the transmission of light \textit{in~vacuo}
is satisfied both for the reference-body~$K$ and for the
reference-body~$K'$. A light-signal is sent along the
\index{Light-signal}%
positive $x$-axis, and this light-stimulus advances in
\index{Light-stimulus}%
accordance with the equation
\[
x = ct,
\]
\PageSep{34}
\ie\ with the velocity~$c$. According to the equations of
the Lorentz transformation, this simple relation between
$x$~and~$t$ involves a relation between $x'$~and~$t'$. In point
of fact, if we substitute for~$x$ the value~$ct$ in the first
and fourth equations of the Lorentz transformation,
we obtain:
\begin{align*}
x' &= \frac{(c - v)t}{\sqrt{1 - \dfrac{v^{2}}{c^{2}}}}\Add{,} \\
t' &= \frac{\left(1 - \dfrac{v}{c}\right)t}{\sqrt{1 - \dfrac{v^{2}}{c^{2}}}},
\end{align*}
from which, by division, the expression
\[
x' = ct'
\]
immediately follows. If referred to the system~$K'$, the
propagation of light takes place according to this
equation. We thus see that the velocity of transmission
relative to the reference-body~$K'$ is also equal to~$c$. The
same result is obtained for rays of light advancing in
any other direction whatsoever. Of course this is not
surprising, since the equations of the Lorentz transformation
were derived conformably to this point of
view.
\PageSep{35}


\Chapter{XII}{The Behaviour of Measuring-Rods and
Clocks in Motion}

\First{I place} a metre-rod in the $x'$-axis of~$K'$ in such a
manner that one end (the beginning) coincides with
the point $x' = 0$, whilst the other end (the end of the
rod) coincides with the point $x' = 1$. What is the length
of the metre-rod relatively to the system~$K$? In order
to learn this, we need only ask where the beginning of the
rod and the end of the rod lie with respect to~$K$ at a
particular time~$t$ of the system~$K$. By means of the first
equation of the Lorentz transformation the values of
these two points at the time $t = 0$ can be shown to be
\begin{align*}
x_{\text{(beginning of rod)}}
  &= 0�\sqrt{1 - \frac{v^{2}}{c^{2}}}\Add{,} \\
x_{\text{(end of rod)}}
  &= 1�\sqrt{1 - \frac{v^{2}}{c^{2}}},
\end{align*}
the distance between the points being~$\sqrt{1 - \dfrac{v^{2}}{c^{2}}}$. But
the metre-rod is moving with the velocity~$v$ relative to~$K$.
It therefore follows that the length of a rigid metre-rod
moving in the direction of its length with a velocity~$v$
is $\sqrt{1 - v^{2}/c^{2}}$~of a metre. The rigid rod is thus
shorter when in motion than when at rest, and the
more quickly it is moving, the shorter is the rod. For
the velocity $v = c$ we should have $\sqrt{1 - v^{2}/c^{2}} = 0$, and
for still greater velocities the square-root becomes
\PageSep{36}
imaginary. From this we conclude that in the theory
of relativity the velocity~$c$ plays the part of a limiting
\index{Limiting velocity ($c$)}%
velocity, which can neither be reached nor exceeded
by any real body.

Of course this feature of the velocity~$c$ as a limiting
velocity also clearly follows from the equations of the
Lorentz transformation, for these become meaningless
if we choose values of~$v$ greater than~$c$.

If, on the contrary, we had considered a metre-rod
at rest in the $x$-axis with respect to~$K$, then we should
have found that the length of the rod as judged from~$K'$
would have been~$\sqrt{1 - v^{2}/c^{2}}$; this is quite in accordance
with the principle of relativity which forms the
basis of our considerations.

\textit{A~priori} it is quite clear that we must be able to
learn something about the physical behaviour of measuring-rods
and clocks from the equations of transformation,
for the magnitudes $x$,~$y$, $z$,~$t$, are nothing more nor
less than the results of measurements obtainable by
means of measuring-rods and clocks. If we had based
our considerations on the Galilei transformation we
\index{Galilei!transformation}%
should not have obtained a contraction of the rod as a
consequence of its motion.

Let us now consider a seconds-clock which is permanently
\index{Seconds-clock}%
situated at the origin ($x' = 0$) of~$K'$. $t' = 0$
and $t' = 1$ are two successive ticks of this clock. The
first and fourth equations of the Lorentz transformation
give for these two ticks:
\begin{align*}
t &= 0 \\
\intertext{and}
t &= \frac{1}{\sqrt{1 - \dfrac{v^{2}}{c^{2}}}}.
\end{align*}
\PageSep{37}

As judged from~$K$, the clock is moving with the
velocity~$v$; as judged from this reference-body, the
\index{Reference-body}%
time which elapses between two strokes of the clock
is not one second, but $\dfrac{1}{\sqrt{1 - \dfrac{v^{2}}{c^{2}}}}$~seconds, \ie\ a somewhat
larger time. As a consequence of its motion
the clock goes more slowly than when at rest. Here
also the velocity~$c$ plays the part of an unattainable
limiting velocity.
\index{Limiting velocity ($c$)}%
\PageSep{38}


\Chapter{XIII}{Theorem of the Addition of Velocities.
The Experiment of Fizeau}
\index{Addition of velocities}%

\First{Now} in practice we can move clocks and
measuring-rods only with velocities that are
small compared with the velocity of light; hence
we shall hardly be able to compare the results of the
previous section directly with the reality. But, on the
other hand, these results must strike you as being very
singular, and for that reason I shall now draw another
conclusion from the theory, one which can easily be
derived from the foregoing considerations, and which
has been most elegantly confirmed by experiment.

In \Sectionref{VI} we derived the theorem of the addition
of velocities in one direction in the form which also
results from the hypotheses of classical mechanics. This
theorem can also be deduced readily from the Galilei
\index{Galilei!transformation}%
transformation (\Sectionref{XI}). In place of the man
walking inside the carriage, we introduce a point moving
relatively to the co-ordinate system~$K'$ in accordance
with the equation
\[
x' = wt'.
\]
By means of the first and fourth equations of the Galilei
transformation we can express $x'$~and~$t'$ in terms of $x$~and~$t$,
and we then obtain
\[
x = (v + w)t.
\]
\PageSep{39}
This equation expresses nothing else than the law of
motion of the point with reference to the system~$K$
(of the man with reference to the embankment). We
denote this velocity by the symbol~$W$, and we then
obtain, as in \Sectionref{VI},
\[
W = v + w.
\Tag{(A)}
\]

But we can carry out this consideration just as well
on the basis of the theory of relativity. In the equation
\[
x' = wt'
\]
we must then express $x'$~and~$t'$ in terms of $x$~and~$t$, making
use of the first and fourth equations of the \emph{Lorentz
\index{Lorentz, H. A.!transformation}%
transformation}. Instead of the equation~\Eqref{(A)} we then
obtain the equation
\[
W = \frac{v + w}{1 + \dfrac{vw}{c^{2}}},
\Tag{(B)}
\]
which corresponds to the theorem of addition for
velocities in one direction according to the theory of
relativity. The question now arises as to which of these
two theorems is the better in accord with experience. On
this point we are enlightened by a most important experiment
which the brilliant physicist Fizeau performed more
\index{Fizeau}%
\index{Fizeau!experiment of}%
than half a century ago, and which has been repeated
since then by some of the best experimental physicists,
so that there can be no doubt about its result. The
experiment is concerned with the following question.
Light travels in a motionless liquid with a particular
velocity~$w$. How quickly does it travel in the direction
of the arrow in the tube~$T$ (see the accompanying diagram,
\Figref{3}) when the liquid above mentioned is flowing
through the tube with a velocity~$v$?
\PageSep{40}

In accordance with the principle of relativity we shall
\index{Propagation of light!in liquid}%
certainly have to take for granted that the propagation
of light always takes place with the same velocity~$w$
\emph{with respect to the liquid}, whether the latter is in motion
with reference to other bodies or not. The velocity
of light relative to the liquid and the velocity of the
latter relative to the tube are thus known, and we
require the velocity of light relative to the tube.

It is clear that we have the problem of \Sectionref{VI}
again before us. The tube plays the part of the railway
embankment or of the co-ordinate system~$K$, the liquid
plays the part of the carriage or of the co-ordinate
system~$K'$, and finally, the light plays the part of the
%[Illustration: Fig. 3.]
\Figure[2in]{040}
man walking along the carriage, or of the moving point
in the present section. If we denote the velocity of the
light relative to the tube by~$W$, then this is given
by the equation \Eqref{(A)}~or~\Eqref{(B)}, according as the Galilei
transformation or the Lorentz transformation corresponds
to the facts. Experiment\footnote
  {Fizeau found $W = w + v\left(1 - \dfrac{1}{n^{2}}\right)$, where $n = \dfrac{c}{w}$ is the index of
  refraction of the liquid. On the other hand, owing to the smallness
  of~$\dfrac{vw}{c^{2}}$ as compared with~$1$, we can replace~\Eqref{(B)} in the first
  place by $W = (w + v) \left(1 - \dfrac{vw}{c^{2}}\right)$, or to the same order of approximation
  by $w + v \left(1 - \dfrac{1}{n^{2}}\right)$, which agrees with Fizeau's result.}
decides in favour
of equation~\Eqref{(B)} derived from the theory of relativity, and
the agreement is, indeed, very exact. According to
\PageSep{41}
recent and most excellent measurements by Zeeman, the
\index{Zeeman}%
influence of the velocity of flow~$v$ on the propagation of
light is represented by formula~\Eqref{(B)} to within one per
cent. %[** TN: [sic] two words]

Nevertheless we must now draw attention to the fact
that a theory of this phenomenon was given by H.~A.
Lorentz long before the statement of the theory of
\index{Lorentz, H. A.}%
relativity. This theory was of a purely electrodynamical
nature, and was obtained by the use of particular
hypotheses as to the electromagnetic structure of matter.
This circumstance, however, does not in the least
diminish the conclusiveness of the experiment as a
crucial test in favour of the theory of relativity, for the
electrodynamics of Maxwell-Lorentz, on which the
\index{Electrodynamics}%
\index{Maxwell}%
original theory was based, in no way opposes the theory
of relativity. Rather has the latter been developed
from electrodynamics as an astoundingly simple combination
and generalisation of the hypotheses, formerly
independent of each other, on which electrodynamics
was built.
\PageSep{42}


\Chapter{XIV}{The Heuristic Value of the Theory of
Relativity}
\index{Heuristic value of relativity}%

\First{Our} train of thought in the foregoing pages can be
epitomised in the following manner. Experience
has led to the conviction that, on the one hand,
the principle of relativity holds true, and that on the
other hand the velocity of transmission of light \textit{in~vacuo}
has to be considered equal to a constant~$c$. By uniting
these two postulates we obtained the law of transformation
for the rectangular co-ordinates $x$,~$y$,~$z$ and the time~$t$
of the events which constitute the processes of nature.
\index{Processes of Nature}%
In this connection we did not obtain the Galilei transformation,
\index{Galilei!transformation}%
but, differing from classical mechanics,
the \emph{Lorentz transformation}.
\index{Lorentz, H. A.!transformation}%

The law of transmission of light, the acceptance of
which is justified by our actual knowledge, played an
important part in this process of thought. Once in
possession of the Lorentz transformation, however,
we can combine this with the principle of relativity,
and sum up the theory thus:

Every general law of nature must be so constituted
that it is transformed into a law of exactly the same
form when, instead of the space-time variables $x$,~$y$, $z$,~$t$
of the original co-ordinate system~$K$, we introduce new
space-time variables $x'$,~$y'$, $z'$,~$t'$ of a co-ordinate system~$K'$.
\PageSep{43}
In this connection the relation between the
ordinary and the accented magnitudes is given by the
Lorentz transformation. Or, in brief: General laws
of nature are co-variant with respect to Lorentz transformations.
\index{Covariant@{Co-variant}}%

This is a definite mathematical condition that the
theory of relativity demands of a natural law, and in
virtue of this, the theory becomes a valuable heuristic aid
in the search for general laws of nature. If a general
law of nature were to be found which did not satisfy
this condition, then at least one of the two fundamental
assumptions of the theory would have been disproved.
Let us now examine what general results the latter
theory has hitherto evinced.
\PageSep{44}


\Chapter{XV}{General Results of the Theory}

\First{It} is clear from our previous considerations that the
(special) theory of relativity has grown out of electrodynamics
\index{Electrodynamics}%
and optics. In these fields it has not
\index{Optics}%
appreciably altered the predictions of theory, but it
has considerably simplified the theoretical structure,
\ie\ the derivation of laws, and---what is incomparably
\index{Derivation of laws}%
more important---it has considerably reduced the
number of independent hypotheses forming the basis of
\index{Basis of theory}%
theory. The special theory of relativity has rendered
the Maxwell-Lorentz theory so plausible, that the latter
\index{Lorentz, H. A.}%
\index{Maxwell}%
would have been generally accepted by physicists
even if experiment had decided less unequivocally in its
favour.

Classical mechanics required to be modified before it
\index{Classical mechanics}%
could come into line with the demands of the special
theory of relativity. For the main part, however,
this modification affects only the laws for rapid motions,
in which the velocities of matter~$v$ are not very small as
compared with the velocity of light. We have experience
of such rapid motions only in the case of electrons
\index{Electron}%
and ions; for other motions the variations from the laws
\index{Ions}%
of classical mechanics are too small to make themselves
evident in practice. We shall not consider the motion
\index{Motion!of heavenly bodies}%
of stars until we come to speak of the general theory of
relativity. In accordance with the theory of relativity
\PageSep{45}
the kinetic energy of a material point of mass~$m$ is no
\index{Kinetic energy}%
longer given by the well-known expression
\[
m\frac{v^{2}}{2}\Change{.}{,}
\]
but by the expression
\[
\frac{mc^{2}}{\sqrt{1 - \dfrac{v^{2}}{c^{2}}}}.
\]
This expression approaches infinity as the velocity~$v$
approaches the velocity of light~$c$. The velocity must
therefore always remain less than~$c$, however great may
be the energies used to produce the acceleration. If
we develop the expression for the kinetic energy in the
form of a series, we obtain
\[
mc^{2} + m\frac{v^{2}}{2} + \frac{3}{8}m\frac{v^4}{c^{2}} + \dots.
\]

When $\dfrac{v^{2}}{c^{2}}$ is small compared with unity, the third
of these terms is always small in comparison with the
second, which last is alone considered in classical
mechanics. The first term~$mc^{2}$ does not contain
the velocity, and requires no consideration if we are only
dealing with the question as to how the energy of a
point-mass depends on the velocity. We shall speak
\index{Point-mass, energy of}%
of its essential significance later.

The most important result of a general character to
\index{Conservation of energy}%
\index{Conservation of energy!mass}%
which the special theory of relativity has led is concerned
with the conception of mass. Before the advent of
\index{Conception of mass}%
relativity, physics recognised two conservation laws of
fundamental importance, namely, the law of the conservation
of energy and the law of the conservation of
mass; these two fundamental laws appeared to be quite
\PageSep{46}
independent of each other. By means of the theory of
relativity they have been united into one law. We shall
now briefly consider how this unification came about,
and what meaning is to be attached to it.

The principle of relativity requires that the law of the
conservation of energy should hold not only with reference
to a co-ordinate system~$K$, but also with respect
to every co-ordinate system~$K'$ which is in a state of
uniform motion of translation relative to~$K$, or, briefly,
relative to every ``Galileian'' system of co-ordinates.
\index{Galileian system of co-ordinates}%
In contrast to classical mechanics, the Lorentz transformation
is the deciding factor in the transition from
one such system to another.

By means of comparatively simple considerations
we are led to draw the following conclusion from
these premises, in conjunction with the fundamental
equations of the electrodynamics of Maxwell: A body
\index{Maxwell!fundamental equations}%
\index{Absorption of energy}%
moving with the velocity~$v$, which absorbs\footnote
  {$E_{0}$~is the energy taken up, as judged from a co-ordinate
  system moving with the body.}
an amount
of energy~$E_{0}$ in the form of radiation without suffering
\index{Radiation}%
an alteration in velocity in the process, has, as a consequence,
its energy increased by an amount
\[
\frac{E_{0}}{\sqrt{1 - \dfrac{v^{2}}{c^{2}}}}.
\]

In consideration of the expression given above for the
kinetic energy of the body, the required energy of the
body comes out to be
\[
\frac{\left(m + \dfrac{E_{0}}{c^{2}}\right)c^{2}}
     {\sqrt{1 - \dfrac{v^{2}}{c^{2}}}}.
\]
\PageSep{47}

Thus the body has the same energy as a body of mass
$\left(m + \dfrac{E_{0}}{c^{2}}\right)$ moving with the velocity~$v$. Hence we can
say: If a body takes up an amount of energy~$E_{0}$, then
its inertial mass increases by an amount~$\dfrac{E_{0}}{c^{2}}$; the
\index{Inertial mass}%
inertial mass of a body is not a constant, but varies
according to the change in the energy of the body.
The inertial mass of a system of bodies can even be
regarded as a measure of its energy. The law of the
conservation of the mass of a system becomes identical
with the law of the conservation of energy, and is only
\index{Conservation of energy!mass}%
valid provided that the system neither takes up nor sends
out energy. Writing the expression for the energy in
the form
\[
\frac{mc^{2} + E_{0}}{\sqrt{1 - \dfrac{v^{2}}{c^{2}}}},
\]
we see that the term~$mc^{2}$, which has hitherto attracted
our attention, is nothing else than the energy possessed
by the body\footnote
  {As judged from a co-ordinate system moving with the body.}
before it absorbed the energy~$E_{0}$.

A direct comparison of this relation with experiment
is not possible at the present time, owing to the fact that
the changes in energy~$E_{0}$ to which we can subject a
system are not large enough to make themselves
perceptible as a change in the inertial mass of the
system. $\dfrac{E_{0}}{c^{2}}$~is too small in comparison with the mass~$m$,
which was present before the alteration of the energy.
It is owing to this circumstance that classical mechanics
was able to establish successfully the conservation of
mass as a law of independent validity.
\PageSep{48}

Let me add a final remark of a fundamental nature.
The success of the Faraday-Maxwell interpretation of
\index{Faraday}%
\index{Maxwell|(}%
electromagnetic action at a distance resulted in physicists
\index{Action at a distance}%
becoming convinced that there are no such things as
instantaneous actions at a distance (not involving an
intermediary medium) of the type of Newton's law of
\index{Newton's!law of gravitation}%
gravitation. According to the theory of relativity,
action at a distance with the velocity of light always
takes the place of instantaneous action at a distance or
of action at a distance with an infinite velocity of transmission.
This is connected with the fact that the
velocity~$c$ plays a fundamental r�le in this theory. In
\Partref{II} we shall see in what way this result becomes
modified in the general theory of relativity.
\PageSep{49}


\Chapter{XVI}{Experience and the Special Theory of
Relativity}
\index{Experience}%

\First{To} what extent is the special theory of relativity
supported by experience? This question is not
easily answered for the reason already mentioned
in connection with the fundamental experiment of Fizeau.
\index{Fizeau}%
The special theory of relativity has crystallised out
from the Maxwell-Lorentz theory of electromagnetic
\index{Lorentz, H. A.}%
phenomena. Thus all facts of experience which support
the electromagnetic theory also support the theory of
\index{Electromagnetic theory}%
relativity. As being of particular importance, I mention
here the fact that the theory of relativity enables us to
predict the effects produced on the light reaching us
from the fixed stars. These results are obtained in an
exceedingly simple manner, and the effects indicated,
which are due to the relative motion of the earth with
reference to those fixed stars, are found to be in accord
with experience. We refer to the yearly movement of
the apparent position of the fixed stars resulting from the
motion of the earth round the sun (aberration), and to the
\index{Aberration}%
influence of the radial components of the relative
motions of the fixed stars with respect to the earth on
the colour of the light reaching us from them. The
\PageSep{50}
latter effect manifests itself in a slight displacement
of the spectral lines of the light transmitted to us from
a fixed star, as compared with the position of the same
spectral lines when they are produced by a terrestrial
source of light (Doppler principle). The experimental
\index{Doppler principle}%
arguments in favour of the Maxwell-Lorentz theory,
\index{Lorentz, H. A.|(}%
which are at the same time arguments in favour of the
theory of relativity, are too numerous to be set forth
here. In reality they limit the theoretical possibilities
to such an extent, that no other theory than that of
Maxwell and Lorentz has been able to hold its own when
tested by experience.

But there are two classes of experimental facts
hitherto obtained which can be represented in the
Maxwell-Lorentz theory only by the introduction of an
\index{Maxwell|)}%
auxiliary hypothesis, which in itself---\ie\ without
making use of the theory of relativity---appears extraneous.

It is known that cathode rays and the so-called
\index{beta-rays@{$\beta$-rays}}%
\index{Cathode rays}%
$\beta$-rays emitted by radioactive substances consist of
\index{Radioactive substances}%
negatively electrified particles (electrons) of very small
inertia and large velocity. By examining the deflection
of these rays under the influence of electric and magnetic
fields, we can study the law of motion of these particles
very exactly.

In the theoretical treatment of these electrons, we are
faced with the difficulty that electrodynamic theory of
itself is unable to give an account of their nature. For
since electrical masses of one sign repel each other, the
negative electrical masses constituting the electron would
\index{Electron}%
necessarily be scattered under the influence of their
mutual repulsions, unless there are forces of another
kind operating between them, the nature of which has
\PageSep{51}
hitherto remained obscure to us.\footnote
  {The general theory of relativity renders it likely that the
  electrical masses of an electron are held together by gravitational
\index{Electron!electrical masses of}%
  forces.}
If we now assume
that the relative distances between the electrical masses
constituting the electron remain unchanged during the
motion of the electron (rigid connection in the sense of
classical mechanics), we arrive at a law of motion of the
electron which does not agree with experience. Guided
by purely formal points of view, H.~A.~Lorentz was the
first to introduce the hypothesis that the particles
constituting the electron experience a contraction
in the direction of motion in consequence of that motion,
the amount of this contraction being proportional to
the expression~$\sqrt{1 - \dfrac{v^{2}}{c^{2}}}$. This hypothesis, which is
not justifiable by any electrodynamical facts, supplies us
then with that particular law of motion which has
been confirmed with great precision in recent years.

The theory of relativity leads to the same law of
motion, without requiring any special hypothesis whatsoever
as to the structure and the behaviour of the
electron. We arrived at a similar conclusion in \Sectionref{XIII}
in connection with the experiment of Fizeau, the
\index{Fizeau}%
result of which is foretold by the theory of relativity
without the necessity of drawing on hypotheses as to
the physical nature of the liquid.

The second class of facts to which we have alluded
has reference to the question whether or not the motion
of the earth in space can be made perceptible in terrestrial
experiments. We have already remarked in \Sectionref{V}
that all attempts of this nature led to a negative result.
Before the theory of relativity was put forward, it was
\PageSep{52}
difficult to become reconciled to this negative result,
for reasons now to be discussed. The inherited
prejudices about time and space did not allow any
\index{Time!conception of}%
\index{Space}%
doubt to arise as to the prime importance of the
Galilei transformation for changing over from one
\index{Galilei!transformation}%
body of reference to another. Now assuming that the
Maxwell-Lorentz equations hold for a reference-body~$K$,
\index{Maxwell}%
we then find that they do not hold for a reference-body~$K'$
moving uniformly with respect to~$K$, if we
assume that the relations of the Galileian transformation
exist between the co-ordinates of $K$~and~$K'$. It
thus appears that of all Galileian co-ordinate systems
one~($K$) corresponding to a particular state of motion
is physically unique. This result was interpreted
physically by regarding $K$ as at rest with respect to a
hypothetical �ther of space. On the other hand,
all co-ordinate systems~$K'$ moving relatively to~$K$ were
to be regarded as in motion with respect to the �ther.
\index{Aether}%
\index{Aether!-drift}%
To this motion of~$K'$ against the �ther (``�ther-drift''
relative to~$K'$) were assigned the more complicated
laws which were supposed to hold relative to~$K'$.
Strictly speaking, such an �ther-drift ought also to be
assumed relative to the earth, and for a long time the
efforts of physicists were devoted to attempts to detect
the existence of an �ther-drift at the earth's surface.

In one of the most notable of these attempts Michelson
\index{Michelson|(}%
devised a method which appears as though it must be
decisive. Imagine two mirrors so arranged on a rigid
body that the reflecting surfaces face each other. A
ray of light requires a perfectly definite time~$T$ to pass
from one mirror to the other and back again, if the whole
system be at rest with respect to the �ther. It is found
by calculation, however, that a slightly different time~$T'$
\PageSep{53}
is required for this process, if the body, together with
the mirrors, be moving relatively to the �ther. And
\index{Aether!-drift}%
yet another point: it is shown by calculation that for
a given velocity~$v$ with reference to the �ther, this
time~$T'$ is different when the body is moving perpendicularly
to the planes of the mirrors from that resulting
when the motion is parallel to these planes. Although
the estimated difference between these two times is
exceedingly small, Michelson and Morley performed an
\index{Morley}%
experiment involving interference in which this difference
should have been clearly detectable. But the experiment
gave a negative result---a fact very perplexing
to physicists. Lorentz and FitzGerald rescued the
\index{FitzGerald}%
\index{Lorentz, H. A.|)}%
theory from this difficulty by assuming that the motion
of the body relative to the �ther produces a contraction
of the body in the direction of motion, the amount of contraction
being just sufficient to compensate for the difference
in time mentioned above. Comparison with the
discussion in \Sectionref{XII} shows that also from the standpoint
of the theory of relativity this solution of the
difficulty was the right one. But on the basis of the
theory of relativity the method of interpretation is
incomparably more satisfactory. According to this
theory there is no such thing as a ``specially favoured''
(unique) co-ordinate system to occasion the introduction
of the �ther-idea, and hence there can be no �ther-drift,
nor any experiment with which to demonstrate it.
Here the contraction of moving bodies follows from
the two fundamental principles of the theory without
the introduction of particular hypotheses; and as the
prime factor involved in this contraction we find, not
the motion in itself, to which we cannot attach any
meaning, but the motion with respect to the body of
\PageSep{54}
reference chosen in the particular case in point. Thus
for a co-ordinate system moving with the earth the
mirror system of Michelson and Morley is not shortened,
\index{Michelson|)}%
\index{Morley}%
but it \emph{is} shortened for a co-ordinate system which is at
rest relatively to the sun.
\PageSep{55}


\Chapter{XVII}{Minkowski's Four-dimensional Space}
\index{Minkowski|(}%
\index{Space}%

\First{The} non-mathematician is seized by a mysterious
shuddering when he hears of ``four-dimensional''
things, by a feeling not unlike that awakened by
thoughts of the occult. And yet there is no more
common-place statement than that the world in which
\index{World}%
we live is a four-dimensional space-time continuum.
\index{Continuum}%

Space is a three-dimensional continuum. By this
\index{Space co-ordinates}%
\index{Three-dimensional}%
\index{Time!coordinate@{co-ordinate}}%
we mean that it is possible to describe the position of a
point (at rest) by means of three numbers (co-ordinates)
$x$,~$y$,~$z$, and that there is an indefinite number of points
in the neighbourhood of this one, the position of which
can be described by co-ordinates such as $x_{1}$,~$y_{1}$,~$z_{1}$, which
may be as near as we choose to the respective values of
the co-ordinates $x$,~$y$,~$z$ of the first point. In virtue of the
latter property we speak of a ``continuum,'' and owing
to the fact that there are three co-ordinates we speak of
it as being ``three-dimensional.''

Similarly, the world of physical phenomena which was
briefly called ``world'' by Minkowski is naturally
four-dimensional in the space-time sense. For it is
composed of individual events, each of which is described
by four numbers, namely, three space
co-ordinates $x$,~$y$,~$z$ and a time co-ordinate, the time-value~$t$.
The ``world'' is in this sense also a continuum;
for to every event there are as many ``neighbouring''
\PageSep{56}
events (realised or at least thinkable) as we care to
choose, the co-ordinates $x_{1}$,~$y_{1}$, $z_{1}$,~$t_{1}$ of which differ
by an indefinitely small amount from those of the
event $x$,~$y$, $z$,~$t$ originally considered. That we have not
been accustomed to regard the world in this sense as a
\index{World}%
four-dimensional continuum is due to the fact that in
physics, before the advent of the theory of relativity,
time played a different and more independent r�le, as
compared with the space co-ordinates. It is for this
reason that we have been in the habit of treating time
as an independent continuum. As a matter of fact,
according to classical mechanics, time is absolute,
\ie\ it is independent of the position and the condition
of motion of the system of co-ordinates. We see this
expressed in the last equation of the Galileian transformation
($t' = t$).

The four-dimensional mode of consideration of the
``world'' is natural on the theory of relativity, since
according to this theory time is robbed of its independence.
This is shown by the fourth equation of the
Lorentz transformation:
\[
t' = \frac{t - \dfrac{v}{c^{2}}x}{\sqrt{1 - \dfrac{v^{2}}{c^{2}}}}.
\]
Moreover, according to this equation the time difference~$\Delta t'$
\index{Space!interval@{-interval}}%
\index{Time-interval}%
of two events with respect to~$K'$ does not in general
vanish, even when the time difference~$\Delta t$ of the same
events with reference to~$K$ vanishes. Pure ``space-distance''
of two events with respect to~$K$ results in
``time-distance'' of the same events with respect to~$K'$.
But the discovery of Minkowski, which was of importance
\PageSep{57}
for the formal development of the theory of relativity,
does not lie here. It is to be found rather in
the fact of his recognition that the four-dimensional
space-time continuum of the theory of relativity, in its
\index{Continuum!three-dimensional}%
most essential formal properties, shows a pronounced
relationship to the three-dimensional continuum of
Euclidean geometrical space.\footnote
  {Cf.\ the somewhat more detailed discussion in \Appendixref{II}.}
In order to give due
prominence to this relationship, however, we must
replace the usual time co-ordinate~$t$ by an imaginary
magnitude~$\sqrt{-1}�ct$ proportional to it. Under these
conditions, the natural laws satisfying the demands of
the (special) theory of relativity assume mathematical
forms, in which the time co-ordinate plays exactly the
same r�le as the three space co-ordinates. Formally,
these four co-ordinates correspond exactly to the three
space co-ordinates in Euclidean geometry. It must be
\index{Euclidean geometry}%
\index{Euclidean space}%
clear even to the non-mathematician that, as a consequence
of this purely formal addition to our knowledge,
the theory perforce gained clearness in no mean
measure.

These inadequate remarks can give the reader only a
vague notion of the important idea contributed by Minkowski.
Without it the general theory of relativity, of
which the fundamental ideas are developed in the following
pages, would perhaps have got no farther than its
long clothes. Minkowski's work is doubtless difficult of
\index{Minkowski|)}%
access to anyone inexperienced in mathematics, but
since it is not necessary to have a very exact grasp of
this work in order to understand the fundamental ideas
of either the special or the general theory of relativity,
I shall at present leave it here, and shall revert to it
only towards the end of \Partref{II}.
\index{Special theory of relativity|)}%
\PageSep{58}
% [Blank page]
\PageSep{59}


\Part{II}{The General Theory of Relativity}{General Theory of Relativity}
\index{General theory of relativity|(}%

\Chapter{XVIII}{Special and General Principle of
Relativity}
\index{Laws of Galilei-Newton!of Nature}%

\First{The} basal principle, which was the pivot of all
our previous considerations, was the \emph{special}
principle of relativity, \ie\ the principle of the
physical relativity of all \emph{uniform} motion. Let us once
\index{Uniform translation}%
more analyse its meaning carefully.

It was at all times clear that, from the point of view
of the idea it conveys to us, every motion must only
be considered as a relative motion. Returning to the
illustration we have frequently used of the embankment
and the railway carriage, we can express the fact of the
motion here taking place in the following two forms,
both of which are equally justifiable:
\begin{itemize}
\item[\itema] The carriage is in motion relative to the embankment.

\item[\itemb] The embankment is in motion relative to the
  carriage.
\end{itemize}

In \itema~the embankment, in \itemb~the carriage, serves as
the body of reference in our statement of the motion
taking place. If it is simply a question of detecting
\PageSep{60}
or of describing the motion involved, it is in principle
\index{Motion}%
immaterial to what reference-body we refer the motion.
\index{Reference-body}%
As already mentioned, this is self-evident, but it must
not be confused with the much more comprehensive
statement called ``the principle of relativity,'' which
\index{Principle of relativity}%
we have taken as the basis of our investigations.

The principle we have made use of not only maintains
that we may equally well choose the carriage or the
embankment as our reference-body for the description
of any event (for this, too, is self-evident). Our principle
rather asserts what follows: If we formulate the general
laws of nature as they are obtained from experience,
\index{Experience}%
by making use of
\begin{itemize}
\item[\itema] the embankment as reference-body,
\item[\itemb] the railway carriage as reference-body,
\end{itemize}
then these general laws of nature (\eg\ the laws of
mechanics or the law of the propagation of light \textit{in~vacuo})
have exactly the same form in both cases. This can
also be expressed as follows: For the \emph{physical} description
of natural processes, neither of the reference-bodies
$K$,~$K'$ is unique (lit.\ ``specially marked out'') as
compared with the other. Unlike the first, this latter
statement need not of necessity hold \textit{a~priori}; it is
not contained in the conceptions of ``motion'' and
``reference-body'' and derivable from them; only
\emph{experience} can decide as to its correctness or incorrectness.

Up to the present, however, we have by no means
maintained the equivalence of \emph{all} bodies of reference~$K$
in connection with the formulation of natural laws.
Our course was more on the following lines. In the
first place, we started out from the assumption that
there exists a reference-body~$K$, whose condition of
\PageSep{61}
\index{Law of inertia}%
motion is such that the Galileian law holds with respect
to it: A particle left to itself and sufficiently far removed
from all other particles moves uniformly in a straight
line. With reference to~$K$ (Galileian reference-body) the
laws of nature were to be as simple as possible. But
in addition to~$K$, all bodies of reference~$K'$ should be
given preference in this sense, and they should be exactly
equivalent to~$K$ for the formulation of natural laws,
provided that they are in a state of \emph{uniform rectilinear
and non-rotary motion} with respect to~$K$; all these
bodies of reference are to be regarded as Galileian
reference-bodies. The validity of the principle of
relativity was assumed only for these reference-bodies,
but not for others (\eg\ those possessing motion of a
different kind). In this sense we speak of the \emph{special}
principle of relativity, or special theory of relativity.

In contrast to this we wish to understand by the
``general principle of relativity'' the following statement:
All bodies of reference $K$,~$K'$,~etc., are equivalent
for the description of natural phenomena (formulation of
the general laws of nature), whatever may be their
state of motion. But before proceeding farther, it
ought to be pointed out that this formulation must be
replaced later by a more abstract one, for reasons which
will become evident at a later stage.

Since the introduction of the special principle of
relativity has been justified, every intellect which
strives after generalisation must feel the temptation
to venture the step towards the general principle of
relativity. But a simple and apparently quite reliable
consideration seems to suggest that, for the present
at any rate, there is little hope of success in such an
attempt. Let us imagine ourselves transferred to our
\PageSep{62}
\index{Law of inertia}%
old friend the railway carriage, which is travelling at a
uniform rate. As long as it is moving uniformly, the
occupant of the carriage is not sensible of its motion,
and it is for this reason that he can without reluctance
interpret the facts of the case as indicating that the
carriage is at rest but the embankment in motion.
Moreover, according to the special principle of relativity,
this interpretation is quite justified also from a physical
point of view.

If the motion of the carriage is now changed into a
non-uniform motion, as for instance by a powerful
\index{Non-uniform motion}%
application of the brakes, then the occupant of the
carriage experiences a correspondingly powerful jerk
forwards. The retarded motion is manifested in the
mechanical behaviour of bodies relative to the person
in the railway carriage. The mechanical behaviour is
different from that of the case previously considered,
and for this reason it would appear to be impossible
that the same mechanical laws hold relatively to the non-uniformly
moving carriage, as hold with reference to the
carriage when at rest or in uniform motion. At all
events it is clear that the Galileian law does not hold
with respect to the non-uniformly moving carriage.
Because of this, we feel compelled at the present juncture
to grant a kind of absolute physical reality to non-uniform
motion, in opposition to the general principle
of relativity. But in what follows we shall soon see
that this conclusion cannot be maintained.
\PageSep{63}


\Chapter{XIX}{The Gravitational Field}

``\First{If} we pick up a stone and then let it go, why does it
fall to the ground?'' The usual answer to this
question is: ``Because it is attracted by the earth.''
Modern physics formulates the answer rather differently
for the following reason. As a result of the more careful
study of electromagnetic phenomena, we have come
to regard action at a distance as a process impossible
without the intervention of some intermediary medium.
If, for instance, a magnet attracts a piece of iron, we
cannot be content to regard this as meaning that the
magnet acts directly on the iron through the intermediate
empty space, but we are constrained to imagine---after
the manner of Faraday---that the magnet
\index{Faraday}%
always calls into being something physically real in
the space around it, that something being what we call a
``magnetic field.'' In its turn this magnetic field
\index{Magnetic field}%
operates on the piece of iron, so that the latter strives
to move towards the magnet. We shall not discuss
here the justification for this incidental conception,
which is indeed a somewhat arbitrary one. We shall
only mention that with its aid electromagnetic phenomena
can be theoretically represented much more
satisfactorily than without it, and this applies particularly
\index{Electromagnetic theory!waves}%
to the transmission of electromagnetic waves.
\PageSep{64}
The effects of gravitation also are regarded in an analogous
\index{Gravitation}%
manner.

The action of the earth on the stone takes place indirectly.
The earth produces in its surroundings a
gravitational field, which acts on the stone and produces
\index{Gravitational field}%
its motion of fall. As we know from experience, the
intensity of the action on a body diminishes according
to a quite definite law, as we proceed farther and farther
away from the earth. From our point of view this
means: The law governing the properties of the gravitational
field in space must be a perfectly definite one, in
order correctly to represent the diminution of gravitational
action with the distance from operative bodies.
It is something like this: The body (\eg\ the earth) produces
a field in its immediate neighbourhood directly;
the intensity and direction of the field at points farther
removed from the body are thence determined by
the law which governs the properties in space of the
gravitational fields themselves.

In contrast to electric and magnetic fields, the gravitational
field exhibits a most remarkable property, which
is of fundamental importance for what follows. Bodies
which are moving under the sole influence of a gravitational
field receive an acceleration, \emph{which does not in the
\index{Acceleration}%
least depend either on the material or on the physical
state of the body}. For instance, a piece of lead and
a piece of wood fall in exactly the same manner in a
gravitational field (\textit{in~vacuo}), when they start off from
rest or with the same initial velocity. This law, which
holds most accurately, can be expressed in a different
form in the light of the following consideration.

According to Newton's law of motion, we have
\index{Newton's!law of motion}%
\[
(\text{Force}) = (\text{inertial mass}) � (\text{acceleration}),
\]
\PageSep{65}
where the ``inertial mass'' is a characteristic constant
\index{Inertial mass}%
of the accelerated body. If now gravitation is the
cause of the acceleration, we then have
%[** TN: Re-breaking next two displayed equations]
\begin{multline*}
(\text{Force})
  = (\text{gravitational mass}) \\
  � (\text{intensity of the gravitational field}),
\index{Gravitational mass}%
\end{multline*}
where the ``gravitational mass'' is likewise a characteristic
constant for the body. From these two relations
follows:
\begin{multline*}
(\text{acceleration})
  = \frac{(\text{gravitational mass})}{(\text{inertial mass})} \\
  � (\text{intensity of the gravitational field}).
\end{multline*}

If now, as we find from experience, the acceleration is
to be independent of the nature and the condition of the
body and always the same for a given gravitational
field, then the ratio of the gravitational to the inertial
mass must likewise be the same for all bodies. By a
suitable choice of units we can thus make this ratio
equal to unity. We then have the following law:
The \emph{gravitational} mass of a body is equal to its \emph{inertial}
mass.

It is true that this important law had hitherto been
recorded in mechanics, but it had not been \emph{interpreted}.
A satisfactory interpretation can be obtained only if we
recognise the following fact: \emph{The same} quality of a
body manifests itself according to circumstances as
``inertia'' or as ``weight'' (lit.\ ``heaviness''). In the
\index{Inertia}%
\index{Weight (heaviness)}%
following section we shall show to what extent this is
actually the case, and how this question is connected
with the general postulate of relativity.
\PageSep{66}


\Chapter{XX}{The Equality of Inertial and Gravitational
Mass as an Argument for the
General Postulate of Relativity}

\First{We} imagine a large portion of empty space, so far
removed from stars and other appreciable
masses, that we have before us approximately
the conditions required by the fundamental law of Galilei.
It is then possible to choose a Galileian reference-body for
this part of space (world), relative to which points at
rest remain at rest and points in motion continue permanently
in uniform rectilinear motion. As reference-body
let us imagine a spacious chest resembling a room
\index{Chest}%
with an observer inside who is equipped with apparatus.
Gravitation naturally does not exist for this observer.
He must fasten himself with strings to the floor,
otherwise the slightest impact against the floor will
cause him to rise slowly towards the ceiling of the
room.

To the middle of the lid of the chest is fixed externally
a hook with rope attached, and now a ``being'' (what
\index{Being@{``Being''}}%
kind of a being is immaterial to us) begins pulling at
this with a constant force. The chest together with the
observer then begin to move ``upwards'' with a
uniformly accelerated motion. In course of time their
velocity will reach unheard-of values---provided that
\PageSep{67}
we are viewing all this from another reference-body
which is not being pulled with a rope.

But how does the man in the chest regard the process?
The acceleration of the chest will be transmitted to him
\index{Acceleration}%
by the reaction of the floor of the chest. He must
therefore take up this pressure by means of his legs if
he does not wish to be laid out full length on the floor.
He is then standing in the chest in exactly the same way
as anyone stands in a room of a house on our earth.
If he release a body which he previously had in his
hand, the acceleration of the chest will no longer be
transmitted to this body, and for this reason the body
will approach the floor of the chest with an accelerated
relative motion. The observer will further convince
himself \emph{that the acceleration of the body towards the floor
of the chest is always of the same magnitude, whatever
kind of body he may happen to use for the experiment}.

Relying on his knowledge of the gravitational field
\index{Gravitational field}%
(as it was discussed in the preceding section), the man
in the chest will thus come to the conclusion that he
and the chest are in a gravitational field which is constant
with regard to time. Of course he will be puzzled for
a moment as to why the chest does not fall, in this
gravitational field. Just then, however, he discovers
the hook in the middle of the lid of the chest and the
rope which is attached to it, and he consequently comes
to the conclusion that the chest is suspended at rest in
the gravitational field.

Ought we to smile at the man and say that he errs
in his conclusion? I do not believe we ought to if we
wish to remain consistent; we must rather admit that
his mode of grasping the situation violates neither reason
nor known mechanical laws. Even though it is being
\PageSep{68}
accelerated with respect to the ``Galileian space''
first considered, we can nevertheless regard the chest
as being at rest. We have thus good grounds for
extending the principle of relativity to include bodies
of reference which are accelerated with respect to each
other, and as a result we have gained a powerful argument
for a generalised postulate of relativity.

We must note carefully that the possibility of this
mode of interpretation rests on the fundamental
property of the gravitational field of giving all bodies
\index{Gravitational mass}%
the same acceleration, or, what comes to the same thing,
on the law of the equality of inertial and gravitational
mass. If this natural law did not exist, the man in
the accelerated chest would not be able to interpret
the behaviour of the bodies around him on the supposition
of a gravitational field, and he would not be justified
on the grounds of experience in supposing his reference-body
to be ``at rest.''

Suppose that the man in the chest fixes a rope to the
inner side of the lid, and that he attaches a body to the
free end of the rope. The result of this will be to stretch
the rope so that it will hang ``vertically'' downwards.
If we ask for an opinion of the cause of tension in the
rope, the man in the chest will say: ``The suspended
body experiences a downward force in the gravitational
field, and this is neutralised by the tension of the rope;
what determines the magnitude of the tension of the
rope is the \emph{gravitational mass} of the suspended body.''
On the other hand, an observer who is poised freely in
space will interpret the condition of things thus: ``The
rope must perforce take part in the accelerated motion
of the chest, and it transmits this motion to the body
attached to it. The tension of the rope is just large
\PageSep{69}
enough to effect the acceleration of the body. That
which determines the magnitude of the tension of the
rope is the \emph{inertial mass} of the body.'' Guided by
\index{Inertial mass}%
this example, we see that our extension of the principle
of relativity implies the \emph{necessity} of the law of the
equality of inertial and gravitational mass. Thus we
have obtained a physical interpretation of this law.

From our consideration of the accelerated chest we
see that a general theory of relativity must yield important
results on the laws of gravitation. In point of
\index{Gravitation}%
fact, the systematic pursuit of the general idea of relativity
has supplied the laws satisfied by the gravitational
field. Before proceeding farther, however, I
must warn the reader against a misconception suggested
by these considerations. A gravitational field exists
for the man in the chest, despite the fact that there was
no such field for the co-ordinate system first chosen.
Now we might easily suppose that the existence of a
gravitational field is always only an \emph{apparent} one. We
might also think that, regardless of the kind of gravitational
field which may be present, we could always
choose another reference-body such that \emph{no} gravitational
field exists with reference to it. This is by no means
true for all gravitational fields, but only for those of
quite special form. It is, for instance, impossible to
choose a body of reference such that, as judged from it,
the gravitational field of the earth (in its entirety)
vanishes.

We can now appreciate why that argument is not
convincing, which we brought forward against the
general principle of relativity at the end of \Sectionref{XVIII}.
It is certainly true that the observer in the railway
carriage experiences a jerk forwards as a result of the
\PageSep{70}
application of the brake, and that he recognises in this the
non-uniformity of motion (retardation) of the carriage.
But he is compelled by nobody to refer this jerk to a
``real'' acceleration (retardation) of the carriage. He
\index{Acceleration}%
might also interpret his experience thus: ``My body of
reference (the carriage) remains permanently at rest.
With reference to it, however, there exists (during the
period of application of the brakes) a gravitational
field which is directed forwards and which is variable
with respect to time. Under the influence of this field,
the embankment together with the earth moves non-uniformly
in such a manner that their original velocity
in the backwards direction is continuously reduced.''
\PageSep{71}


\Chapter{XXI}{In what Respects are the Foundations
of Classical Mechanics and of the
Special Theory of Relativity unsatisfactory?}
\index{Classical mechanics}%
\index{Laws of Galilei-Newton!of Nature}%

\First{We} have already stated several times that
classical mechanics starts out from the following
law: Material particles sufficiently far
removed from other material particles continue to
move uniformly in a straight line or continue in a
state of rest. We have also repeatedly emphasised
that this fundamental law can only be valid for
bodies of reference~$K$ which possess certain unique
states of motion, and which are in uniform translational
motion relative to each other. Relative to other reference-bodies~$K$
the law is not valid. Both in classical
mechanics and in the special theory of relativity we
therefore differentiate between reference-bodies~$K$
relative to which the recognised ``laws of nature'' can
be said to hold, and reference-bodies~$K$ relative to which
these laws do not hold.

But no person whose mode of thought is logical can
rest satisfied with this condition of things. He asks:
``How does it come that certain reference-bodies (or
their states of motion) are given priority over other
reference-bodies (or their states of motion)? \emph{What is
\PageSep{72}
the reason for this preference?}\Change{}{''} In order to show clearly
what I mean by this question, I shall make use of a
comparison.

I am standing in front of a gas range. Standing
alongside of each other on the range are two pans so
much alike that one may be mistaken for the other.
Both are half full of water. I notice that steam is being
emitted continuously from the one pan, but not from the
other. I am surprised at this, even if I have never seen
either a gas range or a pan before. But if I now notice
a luminous something of bluish colour under the first
pan but not under the other, I cease to be astonished,
even if I have never before seen a gas flame. For I
can only say that this bluish something will cause the
emission of the steam, or at least \emph{possibly} it may do so.
If, however, I notice the bluish something in neither
case, and if I observe that the one continuously emits
steam whilst the other does not, then I shall remain
astonished and dissatisfied until I have discovered
some circumstance to which I can attribute the different
behaviour of the two pans.

Analogously, I seek in vain for a real something in
classical mechanics (or in the special theory of relativity)
to which I can attribute the different behaviour
of bodies considered with respect to the reference-systems
$K$~and~$K'$.\footnote
  {The objection is of importance more especially when the state
  of motion of the reference-body is of such a nature that it does
  not require any external agency for its maintenance, \eg\ in
  the case when the reference-body is rotating uniformly.}
Newton saw this objection and
\index{Newton}%
attempted to invalidate it, but without success. But
E.~Mach recognised it most clearly of all, and because
\index{Mach, E.}%
of this objection he claimed that mechanics must be
\PageSep{73}
placed on a new basis. It can only be got rid of by
means of a physics which is conformable to the general
principle of relativity, since the equations of such a
theory hold for every body of reference, whatever
may be its state of motion.
\PageSep{74}


\Chapter{XXII}{A Few Inferences from the General
Principle of Relativity}

\First{The} considerations of \Sectionref{XX} show that the
general principle of relativity puts us in a position
to derive properties of the gravitational field in a
\index{Gravitational field}%
purely theoretical manner. Let us suppose, for instance,
that we know the space-time ``course'' for any natural
process whatsoever, as regards the manner in which it
takes place in the Galileian domain relative to a
Galileian body of reference~$K$. By means of purely
theoretical operations (\ie\ simply by calculation) we are
then able to find how this known natural process
appears, as seen from a reference-body~$K'$ which is
accelerated relatively to~$K$. But since a gravitational
field exists with respect to this new body of reference~$K'$,
our consideration also teaches us how the gravitational
field influences the process studied.

For example, we learn that a body which is in a state
of uniform rectilinear motion with respect to~$K$ (in
accordance with the law of Galilei) is executing an
accelerated and in general curvilinear motion with
\index{Curvilinear motion}%
respect to the accelerated reference-body~$K'$ (chest).
This acceleration or curvature corresponds to the influence
on the moving body of the gravitational field
prevailing relatively to~$K'$. It is known that a gravitational
field influences the movement of bodies in this
\PageSep{75}
way, so that our consideration supplies us with nothing
essentially new.

However, we obtain a new result of fundamental
\index{Propagation of light!in gravitational fields}%
importance when we carry out the analogous consideration
for a ray of light. With respect to the Galileian
reference-body~$K$, such a ray of light is transmitted
rectilinearly with the velocity~$c$. It can easily be shown
that the path of the same ray of light is no longer a
straight line when we consider it with reference to the
accelerated chest (reference-body~$K'$). From this we
conclude, \emph{that, in general, rays of light are propagated
curvilinearly in gravitational fields}. In two respects
this result is of great importance.

In the first place, it can be compared with the reality.
Although a detailed examination of the question shows
that the curvature of light rays required by the general
theory of relativity is only exceedingly small for the
gravitational fields at our disposal in practice, its estimated
magnitude for light rays passing the sun at
grazing incidence is nevertheless $1.7$~seconds of arc.
This ought to manifest itself in the following way.
As seen from the earth, certain fixed stars appear to be
in the neighbourhood of the sun, and are thus capable
of observation during a total eclipse of the sun. At such
times, these stars ought to appear to be displaced
outwards from the sun by an amount indicated above,
as compared with their apparent position in the sky
when the sun is situated at another part of the heavens.
The examination of the correctness or otherwise of this
deduction is a problem of the greatest importance, the
early solution of which is to be expected of astronomers.\footnote
  {By means of the star photographs of two expeditions equipped
  by a Joint Committee of the Royal and Royal Astronomical
  Societies, the existence of the deflection of light demanded by
  theory was confirmed during the solar eclipse of 29th~May, 1919.
\index{Solar eclipse}%
  (Cf.\ \Appendixref{III}.)}
\PageSep{76}

In the second place our result shows that, according
to the general theory of relativity, the law of the constancy
of the velocity of light \textit{in~vacuo}, which constitutes
\index{Velocity of light}%
one of the two fundamental assumptions in the
special theory of relativity and to which we have
already frequently referred, cannot claim any unlimited
validity. A curvature of rays of light can only take
place when the velocity of propagation of light varies
with position. Now we might think that as a consequence
of this, the special theory of relativity and with
it the whole theory of relativity would be laid in the
dust. But in reality this is not the case. We can only
conclude that the special theory of relativity cannot
claim an unlimited domain of validity; its results
hold only so long as we are able to disregard the influences
of gravitational fields on the phenomena
(\eg\ of light).

Since it has often been contended by opponents of
the theory of relativity that the special theory of
relativity is overthrown by the general theory of relativity,
it is perhaps advisable to make the facts of the
case clearer by means of an appropriate comparison.
Before the development of electrodynamics the laws
\index{Electrodynamics}%
of electrostatics were looked upon as the laws of
\index{Electrostatics}%
electricity. At the present time we know that
\index{Electricity}%
electric fields can be derived correctly from electrostatic
considerations only for the case, which is never
strictly realised, in which the electrical masses are quite
at rest relatively to each other, and to the co-ordinate
system. Should we be justified in saying that for this
\PageSep{77}
reason electrostatics is overthrown by the field-equations
of Maxwell in electrodynamics? Not in the least.
\index{Maxwell!fundamental equations}%
Electrostatics is contained in electrodynamics as a
limiting case; the laws of the latter lead directly to
those of the former for the case in which the fields are
invariable with regard to time. No fairer destiny
could be allotted to any physical theory, than that it
should of itself point out the way to the introduction
of a more comprehensive theory, in which it lives on
as a limiting case.

In the example of the transmission of light just dealt
with, we have seen that the general theory of relativity
enables us to derive theoretically the influence of a
gravitational field on the course of natural processes,
\index{Gravitational field}%
the laws of which are already known when a gravitational
field is absent. But the most attractive problem,
to the solution of which the general theory of relativity
supplies the key, concerns the investigation of the laws
satisfied by the gravitational field itself. Let us consider
this for a moment.

We are acquainted with space-time domains which
behave (approximately) in a ``Galileian'' fashion under
suitable choice of reference-body, \ie\ domains in which
gravitational fields are absent. If we now refer such
a domain to a reference-body~$K'$ possessing any kind
of motion, then relative to~$K'$ there exists a gravitational
field which is variable with respect to space and
time.\footnote
  {This follows from a generalisation of the discussion in \Sectionref{XX}.}
The character of this field will of course depend
on the motion chosen for~$K'$. According to the general
theory of relativity, the general law oi the gravitational
field must be satisfied for all gravitational fields obtainable
\PageSep{78}
in this way. Even though by no means all gravitational
fields can be produced in this way, yet we may
entertain the hope that the general law of gravitation
\index{Gravitation}%
will be derivable from such gravitational fields of a
special kind. This hope has been realised in the most
beautiful manner. But between the clear vision of
this goal and its actual realisation it was necessary to
surmount a serious difficulty, and as this lies deep at
the root of things, I dare not withhold it from the reader.
We require to extend our ideas of the space-time continuum
\index{Continuum!space-time}%
still farther.
\PageSep{79}


\Chapter{XXIII}{Behaviour of Clocks and Measuring-Rods
on a Rotating Body of Reference}

\First{Hitherto} I have purposely refrained from
speaking about the physical interpretation of
space- and time-data in the case of the general
theory of relativity. As a consequence, I am guilty of a
certain slovenliness of treatment, which, as we know
from the special theory of relativity, is far from being
unimportant and pardonable. It is now high time that
we remedy this defect; but I would mention at the
outset, that this matter lays no small claims on the
patience and on the power of abstraction of the reader.

We start off again from quite special cases, which we
\index{Galileian system of co-ordinates}%
have frequently used before. Let us consider a space-time
domain in which no gravitational field exists
relative to a reference-body~$K$ whose state of motion
\index{Reference-body!rotating}%
has been suitably chosen. $K$~is then a Galileian reference-body
as regards the domain considered, and the
results of the special theory of relativity hold relative
to~$K$. Let us suppose the same domain referred to a
second body of reference~$K'$, which is rotating uniformly
with respect to~$K$. In order to fix our ideas, we shall
imagine~$K'$ to be in the form of a plane circular disc,
which rotates uniformly in its own plane about its
centre. An observer who is sitting eccentrically on the
\PageSep{80}
disc~$K'$ is sensible of a force which acts outwards in a
radial direction, and which would be interpreted as an
effect of inertia (centrifugal force) by an observer who
\index{Centrifugal force}%
was at rest with respect to the original reference-body~$K$.
But the observer on the disc may regard his disc as a
reference-body which is ``at rest''; on the basis of the
general principle of relativity he is justified in doing this.
The force acting on himself, and in fact on all other
bodies which are at rest relative to the disc, he regards
as the effect of a gravitational field. Nevertheless,
the space-distribution of this gravitational field is of a
kind that would not be possible on Newton's theory of
\index{Newton's!law of gravitation}%
gravitation.\footnote
  {The field disappears at the centre of the disc and increases
  proportionally to the distance from the centre as we proceed
  outwards.}
But since the observer believes in the
general theory of relativity, this does not disturb him;
he is quite in the right when he believes that a general
law of gravitation can be formulated---a law which not
only explains the motion of the stars correctly, but
also the field of force experienced by himself.

The observer performs experiments on his circular
disc with clocks and measuring-rods. In doing so, it
\index{Clocks}%
\index{Measuring-rod}%
is his intention to arrive at exact definitions for the
signification of time- and space-data with reference
to the circular disc~$K'$, these definitions being based on
his observations. What will be his experience in this
enterprise?

To start with, he places one of two identically constructed
clocks at the centre of the circular disc, and the
other on the edge of the disc, so that they are at rest
relative to it. We now ask ourselves whether both
clocks go at the same rate from the standpoint of the
\PageSep{81}
non-rotating Galileian reference-body~$K$. As judged
from this body, the clock at the centre of the disc has
no velocity, whereas the clock at the edge of the disc
is in motion relative to~$K$ in consequence of the rotation.
\index{Rotation}%
According to a result obtained in \Sectionref{XII}, it follows
that the latter clock goes at a rate permanently slower
than that of the clock at the centre of the circular disc,
\ie\ as observed from~$K$. It is obvious that the same effect
would be noted by an observer whom we will imagine
sitting alongside his clock at the centre of the circular
disc. Thus on our circular disc, or, to make the case
more general, in every gravitational field, a clock will
go more quickly or less quickly, according to the position
in which the clock is situated (at rest). For this reason
it is not possible to obtain a reasonable definition of time
with the aid of clocks which are arranged at rest with
\index{Clocks}%
respect to the body of reference. A similar difficulty
presents itself when we attempt to apply our earlier
definition of simultaneity in such a case, but I do not
\index{Simultaneity}%
wish to go any farther into this question.

Moreover, at this stage the definition of the space
\index{Space co-ordinates}%
co-ordinates also presents insurmountable difficulties.
If the observer applies his standard measuring-rod
\index{Measuring-rod}%
(a rod which is short as compared with the radius of
the disc) tangentially to the edge of the disc, then, as
judged from the Galileian system, the length of this rod
will be less than~$1$, since, according to \Sectionref{XII}, moving
bodies suffer a shortening in the direction of the motion.
On the other hand, the measuring-rod will not experience
a shortening in length, as judged from~$K$, if it is applied
to the disc in the direction of the radius. If, then, the
observer first measures the circumference of the disc
with his measuring-rod and then the diameter of the
\PageSep{82}
disc, on dividing the one by the other, he will not obtain
as quotient the familiar number $\pi = 3.14\dots$, but
a larger number,\footnote
  {Throughout this consideration we have to use the Galileian
  (non-rotating) system~$K$ as reference-body, since we may only
  assume the validity of the results of the special theory of relativity
  relative to~$K$ (relative to~$K'$ a gravitational field prevails).}
whereas of course, for a disc which is
at rest with respect to~$K$, this operation would yield~$\pi$
\index{Value of $\pi$}%
exactly. This proves that the propositions of Euclidean
\index{Euclidean geometry}%
geometry cannot hold exactly on the rotating disc, nor
in general in a gravitational field, at least if we attribute
the length~$1$ to the rod in all positions and in every
orientation. Hence the idea of a straight line also loses
\index{Straight line}%
its meaning. We are therefore not in a position to
define exactly the co-ordinates $x$,~$y$,~$z$ relative to the
disc by means of the method used in discussing the
special theory, and as long as the co-ordinates and times
of events have not been defined, we cannot assign an
exact meaning to the natural laws in which these occur.

Thus all our previous conclusions based on general
relativity would appear to be called in question. In
reality we must make a subtle detour in order to be
able to apply the postulate of general relativity exactly.
I shall prepare the reader for this in the
following paragraphs.
\PageSep{83}


\Chapter{XXIV}{Euclidean and Non-Euclidean Continuum}
\index{Continuum}%

\First{The} surface of a marble table is spread out in front
of me. I can get from any one point on this
table to any other point by passing continuously
from one point to a ``neighbouring'' one, and repeating
this process a (large) number of times, or, in other words,
by going from point to point without executing ``jumps.''
I am sure the reader will appreciate with sufficient
clearness what I mean here by ``neighbouring'' and by
``jumps'' (if he is not too pedantic). We express this
property of the surface by describing the latter as a
continuum.

Let us now imagine that a large number of little rods
of equal length have been made, their lengths being
small compared with the dimensions of the marble
slab. When I say they are of equal length, I mean that
one can be laid on any other without the ends overlapping.
We next lay four of these little rods on the
marble slab so that they constitute a quadrilateral
figure (a square), the diagonals of which are equally
long. To ensure the equality of the diagonals, we make
use of a little testing-rod. To this square we add
similar ones, each of which has one rod in common
with the first. We proceed in like manner with each of
these squares until finally the whole marble slab is
\PageSep{84}
laid out with squares. The arrangement is such, that
each side of a square belongs to two squares and each
corner to four squares.

It is a veritable wonder that we can carry out this
business without getting into the greatest difficulties.
We only need to think of the following. If at any
moment three squares meet at a corner, then two sides
of the fourth square are already laid, and, as a consequence,
the arrangement of the remaining two sides of
the square is already completely determined. But I
am now no longer able to adjust the quadrilateral so
that its diagonals may be equal. If they are equal
of their own accord, then this is an especial favour
of the marble slab and of the little rods, about which I
can only be thankfully surprised. We must needs
experience many such surprises if the construction is to
be successful.

If everything has really gone smoothly, then I say
that the points of the marble slab constitute a Euclidean
\index{Distance (line-interval)}%
\index{Continuum!Euclidean}%
continuum with respect to the little rod, which has been
used as a ``distance'' (line-interval). By choosing
one corner of a square as ``origin,'' I can characterise
every other corner of a square with reference to this
origin by means of two numbers. I only need state
how many rods I must pass over when, starting from the
origin, I proceed towards the ``right'' and then ``upwards,''
in order to arrive at the corner of the square
under consideration. These two numbers are then the
``Cartesian co-ordinates'' of this corner with reference
\index{Cartesian system of co-ordinates}%
to the ``Cartesian co-ordinate system'' which is determined
by the arrangement of little rods.

By making use of the following modification of this
abstract experiment, we recognise that there must also
\PageSep{85}
\index{Measurement of length}%
be cases in which the experiment would be unsuccessful.
We shall suppose that the rods ``expand'' by an amount
proportional to the increase of temperature. We heat
the central part of the marble slab, but not the periphery,
in which case two of our little rods can still be
brought into coincidence at every position on the table.
But our construction of squares must necessarily come
into disorder during the heating, because the little rods
on the central region of the table expand, whereas
those on the outer part do not.

With reference to our little rods---defined as unit
lengths---the marble slab is no longer a Euclidean continuum,
and we are also no longer in the position of defining
Cartesian co-ordinates directly with their aid,
since the above construction can no longer be carried
out. But since there are other things which are not
influenced in a similar manner to the little rods (or
perhaps not at all) by the temperature of the table, it is
possible quite naturally to maintain the point of view
that the marble slab is a ``Euclidean continuum.''
This can be done in a satisfactory manner by making a
more subtle stipulation about the measurement or the
comparison of lengths.

But if rods of every kind (\ie\ of every material) were
to behave \emph{in the same way} as regards the influence of
temperature when they are on the variably heated
marble slab, and if we had no other means of detecting
the effect of temperature than the geometrical behaviour
of our rods in experiments analogous to the one
described above, then our best plan would be to assign
the distance \emph{one} to two points on the slab, provided that
the ends of one of our rods could be made to coincide
with these two points; for how else should we define
\PageSep{86}
the distance without our proceeding being in the highest
measure grossly arbitrary? The method of Cartesian
co-ordinates must then be discarded, and replaced by
another which does not assume the validity of Euclidean
\index{Continuum!Euclidean}%
\index{Continuum!non-Euclidean}%
\index{Euclidean geometry}%
\index{Euclidean space}%
geometry for rigid bodies.\footnote
  {Mathematicians have been confronted with our problem in the
  following form. If we are given a surface (\eg\ an ellipsoid) in
  Euclidean three-dimensional space, then there exists for this
  surface a two-dimensional geometry, just as much as for a plane
  surface. Gauss undertook the task of treating this two-dimensional
\index{Gauss}%
  geometry from first principles, without making use of the
  fact that the surface belongs to a Euclidean continuum of
  three dimensions. If we imagine constructions to be made with
  rigid rods \emph{in the surface} (similar to that above with the marble
  slab), we should find that different laws hold for these from those
  resulting on the basis of Euclidean plane geometry. The surface
  is not a Euclidean continuum with respect to the rods, and we
  cannot define Cartesian co-ordinates \emph{in the surface}. Gauss
  indicated the principles according to which we can treat the
  geometrical relationships in the surface, and thus pointed out
  the way to the method of Riemann of treating multi-dimensional,
\index{Riemann}%
  non-Euclidean \textit{continua}. Thus it is that mathematicians
  long ago solved the formal problems to which we are led by the
  general postulate of relativity.}
The reader will notice that
the situation depicted here corresponds to the one
brought about by the general postulate of relativity
(\Sectionref{XXIII}).
\PageSep{87}


\Chapter{XXV}{Gaussian Co-ordinates}

\First{According} to Gauss, this combined analytical
\index{Gauss}%
and geometrical mode of handling the problem
can be arrived at in the following way. We
imagine a system of arbitrary curves (see \Figref{4})
drawn on the surface of the table. These we designate
as $u$-curves, and we indicate each of them by
means of a number. The curves $u = 1$, $u = 2$ and
$u = 3$ are drawn in the diagram. Between the curves
$u = 1$ and $u = 2$ we must imagine an infinitely large
number to be drawn, all of which correspond
%[Illustration: Fig. 4.]
\WFigure{2in}{087}
to real
numbers lying between $1$~and~$2$. We have then
a system of $u$-curves, and
this ``infinitely dense'' system
covers the whole surface
of the table. These
$u$-curves must not intersect
each other, and through each
point of the surface one and
only one curve must pass.
Thus a perfectly definite
value of~$u$ belongs to every point on the surface of the
marble slab. In like manner we imagine a system of
$v$-curves drawn on the surface. These satisfy the same
conditions as the $u$-curves, they are provided with numbers
\PageSep{88}
in a corresponding manner, and they may likewise
be of arbitrary shape. It follows that a value of~$u$ and
a value of~$v$ belong to every point on the surface of the
table. We call these two numbers the co-ordinates
of the surface of the table (Gaussian co-ordinates).
\index{Gaussian co-ordinates|(}%
For example, the point~$P$ in the diagram has the Gaussian
co-ordinates $u = 3$, $v = 1$. Two neighbouring points $P$
and~$P'$ on the surface then correspond to the co-ordinates
\begin{align*}
&P:  &&u, v \\
&P': &&u + du, v + dv,
\end{align*}
where $du$~and~$dv$ signify very small numbers. In a
similar manner we may indicate the distance (line-interval)
\index{Distance (line-interval)}%
between $P$~and~$P'$, as measured with a little
rod, by means of the very small number~$ds$. Then
according to Gauss we have
\[
ds^{2} = g_{11}\, du^{2} + 2g_{12}\, du\, dv + g_{22}\, dv^{2},
\]
where $g_{11}$,~$g_{12}$,~$g_{22}$, are magnitudes which depend in a
perfectly definite way on $u$~and~$v$. The magnitudes $g_{11}$,~$g_{12}$
and~$g_{22}$ determine the behaviour of the rods relative
to the $u$-curves and $v$-curves, and thus also relative
to the surface of the table. For the case in which the
points of the surface considered form a Euclidean continuum
\index{Continuum!Euclidean}%
with reference to the measuring-rods, but
only in this case, it is possible to draw the $u$-curves
and $v$-curves and to attach numbers to them, in such a
manner, that we simply have:
\[
ds^{2} = du^{2} + dv^{2}.
\]
Under these conditions, the $u$-curves and $v$-curves are
straight lines in the sense of Euclidean geometry, and
\index{Euclidean geometry}%
\index{Straight line}%
they are perpendicular to each other. Here the Gaussian
co-ordinates are simply Cartesian ones. It is clear
\PageSep{89}
that Gauss co-ordinates are nothing more than an
association of two sets of numbers with the points of
the surface considered, of such a nature that numerical
values differing very slightly from each other are
associated with neighbouring points ``in space.''

So far, these considerations hold for a continuum
\index{Continuum!four-dimensional}%
of two dimensions. But the Gaussian method can be
applied also to a continuum of three, four or more
dimensions. If, for instance, a continuum of four
dimensions be supposed available, we may represent
it in the following way. With every point of the
continuum we associate arbitrarily four numbers, $x_{1}$,~$x_{2}$,
$x_{3}$,~$x_{4}$, which are known as ``co-ordinates.'' Adjacent
points correspond to adjacent values of the co-ordinates.
If a distance~$ds$ is associated with the adjacent points
\index{Adjacent points}%
$P$~and~$P'$, this distance being measurable and well-defined
from a physical point of view, then the following
formula holds:
\[
ds^{2} = g_{11}\, {dx_{1}}^{2}
    + 2g_{12}\, dx_{1}\, dx_{2} \Add{+} \dots
    +  g_{44}\, {dx_{4}}^{2},
\]
where the magnitudes $g_{11}$,~etc., have values which vary
with the position in the continuum. Only when the
continuum is a Euclidean one is it possible to associate
the co-ordinates $x_{1}$\Add{,}\ldots\Add{,}~$x_{4}$ with the points of the
continuum so that we have simply
\[
ds^{2} = {dx_{1}}^{2} + {dx_{2}}^{2} + {dx_{3}}^{2} + {dx_{4}}^{2}.
\]
In this case relations hold in the four-dimensional
continuum which are analogous to those holding in our
three-dimensional measurements.

However, the Gauss treatment for~$ds^{2}$ which we have
given above is not always possible. It is only possible
when sufficiently small regions of the continuum under
consideration may be regarded as Euclidean continua.
\PageSep{90}
For example, this obviously holds in the case of the
marble slab of the table and local variation of temperature.
The temperature is practically constant for a small
part of the slab, and thus the geometrical behaviour of
the rods is \emph{almost} as it ought to be according to the
rules of Euclidean geometry. Hence the imperfections
\index{Continuum!non-Euclidean}%
of the construction of squares in the previous section
do not show themselves clearly until this construction
is extended over a considerable portion of the surface
of the table.

We can sum this up as follows: Gauss invented a
\index{Gauss}%
method for the mathematical treatment of continua in
general, in which ``size-relations'' (``distances'' between
\index{Size-relations}%
neighbouring points) are defined. To every point of a
continuum are assigned as many numbers (Gaussian co-ordinates)
as the continuum has dimensions. This is
done in such a way, that only one meaning can be attached
to the assignment, and that numbers (Gaussian co-ordinates)
\index{Gaussian co-ordinates|)}%
which differ by an indefinitely small amount
are assigned to adjacent points. The Gaussian co-ordinate
system is a logical generalisation of the Cartesian
co-ordinate system. It is also applicable to non-Euclidean
continua, but only when, with respect to the defined
``size'' or ``distance,'' small parts of the continuum
under consideration behave more nearly like a Euclidean
system, the smaller the part of the continuum under
our notice.
\PageSep{91}


\Chapter{XXVI}{The Space-Time Continuum of the Special
Theory of Relativity considered as
a Euclidean Continuum}
\index{Continuum!four-dimensional}%
\index{Continuum!space-time|(}%

\First{We} are now in a position to formulate more
exactly the idea of Minkowski, which was
\index{Minkowski}%
only vaguely indicated in \Sectionref{XVII}.
In accordance with the special theory of relativity,
certain co-ordinate systems are given preference
for the description of the four-dimensional, space-time
continuum. We called these ``Galileian co-ordinate
\index{Galileian system of co-ordinates}%
systems.'' For these systems, the four co-ordinates
$x$,~$y$, $z$,~$t$, which determine an event or---in other
words---a point of the four-dimensional continuum, are
defined physically in a simple manner, as set forth in
detail in the first part of this book. For the transition
from one Galileian system to another, which is moving
uniformly with reference to the first, the equations of
the Lorentz transformation are valid. These last
\index{Lorentz, H. A.!transformation}%
form the basis for the derivation of deductions from the
special theory of relativity, and in themselves they are
nothing more than the expression of the universal
validity of the law of transmission of light for all Galileian
\index{Propagation of light}%
systems of reference.

Minkowski found that the Lorentz transformations
satisfy the following simple conditions. Let us consider
\PageSep{92}
two neighbouring events, the relative position of which
in the four-dimensional continuum is given with respect
\index{Continuum!four-dimensional}%
to a Galileian reference-body~$K$ by the space co-ordinate
\index{Coordinate@{Co-ordinate}!differences}%
\index{Coordinate@{Co-ordinate}!differentials}%
differences $dx$,~$dy$,~$dz$ and the time-difference~$dt$. With
reference to a second Galileian system we shall suppose
that the corresponding differences for these two events
are $dx'$,~$dy'$, $dz'$,~$dt'$. Then these magnitudes always
fulfil the condition\footnote
  {Cf.\ Appendices I~and~II\@. The relations which are derived
  there for the co-ordinates themselves are valid also for co-ordinate
  \emph{differences}, and thus also for co-ordinate differentials
  (indefinitely small differences).}
\[
dx^{2} + dy^{2} + dz^{2} - c^{2}\, dt^{2}
  = dx'^{2} + dy'^{2} + dz'^{2} - c^{2}\, dt'^{2}.
\]

The validity of the Lorentz transformation follows
from this condition. We can express this as follows:
The magnitude
\[
ds^{2} = dx^{2} + dy^{2} + dz^{2} - c^{2}\, dt^{2},
\]
which belongs to two adjacent points of the four-dimensional
space-time continuum, has the same value
for all selected (Galileian) reference-bodies. If we replace
$x$,~$y$, $z$,~$\sqrt{-1}\,ct$, by $x_{1}$,~$x_{2}$, $x_{3}$,~$x_{4}$, we also obtain the
result that
\[
ds^{2} = {dx_{1}}^{2} + {dx_{2}}^{2} + {dx_{3}}^{2} + {dx_{4}}^{2}
\]
is independent of the choice of the body of reference.
We call the magnitude~$ds$ the ``distance'' apart of the
two events or four-dimensional points.

Thus, if we choose as time-variable the imaginary
variable~$\sqrt{-1}\,ct$ instead of the real quantity~$t$, we can
regard the space-time continuum---in accordance with
the special theory of relativity---as a ``Euclidean''
\index{Continuum!Euclidean}%
four-dimensional continuum, a result which follows
from the considerations of the preceding section.
\PageSep{93}


\Chapter{XXVII}{The Space-Time Continuum of the
General Theory of Relativity is
not a Euclidean Continuum}

\First{In} the first part of this book we were able to make use
of space-time co-ordinates which allowed of a simple
and direct physical interpretation, and which, according
to \Sectionref{XXVI}, can be regarded as four-dimensional
Cartesian co-ordinates. This was possible on the basis
of the law of the constancy of the velocity of light. But
according to \Sectionref{XXI}, the general theory of relativity
cannot retain this law. On the contrary, we arrived at
the result that according to this latter theory the
velocity of light must always depend on the co-ordinates
when a gravitational field is present. In connection
\index{Gravitational field}%
with a specific illustration in \Sectionref{XXIII}, we found
that the presence of a gravitational field invalidates the
definition of the co-ordinates and the time, which led us
to our objective in the special theory of relativity.

In view of the results of these considerations we are
led to the conviction that, according to the general
principle of relativity, the space-time continuum cannot
be regarded as a Euclidean one, but that here we have
the general case, corresponding to the marble slab with
local variations of temperature, and with which we
made acquaintance as an example of a two-dimensional
\PageSep{94}
continuum. Just as it was there impossible to construct
\index{Continuum!two-dimensional}%
\index{Continuum!four-dimensional}%
a Cartesian co-ordinate system from equal rods, so
here it is impossible to build up a system (reference-body)
from rigid bodies and clocks, which shall be of
\index{Clocks}%
such a nature that measuring-rods and clocks, arranged
\index{Measuring-rod}%
rigidly with respect to one another, shall indicate position
and time directly. Such was the essence of the
difficulty with which we were confronted in \Sectionref{XXIII}.

But the considerations of Sections \Srefno{XXV}~and~\Srefno{XXVI}
show us the way to surmount this difficulty. We refer the
four-dimensional space-time continuum in an arbitrary
manner to Gauss co-ordinates. We assign to every
\index{Gaussian co-ordinates}%
point of the continuum (event) four numbers, $x_{1}$,~$x_{2}$,
$x_{3}$,~$x_{4}$ (co-ordinates), which have not the least direct
physical significance, but only serve the purpose of
numbering the points of the continuum in a definite
but arbitrary manner. This arrangement does not even
need to be of such a kind that we must regard $x_{1}$,~$x_{2}$,~$x_{3}$ as
``space'' co-ordinates and $x_{4}$~as a ``time'' co-ordinate.

The reader may think that such a description of the
world would be quite inadequate. What does it mean
to assign to an event the particular co-ordinates $x_{1}$,~$x_{2}$,
$x_{3}$,~$x_{4}$, if in themselves these co-ordinates have no
significance? More careful consideration shows, however,
that this anxiety is unfounded. Let us consider,
for instance, a material point with any kind of motion.
If this point had only a momentary existence without
duration, then it would be described in space-time by a
single system of values  $x_{1}$,~$x_{2}$, $x_{3}$,~$x_{4}$. Thus its permanent
existence must be characterised by an infinitely large
number of such systems of values, the co-ordinate values
of which are so close together as to give continuity;
\PageSep{95}
corresponding to the material point, we thus have a
(uni-dimensional) line in the four-dimensional continuum.
\index{Continuity}%
In the same way, any such lines in our continuum
correspond to many points in motion. The only statements
having regard to these points which can claim
a physical existence are in reality the statements about
their encounters. In our mathematical treatment,
such an encounter is expressed in the fact that the two
lines which represent the motions of the points in
question have a particular system of co-ordinate values,
$x_{1}$,~$x_{2}$, $x_{3}$,~$x_{4}$, in common. After mature consideration
the reader will doubtless admit that in reality such
encounters constitute the only actual evidence of a
time-space nature with which we meet in physical
statements.

When we were describing the motion of a material
\index{Encounter (space-time coincidence)}%
point relative to a body of reference, we stated
nothing more than the encounters of this point with
particular points of the reference-body. We can also
determine the corresponding values of the time by the
observation of encounters of the body with clocks, in
\index{Clocks}%
conjunction with the observation of the encounter of the
hands of clocks with particular points on the dials.
It is just the same in the case of space-measurements by
means of measuring-rods, as a little consideration will
show.

The following statements hold generally: Every
physical description resolves itself into a number of
statements, each of which refers to the space-time
coincidence of two events $A$~and~$B$. In terms of
Gaussian co-ordinates, every such statement is expressed
by the agreement of their four co-ordinates  $x_{1}$,~$x_{2}$, $x_{3}$,~$x_{4}$.
Thus in reality, the description of the time-space
\PageSep{96}
continuum by means of Gauss co-ordinates completely
\index{Gaussian co-ordinates|(}%
replaces the description with the aid of a body of reference,
without suffering from the defects of the latter
mode of description; it is not tied down to the Euclidean
character of the continuum which has to be represented.
\index{Continuum!space-time|)}%
\PageSep{97}


\Chapter{XXVIII}{Exact Formulation of the General
Principle of Relativity}
\index{General theory of relativity}%

\First{We} are now in a position to replace the provisional
formulation of the general principle
of relativity given in \Sectionref{XVIII} by
an exact formulation. The form there used, ``All
bodies of reference $K$,~$K'$,~etc., are equivalent for
the description of natural phenomena (formulation of
the general laws of nature), whatever may be their
state of motion,'' cannot be maintained, because the
use of rigid reference-bodies, in the sense of the method
followed in the special theory of relativity, is in general
not possible in space-time description. The Gauss
co-ordinate system has to take the place of the body of
reference. The following statement corresponds to the
fundamental idea of the general principle of relativity:
``\emph{All Gaussian co-ordinate systems are essentially equivalent
for the formulation of the general laws of nature.}''

We can state this general principle of relativity in still
another form, which renders it yet more clearly intelligible
than it is when in the form of the natural
extension of the special principle of relativity. According
to the special theory of relativity, the equations
which express the general laws of nature pass over into
equations of the same form when, by making use of the
Lorentz transformation, we replace the space-time
\index{Lorentz, H. A.!transformation}%
\PageSep{98}
variables $x$,~$y$, $z$,~$t$, of a (Galileian) reference-body~$K$
by the space-time variables $x'$,~$y'$, $z'$,~$t'$, of a new reference-body~$K'$.
According to the general theory
of relativity, on the other hand, by application of
\emph{arbitrary substitutions} of the Gauss variables $x_{1}$,~$x_{2}$, $x_{3}$,~$x_{4}$,
\index{Arbitrary substitutions}%
the equations must pass over into equations of the same
form; for every transformation (not only the Lorentz
\index{Lorentz, H. A.!transformation}%
transformation) corresponds to the transition of one
Gauss co-ordinate system into another.

If we desire to adhere to our ``old-time'' three-dimensional
\index{Law of inertia}%
view of things, then we can characterise
the development which is being undergone by the
fundamental idea of the general theory of relativity
as follows: The special theory of relativity has reference
to Galileian domains, \ie\ to those in which no gravitational
field exists. In this connection a Galileian reference-body
\index{Galileian system of co-ordinates}%
serves as body of reference, \ie\ a rigid
body the state of motion of which is so chosen that the
Galileian law of the uniform rectilinear motion of
``isolated'' material points holds relatively to it.

Certain considerations suggest that we should refer
the same Galileian domains to \emph{non-Galileian} reference-bodies
\index{Non-Galileian reference-bodies}%
also. A gravitational field of a special kind is
\index{Gravitational field}%
then present with respect to these bodies (cf.\ Sections \Srefno{XX}
and~\Srefno{XXIII}).

In gravitational fields there are no such things as rigid
\index{Time!in Physics}%
bodies with Euclidean properties; thus the fictitious rigid
body of reference is of no avail in the general theory of
relativity. The motion of clocks is also influenced by
\index{Clocks|(}%
gravitational fields, and in such a way that a physical
definition of time which is made directly with the aid of
clocks has by no means the same degree of plausibility
as in the special theory of relativity.
\PageSep{99}
\index{Laws of Galilei-Newton!of Nature}%
\index{Time!coordinate@{co-ordinate}}%

For this reason non-rigid reference-bodies are used,
which are as a whole not only moving in any way
whatsoever, but which also suffer alterations in form
\textit{ad~lib.}\ during their motion. Clocks, for which the law of
motion is of any kind, however irregular, serve for the
definition of time. We have to imagine each of these
clocks fixed at a point on the non-rigid reference-body.
\index{Reference-mollusk|(}%
These clocks satisfy only the one condition, that the
``readings'' which are observed simultaneously on
adjacent clocks (in space) differ from each other by an
\index{Space!point@{-point}}%
indefinitely small amount. This non-rigid reference-body,
which might appropriately be termed a ``reference-mollusk,''
is in the main equivalent to a Gaussian four-dimensional
co-ordinate system chosen arbitrarily.
That which gives the ``mollusk'' a certain comprehensibleness
as compared with the Gauss co-ordinate
system is the (really unjustified) formal retention of
the separate existence of the space co-ordinates as
\index{Space co-ordinates}%
opposed to the time co-ordinate. Every point on the
mollusk is treated as a space-point, and every material
point which is at rest relatively to it as at rest, so long as
the mollusk is considered as reference-body. The
general principle of relativity requires that all these
mollusks can be used as reference-bodies with equal
right and equal success in the formulation of the general
laws of nature; the laws themselves must be quite
independent of the choice of mollusk.

The great power possessed by the general principle
of relativity lies in the comprehensive limitation which
is imposed on the laws of nature in consequence of what
we have seen above.
\PageSep{100}


\Chapter{XXIX}{The Solution of the Problem of Gravitation
on the Basis of the General
Principle of Relativity}

\First{If} the reader has followed all our previous considerations,
he will have no further difficulty in
understanding the methods leading to the solution
of the problem of gravitation.

We start off from a consideration of a Galileian
domain, \ie\ a domain in which there is no gravitational
field relative to the Galileian reference-body~$K$. The
\index{Galileian system of co-ordinates}%
behaviour of measuring-rods and clocks with reference
\index{Measuring-rod}%
to~$K$ is known from the special theory of relativity,
likewise the behaviour of ``isolated'' material points;
the latter move uniformly and in straight lines.

Now let us refer this domain to a random Gauss co-ordinate
system or to a ``mollusk'' as reference-body~$K'$.
Then with respect to~$K'$ there is a gravitational
field~$G$ (of a particular kind). We learn the behaviour
of measuring-rods and clocks and also of freely-moving
material points with reference to~$K'$ simply by mathematical
transformation. We interpret this behaviour
as the behaviour of measuring-rods, clocks and material
\index{Clocks|)}%
points under the influence of the gravitational field~$G$.
\index{Gravitational field}%
Hereupon we introduce a hypothesis: that the influence
of the gravitational field on measuring-rods,
\index{Gaussian co-ordinates|)}%
\PageSep{101}
clocks and freely-moving material points continues to
take place according to the same laws, even in the case
when the prevailing gravitational field is \emph{not} derivable
\index{Gravitational field}%
from the Galileian special case, simply by means of a
transformation of co-ordinates.

The next step is to investigate the space-time
behaviour of the gravitational field~$G$, which was derived
from the Galileian special case simply by transformation
of the co-ordinates. This behaviour is formulated
in a law, which is always valid, no matter how the
\index{Matter}%
reference-body (mollusk) used in the description may
\index{Reference-mollusk|)}%
be chosen.

This law is not yet the \emph{general} law of the gravitational
field, since the gravitational field under consideration is
of a special kind. In order to find out the general
law-of-field of gravitation we still require to obtain a
generalisation of the law as found above. This can be
obtained without caprice, however, by taking into
consideration the following demands:
\begin{itemize}
\item[\itema] The required generalisation must likewise satisfy
  the general postulate of relativity.

\item[\itemb] If there is any matter in the domain under consideration,
  only its inertial mass, and thus
\index{Inertial mass}%
  according to \Sectionref{XV} only its energy is of
  importance for its effect in exciting a field.

\item[\itemc] Gravitational field and matter together must
  satisfy the law of the conservation of energy
\index{Conservation of energy}%
\index{Conservation of energy!impulse}%
\index{Kinetic energy}%
  (and of impulse).
\end{itemize}

Finally, the general principle of relativity permits
us to determine the influence of the gravitational field
on the course of all those processes which take place
according to known laws when a gravitational field is
\PageSep{102}
absent, \ie\ which have already been fitted into the
frame of the special theory of relativity. In this connection
we proceed in principle according to the method
which has already been explained for measuring-rods,
\index{Measuring-rod}%
clocks and freely-moving material points.
\index{Clocks}%

The theory of gravitation derived in this way from
\index{Gravitation}%
the general postulate of relativity excels not only in
its beauty; nor in removing the defect attaching to
classical mechanics which was brought to light in \Sectionref{XXI};
\index{Classical mechanics}%
nor in interpreting the empirical law of the equality
of inertial and gravitational mass; but it has also
\index{Gravitational mass}%
\index{Inertial mass}%
already explained a result of observation in astronomy,
\index{Astronomy}%
against which classical mechanics is powerless.

If we confine the application of the theory to the
case where the gravitational fields can be regarded as
being weak, and in which all masses move with respect
to the co-ordinate system with velocities which are
small compared with the velocity of light, we then obtain
as a first approximation the Newtonian theory. Thus
the latter theory is obtained here without any particular
assumption, whereas Newton had to introduce the
\index{Newton}%
hypothesis that the force of attraction between mutually
attracting material points is inversely proportional to
the square of the distance between them. If we increase
the accuracy of the calculation, deviations from
the theory of Newton make their appearance, practically
all of which must nevertheless escape the test of
observation owing to their smallness.

We must draw attention here to one of these deviations.
According to Newton's theory, a planet moves
round the sun in an ellipse, which would permanently
maintain its position with respect to the fixed stars,
if we could disregard the motion of the fixed stars
\index{Motion!of heavenly bodies}%
\PageSep{103}
themselves and the action of the other planets under
consideration. Thus, if we correct the observed motion
of the planets for these two influences, and if Newton's
theory be strictly correct, we ought to obtain for the
orbit of the planet an ellipse, which is fixed with reference
to the fixed stars. This deduction, which can
be tested with great accuracy, has been confirmed
for all the planets save one, with the precision that is
capable of being obtained by the delicacy of observation
attainable at the present time. The sole exception
is Mercury, the planet which lies nearest the sun. Since
\index{Mercury}%
\index{Mercury!orbit of}%
the time of Leverrier, it has been known that the ellipse
\index{Leverrier}%
corresponding to the orbit of Mercury, after it has been
corrected for the influences mentioned above, is not
stationary with respect to the fixed stars, but that it
rotates exceedingly slowly in the plane of the orbit
and in the sense of the orbital motion. The value
obtained for this rotary movement of the orbital ellipse
was $43$~seconds of arc per~century, an amount ensured
to be correct to within a few seconds of arc. This
effect can be explained by means of classical mechanics
\index{Classical mechanics}%
only on the assumption of hypotheses which have
little probability, and which were devised solely for
this purpose.

On the basis of the general theory of relativity, it
is found that the ellipse of every planet round the sun
must necessarily rotate in the manner indicated above;
that for all the planets, with the exception of Mercury,
this rotation is too small to be detected with the delicacy
of observation possible at the present time; but that in
the case of Mercury it must amount to $43$~seconds of
arc per century, a result which is strictly in agreement
with observation.
\PageSep{104}

Apart from this one, it has hitherto been possible to
make only two deductions from the theory which admit
of being tested by observation, to wit, the curvature
\index{Curvature of light-rays}%
of light rays by the gravitational field of the sun,\footnote
  {Observed by Eddington and others in~1919. (Cf.\ \Appendixref{III}.)}
\index{Eddington}%
and a displacement of the spectral lines of light reaching
\index{Displacement of spectral lines}%
us from large stars, as compared with the corresponding
lines for light produced in an analogous manner terrestrially
(\ie\ by the same kind of molecule). I do not
doubt that these deductions from the theory will be
confirmed also.
\index{General theory of relativity|)}%
\PageSep{105}


\Part{III}{Considerations on the Universe as
a Whole}{Considerations on the Universe}

\Chapter{XXX}{Cosmological Difficulties of Newton's
Theory}
\index{Newton}%

\First{Apart} from the difficulty discussed in \Sectionref{XXI},
there is a second fundamental difficulty
attending classical celestial mechanics, which,
\index{Celestial mechanics}%
to the best of my knowledge, was first discussed in
detail by the astronomer Seeliger. If we ponder over
\index{Seeliger}%
the question as to how the universe, considered as a
whole, is to be regarded, the first answer that suggests
itself to us is surely this: As regards space (and time)
\index{Space}%
\index{Time!conception of}%
the universe is infinite. There are stars everywhere,
so that the density of matter, although very variable
in detail, is nevertheless on the average everywhere the
same. In other words: However far we might travel
through space, we should find everywhere an attenuated
swarm of fixed stars of approximately the same kind
and density.

This view is not in harmony with the theory of
Newton. The latter theory rather requires that the
universe should have a kind of centre in which the
\PageSep{106}
density of the stars is a maximum, and that as we
proceed outwards from this centre the group-density
\index{Group-density of stars}%
of the stars should diminish, until finally, at great
distances, it is succeeded by an infinite region of emptiness.
The stellar universe ought to be a finite island in
\index{Stellar universe}%
the infinite ocean of space.\footnote
  {\textit{Proof}---According to the theory of Newton, the number of
  ``lines of force'' which come from infinity and terminate in a
\index{Lines of force}%
  mass~$m$ is proportional to the mass~$m$. If, on the average, the
  mass-density~$\rho_{0}$ is constant throughout the universe, then a
  sphere of volume~$V$ will enclose the average mass~$\rho_{0}V$. Thus
  the number of lines of force passing through the surface~$F$ of the
  sphere into its interior is proportional to~$\rho_{0}V$. For unit area
  of the surface of the sphere the number of lines of force which
  enters the sphere is thus proportional to~$\rho_{0}\dfrac{V}{F}$ or to~$\rho_{0}R$. Hence
  the intensity of the field at the surface would ultimately become
  infinite with increasing radius~$R$ of the sphere, which is impossible.}

This conception is in itself not very satisfactory.
It is still less satisfactory because it leads to the result
that the light emitted by the stars and also individual
stars of the stellar system are perpetually passing out
into infinite space, never to return, and without ever
again coming into interaction with other objects of
nature. Such a finite material universe would be
destined to become gradually but systematically impoverished.

In order to escape this dilemma, Seeliger suggested a
\index{Intensity of gravitational field}%
\index{Seeliger}%
modification of Newton's law, in which he assumes that
\index{Newton's!law of gravitation}%
for great distances the force of attraction between two
masses diminishes more rapidly than would result from
the inverse square law. In this way it is possible for the
mean density of matter to be constant everywhere,
even to infinity, without infinitely large gravitational
fields being produced. We thus free ourselves from the
\PageSep{107}
distasteful conception that the material universe ought
to possess something of the nature of a centre. Of
course we purchase our emancipation from the fundamental
difficulties mentioned, at the cost of a modification
and complication of Newton's law which has
neither empirical nor theoretical foundation. We can
imagine innumerable laws which would serve the same
purpose, without our being able to state a reason why
one of them is to be preferred to the others; for any
one of these laws would be founded just as little on
more general theoretical principles as is the law of
Newton.
\PageSep{108}


\Chapter{XXXI}{The Possibility of a ``Finite'' and yet
``Unbounded'' Universe}
\index{Universe (World) structure of}%

\First{But} speculations on the structure of the universe
also move in quite another direction. The
development of non-Euclidean geometry led to
\index{Euclidean geometry}%
\index{Non-Euclidean geometry}%
the recognition of the fact, that we can cast doubt on the
\emph{infiniteness} of our space without coming into conflict
with the laws of thought or with experience (Riemann,
\index{Riemann}%
Helmholtz). These questions have already been treated
\index{Helmholtz}%
in detail and with unsurpassable lucidity by Helmholtz
and Poincar�, whereas I can only touch on them
\index{Poincare@{Poincar�}}%
briefly here.

In the first place, we imagine an existence in two-dimensional
\index{Being@{``Being''}}%
\index{Space!two-dimensional}%
space. Flat beings with flat implements,
and in particular flat rigid measuring-rods, are free to
move in a \emph{plane}. For them nothing exists outside of
\index{Plane}%
this plane: that which they observe to happen to
themselves and to their flat ``things'' is the all-inclusive
reality of their plane. In particular, the constructions
of plane Euclidean geometry can be carried out by
means of the rods, \eg\ the lattice construction, considered
\index{Lattice}%
in \Sectionref{XXIV}. In contrast to ours, the universe of
these beings is two-dimensional; but, like ours, it extends
to infinity. In their universe there is room for an
infinite number of identical squares made up of rods,
\PageSep{109}
\ie\ its volume (surface) is infinite. If these beings say
their universe is ``plane,'' there is sense in the statement,
\index{Plane}%
\index{Universe!Euclidean}%
because they mean that they can perform the constructions
of plane Euclidean geometry with their rods.
\index{Euclidean geometry}%
In this connection the individual rods always represent
\index{Distance (line-interval)}%
the same distance, independently of their position.

Let us consider now a second two-dimensional existence,
but this time on a spherical surface instead of on
\index{Spherical!surface}%
a plane. The flat beings with their measuring-rods
and other objects fit exactly on this surface and they
are unable to leave it. Their whole universe of observation
extends exclusively over the surface of the sphere.
Are these beings able to regard the geometry of their
universe as being plane geometry and their rods withal
as the realisation of ``distance''? They cannot do
this. For if they attempt to realise a straight line, they
\index{Straight line}%
will obtain a curve, which we ``three-dimensional
beings'' designate as a great circle, \ie\ a self-contained
line of definite finite length, which can be measured
up by means of a measuring-rod. Similarly, this
universe has a finite area that can be compared with the
area of a square constructed with rods. The great
charm resulting from this consideration lies in the
recognition of the fact that \emph{the universe of these beings is
finite and yet has no limits}.

But the spherical-surface beings do not need to go
on a world-tour in order to perceive that they are not
\index{World}%
living in a Euclidean universe. They can convince
themselves of this on every part of their ``world,''
provided they do not use too small a piece of it. Starting
from a point, they draw ``straight lines'' (arcs of circles
as judged in three-dimensional space) of equal length
in all directions. They will call the line joining the
\PageSep{110}
free ends of these lines a ``circle.'' For a plane surface,
the ratio of the circumference of a circle to its diameter,
both lengths being measured with the same rod, is,
according to Euclidean geometry of the plane, equal to
a constant value~$\pi$, which is independent of the diameter
\index{Value of $\pi$}%
of the circle. On their spherical surface our flat beings
would find for this ratio the value
\[
\pi = \frac{\sin\left(\dfrac{r}{R}\right)}{\left(\dfrac{r}{R}\right)},
\]
\ie\ a smaller value than~$\pi$, the difference being the
more considerable, the greater is the radius of the
circle in comparison with the radius~$R$ of the ``world-sphere.''
\index{World!sphere@{-sphere}}%
By means of this relation the spherical beings
can determine the radius of their universe (``world''),
even when only a relatively small part of their world-sphere
is available for their measurements. But if this
part is very small indeed, they will no longer be able to
demonstrate that they are on a spherical ``world'' and
not on a Euclidean plane, for a small part of a spherical
surface differs only slightly from a piece of a plane of
the same size.

Thus if the spherical-surface beings are living on a
planet of which the solar system occupies only a negligibly
small part of the spherical universe, they have no means
of determining whether they are living in a finite or in
an infinite universe, because the ``piece of universe''
to which they have access is in both cases practically
plane, or Euclidean. It follows directly from this
discussion, that for our sphere-beings the circumference
of a circle first increases with the radius until the ``circumference
\PageSep{111}
\index{Universe (World) structure of!circumference of}%
of the universe'' is reached, and that it
\index{Universe!Euclidean}%
\index{Universe!spherical}%
thenceforward gradually decreases to zero for still
further increasing values of the radius. During this
process the area of the circle continues to increase
more and more, until finally it becomes equal to the
total area of the whole ``world-sphere.''
\index{World!sphere@{-sphere}}%

Perhaps the reader will wonder why we have placed
our ``beings'' on a sphere rather than on another closed
surface. But this choice has its justification in the fact
that, of all closed surfaces, the sphere is unique in possessing
the property that all points on it are equivalent. I
admit that the ratio of the circumference~$c$ of a circle
to its radius~$r$ depends on~$r$, but for a given value of~$r$
it is the same for all points of the ``world-sphere'';
in other words, the ``world-sphere'' is a ``surface of
constant curvature.''

To this two-dimensional sphere-universe there is a
three-dimensional analogy, namely, the three-dimensional
spherical space which was discovered by Riemann. Its
\index{Riemann}%
points are likewise all equivalent. It possesses a finite
volume, which is determined by its ``radius'' ($2\pi^{2}R^{3}$).
Is it possible to imagine a spherical space? To imagine
a space means nothing else than that we imagine an
epitome of our ``space'' experience, \ie\ of experience
that we can have in the movement of ``rigid'' bodies.
In this sense we \emph{can} imagine a spherical space.

Suppose we draw lines or stretch strings in all directions
from a point, and mark off from each of these
the distance~$r$ with a measuring-rod. All the free end-points
\index{Measuring-rod}%
of these lengths lie on a spherical surface. We
\index{Spherical!space}%
can specially measure up the area~($F$) of this surface
by means of a square made up of measuring-rods. If
the universe is Euclidean, then $F = 4\pi r^{2}$; if it is spherical,
\PageSep{112}
then $F$~is always less than~$4\pi r^{2}$. With increasing
values of~$r$, $F$~increases from zero up to a maximum
value which is determined by the ``world-radius,'' but
\index{World!radius@{-radius}}%
for still further increasing values of~$r$, the area gradually
diminishes to zero. At first, the straight lines which
radiate from the starting point diverge farther and
farther from one another, but later they approach
each other, and finally they run together again at a
``counter-point'' to the starting point. Under such
\index{Counter-Point}%
conditions they have traversed the whole spherical
space. It is easily seen that the three-dimensional
spherical space is quite analogous to the two-dimensional
spherical surface. It is finite (\ie\ of finite volume), and
\index{Spherical!space}%
has no bounds.

It may be mentioned that there is yet another kind
of curved space: ``elliptical space.'' It can be regarded
\index{Elliptical space}%
as a curved space in which the two ``counter-points''
are identical (indistinguishable from each other). An
elliptical universe can thus be considered to some
\index{Universe!elliptical}%
extent as a curved universe possessing central symmetry.

It follows from what has been said, that closed spaces
without limits are conceivable. From amongst these,
the spherical space (and the elliptical) excels in its
simplicity, since all points on it are equivalent. As a
result of this discussion, a most interesting question
arises for astronomers and physicists, and that is
whether the universe in which we live is infinite, or
whether it is finite in the manner of the spherical universe.
Our experience is far from being sufficient to
enable us to answer this question. But the general
theory of relativity permits of our answering it with a
moderate degree of certainty, and in this connection the
difficulty mentioned in \Sectionref{XXX} finds its solution.
\PageSep{113}


\Chapter{XXXII}{The Structure of Space according to
the General Theory of Relativity}
\index{Motion!of heavenly bodies}%
\index{Universe (World) structure of}%

\First{According} to the general theory of relativity,
the geometrical properties of space are not independent,
but they are determined by matter.
Thus we can draw conclusions about the geometrical
structure of the universe only if we base our considerations
on the state of the matter as being something
that is known. We know from experience that, for a
suitably chosen co-ordinate system, the velocities of
the stars are small as compared with the velocity of
transmission of light. We can thus as a rough approximation
arrive at a conclusion as to the nature of
the universe as a whole, if we treat the matter as being
at rest.

We already know from our previous discussion that the
behaviour of measuring-rods and clocks is influenced by
\index{Clocks}%
\index{Measuring-rod}%
gravitational fields, \ie\ by the distribution of matter.
\index{Gravitational field}%
This in itself is sufficient to exclude the possibility of
the exact validity of Euclidean geometry in our universe.
\index{Euclidean geometry}%
But it is conceivable that our universe differs
only slightly from a Euclidean one, and this notion
seems all the more probable, since calculations show
that the metrics of surrounding space is influenced only
to an exceedingly small extent by masses even of the
\PageSep{114}
magnitude of our sun. We might imagine that, as
regards geometry, our universe behaves analogously
\index{Universe!elliptical}%
\index{Universe!space expanse (radius) of}%
\index{Universe!spherical}%
to a surface which is irregularly curved in its individual
parts, but which nowhere departs appreciably from a
plane: something like the rippled surface of a lake.
Such a universe might fittingly be called a quasi-Euclidean
universe. As regards its space it would be
infinite. But calculation shows that in a quasi-Euclidean
universe the average density of matter
would necessarily be \emph{nil}. Thus such a universe could
not be inhabited by matter everywhere; it would
present to us that unsatisfactory picture which we
portrayed in \Sectionref{XXX}.

If we are to have in the universe an average density
of matter which differs from zero, however small may
be that difference, then the universe cannot be quasi-Euclidean.
\index{Quasi-Euclidean universe}%
On the contrary, the results of calculation
indicate that if matter be distributed uniformly, the
universe would necessarily be spherical (or elliptical).
Since in reality the detailed distribution of matter is
not uniform, the real universe will deviate in individual
parts from the spherical, \ie\ the universe will be quasi-spherical.
\index{Quasi-spherical universe}%
But it will be necessarily finite. In fact, the
theory supplies us with a simple connection\footnote
  {For the ``radius''~$R$ of the universe we obtain the equation
  \[
  R^{2} = \frac{2}{\kappa \rho}.
  \]
  The use of the C.G.S. system in this equation gives $\dfrac{2}{\kappa} = 1.08 � 10^{27}$;
is the average density of the matter.}
between
the space-expanse of the universe and the average
density of matter in it.
\PageSep{115}


\Appendix{I}{Simple Derivation of the Lorentz
Transformation}{[Supplementary to \Sectionref{XI}]}
\index{Lorentz, H. A.!transformation}%

\First{For} the relative orientation of the co-ordinate
systems indicated in \Figref{2}, the $x$-axes of both
systems permanently coincide. In the present
case we can divide the problem into parts by considering
first only events which are localised on the $x$-axis. Any
such event is represented with respect to the co-ordinate
system~$K$ by the abscissa~$x$ and the time~$t$, and with
respect to the system~$K'$ by the abscissa~$x'$ and the
time~$t'$. We require to find $x'$~and~$t'$ when $x$~and~$t$ are
given.

A light-signal, which is proceeding along the positive
\index{Light-signal}%
axis of~$x$, is transmitted according to the equation
\[
x = ct
\]
or
\[
x - ct = 0.
\Tag{(1)}
\]
Since the same light-signal has to be transmitted relative
to~$K'$ with the velocity~$c$, the propagation relative to
the system~$K'$ will be represented by the analogous
formula
\[
x' - ct' = 0.
\Tag{(2)}
\]
Those space-time points (events) which satisfy~\Eqref{(1)} must
\PageSep{116}
also satisfy~\Eqref{(2)}. Obviously this will be the case when
the relation
\[
(x' - ct') = \lambda(x - ct)\Change{.}{}
\Tag{(3)}
\]
is fulfilled in general, where $\lambda$~indicates a constant; for,
according to~\Eqref{(3)}, the disappearance of~$(x - ct)$ involves
the disappearance of~$(x' - ct')$.

If we apply quite similar considerations to light rays
which are being transmitted along the negative $x$-axis,
we obtain the condition
\[
(x' + ct') = \mu(x + ct).
\Tag{(4)}
\]

By adding (or subtracting) equations \Eqref{(3)}~and~\Eqref{(4)}, and
introducing for convenience the constants $a$~and~$b$ in
place of the constants $\lambda$~and~$\mu$, where
\begin{align*}
a &= \frac{\lambda + \mu}{2}
\intertext{and}
b &= \frac{\lambda - \mu}{2},
\end{align*}
we obtain the equations
\[
\left.
\begin{aligned}
x'  &= ax - bct\Add{,} \\
ct' &= act - bx.
\end{aligned}
\right\}
\Tag{(5)}
\]

We should thus have the solution of our problem,
if the constants $a$~and~$b$ were known. These result
from the following discussion.

For the origin of~$K'$ we have permanently $x' = 0$, and
hence according to the first of the equations~\Eqref{(5)}
\[
x = \frac{bc}{a} t.
\]

If we call~$v$ the velocity with which the origin of~$K'$ is
moving relative to~$K$, we then have
\[
v = \frac{bc}{a}.
\Tag{(6)}
\]
\PageSep{117}

The same value~$v$ can be obtained from equation~\Eqref{(5)},
if we calculate the velocity of another point of~$K'$
relative to~$K$, or the velocity (directed towards the
\index{Relative!velocity}%
negative $x$-axis) of a point of~$K$ with respect to~$K'$. In
short, we can designate~$v$ as the relative velocity of the
two systems.

Furthermore, the principle of relativity teaches us
that, as judged from~$K$, the length of a unit measuring-rod
\index{Measuring-rod}%
which is at rest with reference to~$K'$ must be exactly
the same as the length, as judged from~$K'$, of a unit
measuring-rod which is at rest relative to~$K$. In order
to see how the points of the $x'$-axis appear as viewed
from~$K$, we only require to take a ``snapshot'' of~$K'$
\index{Instantaneous photograph (snapshot)}%
from~$K$; this means that we have to insert a particular
value of~$t$ (time of~$K$), \eg\ $t = 0$. For this value of~$t$
we then obtain from the first of the equations~\Eqref{(5)}
\[
x' = ax.
\]

Two points of the $x'$-axis which are separated by the
distance $\Delta x' = 1$ when measured in the $K'$~system are
thus separated in our instantaneous photograph by the
distance
\[
\Delta x = \frac{1}{a}.
\Tag{(7)}
\]

But if the snapshot be taken from~$K'$\Change{}{ }($t' = 0$), and if
we eliminate~$t$ from the equations~\Eqref{(5)}, taking into
account the expression~\Eqref{(6)}, we obtain
\[
x' = a\left(1 - \frac{v^{2}}{c^{2}}\right)x.
\]

From this we conclude that two points on the $x$-axis
and separated by the distance~$1$ (relative to~$K$) will
be represented on our snapshot by the distance
\[
\Delta x' = a\left(1 - \frac{v^{2}}{c^{2}}\right).
\Tag{(7a)}
\]
\PageSep{118}

But from what has been said, the two snapshots
must be identical; hence $\Delta x$~in~\Eqref{(7)} must be equal to
$\Delta x'$~in~\Eqref{(7a)}, so that we obtain
\[
a^{2} = \frac{1}{1 - \dfrac{v^{2}}{c^{2}}}.
\Tag{(7b)}
\]

The equations \Eqref{(6)}~and~\Eqref{(7b)} determine the constants $a$~and~$b$.
By inserting the values of these constants in~\Eqref{(5)},
we obtain the first and the fourth of the equations
given in \Sectionref{XI}.
\[
\left.
\begin{aligned}
x' &= \frac{x - vt}{\sqrt{1 - \dfrac{v^{2}}{c^{2}}}}\Add{,} \\
t' &= \frac{t - \dfrac{v}{c^{2}}x}{\sqrt{1 - \dfrac{v^{2}}{c^{2}}}}.
\end{aligned}
\right\}
\Tag{(8)}
\]

Thus we have obtained the Lorentz transformation
\index{Lorentz, H. A.!transformation}%
for events on the $x$-axis. It satisfies the condition
\[
x'^{2} - c^{2} t'^{2} = x^{2} - c^{2} t^{2}.
\Tag{(8a)}
\]

The extension of this result, to include events which
take place outside the $x$-axis, is obtained by retaining
equations~\Eqref{(8)} and supplementing them by the relations
\[
\left.
\begin{aligned}
y' &= y\Add{,} \\
z' &= z.
\end{aligned}
\right\}
\Tag{(9)}
\]
In this way we satisfy the postulate of the constancy of
the velocity of light \textit{in~vacuo} for rays of light of arbitrary
\index{Velocity of light}%
direction, both for the system~$K$ and for the system~$K'$.
This may be shown in the following manner.

We suppose a light-signal sent out from the origin
\index{Light-signal}%
of~$K$ at the time $t = 0$. It will be propagated according
to the equation
\[
r = \sqrt{x^{2} + y^{2} + z^{2}} = ct,
\]
\PageSep{119}
or, if we square this equation, according to the equation
\[
x^{2} + y^{2} + z^{2} - c^{2} t^{2} = 0.
\Tag{(10)}
\]

It is required by the law of propagation of light, in
\index{Propagation of light}%
conjunction with the postulate of relativity, that the
transmission of the signal in question should take place---as
judged from~$K'$---in accordance with the corresponding
formula
\[
r' = ct',
\]
or,
\[
x'^{2} + y'^{2} + z'^{2} - c^{2} t'^{2} = 0.
\Tag{(10a)}
\]
In order that equation~\Eqref{(10a)} may be a consequence of
equation~\Eqref{(10)}, we must have
\[
x'^{2} + y'^{2} + z'^{2} - c^{2} t'^{2}
   = \sigma(x^{2} + y^{2} + z^{2} - c^{2} t^{2}).
\Tag{(11)}
\]

Since equation~\Eqref{(8a)} must hold for points on the
$x$-axis, we thus have $\sigma = 1$. It is easily seen that the
Lorentz transformation really satisfies equation~\Eqref{(11)}
\index{Lorentz, H. A.!transformation}%
for $\sigma = 1$; for \Eqref{(11)}~is a consequence of \Eqref{(8a)}~and~\Eqref{(9)},
and hence also of \Eqref{(8)}~and~\Eqref{(9)}. We have thus derived
the Lorentz transformation.

The Lorentz transformation represented by \Eqref{(8)}~and~\Eqref{(9)}
still requires to be generalised. Obviously it is
immaterial whether the axes of~$K'$ be chosen so that
they are spatially parallel to those of~$K$. It is also not
essential that the velocity of translation of~$K'$ with
respect to~$K$ should be in the direction of the $x$-axis.
A simple consideration shows that we are able to
construct the Lorentz transformation in this general
sense from two kinds of transformations, viz.\ from
Lorentz transformations in the special sense and from
purely spatial transformations, which corresponds to
the replacement of the rectangular co-ordinate system
\PageSep{120}
by a new system with its axes pointing in other
directions.

Mathematically, we can characterise the generalised
Lorentz transformation thus:
\index{Lorentz, H. A.!transformation!(generalised)}%

It expresses $x'$,~$y'$, $z'$,~$t'$, in terms of linear homogeneous
functions of $x$,~$y$, $z$,~$t$, of such a kind that the relation
\[
x'^{2} + y'^{2} + z'^{2} - c^{2} t'^{2}
  = x^{2} + y^{2} + z^{2} - c^{2} t^{2}
\Tag{(11a)}
\]
is satisfied identically. That is to say: If we substitute
their expressions in $x$,~$y$, $z$,~$t$, in place of $x'$,~$y'$,
$z'$,~$t'$, on the left-hand side, then the left-hand side of~\Eqref{(11a)}
agrees with the right-hand side.
\PageSep{121}


\Appendix{II}{Minkowski's Four-dimensional Space
(``World'')}{[Supplementary to \Sectionref{XVII}]}

\First{We} can characterise the Lorentz transformation
\index{Lorentz, H. A.!transformation}%
still more simply if we introduce the imaginary~$\sqrt{-1}�ct$
in place of~$t$, as time-variable. If, in
accordance with this, we insert
\begin{align*}
x_{1} &= x\Add{,} \\
x_{2} &= y\Add{,} \\
x_{3} &= z\Add{,} \\
x_{4} &= \sqrt{-1}�ct,
\end{align*}
and similarly for the accented system~$K'$, then the
condition which is identically satisfied by the transformation
can be expressed thus:
\[
x_{1}'^{2} + x_{2}'^{2} + x_{3}'^{2} + x_{4}'^{2}
  = {x_{1}}^{2} + {x_{2}}^{2} + {x_{3}}^{2} + {x_{4}}^{2}.
\Tag{(12)}
\]

That is, by the afore-mentioned choice of ``co-ordinates,''
\Eqref{(11a)}~is transformed into this equation.

We see from~\Eqref{(12)} that the imaginary time co-ordinate~$x_{4}$
\index{Cartesian system of co-ordinates}%
\index{Euclidean geometry}%
\index{Euclidean space}%
\index{Space!three-dimensional}%
\index{Time!in Physics}%
enters into the condition of transformation in exactly
the same way as the space co-ordinates $x_{1}$,~$x_{2}$,~$x_{3}$. It
is due to this fact that, according to the theory of
\PageSep{122}
relativity, the ``time''~$x_{4}$ enters into natural laws in the
same form as the space co-ordinates $x_{1}$,~$x_{2}$,~$x_{3}$.

A four-dimensional continuum described by the
\index{Continuum!four-dimensional}%
``co-or\-di\-nates''  $x_{1}$,~$x_{2}$, $x_{3}$,~$x_{4}$, was called ``world'' by
\index{World}%
\index{World!point@{-point}}%
Minkowski, who also termed a point-event a ``world-point.''
\index{Minkowski}%
From a ``happening'' in three-dimensional
space, physics becomes, as it were, an ``existence'' in
the four-dimensional ``world.''

This four-dimensional ``world'' bears a close similarity
to the three-dimensional ``space'' of (Euclidean)
analytical geometry. If we introduce into the latter a
new Cartesian co-ordinate system $(x_{1}', x_{2}', x_{3}')$ with
the same origin, then $x_{1}'$,~$x_{2}'$,~$x_{3}'$, are linear homogeneous
functions of $x_{1}$,~$x_{2}$,~$x_{3}$, which identically satisfy the
equation
\[
x_{1}'^{2} + x_{2}'^{2} + x_{3}'^{2}
  = {x_{1}}^{2} + {x_{2}}^{2} + {x_{3}}^{2}.
\]
The analogy with~\Eqref{(12)} is a complete one. We can
regard Minkowski's ``world'' in a formal manner as a
four-dimensional Euclidean space (with imaginary
time co-ordinate); the Lorentz transformation corresponds
to a ``rotation'' of the co-ordinate system in the
\index{Rotation}%
four-dimensional ``world.''
\PageSep{123}


\Appendix{III}{The Experimental Confirmation of the
General Theory of Relativity}{}
\index{Theory}%

\First{From} a systematic theoretical point of view, we
may imagine the process of evolution of an empirical
science to be a continuous process of induction.
\index{Induction}%
Theories are evolved and are expressed in
short compass as statements of a large number of individual
observations in the form of empirical laws,
\index{Empirical laws}%
from which the general laws can be ascertained by comparison.
Regarded in this way, the development of a
science bears some resemblance to the compilation of a
classified catalogue. It is, as it were, a purely empirical
enterprise.

But this point of view by no means embraces the whole
of the actual process; for it slurs over the important
part played by intuition and deductive thought in the
\index{Deductive thought}%
\index{Intuition}%
development of an exact science. As soon as a science
has emerged from its initial stages, theoretical advances
are no longer achieved merely by a process of arrangement.
Guided by empirical data, the investigator
rather develops a system of thought which, in general,
is built up logically from a small number of fundamental
assumptions, the so-called axioms. We call such a
\index{Axioms}%
system of thought a \emph{theory}. The theory finds the
\PageSep{124}
\index{Classical mechanics}%
\index{Darwinian theory}%
justification for its existence in the fact that it correlates
a large number of single observations, and it is just here
that the ``truth'' of the theory lies.
\index{Theory!truth of}%

Corresponding to the same complex of empirical data,
there may be several theories, which differ from one
another to a considerable extent. But as regards the
deductions from the theories which are capable of
being tested, the agreement between the theories may
be so complete, that it becomes difficult to find such
deductions in which the two theories differ from each
other. As an example, a case of general interest is
available in the province of biology, in the Darwinian
\index{Biology}%
theory of the development of species by selection in
the struggle for existence, and in the theory of development
which is based on the hypothesis of the hereditary
transmission of acquired characters.

We have another instance of far-reaching agreement
between the deductions from two theories in Newtonian
mechanics on the one hand, and the general theory of
relativity on the other. This agreement goes so far,
that up to the present we have been able to find only
a few deductions from the general theory of relativity
which are capable of investigation, and to which the
physics of pre-relativity days does not also lead, and
this despite the profound difference in the fundamental
assumptions of the two theories. In what follows, we
shall again consider these important deductions, and we
shall also discuss the empirical evidence appertaining to
them which has hitherto been obtained.


\Subsection{a}{Motion of the Perihelion of Mercury}
\index{Perihelion of Mercury|(}%

According to Newtonian mechanics and Newton's
\index{Newton's!law of gravitation}%
law of gravitation, a planet which is revolving round the
\PageSep{125}
sun would describe an ellipse round the latter, or, more
correctly, round the common centre of gravity of the
sun and the planet. In such a system, the sun, or the
common centre of gravity, lies in one of the foci of the
orbital ellipse in such a manner that, in the course of a
planet-year, the distance sun-planet grows from a
minimum to a maximum, and then decreases again to
a minimum. If instead of Newton's law we insert a
\index{Newton}%
somewhat different law of attraction into the calculation,
we find that, according to this new law, the motion
would still take place in such a manner that the distance
sun-planet exhibits periodic variations; but in this
case the angle described by the line joining sun and
planet during such a period (from perihelion---closest
proximity to the sun---to perihelion) would differ from~$360�$.
The line of the orbit would not then be a closed
one, but in the course of time it would fill up an annular
part of the orbital plane, viz.\ between the circle of
least and the circle of greatest distance of the planet from
the sun.

According also to the general theory of relativity,
which differs of course from the theory of Newton, a
small variation from the Newton-Kepler motion of a
\index{Kepler}%
planet in its orbit should take place, and in such a way,
that the angle described by the radius sun-planet
between one perihelion and the next should exceed that
corresponding to one complete revolution by an amount
given by
\[
+\frac{24\pi^{3} a^{2}}{T^{2} c^{2} (1-e^{2})}.
\]

(\NB---One complete revolution corresponds to the
angle~$2\pi$ in the absolute angular measure customary in
physics, and the above expression gives the amount by
\PageSep{126}
which the radius sun-planet exceeds this angle during
the interval between one perihelion and the next.)
In this expression $a$~represents the major semi-axis of
the ellipse, $e$~its eccentricity, $c$~the velocity of light, and
$T$~the period of revolution of the planet. Our result
may also be stated as follows: According to the general
theory of relativity, the major axis of the ellipse rotates
round the sun in the same sense as the orbital motion
of the planet. Theory requires that this rotation should
amount to $43$~seconds of arc per~century for the planet
Mercury, but for the other planets of our solar system its
\index{Mercury}%
\index{Mercury!orbit of}%
magnitude should be so small that it would necessarily
escape detection.\footnote
  {Especially since the next planet Venus has an orbit that is
\index{Venus}%
  almost an exact circle, which makes it more difficult to locate
  the perihelion with precision.}

In point of fact, astronomers have found that the
theory of Newton does not suffice to calculate the
observed motion of Mercury with an exactness corresponding
to that of the delicacy of observation attainable
at the present time. After taking account of all
the disturbing influences exerted on Mercury by the
remaining planets, it was found (Leverrier---1859---and
\index{Leverrier}%
Newcomb---1895) that an unexplained perihelial
\index{Newcomb}%
movement of the orbit of Mercury remained over, the
amount of which does not differ sensibly from the above-mentioned
$+43$~seconds of arc per~century. The uncertainty
of the empirical result amounts to a few
seconds only.
\index{Perihelion of Mercury|)}%


\Subsection{b}{Deflection of Light by a Gravitational
Field}

In \Sectionref{XXII} it has been already mentioned that,
\PageSep{127}
according to the general theory of relativity, a ray of
light will experience a curvature of its path when passing
\index{Curvature of light-rays}%
\index{Curvature of light-rays!space}%
through a gravitational field, this curvature being similar
to that experienced by the path of a body which is
projected through a gravitational field. As a result of
this theory, we should expect that a ray of light which
is passing close to a heavenly body would be deviated
towards the latter. For a ray of light which passes the
sun at a distance of $\Delta$~sun-radii from its centre, the
angle of deflection~($\alpha$) should amount to
\[
\alpha = \frac{\text{$1.7$~seconds of arc}}{\Delta}.
\]
It may be added that, according to the theory, half of
this deflection is produced by the
Newtonian field of attraction of the
sun, and the other half by the geometrical
modification (``curvature'')
of space caused by the sun.

%[Illustration: Fig. 5.]
\WFigure{1in}{127}
This result admits of an experimental
\index{Solar eclipse}%
test by means of the photographic
registration of stars during
a total eclipse of the sun. The only
reason why we must wait for a total
eclipse is because at every other
time the atmosphere is so strongly
illuminated by the light from the
sun that the stars situated near the
sun's disc are invisible. The predicted effect can be
seen clearly from the accompanying diagram. If the
sun~($S$) were not present, a star which is practically
infinitely distant would be seen in the direction~$D_{1}$, as
observed from the earth. But as a consequence of the
\PageSep{128}
deflection of light from the star by the sun, the star
will be seen in the direction~$D_{2}$, \ie\ at a somewhat
greater distance from the centre of the sun than corresponds
to its real position.

In practice, the question is tested in the following
way. The stars in the neighbourhood of the sun are
photographed during a solar eclipse. In addition, a
\index{Solar eclipse}%
\index{Stellar universe!photographs}%
second photograph of the same stars is taken when the
sun is situated at another position in the sky, \ie\ a few
months earlier or later. As compared with the standard
photograph, the positions of the stars on the eclipse-photograph
ought to appear displaced radially outwards
(away from the centre of the sun) by an amount
corresponding to the angle~$\alpha$.

We are indebted to the Royal Society and to the
Royal Astronomical Society for the investigation of
this important deduction. Undaunted by the war and
by difficulties of both a material and a psychological
nature aroused by the war, these societies equipped
two expeditions---to Sobral (Brazil), and to the island of
Principe (West Africa)---and sent several of Britain's
most celebrated astronomers (Eddington, Cottingham,
\index{Cottingham}%
\index{Eddington}%
Crommelin, Davidson), in order to obtain photographs
\index{Crommelin}%
\index{Davidson}%
of the solar eclipse of 29th~May, 1919. The relative
discrepancies to be expected between the stellar photographs
obtained during the eclipse and the comparison
photographs amounted to a few hundredths of a millimetre
only. Thus great accuracy was necessary in
making the adjustments required for the taking of the
photographs, and in their subsequent measurement.

The results of the measurements confirmed the theory
in a thoroughly satisfactory manner. The rectangular
components of the observed and of the calculated
\PageSep{129}
deviations of the stars (in seconds of arc) are set forth
in the following table of results:
\[
\begin{array}{@{}c*{2}{>{\quad}cc}@{}}
%[** TN: Re-break first column heading to improve overall width]
\ColHead{1}{Number of}{Number of\\ the Star.} &
\ColHead{2}{Observed. Calculated.}{First Co-ordinate. \\[2pt]
$\overbrace{\text{Observed. Calculated.}}$} &
\ColHead{2}{Observed. Calculated.}{Second Co-ordinate. \\[2pt]
$\overbrace{\text{Observed. Calculated.}}$} \\
11 & -0.19 & -0.22 & +0.16 & +0.02 \\
\Z5 & +0.29 & +0.31 & -0.46 & -0.43 \\
\Z4 & +0.11 & +0.10 & +0.83 & +0.74 \\
\Z3 & +0.20 & +0.12 & +1.00 & +0.87 \\
\Z6 & +0.10 & +0.04 & +0.57 & +0.40 \\
10 & -0.08 & +0.09 & +0.35 & +0.32 \\
\Z2 & +0.95 & +0.85 & -0.27 & -0.09
\end{array}
\]

\Subsection{c}{Displacement of Spectral Lines towards
the Red}
\index{Displacement of spectral lines}%

In \Sectionref{XXIII} it has been shown that in a system~$K'$
which is in rotation with regard to a Galileian system~$K$,
clocks of identical construction, and which are considered
\index{Clocks}%
\index{Clocks!rate of}%
at rest with respect to the rotating reference-body,
go at rates which are dependent on the positions
of the clocks. We shall now examine this dependence
quantitatively. A clock, which is situated at a distance~$r$
from the centre of the disc, has a velocity relative to~$K$
which is given by
\[
v = \omega r,
\]
where $\omega$~represents the angular velocity of rotation of the
disc~$K'$ with respect to~$K$. If $\nu_{0}$~represents the number
of ticks of the clock per unit time (``rate'' of the clock)
relative to~$K$ when the clock is at rest, then the ``rate''
of the clock~($\nu$) when it is moving relative to~$K$ with
a velocity~$v$, but at rest with respect to the disc, will,
in accordance with \Sectionref{XII}, be given by
\[
\nu = \nu_{0} \sqrt{1 - \frac{v^{2}}{c^{2}}},
\]
\PageSep{130}
or with sufficient accuracy by
\[
\nu = \nu_{0} \left(1 - \tfrac{1}{2}\, \frac{v^{2}}{c^{2}}\right).
\]
This expression may also be stated in the following
form:
\[
\nu = \nu_{0} \left(1 - \frac{1}{c^{2}}\, \frac{\omega^{2} r^{2}}{2}\right).
\]
If we represent the difference of potential of the centrifugal
force between the position of the clock and the
centre of the disc by~$\phi$, \ie\ the work, considered negatively,
which must be performed on the unit of mass
against the centrifugal force in order to transport it
\index{Centrifugal force}%
from the position of the clock on the rotating disc to
the centre of the disc, then we have
\[
\phi = -\frac{\omega^{2} r^{2}}{2}.
\]
From this it follows that
\[
\nu = \nu_{0} \left(1 + \frac{\phi}{c^{2}}\right).
\]
In the first place, we see from this expression that two
clocks of identical construction will go at different rates
when situated at different distances from the centre of
the disc. This result is also valid from the standpoint
of an observer who is rotating with the disc.

Now, as judged from the disc, the latter is in a gravitational
\index{Gravitational field!potential of}%
field of potential~$\phi$, hence the result we have
obtained will hold quite generally for gravitational
fields. Furthermore, we can regard an atom which is
emitting spectral lines as a clock, so that the following
statement will hold:

\emph{An atom absorbs or emits light of a frequency which is
\PageSep{131}
dependent on the potential of the gravitational field in
\index{Gravitational field!potential of}%
which it is situated.}

The frequency of an atom situated on the surface of a
\index{Frequency of atom}%
heavenly body will be somewhat less than the frequency
of an atom of the same element which is situated in free
space (or on the surface of a smaller celestial body).
Now $\phi = -K\dfrac{M}{r}$, where $K$~is Newton's constant of
\index{Newton's!constant of gravitation}%
gravitation, and $M$~is the mass of the heavenly body.
Thus a displacement towards the red ought to take place
for spectral lines produced at the surface of stars as
compared with the spectral lines of the same element
produced at the surface of the earth, the amount of this
displacement being
\[
\frac{\nu_{0} - \nu}{\nu_{0}} = \frac{K}{c^{2}}\, \frac{M}{r}.
\]

For the sun, the displacement towards the red predicted
by theory amounts to about two millionths of
the wave-length. A trustworthy calculation is not
possible in the case of the stars, because in general
neither the mass~$M$ nor the radius~$r$ is known.

It is an open question whether or not this effect
exists, and at the present time astronomers are working
with great zeal towards the solution. Owing to the
smallness of the effect in the case of the sun, it is difficult
to form an opinion as to its existence. Whereas
Grebe and Bachem (Bonn), as a result of their own
\index{Bachem}%
\index{Grebe}%
measurements and those of Evershed and Schwarzschild
\index{Evershed}%
\index{Schwarzschild}%
on the cyanogen bands, have placed the existence of
\index{Cyanogen bands}%
the effect almost beyond doubt, other investigators,
particularly St.~John, have been led to the opposite
\index{St. John@{St.\ John}}%
opinion in consequence of their measurements.
\PageSep{132}

Mean displacements of lines towards the less refrangible
end of the spectrum are certainly revealed by
statistical investigations of the fixed stars; but up
to the present the examination of the available data
does not allow of any definite decision being arrived at,
as to whether or not these displacements are to be
referred in reality to the effect of gravitation. The
results of observation have been collected together,
and discussed in detail from the standpoint of the
question which has been engaging our attention here,
in a paper by E.~Freundlich entitled ``Zur Pr�fung der
allgemeinen Relativit�ts-Theorie'' (\textit{Die Naturwissenschaften},
1919, No.~35, p.~520: Julius Springer, Berlin).

At all events, a definite decision will be reached during
the next few years. If the displacement of spectral
lines towards the red by the gravitational potential
does not exist, then the general theory of relativity
will be untenable. On the other hand, if the cause of
the displacement of spectral lines be definitely traced
to the gravitational potential, then the study of this
displacement will furnish us with important information
\index{Mass of heavenly bodies}%
as to the mass of the heavenly bodies.
\PageSep{133}


\backmatter
\BookMark{-1}{Back Matter}
\Bibliography{WORKS IN ENGLISH ON EINSTEIN'S THEORY}

\Bibsection{Introductory}

\Bibitem{The Foundations of Einstein's Theory of Gravitation}
{Erwin Freundlich (translation by H.~L.~Brose).
Camb.\ Univ.\ Press, 1920.}

\Bibitem{Space and Time in Contemporary Physics}{Moritz Schlick
(translation by H.~L.~Brose). Clarendon Press,
Oxford, 1920.}


\Bibsection{The Special Theory}

\Bibitem{The Principle of Relativity}{E.~Cunningham. Camb.\
Univ.\ Press.}

\Bibitem{Relativity and the Electron Theory}{E.~Cunningham, Monographs
on Physics. Longmans, Green \&~Co.}

\Bibitem{The Theory of Relativity}{L.~Silberstein. Macmillan \&~Co.}

\Bibitem{The Space-Time Manifold of Relativity}{E.~B.~Wilson
and G.~N.~Lewis, \textit{Proc.\ Amer.\ Soc.\ Arts \&~Science},
vol.~xlviii., No.~11, 1912.}


\Bibsection{The General Theory}

\Bibitem{Report on the Relativity Theory of Gravitation}{A.~S.
Eddington. Fleetway Press Ltd., Fleet Street,
London.}
\PageSep{134}

\Bibitem{On Einstein's Theory of Gravitation and its Astronomical
Consequences}{W.~de~Sitter, \textit{M.~N.~Roy.\ Astron.\
Soc.},~lxxvi.\ p.~699, 1916; lxxvii.\ p.~155, 1916; lxxviii.\
p.~3, 1917.}

\Bibitem{On Einstein's Theory of Gravitation}{H.~A.~Lorentz, \textit{Proc.\
Amsterdam Acad.}, vol.~xix. p.~1341, 1917.}

\Bibitem{Space, Time and Gravitation}{W.~de~Sitter: \textit{The
Observatory}, No.~505, p.~412. Taylor \&~Francis, Fleet
Street, London.}

\Bibitem{The Total Eclipse of 29th~May, 1919, and the Influence of
Gravitation on Light}{A.~S.~Eddington, \textit{ibid.},
March~1919.}

\Bibitem{Discussion on the Theory of Relativity}{\textit{M.~N.~Roy.\ Astron.\
Soc.}, vol.~lxxx.\ No.~2, p.~96, December~1919.}

\Bibitem{The Displacement of Spectrum Lines and the Equivalence
Hypothesis}{W.~G.~Duffield, \textit{M.~N.~Roy.\ Astron.\ Soc.},
vol.~lxxx.\Change{;}{} No.~3, p.~262, 1920.}

\Bibitem{Space, Time and Gravitation}{A.~S.~Eddington, Camb.\ Univ.\
Press, 1920.}


\Bibsection{Also, Chapters in}

\Bibitem{The Mathematical Theory of Electricity and Magnetism}
{J.~H. Jeans (4th~edition). Camb.\ Univ.\ Press, 1920.}

\Bibitem{The Electron Theory of Matter}{O.~W.~Richardson. Camb.\
Univ.\ Press.}
\PageSep{135}
\printindex % [** TN: Auto-generate the index]
\iffalse %%%% Start of index text %%%%
INDEX

Aberration 49

Absorption of energy 46

Acceleration 64, 67, 70

Action at a distance 48

Addition of velocities 16, 38

Adjacent points 89

Aether 52
  drift@{-drift}#drift 52, 53

Arbitrary substitutions 98

Astronomy 7, 102

Astronomical day 11

Axioms 2, 123
  truth of 2

Bachem 131

Basis of theory 44

Being@{``Being''}#Being 66, 108

beta-rays@{$\beta$-rays}#rays 50

Biology 124

Cartesian system of co-ordinates 7, 84, 122

Cathode rays 50

Celestial mechanics 105

Centrifugal force 80, 130

Chest 66

Classical mechanics 9, 13, 14, 16, 30, 44, 71, 102, 103, 124
  truth of 13

Clocks 10, 23, 80, 81, 94, 95, 98-100, 102, 113, 129
  rate of 129

Conception of mass 45
  position 6

Conservation of energy 45, 101
  impulse 101
  mass 45, 47

Continuity 95

Continuum 55, 83
  two-dimensional 94
  three-dimensional 57
  four-dimensional 89, 91, 92, 94, 122
  space-time 78, 91-96
  Euclidean 84, 86, 88, 92
  non-Euclidean 86, 90

Coordinate@{Co-ordinate}#Co-ordinate
  differences 92
  differentials 92
  planes 32

Cottingham 128

Counter-Point 112

Covariant@{Co-variant}#Co-variant 43

Crommelin 128

Curvature of light-rays 104, 127
  space 127

Curvilinear motion 74

Cyanogen bands 131

Darwinian theory 124

Davidson 128

Deductive thought 123

Derivation of laws 44

Desitter@{De Sitter}#De Sitter 17

Displacement of spectral lines 104, 129

Distance (line-interval) 3, 5, 8, 28, 29, 84, 88, 109
  physical interpretation of 5
  relativity of 28

Doppler principle 50 %.

Double stars 17

Eclipse of star 17

Eddington 104, 128
%\PageSep{136}

Electricity 76

Electrodynamics 13, 19, 41, 44, 76

Electromagnetic theory 49
  waves 63

Electron 44, 50 %.
  electrical masses of 51

Electrostatics 76

Elliptical space 112

Empirical laws 123

Encounter (space-time coincidence) 95

Equivalent 14

Euclidean geometry 1, 2, 57, 82, 86, 88, 108, 109, 113, 122
  propositions of 3, 8

%[** TN: Add explicit "Euclidean" heading]
Euclidean space 57, 86, 122

Evershed 131

Experience 49, 60

Faraday 48, 63

FitzGerald 53

Fixed stars 11

Fizeau 39, 49, 51
  experiment of 39

Frequency of atom 131

Galilei 11
  transformation 33, 36, 38, 42, 52

Galileian system of co-ordinates
  11, 13, 14, 46, 79, 91, 98, 100

Gauss 86, 87, 90

Gaussian co-ordinates 88-90, 94, 96-100

General theory of relativity 59-104, 97

Geometrical ideas 2, 3
  propositions 1
  truth of 2-4

Gravitation 64, 69, 78, 102

Gravitational field 64, 67, 74, 77, 93, 98, 100, 101, 113
  potential of 130, 131

%[** TN: Add explicit "Gravitational" heading]
Gravitational mass 65, 68, 102

Grebe 131

Group-density of stars 106

Helmholtz 108

Heuristic value of relativity#Heuristic 42

Induction 123

Inertia 65

Inertial mass 47, 65, 69, 101, 102

Instantaneous photograph (snapshot) 117

Intensity of gravitational field 106

Intuition 123

Ions 44

Kepler 125

Kinetic energy 45, 101

Lattice 108

Law of inertia 11, 61, 62, 98

Laws of Galilei-Newton 13
  of Nature 60, 71, 99

Leverrier 103, 126

Light-signal 33, 115, 118

Light-stimulus 33

Limiting velocity ($c$)#Limiting 36, 37

Lines of force 106

Lorentz, H. A.#Lorentz 19, 41, 44, 49, 50-53
  transformation 33, 39, 42, 91, 97, 98, 115, 118, 119, 121
    (generalised) 120

Mach, E.#Mach 72

Magnetic field 63

Manifold|see{Continuum} 0

Mass of heavenly bodies 132

Matter 101

Maxwell 41, 44, 48-50, 52
  fundamental equations 46, 77

Measurement of length 85

Measuring-rod 5, 6, 28, 80, 81, 94, 100, 102, 111, 113, 117

Mercury 103, 126
  orbit of 103, 126

Michelson 52-54

Minkowski 55-57, 91, 122
%\PageSep{137}

Morley 53, 54

Motion 14, 60
  of heavenly bodies 13, 15, 44, 102, 113

Newcomb 126

Newton 11, 72, 102, 105, 125

Newton's
  constant of gravitation 131
  law of gravitation 48, 80, 106, 124
  law of motion 64

Non-Euclidean geometry 108

Non-Galileian reference-bodies 98

Non-uniform motion 62

Optics 13, 19, 44

Organ-pipe, note of 14

Parabola 9, 10

Path-curve 10

Perihelion of Mercury 124-126

Physics 7
  of measurement 7

Place specification 5, 6

Plane 1, 108, 109

Poincare@{Poincar�}#Poincar� 108

Point 1

Point-mass, energy of#Point-mass 45

Position 9

Principle of relativity 13-15, 19, 20, 60

Processes of Nature 42

Propagation of light 17, 19, 20, 32, 91, 119
  in liquid 40
  in gravitational fields 75

Quasi-Euclidean universe 114

Quasi-spherical universe 114

Radiation 46

Radioactive substances 50

Reference-body 5, 7, 9-11, 18, 23, 25, 26, 37, 60
  rotating 79

%[** TN: Add explicit "Reference-" heading]
Reference-mollusk 99-101

Relative
  position 3
  velocity 117

Rest 14

Riemann 86, 108, 111

Rotation 81, 122

Schwarzschild 131

Seconds-clock 36

Seeliger 105, 106

Simultaneity 22, 24-26, 81
  relativity of 26

Size-relations 90

Solar eclipse 75, 127, 128

Space 9, 52, 55, 105
  conception of 19

Space co-ordinates 55, 81, 99

Space
  interval@{-interval}#interval 30, 56
  point@{-point}#point 99
  two-dimensional 108
  three-dimensional 122

Special theory of relativity 1-57, 20

Spherical
  surface 109
  space 111, 112

St. John@{St.\ John}#St.~John 131

Stellar universe 106
  photographs 128

Straight line 1-3, 9, 10, 82, 88, 109

System of co-ordinates 5, 10, 11

Terrestrial space 15

Theory 123
  truth of 124

Three-dimensional 55

Time
  conception of 19, 52, 105
  coordinate@{co-ordinate}#co-ordinate 55, 99
  in Physics 21, 98, 122
  of an event 24, 26

Time-interval 30, 56

Trajectory 10

Truth@{``Truth''}#Truth 2

Uniform translation 12, 59

Universe (World) structure of 108, 113
  circumference of 111
%\PageSep{138}

Universe
  elliptical 112, 114
  Euclidean 109, 111
  space expanse (radius) of 114
  spherical 111, 114

Value of $\pi$#$\pi$ 82, 110

Velocity of light 10, 17, 18, 76, 118

Venus 126

Weight (heaviness) 65

World 55, 56, 109, 122

World
  point@{-point}#point 122
  radius@{-radius}#radius 112
  sphere@{-sphere}#sphere 110, 111

Zeeman 41
\fi %%%% End of index text %%%%
\PageSep{139}
% [Blank page]
\PageSep{140}
\ifthenelse{\boolean{ForPrinting}}{\cleardoublepage\null}{}
\newpage
\begin{CenterPage}
  \scriptsize
  PRINTED BY \\[2pt]
  MORRISON AND GIBB LIMITED \\[2pt]
  EDINBURGH
\end{CenterPage}
%%%%%%%%%%%%%%%%%%%%%%%%% GUTENBERG LICENSE %%%%%%%%%%%%%%%%%%%%%%%%%%

\cleardoublepage
\BookMark{0}{PG License}
\SetEvenHead{Licensing}
\SetOddHead{Licensing}
\pagenumbering{Roman}
\begin{PGtext}
End of the Project Gutenberg EBook of Relativity: The Special and the
General Theory, by Albert Einstein

*** END OF THIS PROJECT GUTENBERG EBOOK RELATIVITY ***

***** This file should be named 36114-pdf.pdf or 36114-pdf.zip *****
This and all associated files of various formats will be found in:
        http://www.gutenberg.org/3/6/1/1/36114/

Produced by Andrew D. Hwang. (This ebook was produced using
OCR text generously provided by the University of Toronto
Robarts Library through the Internet Archive.)


Updated editions will replace the previous one--the old editions
will be renamed.

Creating the works from public domain print editions means that no
one owns a United States copyright in these works, so the Foundation
(and you!) can copy and distribute it in the United States without
permission and without paying copyright royalties.  Special rules,
set forth in the General Terms of Use part of this license, apply to
copying and distributing Project Gutenberg-tm electronic works to
protect the PROJECT GUTENBERG-tm concept and trademark.  Project
Gutenberg is a registered trademark, and may not be used if you
charge for the eBooks, unless you receive specific permission.  If you
do not charge anything for copies of this eBook, complying with the
rules is very easy.  You may use this eBook for nearly any purpose
such as creation of derivative works, reports, performances and
research.  They may be modified and printed and given away--you may do
practically ANYTHING with public domain eBooks.  Redistribution is
subject to the trademark license, especially commercial
redistribution.



*** START: FULL LICENSE ***

THE FULL PROJECT GUTENBERG LICENSE
PLEASE READ THIS BEFORE YOU DISTRIBUTE OR USE THIS WORK

To protect the Project Gutenberg-tm mission of promoting the free
distribution of electronic works, by using or distributing this work
(or any other work associated in any way with the phrase "Project
Gutenberg"), you agree to comply with all the terms of the Full Project
Gutenberg-tm License (available with this file or online at
http://gutenberg.org/license).


Section 1.  General Terms of Use and Redistributing Project Gutenberg-tm
electronic works

1.A.  By reading or using any part of this Project Gutenberg-tm
electronic work, you indicate that you have read, understand, agree to
and accept all the terms of this license and intellectual property
(trademark/copyright) agreement.  If you do not agree to abide by all
the terms of this agreement, you must cease using and return or destroy
all copies of Project Gutenberg-tm electronic works in your possession.
If you paid a fee for obtaining a copy of or access to a Project
Gutenberg-tm electronic work and you do not agree to be bound by the
terms of this agreement, you may obtain a refund from the person or
entity to whom you paid the fee as set forth in paragraph 1.E.8.

1.B.  "Project Gutenberg" is a registered trademark.  It may only be
used on or associated in any way with an electronic work by people who
agree to be bound by the terms of this agreement.  There are a few
things that you can do with most Project Gutenberg-tm electronic works
even without complying with the full terms of this agreement.  See
paragraph 1.C below.  There are a lot of things you can do with Project
Gutenberg-tm electronic works if you follow the terms of this agreement
and help preserve free future access to Project Gutenberg-tm electronic
works.  See paragraph 1.E below.

1.C.  The Project Gutenberg Literary Archive Foundation ("the Foundation"
or PGLAF), owns a compilation copyright in the collection of Project
Gutenberg-tm electronic works.  Nearly all the individual works in the
collection are in the public domain in the United States.  If an
individual work is in the public domain in the United States and you are
located in the United States, we do not claim a right to prevent you from
copying, distributing, performing, displaying or creating derivative
works based on the work as long as all references to Project Gutenberg
are removed.  Of course, we hope that you will support the Project
Gutenberg-tm mission of promoting free access to electronic works by
freely sharing Project Gutenberg-tm works in compliance with the terms of
this agreement for keeping the Project Gutenberg-tm name associated with
the work.  You can easily comply with the terms of this agreement by
keeping this work in the same format with its attached full Project
Gutenberg-tm License when you share it without charge with others.

1.D.  The copyright laws of the place where you are located also govern
what you can do with this work.  Copyright laws in most countries are in
a constant state of change.  If you are outside the United States, check
the laws of your country in addition to the terms of this agreement
before downloading, copying, displaying, performing, distributing or
creating derivative works based on this work or any other Project
Gutenberg-tm work.  The Foundation makes no representations concerning
the copyright status of any work in any country outside the United
States.

1.E.  Unless you have removed all references to Project Gutenberg:

1.E.1.  The following sentence, with active links to, or other immediate
access to, the full Project Gutenberg-tm License must appear prominently
whenever any copy of a Project Gutenberg-tm work (any work on which the
phrase "Project Gutenberg" appears, or with which the phrase "Project
Gutenberg" is associated) is accessed, displayed, performed, viewed,
copied or distributed:

This eBook is for the use of anyone anywhere at no cost and with
almost no restrictions whatsoever.  You may copy it, give it away or
re-use it under the terms of the Project Gutenberg License included
with this eBook or online at www.gutenberg.org

1.E.2.  If an individual Project Gutenberg-tm electronic work is derived
from the public domain (does not contain a notice indicating that it is
posted with permission of the copyright holder), the work can be copied
and distributed to anyone in the United States without paying any fees
or charges.  If you are redistributing or providing access to a work
with the phrase "Project Gutenberg" associated with or appearing on the
work, you must comply either with the requirements of paragraphs 1.E.1
through 1.E.7 or obtain permission for the use of the work and the
Project Gutenberg-tm trademark as set forth in paragraphs 1.E.8 or
1.E.9.

1.E.3.  If an individual Project Gutenberg-tm electronic work is posted
with the permission of the copyright holder, your use and distribution
must comply with both paragraphs 1.E.1 through 1.E.7 and any additional
terms imposed by the copyright holder.  Additional terms will be linked
to the Project Gutenberg-tm License for all works posted with the
permission of the copyright holder found at the beginning of this work.

1.E.4.  Do not unlink or detach or remove the full Project Gutenberg-tm
License terms from this work, or any files containing a part of this
work or any other work associated with Project Gutenberg-tm.

1.E.5.  Do not copy, display, perform, distribute or redistribute this
electronic work, or any part of this electronic work, without
prominently displaying the sentence set forth in paragraph 1.E.1 with
active links or immediate access to the full terms of the Project
Gutenberg-tm License.

1.E.6.  You may convert to and distribute this work in any binary,
compressed, marked up, nonproprietary or proprietary form, including any
word processing or hypertext form.  However, if you provide access to or
distribute copies of a Project Gutenberg-tm work in a format other than
"Plain Vanilla ASCII" or other format used in the official version
posted on the official Project Gutenberg-tm web site (www.gutenberg.org),
you must, at no additional cost, fee or expense to the user, provide a
copy, a means of exporting a copy, or a means of obtaining a copy upon
request, of the work in its original "Plain Vanilla ASCII" or other
form.  Any alternate format must include the full Project Gutenberg-tm
License as specified in paragraph 1.E.1.

1.E.7.  Do not charge a fee for access to, viewing, displaying,
performing, copying or distributing any Project Gutenberg-tm works
unless you comply with paragraph 1.E.8 or 1.E.9.

1.E.8.  You may charge a reasonable fee for copies of or providing
access to or distributing Project Gutenberg-tm electronic works provided
that

- You pay a royalty fee of 20% of the gross profits you derive from
     the use of Project Gutenberg-tm works calculated using the method
     you already use to calculate your applicable taxes.  The fee is
     owed to the owner of the Project Gutenberg-tm trademark, but he
     has agreed to donate royalties under this paragraph to the
     Project Gutenberg Literary Archive Foundation.  Royalty payments
     must be paid within 60 days following each date on which you
     prepare (or are legally required to prepare) your periodic tax
     returns.  Royalty payments should be clearly marked as such and
     sent to the Project Gutenberg Literary Archive Foundation at the
     address specified in Section 4, "Information about donations to
     the Project Gutenberg Literary Archive Foundation."

- You provide a full refund of any money paid by a user who notifies
     you in writing (or by e-mail) within 30 days of receipt that s/he
     does not agree to the terms of the full Project Gutenberg-tm
     License.  You must require such a user to return or
     destroy all copies of the works possessed in a physical medium
     and discontinue all use of and all access to other copies of
     Project Gutenberg-tm works.

- You provide, in accordance with paragraph 1.F.3, a full refund of any
     money paid for a work or a replacement copy, if a defect in the
     electronic work is discovered and reported to you within 90 days
     of receipt of the work.

- You comply with all other terms of this agreement for free
     distribution of Project Gutenberg-tm works.

1.E.9.  If you wish to charge a fee or distribute a Project Gutenberg-tm
electronic work or group of works on different terms than are set
forth in this agreement, you must obtain permission in writing from
both the Project Gutenberg Literary Archive Foundation and Michael
Hart, the owner of the Project Gutenberg-tm trademark.  Contact the
Foundation as set forth in Section 3 below.

1.F.

1.F.1.  Project Gutenberg volunteers and employees expend considerable
effort to identify, do copyright research on, transcribe and proofread
public domain works in creating the Project Gutenberg-tm
collection.  Despite these efforts, Project Gutenberg-tm electronic
works, and the medium on which they may be stored, may contain
"Defects," such as, but not limited to, incomplete, inaccurate or
corrupt data, transcription errors, a copyright or other intellectual
property infringement, a defective or damaged disk or other medium, a
computer virus, or computer codes that damage or cannot be read by
your equipment.

1.F.2.  LIMITED WARRANTY, DISCLAIMER OF DAMAGES - Except for the "Right
of Replacement or Refund" described in paragraph 1.F.3, the Project
Gutenberg Literary Archive Foundation, the owner of the Project
Gutenberg-tm trademark, and any other party distributing a Project
Gutenberg-tm electronic work under this agreement, disclaim all
liability to you for damages, costs and expenses, including legal
fees.  YOU AGREE THAT YOU HAVE NO REMEDIES FOR NEGLIGENCE, STRICT
LIABILITY, BREACH OF WARRANTY OR BREACH OF CONTRACT EXCEPT THOSE
PROVIDED IN PARAGRAPH 1.F.3.  YOU AGREE THAT THE FOUNDATION, THE
TRADEMARK OWNER, AND ANY DISTRIBUTOR UNDER THIS AGREEMENT WILL NOT BE
LIABLE TO YOU FOR ACTUAL, DIRECT, INDIRECT, CONSEQUENTIAL, PUNITIVE OR
INCIDENTAL DAMAGES EVEN IF YOU GIVE NOTICE OF THE POSSIBILITY OF SUCH
DAMAGE.

1.F.3.  LIMITED RIGHT OF REPLACEMENT OR REFUND - If you discover a
defect in this electronic work within 90 days of receiving it, you can
receive a refund of the money (if any) you paid for it by sending a
written explanation to the person you received the work from.  If you
received the work on a physical medium, you must return the medium with
your written explanation.  The person or entity that provided you with
the defective work may elect to provide a replacement copy in lieu of a
refund.  If you received the work electronically, the person or entity
providing it to you may choose to give you a second opportunity to
receive the work electronically in lieu of a refund.  If the second copy
is also defective, you may demand a refund in writing without further
opportunities to fix the problem.

1.F.4.  Except for the limited right of replacement or refund set forth
in paragraph 1.F.3, this work is provided to you 'AS-IS' WITH NO OTHER
WARRANTIES OF ANY KIND, EXPRESS OR IMPLIED, INCLUDING BUT NOT LIMITED TO
WARRANTIES OF MERCHANTIBILITY OR FITNESS FOR ANY PURPOSE.

1.F.5.  Some states do not allow disclaimers of certain implied
warranties or the exclusion or limitation of certain types of damages.
If any disclaimer or limitation set forth in this agreement violates the
law of the state applicable to this agreement, the agreement shall be
interpreted to make the maximum disclaimer or limitation permitted by
the applicable state law.  The invalidity or unenforceability of any
provision of this agreement shall not void the remaining provisions.

1.F.6.  INDEMNITY - You agree to indemnify and hold the Foundation, the
trademark owner, any agent or employee of the Foundation, anyone
providing copies of Project Gutenberg-tm electronic works in accordance
with this agreement, and any volunteers associated with the production,
promotion and distribution of Project Gutenberg-tm electronic works,
harmless from all liability, costs and expenses, including legal fees,
that arise directly or indirectly from any of the following which you do
or cause to occur: (a) distribution of this or any Project Gutenberg-tm
work, (b) alteration, modification, or additions or deletions to any
Project Gutenberg-tm work, and (c) any Defect you cause.


Section  2.  Information about the Mission of Project Gutenberg-tm

Project Gutenberg-tm is synonymous with the free distribution of
electronic works in formats readable by the widest variety of computers
including obsolete, old, middle-aged and new computers.  It exists
because of the efforts of hundreds of volunteers and donations from
people in all walks of life.

Volunteers and financial support to provide volunteers with the
assistance they need, are critical to reaching Project Gutenberg-tm's
goals and ensuring that the Project Gutenberg-tm collection will
remain freely available for generations to come.  In 2001, the Project
Gutenberg Literary Archive Foundation was created to provide a secure
and permanent future for Project Gutenberg-tm and future generations.
To learn more about the Project Gutenberg Literary Archive Foundation
and how your efforts and donations can help, see Sections 3 and 4
and the Foundation web page at http://www.pglaf.org.


Section 3.  Information about the Project Gutenberg Literary Archive
Foundation

The Project Gutenberg Literary Archive Foundation is a non profit
501(c)(3) educational corporation organized under the laws of the
state of Mississippi and granted tax exempt status by the Internal
Revenue Service.  The Foundation's EIN or federal tax identification
number is 64-6221541.  Its 501(c)(3) letter is posted at
http://pglaf.org/fundraising.  Contributions to the Project Gutenberg
Literary Archive Foundation are tax deductible to the full extent
permitted by U.S. federal laws and your state's laws.

The Foundation's principal office is located at 4557 Melan Dr. S.
Fairbanks, AK, 99712., but its volunteers and employees are scattered
throughout numerous locations.  Its business office is located at
809 North 1500 West, Salt Lake City, UT 84116, (801) 596-1887, email
business@pglaf.org.  Email contact links and up to date contact
information can be found at the Foundation's web site and official
page at http://pglaf.org

For additional contact information:
     Dr. Gregory B. Newby
     Chief Executive and Director
     gbnewby@pglaf.org


Section 4.  Information about Donations to the Project Gutenberg
Literary Archive Foundation

Project Gutenberg-tm depends upon and cannot survive without wide
spread public support and donations to carry out its mission of
increasing the number of public domain and licensed works that can be
freely distributed in machine readable form accessible by the widest
array of equipment including outdated equipment.  Many small donations
($1 to $5,000) are particularly important to maintaining tax exempt
status with the IRS.

The Foundation is committed to complying with the laws regulating
charities and charitable donations in all 50 states of the United
States.  Compliance requirements are not uniform and it takes a
considerable effort, much paperwork and many fees to meet and keep up
with these requirements.  We do not solicit donations in locations
where we have not received written confirmation of compliance.  To
SEND DONATIONS or determine the status of compliance for any
particular state visit http://pglaf.org

While we cannot and do not solicit contributions from states where we
have not met the solicitation requirements, we know of no prohibition
against accepting unsolicited donations from donors in such states who
approach us with offers to donate.

International donations are gratefully accepted, but we cannot make
any statements concerning tax treatment of donations received from
outside the United States.  U.S. laws alone swamp our small staff.

Please check the Project Gutenberg Web pages for current donation
methods and addresses.  Donations are accepted in a number of other
ways including checks, online payments and credit card donations.
To donate, please visit: http://pglaf.org/donate


Section 5.  General Information About Project Gutenberg-tm electronic
works.

Professor Michael S. Hart is the originator of the Project Gutenberg-tm
concept of a library of electronic works that could be freely shared
with anyone.  For thirty years, he produced and distributed Project
Gutenberg-tm eBooks with only a loose network of volunteer support.


Project Gutenberg-tm eBooks are often created from several printed
editions, all of which are confirmed as Public Domain in the U.S.
unless a copyright notice is included.  Thus, we do not necessarily
keep eBooks in compliance with any particular paper edition.


Most people start at our Web site which has the main PG search facility:

     http://www.gutenberg.org

This Web site includes information about Project Gutenberg-tm,
including how to make donations to the Project Gutenberg Literary
Archive Foundation, how to help produce our new eBooks, and how to
subscribe to our email newsletter to hear about new eBooks.
\end{PGtext}

% %%%%%%%%%%%%%%%%%%%%%%%%%%%%%%%%%%%%%%%%%%%%%%%%%%%%%%%%%%%%%%%%%%%%%%% %
%                                                                         %
% End of the Project Gutenberg EBook of Relativity: The Special and the   %
% General Theory, by Albert Einstein                                      %
%                                                                         %
% *** END OF THIS PROJECT GUTENBERG EBOOK RELATIVITY ***                  %
%                                                                         %
% ***** This file should be named 36114-t.tex or 36114-t.zip *****        %
% This and all associated files of various formats will be found in:      %
%         http://www.gutenberg.org/3/6/1/1/36114/                         %
%                                                                         %
% %%%%%%%%%%%%%%%%%%%%%%%%%%%%%%%%%%%%%%%%%%%%%%%%%%%%%%%%%%%%%%%%%%%%%%% %

\end{document}
###
@ControlwordReplace = (
  ['\\Preface', 'Preface'],
  ['\\ie', 'i.e.'],
  ['\\eg', 'e.g.'],
  ['\\NB', 'N.B.'],
  ['\\itema', '(a)'],
  ['\\itemb', '(b)'],
  ['\\itemc', '(c)']
  );

@ControlwordArguments = (
  ['\\BookMark', 1, 0, '', '', 1, 0, '', ''],
  ['\\item', 0, 1, '', ' '],
  ['\\Part', 1, 1, '', ' ', 1, 1, '', '', 1, 0, '', ''],
  ['\\Chapter', 0, 0, '', '', 1, 1, '', ' ', 1, 1, '', ''],
  ['\\Section', 1, 1, '', ''],
  ['\\Subsection', 1, 1, '(', ') ', 1, 1, '', ''],
  ['\\SectTitle', 1, 1, '', ''],
  ['\\Appendix', 1, 1, '', ' ', 1, 1, '', ' ', 1, 1, '', ''],
  ['\\Bibliography', 1, 1, 'Bibliography: ', ''],
  ['\\Bibsection', 1, 1, '', ''],
  ['\\Bibitem', 1, 1, '', ' ', 1, 1, '', ''],
  ['\\PubRow', 1, 1, '', ' ', 1, 1, '', ''],
  ['\\Signature', 0, 1, '', ' ', 1, 1, '', ''],
  ['\\Change', 1, 0, '', '', 1, 1, '', ''],
  ['\\Add', 1, 1, '', ''],
  ['\\PageSep', 1, 0, '', ''],
  ['\\Figure', 0, 0, '', '', 1, 1, '', ''],
  ['\\WFigure', 1, 0, '', '', 1, 1, '', ''],
  ['\\Figref', 1, 1, 'Fig. ', ''],
  ['\\Partref', 1, 1, 'Part ', ''],
  ['\\Sectionref', 1, 1, 'Section ', ''],
  ['\\Srefno', 1, 1, '', ''],
  ['\\Appendixref', 1, 1, 'Appendix ', ''],
  ['\\Eqref', 1, 1, '', ''],
  ['\\First', 1, 1, '', '']
  );
###
This is pdfTeXk, Version 3.141592-1.40.3 (Web2C 7.5.6) (format=pdflatex 2010.5.6)  15 MAY 2011 15:31
entering extended mode
 %&-line parsing enabled.
**36114-t.tex
(./36114-t.tex
LaTeX2e <2005/12/01>
Babel <v3.8h> and hyphenation patterns for english, usenglishmax, dumylang, noh
yphenation, arabic, farsi, croatian, ukrainian, russian, bulgarian, czech, slov
ak, danish, dutch, finnish, basque, french, german, ngerman, ibycus, greek, mon
ogreek, ancientgreek, hungarian, italian, latin, mongolian, norsk, icelandic, i
nterlingua, turkish, coptic, romanian, welsh, serbian, slovenian, estonian, esp
eranto, uppersorbian, indonesian, polish, portuguese, spanish, catalan, galicia
n, swedish, ukenglish, pinyin, loaded.
(/usr/share/texmf-texlive/tex/latex/base/book.cls
Document Class: book 2005/09/16 v1.4f Standard LaTeX document class
(/usr/share/texmf-texlive/tex/latex/base/bk12.clo
File: bk12.clo 2005/09/16 v1.4f Standard LaTeX file (size option)
)
\c@part=\count79
\c@chapter=\count80
\c@section=\count81
\c@subsection=\count82
\c@subsubsection=\count83
\c@paragraph=\count84
\c@subparagraph=\count85
\c@figure=\count86
\c@table=\count87
\abovecaptionskip=\skip41
\belowcaptionskip=\skip42
\bibindent=\dimen102
) (/usr/share/texmf-texlive/tex/latex/base/inputenc.sty
Package: inputenc 2006/05/05 v1.1b Input encoding file
\inpenc@prehook=\toks14
\inpenc@posthook=\toks15
(/usr/share/texmf-texlive/tex/latex/base/latin1.def
File: latin1.def 2006/05/05 v1.1b Input encoding file
)) (/usr/share/texmf-texlive/tex/latex/base/ifthen.sty
Package: ifthen 2001/05/26 v1.1c Standard LaTeX ifthen package (DPC)
) (/usr/share/texmf-texlive/tex/latex/amsmath/amsmath.sty
Package: amsmath 2000/07/18 v2.13 AMS math features
\@mathmargin=\skip43
For additional information on amsmath, use the `?' option.
(/usr/share/texmf-texlive/tex/latex/amsmath/amstext.sty
Package: amstext 2000/06/29 v2.01
(/usr/share/texmf-texlive/tex/latex/amsmath/amsgen.sty
File: amsgen.sty 1999/11/30 v2.0
\@emptytoks=\toks16
\ex@=\dimen103
)) (/usr/share/texmf-texlive/tex/latex/amsmath/amsbsy.sty
Package: amsbsy 1999/11/29 v1.2d
\pmbraise@=\dimen104
) (/usr/share/texmf-texlive/tex/latex/amsmath/amsopn.sty
Package: amsopn 1999/12/14 v2.01 operator names
)
\inf@bad=\count88
LaTeX Info: Redefining \frac on input line 211.
\uproot@=\count89
\leftroot@=\count90
LaTeX Info: Redefining \overline on input line 307.
\classnum@=\count91
\DOTSCASE@=\count92
LaTeX Info: Redefining \ldots on input line 379.
LaTeX Info: Redefining \dots on input line 382.
LaTeX Info: Redefining \cdots on input line 467.
\Mathstrutbox@=\box26
\strutbox@=\box27
\big@size=\dimen105
LaTeX Font Info:    Redeclaring font encoding OML on input line 567.
LaTeX Font Info:    Redeclaring font encoding OMS on input line 568.
\macc@depth=\count93
\c@MaxMatrixCols=\count94
\dotsspace@=\muskip10
\c@parentequation=\count95
\dspbrk@lvl=\count96
\tag@help=\toks17
\row@=\count97
\column@=\count98
\maxfields@=\count99
\andhelp@=\toks18
\eqnshift@=\dimen106
\alignsep@=\dimen107
\tagshift@=\dimen108
\tagwidth@=\dimen109
\totwidth@=\dimen110
\lineht@=\dimen111
\@envbody=\toks19
\multlinegap=\skip44
\multlinetaggap=\skip45
\mathdisplay@stack=\toks20
LaTeX Info: Redefining \[ on input line 2666.
LaTeX Info: Redefining \] on input line 2667.
) (/usr/share/texmf-texlive/tex/latex/amsfonts/amssymb.sty
Package: amssymb 2002/01/22 v2.2d
(/usr/share/texmf-texlive/tex/latex/amsfonts/amsfonts.sty
Package: amsfonts 2001/10/25 v2.2f
\symAMSa=\mathgroup4
\symAMSb=\mathgroup5
LaTeX Font Info:    Overwriting math alphabet `\mathfrak' in version `bold'
(Font)                  U/euf/m/n --> U/euf/b/n on input line 132.
)) (/usr/share/texmf-texlive/tex/latex/base/alltt.sty
Package: alltt 1997/06/16 v2.0g defines alltt environment
) (/usr/share/texmf-texlive/tex/latex/tools/array.sty
Package: array 2005/08/23 v2.4b Tabular extension package (FMi)
\col@sep=\dimen112
\extrarowheight=\dimen113
\NC@list=\toks21
\extratabsurround=\skip46
\backup@length=\skip47
) (/usr/share/texmf-texlive/tex/latex/bigfoot/perpage.sty
Package: perpage 2006/07/15 1.12 Reset/sort counters per page
\c@abspage=\count100
) (/usr/share/texmf-texlive/tex/latex/tools/multicol.sty
Package: multicol 2006/05/18 v1.6g multicolumn formatting (FMi)
\c@tracingmulticols=\count101
\mult@box=\box28
\multicol@leftmargin=\dimen114
\c@unbalance=\count102
\c@collectmore=\count103
\doublecol@number=\count104
\multicoltolerance=\count105
\multicolpretolerance=\count106
\full@width=\dimen115
\page@free=\dimen116
\premulticols=\dimen117
\postmulticols=\dimen118
\multicolsep=\skip48
\multicolbaselineskip=\skip49
\partial@page=\box29
\last@line=\box30
\mult@rightbox=\box31
\mult@grightbox=\box32
\mult@gfirstbox=\box33
\mult@firstbox=\box34
\@tempa=\box35
\@tempa=\box36
\@tempa=\box37
\@tempa=\box38
\@tempa=\box39
\@tempa=\box40
\@tempa=\box41
\@tempa=\box42
\@tempa=\box43
\@tempa=\box44
\@tempa=\box45
\@tempa=\box46
\@tempa=\box47
\@tempa=\box48
\@tempa=\box49
\@tempa=\box50
\@tempa=\box51
\c@columnbadness=\count107
\c@finalcolumnbadness=\count108
\last@try=\dimen119
\multicolovershoot=\dimen120
\multicolundershoot=\dimen121
\mult@nat@firstbox=\box52
\colbreak@box=\box53
) (/usr/share/texmf-texlive/tex/latex/base/makeidx.sty
Package: makeidx 2000/03/29 v1.0m Standard LaTeX package
) (/usr/share/texmf-texlive/tex/latex/caption/caption.sty
Package: caption 2007/01/07 v3.0k Customising captions (AR)
(/usr/share/texmf-texlive/tex/latex/caption/caption3.sty
Package: caption3 2007/01/07 v3.0k caption3 kernel (AR)
(/usr/share/texmf-texlive/tex/latex/graphics/keyval.sty
Package: keyval 1999/03/16 v1.13 key=value parser (DPC)
\KV@toks@=\toks22
)
\captionmargin=\dimen122
\captionmarginx=\dimen123
\captionwidth=\dimen124
\captionindent=\dimen125
\captionparindent=\dimen126
\captionhangindent=\dimen127
)) (/usr/share/texmf-texlive/tex/latex/graphics/graphicx.sty
Package: graphicx 1999/02/16 v1.0f Enhanced LaTeX Graphics (DPC,SPQR)
(/usr/share/texmf-texlive/tex/latex/graphics/graphics.sty
Package: graphics 2006/02/20 v1.0o Standard LaTeX Graphics (DPC,SPQR)
(/usr/share/texmf-texlive/tex/latex/graphics/trig.sty
Package: trig 1999/03/16 v1.09 sin cos tan (DPC)
) (/etc/texmf/tex/latex/config/graphics.cfg
File: graphics.cfg 2007/01/18 v1.5 graphics configuration of teTeX/TeXLive
)
Package graphics Info: Driver file: pdftex.def on input line 90.
(/usr/share/texmf-texlive/tex/latex/pdftex-def/pdftex.def
File: pdftex.def 2007/01/08 v0.04d Graphics/color for pdfTeX
\Gread@gobject=\count109
))
\Gin@req@height=\dimen128
\Gin@req@width=\dimen129
) (/usr/share/texmf-texlive/tex/latex/wrapfig/wrapfig.sty
\wrapoverhang=\dimen130
\WF@size=\dimen131
\c@WF@wrappedlines=\count110
\WF@box=\box54
\WF@everypar=\toks23
Package: wrapfig 2003/01/31  v 3.6
) (/usr/share/texmf-texlive/tex/latex/tools/calc.sty
Package: calc 2005/08/06 v4.2 Infix arithmetic (KKT,FJ)
\calc@Acount=\count111
\calc@Bcount=\count112
\calc@Adimen=\dimen132
\calc@Bdimen=\dimen133
\calc@Askip=\skip50
\calc@Bskip=\skip51
LaTeX Info: Redefining \setlength on input line 75.
LaTeX Info: Redefining \addtolength on input line 76.
\calc@Ccount=\count113
\calc@Cskip=\skip52
) (/usr/share/texmf-texlive/tex/latex/fancyhdr/fancyhdr.sty
\fancy@headwidth=\skip53
\f@ncyO@elh=\skip54
\f@ncyO@erh=\skip55
\f@ncyO@olh=\skip56
\f@ncyO@orh=\skip57
\f@ncyO@elf=\skip58
\f@ncyO@erf=\skip59
\f@ncyO@olf=\skip60
\f@ncyO@orf=\skip61
) (/usr/share/texmf-texlive/tex/latex/geometry/geometry.sty
Package: geometry 2002/07/08 v3.2 Page Geometry
\Gm@cnth=\count114
\Gm@cntv=\count115
\c@Gm@tempcnt=\count116
\Gm@bindingoffset=\dimen134
\Gm@wd@mp=\dimen135
\Gm@odd@mp=\dimen136
\Gm@even@mp=\dimen137
\Gm@dimlist=\toks24
(/usr/share/texmf-texlive/tex/xelatex/xetexconfig/geometry.cfg)) (/usr/share/te
xmf-texlive/tex/latex/hyperref/hyperref.sty
Package: hyperref 2007/02/07 v6.75r Hypertext links for LaTeX
\@linkdim=\dimen138
\Hy@linkcounter=\count117
\Hy@pagecounter=\count118
(/usr/share/texmf-texlive/tex/latex/hyperref/pd1enc.def
File: pd1enc.def 2007/02/07 v6.75r Hyperref: PDFDocEncoding definition (HO)
) (/etc/texmf/tex/latex/config/hyperref.cfg
File: hyperref.cfg 2002/06/06 v1.2 hyperref configuration of TeXLive
) (/usr/share/texmf-texlive/tex/latex/oberdiek/kvoptions.sty
Package: kvoptions 2006/08/22 v2.4 Connects package keyval with LaTeX options (
HO)
)
Package hyperref Info: Option `hyperfootnotes' set `false' on input line 2238.
Package hyperref Info: Option `bookmarks' set `true' on input line 2238.
Package hyperref Info: Option `linktocpage' set `false' on input line 2238.
Package hyperref Info: Option `pdfdisplaydoctitle' set `true' on input line 223
8.
Package hyperref Info: Option `pdfpagelabels' set `true' on input line 2238.
Package hyperref Info: Option `bookmarksopen' set `true' on input line 2238.
Package hyperref Info: Option `colorlinks' set `true' on input line 2238.
Package hyperref Info: Hyper figures OFF on input line 2288.
Package hyperref Info: Link nesting OFF on input line 2293.
Package hyperref Info: Hyper index ON on input line 2296.
Package hyperref Info: Plain pages OFF on input line 2303.
Package hyperref Info: Backreferencing OFF on input line 2308.
Implicit mode ON; LaTeX internals redefined
Package hyperref Info: Bookmarks ON on input line 2444.
(/usr/share/texmf-texlive/tex/latex/ltxmisc/url.sty
\Urlmuskip=\muskip11
Package: url 2005/06/27  ver 3.2  Verb mode for urls, etc.
)
LaTeX Info: Redefining \url on input line 2599.
\Fld@menulength=\count119
\Field@Width=\dimen139
\Fld@charsize=\dimen140
\Choice@toks=\toks25
\Field@toks=\toks26
Package hyperref Info: Hyper figures OFF on input line 3102.
Package hyperref Info: Link nesting OFF on input line 3107.
Package hyperref Info: Hyper index ON on input line 3110.
Package hyperref Info: backreferencing OFF on input line 3117.
Package hyperref Info: Link coloring ON on input line 3120.
\Hy@abspage=\count120
\c@Item=\count121
)
*hyperref using driver hpdftex*
(/usr/share/texmf-texlive/tex/latex/hyperref/hpdftex.def
File: hpdftex.def 2007/02/07 v6.75r Hyperref driver for pdfTeX
\Fld@listcount=\count122
)
\c@pp@a@footnote=\count123
\@indexfile=\write3
\openout3 = `36114-t.idx'.

Writing index file 36114-t.idx
\c@figno=\count124
\TmpLen=\skip62
(./36114-t.aux)
\openout1 = `36114-t.aux'.

LaTeX Font Info:    Checking defaults for OML/cmm/m/it on input line 566.
LaTeX Font Info:    ... okay on input line 566.
LaTeX Font Info:    Checking defaults for T1/cmr/m/n on input line 566.
LaTeX Font Info:    ... okay on input line 566.
LaTeX Font Info:    Checking defaults for OT1/cmr/m/n on input line 566.
LaTeX Font Info:    ... okay on input line 566.
LaTeX Font Info:    Checking defaults for OMS/cmsy/m/n on input line 566.
LaTeX Font Info:    ... okay on input line 566.
LaTeX Font Info:    Checking defaults for OMX/cmex/m/n on input line 566.
LaTeX Font Info:    ... okay on input line 566.
LaTeX Font Info:    Checking defaults for U/cmr/m/n on input line 566.
LaTeX Font Info:    ... okay on input line 566.
LaTeX Font Info:    Checking defaults for PD1/pdf/m/n on input line 566.
LaTeX Font Info:    ... okay on input line 566.
(/usr/share/texmf-texlive/tex/latex/ragged2e/ragged2e.sty
Package: ragged2e 2003/03/25 v2.04 ragged2e Package (MS)
(/usr/share/texmf-texlive/tex/latex/everysel/everysel.sty
Package: everysel 1999/06/08 v1.03 EverySelectfont Package (MS)
LaTeX Info: Redefining \selectfont on input line 125.
)
\CenteringLeftskip=\skip63
\RaggedLeftLeftskip=\skip64
\RaggedRightLeftskip=\skip65
\CenteringRightskip=\skip66
\RaggedLeftRightskip=\skip67
\RaggedRightRightskip=\skip68
\CenteringParfillskip=\skip69
\RaggedLeftParfillskip=\skip70
\RaggedRightParfillskip=\skip71
\JustifyingParfillskip=\skip72
\CenteringParindent=\skip73
\RaggedLeftParindent=\skip74
\RaggedRightParindent=\skip75
\JustifyingParindent=\skip76
)
Package caption Info: hyperref package v6.74m (or newer) detected on input line
 566.
(/usr/share/texmf/tex/context/base/supp-pdf.tex
[Loading MPS to PDF converter (version 2006.09.02).]
\scratchcounter=\count125
\scratchdimen=\dimen141
\scratchbox=\box55
\nofMPsegments=\count126
\nofMParguments=\count127
\everyMPshowfont=\toks27
\MPscratchCnt=\count128
\MPscratchDim=\dimen142
\MPnumerator=\count129
\everyMPtoPDFconversion=\toks28
)
-------------------- Geometry parameters
paper: class default
landscape: --
twocolumn: --
twoside: true
asymmetric: --
h-parts: 9.03374pt, 325.215pt, 9.03375pt
v-parts: 4.15848pt, 495.49379pt, 6.23773pt
hmarginratio: 1:1
vmarginratio: 2:3
lines: --
heightrounded: --
bindingoffset: 0.0pt
truedimen: --
includehead: true
includefoot: true
includemp: --
driver: pdftex
-------------------- Page layout dimensions and switches
\paperwidth  343.28249pt
\paperheight 505.89pt
\textwidth  325.215pt
\textheight 433.62pt
\oddsidemargin  -63.23625pt
\evensidemargin -63.23624pt
\topmargin  -68.11151pt
\headheight 12.0pt
\headsep    19.8738pt
\footskip   30.0pt
\marginparwidth 98.0pt
\marginparsep   7.0pt
\columnsep  10.0pt
\skip\footins  10.8pt plus 4.0pt minus 2.0pt
\hoffset 0.0pt
\voffset 0.0pt
\mag 1000
\@twosidetrue \@mparswitchtrue 
(1in=72.27pt, 1cm=28.45pt)
-----------------------
(/usr/share/texmf-texlive/tex/latex/graphics/color.sty
Package: color 2005/11/14 v1.0j Standard LaTeX Color (DPC)
(/etc/texmf/tex/latex/config/color.cfg
File: color.cfg 2007/01/18 v1.5 color configuration of teTeX/TeXLive
)
Package color Info: Driver file: pdftex.def on input line 130.
)
Package hyperref Info: Link coloring ON on input line 566.
(/usr/share/texmf-texlive/tex/latex/hyperref/nameref.sty
Package: nameref 2006/12/27 v2.28 Cross-referencing by name of section
(/usr/share/texmf-texlive/tex/latex/oberdiek/refcount.sty
Package: refcount 2006/02/20 v3.0 Data extraction from references (HO)
)
\c@section@level=\count130
)
LaTeX Info: Redefining \ref on input line 566.
LaTeX Info: Redefining \pageref on input line 566.
(./36114-t.out) (./36114-t.out)
\@outlinefile=\write4
\openout4 = `36114-t.out'.

LaTeX Font Info:    Try loading font information for U+msa on input line 600.
(/usr/share/texmf-texlive/tex/latex/amsfonts/umsa.fd
File: umsa.fd 2002/01/19 v2.2g AMS font definitions
)
LaTeX Font Info:    Try loading font information for U+msb on input line 600.
(/usr/share/texmf-texlive/tex/latex/amsfonts/umsb.fd
File: umsb.fd 2002/01/19 v2.2g AMS font definitions
) [1

{/var/lib/texmf/fonts/map/pdftex/updmap/pdftex.map}] [2] [1

] [2] [3


] [4] [5] [6] [7] (./36114-t.toc [8



] [9])
\tf@toc=\write5
\openout5 = `36114-t.toc'.

[10] [11


] [1

] [2] [3] [4] [5] [6] [7] [8] [9] [10] [11] [12] [13] [14] [15] [16] [17] [18] 
[19] [20] [21] [22] <./images/025.pdf, id=519, 338.26375pt x 50.1875pt>
File: ./images/025.pdf Graphic file (type pdf)
<use ./images/025.pdf> [23 <./images/025.pdf>] [24] [25] [26] [27] [28] <./imag
es/032.pdf, id=581, 194.7275pt x 150.5625pt>
File: ./images/032.pdf Graphic file (type pdf)
<use ./images/032.pdf> [29] [30 <./images/032.pdf>] [31] [32] [33] [34] [35] [3
6] [37] <./images/040.pdf, id=649, 222.8325pt x 39.14626pt>
File: ./images/040.pdf Graphic file (type pdf)
<use ./images/040.pdf> [38 <./images/040.pdf>] [39] [40] [41] [42] [43] [44] [4
5] [46] [47] [48] [49] [50] [51] [52] [53] [54] [55


] [56] [57] [58] [59] [60] [61] [62] [63] [64] [65] [66] [67] [68] [69] [70] [7
1] [72] [73] [74] [75] [76] [77] [78] [79] [80] <./images/087.pdf, id=880, 209.
78375pt x 129.48375pt>
File: ./images/087.pdf Graphic file (type pdf)
<use ./images/087.pdf> [81 <./images/087.pdf>] [82] [83] [84] [85] [86] [87] [8
8] [89] [90] [91] [92] [93] [94] [95] [96] [97] [98


] [99] [100] [101] [102] [103] [104] [105] [106] [107] [108

] [109] [110] [111] [112] [113] [114

] [115] [116

] [117] [118] [119] <./images/127.pdf, id=1169, 99.37125pt x 212.795pt>
File: ./images/127.pdf Graphic file (type pdf)
<use ./images/127.pdf> [120 <./images/127.pdf>] [121] [122] [123] [124] [125] [
126


] (./36114-t.ind [127] [128

] [129] [130] [131] [132]) [133


] [1

] [2] [3] [4] [5] [6] [7] [8] (./36114-t.aux)

 *File List*
    book.cls    2005/09/16 v1.4f Standard LaTeX document class
    bk12.clo    2005/09/16 v1.4f Standard LaTeX file (size option)
inputenc.sty    2006/05/05 v1.1b Input encoding file
  latin1.def    2006/05/05 v1.1b Input encoding file
  ifthen.sty    2001/05/26 v1.1c Standard LaTeX ifthen package (DPC)
 amsmath.sty    2000/07/18 v2.13 AMS math features
 amstext.sty    2000/06/29 v2.01
  amsgen.sty    1999/11/30 v2.0
  amsbsy.sty    1999/11/29 v1.2d
  amsopn.sty    1999/12/14 v2.01 operator names
 amssymb.sty    2002/01/22 v2.2d
amsfonts.sty    2001/10/25 v2.2f
   alltt.sty    1997/06/16 v2.0g defines alltt environment
   array.sty    2005/08/23 v2.4b Tabular extension package (FMi)
 perpage.sty    2006/07/15 1.12 Reset/sort counters per page
multicol.sty    2006/05/18 v1.6g multicolumn formatting (FMi)
 makeidx.sty    2000/03/29 v1.0m Standard LaTeX package
 caption.sty    2007/01/07 v3.0k Customising captions (AR)
caption3.sty    2007/01/07 v3.0k caption3 kernel (AR)
  keyval.sty    1999/03/16 v1.13 key=value parser (DPC)
graphicx.sty    1999/02/16 v1.0f Enhanced LaTeX Graphics (DPC,SPQR)
graphics.sty    2006/02/20 v1.0o Standard LaTeX Graphics (DPC,SPQR)
    trig.sty    1999/03/16 v1.09 sin cos tan (DPC)
graphics.cfg    2007/01/18 v1.5 graphics configuration of teTeX/TeXLive
  pdftex.def    2007/01/08 v0.04d Graphics/color for pdfTeX
 wrapfig.sty    2003/01/31  v 3.6
    calc.sty    2005/08/06 v4.2 Infix arithmetic (KKT,FJ)
fancyhdr.sty    
geometry.sty    2002/07/08 v3.2 Page Geometry
geometry.cfg
hyperref.sty    2007/02/07 v6.75r Hypertext links for LaTeX
  pd1enc.def    2007/02/07 v6.75r Hyperref: PDFDocEncoding definition (HO)
hyperref.cfg    2002/06/06 v1.2 hyperref configuration of TeXLive
kvoptions.sty    2006/08/22 v2.4 Connects package keyval with LaTeX options (HO
)
     url.sty    2005/06/27  ver 3.2  Verb mode for urls, etc.
 hpdftex.def    2007/02/07 v6.75r Hyperref driver for pdfTeX
ragged2e.sty    2003/03/25 v2.04 ragged2e Package (MS)
everysel.sty    1999/06/08 v1.03 EverySelectfont Package (MS)
supp-pdf.tex
   color.sty    2005/11/14 v1.0j Standard LaTeX Color (DPC)
   color.cfg    2007/01/18 v1.5 color configuration of teTeX/TeXLive
 nameref.sty    2006/12/27 v2.28 Cross-referencing by name of section
refcount.sty    2006/02/20 v3.0 Data extraction from references (HO)
 36114-t.out
 36114-t.out
    umsa.fd    2002/01/19 v2.2g AMS font definitions
    umsb.fd    2002/01/19 v2.2g AMS font definitions
./images/025.pdf
./images/032.pdf
./images/040.pdf
./images/087.pdf
./images/127.pdf
 36114-t.ind
 ***********

 ) 
Here is how much of TeX's memory you used:
 5734 strings out of 94074
 81940 string characters out of 1165154
 147634 words of memory out of 1500000
 8516 multiletter control sequences out of 10000+50000
 17695 words of font info for 67 fonts, out of 1200000 for 2000
 645 hyphenation exceptions out of 8191
 34i,18n,44p,464b,649s stack positions out of 5000i,500n,6000p,200000b,5000s
</usr/share/texmf-texlive/fonts/type1/bluesky/cm/cmbx10.pfb></usr/share/texmf
-texlive/fonts/type1/bluesky/cm/cmbx12.pfb></usr/share/texmf-texlive/fonts/type
1/bluesky/cm/cmbxti10.pfb></usr/share/texmf-texlive/fonts/type1/bluesky/cm/cmcs
c10.pfb></usr/share/texmf-texlive/fonts/type1/bluesky/cm/cmex10.pfb></usr/share
/texmf-texlive/fonts/type1/bluesky/cm/cmmi10.pfb></usr/share/texmf-texlive/font
s/type1/bluesky/cm/cmmi12.pfb></usr/share/texmf-texlive/fonts/type1/bluesky/cm/
cmr10.pfb></usr/share/texmf-texlive/fonts/type1/bluesky/cm/cmr12.pfb></usr/shar
e/texmf-texlive/fonts/type1/bluesky/cm/cmr7.pfb></usr/share/texmf-texlive/fonts
/type1/bluesky/cm/cmr8.pfb></usr/share/texmf-texlive/fonts/type1/bluesky/cm/cms
y10.pfb></usr/share/texmf-texlive/fonts/type1/bluesky/cm/cmsy7.pfb></usr/share/
texmf-texlive/fonts/type1/bluesky/cm/cmsy8.pfb></usr/share/texmf-texlive/fonts/
type1/bluesky/cm/cmti10.pfb></usr/share/texmf-texlive/fonts/type1/bluesky/cm/cm
ti12.pfb></usr/share/texmf-texlive/fonts/type1/bluesky/cm/cmtt10.pfb></usr/shar
e/texmf-texlive/fonts/type1/bluesky/cm/cmtt8.pfb>
Output written on 36114-t.pdf (154 pages, 664963 bytes).
PDF statistics:
 1881 PDF objects out of 2073 (max. 8388607)
 401 named destinations out of 1000 (max. 131072)
 418 words of extra memory for PDF output out of 10000 (max. 10000000)

