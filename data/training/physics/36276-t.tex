% %%%%%%%%%%%%%%%%%%%%%%%%%%%%%%%%%%%%%%%%%%%%%%%%%%%%%%%%%%%%%%%%%%%%%%% %
%                                                                         %
% The Project Gutenberg EBook of The Meaning of Relativity, by Albert Einstein
%                                                                         %
% This eBook is for the use of anyone anywhere at no cost and with        %
% almost no restrictions whatsoever.  You may copy it, give it away or    %
% re-use it under the terms of the Project Gutenberg License included     %
% with this eBook or online at www.gutenberg.net                          %
%                                                                         %
%                                                                         %
% Title: The Meaning of Relativity                                        %
%        Four lectures delivered at Princeton University, May, 1921       %
%                                                                         %
% Author: Albert Einstein                                                 %
%                                                                         %
% Translator: Edwin Plimpton Adams                                        %
%                                                                         %
% Release Date: May 29, 2011 [EBook #36276]                               %
%                                                                         %
% Language: English                                                       %
%                                                                         %
% Character set encoding: ISO-8859-1                                      %
%                                                                         %
% *** START OF THIS PROJECT GUTENBERG EBOOK THE MEANING OF RELATIVITY *** %
%                                                                         %
% %%%%%%%%%%%%%%%%%%%%%%%%%%%%%%%%%%%%%%%%%%%%%%%%%%%%%%%%%%%%%%%%%%%%%%% %

\def\ebook{36276}
%%%%%%%%%%%%%%%%%%%%%%%%%%%%%%%%%%%%%%%%%%%%%%%%%%%%%%%%%%%%%%%%%%%%%%
%%                                                                  %%
%% Packages and substitutions:                                      %%
%%                                                                  %%
%% book:     Required.                                              %%
%% inputenc: Latin-1 text encoding. Required.                       %%
%%                                                                  %%
%% ifthen:   Logical conditionals. Required.                        %%
%%                                                                  %%
%% amsmath:  AMS mathematics enhancements. Required.                %%
%% amssymb:  Additional mathematical symbols. Required.             %%
%%                                                                  %%
%% alltt:    Fixed-width font environment. Required.                %%
%% array:    Enhanced tabular features. Required.                   %%
%%                                                                  %%
%% footmisc: Extended footnote capabilities. Required.              %%
%% perpage:  Start footnote numbering on each page. Required.       %%
%%                                                                  %%
%% multicol: Twocolumn environment for index. Required.             %%
%% makeidx:  Indexing. Required.                                    %%
%%                                                                  %%
%% caption:  Caption customization. Required.                       %%
%% graphicx: Standard interface for graphics inclusion. Required.   %%
%%                                                                  %%
%% calc:     Length calculations. Required.                         %%
%% yfonts:   Fraktur symbols. Required.                             %%
%%                                                                  %%
%% fancyhdr: Enhanced running headers and footers. Required.        %%
%%                                                                  %%
%% geometry: Enhanced page layout package. Required.                %%
%% hyperref: Hypertext embellishments for pdf output. Required.     %%
%%                                                                  %%
%%                                                                  %%
%% Producer's Comments:                                             %%
%%                                                                  %%
%%   OCR text for this ebook was obtained on May 14, 2011, from     %%
%%   http://www.archive.org/details/meaningofrelativ00eins.         %%
%%                                                                  %%
%%   Minor changes to the original are noted in this file in three  %%
%%   ways:                                                          %%
%%     1. \Change{}{} for typographical corrections, showing        %%
%%        original and replacement text side-by-side.               %%
%%     2. \Add{} for inconsistent/missing punctuation.              %%
%%     3. [** TN: Note]s for miscellaneous comments.                %%
%%   \Add is implemented in terms of \Change, so redefining \Change %%
%%   will "restore" typographical errors in the original.           %%
%%                                                                  %%
%%                                                                  %%
%% Compilation Flags:                                               %%
%%                                                                  %%
%%   The following behavior may be controlled by boolean flags.     %%
%%                                                                  %%
%%   ForPrinting (false by default):                                %%
%%   If true, compile a print-optimized PDF file: Taller text block,%%
%%   two-sided layout, US Letter paper, black hyperlinks. Default:  %%
%%   screen optimized file (one-sided layout, blue hyperlinks).     %%
%%                                                                  %%
%% PDF pages: 134 (if ForPrinting set to false)                     %%
%% PDF page size: 4.75 x 7"                                         %%
%% PDF bookmarks: created, point to ToC entries                     %%
%% PDF document info: filled in                                     %%
%% Images: 4 pdf diagrams                                           %%
%%                                                                  %%
%% Summary of log file:                                             %%
%% * One underfull box.                                             %%
%%                                                                  %%
%% Compile History:                                                 %%
%%                                                                  %%
%% May, 2011: adhere (Andrew D. Hwang)                              %%
%%            texlive2007, GNU/Linux                                %%
%%                                                                  %%
%% Command block:                                                   %%
%%                                                                  %%
%%     pdflatex x3                                                  %%
%%     makeindex -s relativity.ist # config file specified below    %%
%%     pdflatex x3                                                  %%
%%                                                                  %%
%%                                                                  %%
%% May 2011: pglatex.                                               %%
%%   Compile this project with:                                     %%
%%   pdflatex 36276-t.tex ..... THREE times                         %%
%%   makeindex -s relativity.ist 36276-t.idx                        %%
%%   pdflatex 36276-t.tex ..... THREE times                         %%
%%                                                                  %%
%%   pdfTeXk, Version 3.141592-1.40.3 (Web2C 7.5.6)                 %%
%%                                                                  %%
%%%%%%%%%%%%%%%%%%%%%%%%%%%%%%%%%%%%%%%%%%%%%%%%%%%%%%%%%%%%%%%%%%%%%%
\listfiles
\documentclass[12pt]{book}[2005/09/16]

%%%%%%%%%%%%%%%%%%%%%%%%%%%%% PACKAGES %%%%%%%%%%%%%%%%%%%%%%%%%%%%%%%
\usepackage[latin1]{inputenc}[2006/05/05]

\usepackage{ifthen}[2001/05/26]  %% Logical conditionals

\usepackage{amsmath}[2000/07/18] %% Displayed equations
\usepackage{amssymb}[2002/01/22] %% and additional symbols

\usepackage{alltt}[1997/06/16]   %% boilerplate, credits, license
\usepackage{array}[2005/08/23]   %% extended array/tabular features

\usepackage[symbol,perpage]{footmisc}[2005/03/17]
\usepackage{perpage}[2006/07/15]

\usepackage{multicol}[2006/05/18]
\usepackage{makeidx}[2000/03/29]

\usepackage[font=footnotesize,labelformat=empty]{caption}[2007/01/07]
\usepackage{graphicx}[1999/02/16]%% For diagrams

\usepackage{calc}[2005/08/06]
\usepackage{yfonts}[2003/01/08]

% for running heads
\usepackage{fancyhdr}

%%%%%%%%%%%%%%%%%%%%%%%%%%%%%%%%%%%%%%%%%%%%%%%%%%%%%%%%%%%%%%%%%
%%%% Interlude:  Set up PRINTING (default) or SCREEN VIEWING %%%%
%%%%%%%%%%%%%%%%%%%%%%%%%%%%%%%%%%%%%%%%%%%%%%%%%%%%%%%%%%%%%%%%%

% ForPrinting=true                     false (default)
% Asymmetric margins                   Symmetric margins
% 1 : 1.62 text block aspect ratio     3 : 4 text block aspect ratio
% Black hyperlinks                     Blue hyperlinks
% Start major marker pages recto       No blank verso pages
%
% Chapter-like ``Sections'' start both recto and verso in the scanned
% book. This behavior has been retained.
\newboolean{ForPrinting}

%% UNCOMMENT the next line for a PRINT-OPTIMIZED VERSION of the text %%
%\setboolean{ForPrinting}{true}

%% Initialize values to ForPrinting=false
\newcommand{\Margins}{hmarginratio=1:1}     % Symmetric margins
\newcommand{\HLinkColor}{blue}              % Hyperlink color
\newcommand{\PDFPageLayout}{SinglePage}
\newcommand{\TransNote}{Transcriber's Note}
\newcommand{\TransNoteCommon}{%
  The camera-quality files for this public-domain ebook may be
  downloaded \textit{gratis} at
  \begin{center}
    \texttt{www.gutenberg.org/ebooks/\ebook}.
  \end{center}

  This ebook was produced using OCR text provided by Northeastern
  University's Snell Library through the Internet Archive.
  \bigskip

  Minor typographical corrections and presentational changes have been
  made without comment.
  \bigskip
}

\newcommand{\TransNoteText}{%
  \TransNoteCommon

  This PDF file is optimized for screen viewing, but may easily be
  recompiled for printing. Please consult the preamble of the \LaTeX\
  source file for instructions and other particulars.
}
%% Re-set if ForPrinting=true
\ifthenelse{\boolean{ForPrinting}}{%
  \renewcommand{\Margins}{hmarginratio=2:3} % Asymmetric margins
  \renewcommand{\HLinkColor}{black}         % Hyperlink color
  \renewcommand{\PDFPageLayout}{TwoPageRight}
  \renewcommand{\TransNote}{Transcriber's Note}
  \renewcommand{\TransNoteText}{%
    \TransNoteCommon

    This PDF file is optimized for printing, but may easily be
    recompiled for screen viewing. Please consult the preamble
    of the \LaTeX\ source file for instructions and other particulars.
  }
}{% If ForPrinting=false, don't skip to recto
  \renewcommand{\cleardoublepage}{\clearpage}
}
%%%%%%%%%%%%%%%%%%%%%%%%%%%%%%%%%%%%%%%%%%%%%%%%%%%%%%%%%%%%%%%%%
%%%%  End of PRINTING/SCREEN VIEWING code; back to packages  %%%%
%%%%%%%%%%%%%%%%%%%%%%%%%%%%%%%%%%%%%%%%%%%%%%%%%%%%%%%%%%%%%%%%%

\ifthenelse{\boolean{ForPrinting}}{%
  \setlength{\paperwidth}{8.5in}%
  \setlength{\paperheight}{11in}%
% ~1:1.62
  \usepackage[body={4.5in,7.3in},\Margins]{geometry}[2002/07/08]
}{%
  \setlength{\paperwidth}{4.75in}%
  \setlength{\paperheight}{7in}%
  \raggedbottom
% ~3:4
  \usepackage[body={4.5in,6in},\Margins,includeheadfoot]{geometry}[2002/07/08]
}

\providecommand{\ebook}{00000}    % Overridden during white-washing
\usepackage[pdftex,
  hyperref,
  hyperfootnotes=false,
  pdftitle={The Project Gutenberg eBook \#\ebook: The Meaning of Relativity},
  pdfauthor={Albert Einstein},
  pdfkeywords={Edward Plimpton Adams, Northeastern University,
    Snell Library, The Internet Archive, Andrew D. Hwang},
  pdfstartview=Fit,    % default value
  pdfstartpage=1,      % default value
  pdfpagemode=UseNone, % default value
  bookmarks=true,      % default value
  linktocpage=false,   % default value
  pdfpagelayout=\PDFPageLayout,
  pdfdisplaydoctitle,
  pdfpagelabels=true,
  bookmarksopen=true,
  bookmarksopenlevel=0,
  colorlinks=true,
  linkcolor=\HLinkColor]{hyperref}[2007/02/07]


%%%% Fixed-width environment to format PG boilerplate %%%%
\newenvironment{PGtext}{%
\begin{alltt}
\fontsize{8.1}{9}\ttfamily\selectfont}%
{\end{alltt}}

%% No hrule in page header
\renewcommand{\headrulewidth}{0pt}

% Top-level footnote numbers restart on each page
\MakePerPage{footnote}

\hyphenation{electro-dynamics electro-magnetic electro-motive electro-static}

% Running heads
\newcommand{\FlushRunningHeads}{\clearpage\fancyhf{}\cleardoublepage}
\newcommand{\InitRunningHeads}{%
  \setlength{\headheight}{15pt}
  \pagestyle{fancy}
  \thispagestyle{plain}
  \ifthenelse{\boolean{ForPrinting}}
             {\fancyhead[RO,LE]{\thepage}}
             {\fancyhead[R]{\thepage}}
}

\newcommand{\SetOddHead}[1]{%
  \fancyhead[CO]{\textbf{\MakeUppercase{#1}}}
}

\newcommand{\SetEvenHead}[1]{%
  \fancyhead[CE]{\textbf{\MakeUppercase{#1}}}
}

\newcommand{\BookMark}[2]{\phantomsection\pdfbookmark[#1]{#2}{#2}}

\newcommand{\PGLicenseInit}{%
  \cleardoublepage
  \BookMark{0}{PG License}
  \SetEvenHead{Licensing}
  \SetOddHead{Licensing}
  \pagenumbering{Roman}
}

% ToC formatting
\AtBeginDocument{\renewcommand{\contentsname}%
  {\protect\thispagestyle{plain}%
    \protect\centering\normalfont\large CONTENTS}}

\newcommand{\ToCFont}{\centering\normalfont\normalsize\scshape}
\newcommand{\TableofContents}{%
  \FlushRunningHeads
  \InitRunningHeads
  \SetOddHead{Contents}
  \BookMark{0}{Contents}
  \tableofcontents
}

% For internal bookkeeping
\newboolean{NeedToCPage}
\setboolean{NeedToCPage}{true}

%\ToCLine{Title}{xref}
\newcommand{\ToCLine}[2]{%
  \ifthenelse{\boolean{NeedToCPage}}{%
    \noindent\parbox{\textwidth}{\null\hfill\scriptsize PAGE}\par%
    \setboolean{NeedToCPage}{false}%
  }{}%
  \settowidth{\TmpLen}{\;\pageref{#2}}%
  \noindent\strut\parbox[b]{\textwidth-\TmpLen}{#1\dotfill}%
  \makebox[\TmpLen][r]{\pageref{#2}}%
}

% Index formatting
\makeindex
\newcommand{\PrintIndex}{%
  \cleardoublepage
  \BookMark{-1}{Back Matter}
  \backmatter
  \printindex
}
\newcommand{\etseq}[1]{\hyperpage{#1}~\textit{et seq.}}

\makeatletter
\renewcommand{\@idxitem}{\par\hangindent 30\p@\global\let\idxbrk\nobreak}
\renewcommand\subitem{\idxbrk\@idxitem --- \let\idxbrk\relax}
\renewcommand\subsubitem{\idxbrk\@idxitem --- --- \let\idxbrk\relax}
\renewcommand{\indexspace}{\par\penalty-3000 \vskip 10pt plus5pt minus3pt\relax}

% Index of original prints "A" at start of A entries, etc.
% Write index style configuration code to relativity.ist.
\immediate\openout\@partaux relativity.ist
\immediate\write\@partaux
  {heading_prefix "{\string\\centering "^^J%
   heading_suffix "\string\\endgraf}\string\\nopagebreak\string\n"^^J%
   headings_flag 1 }%
\immediate\closeout\@partaux

\renewenvironment{theindex}{%
  \setlength\columnseprule{0.5pt}\setlength\columnsep{18pt}%
  \cleardoublepage
  \phantomsection
  \label{index}
  \addtocontents{toc}{\protect\ToCLine{\protect\rule{0pt}{36pt}\textsc{Index}}{index}}
  \SetOddHead{Index}
  \BookMark{0}{Index}
% ** N.B. font size
  \begin{multicols}{2}[\subsection*{\centering\normalfont\large INDEX}\small]
    \setlength\parindent{0pt}\setlength\parskip{0pt plus 0.3pt}%
    \thispagestyle{plain}\let\item\@idxitem\raggedright%
  }{%
  \end{multicols}\FlushRunningHeads
}
\makeatother

%\Lecture[continued]{Num}{Title}{Running Head}
\newcommand{\Lecture}[4][]{%
  \ifthenelse{\equal{#2}{I}}{%
    \begin{center}
      \textbf{\Large THE MEANING OF RELATIVITY}
    \end{center}
  }{%
    \FlushRunningHeads
  }%
  \fancyhf{}
  \InitRunningHeads
  \ifthenelse{\equal{#1}{}}{%
    \BookMark{0}{Lecture #2. #3}
  }{%
    \BookMark{0}{Lecture #2. #3 (continued)}
}
  \label{lecture:#2}
  \SetEvenHead{The Meaning of Relativity}
  \SetOddHead{#4}
  \thispagestyle{plain}
  \addtocontents{toc}{\protect\subsection*{\protect\ToCFont Lecture #2}}
  \addtocontents{toc}{\protect\ToCLine{\textsc{#3}#1}{lecture:#2}}
  \section*{\centering\normalfont\small LECTURE #2}
  \subsection*{\centering\normalfont\large\MakeUppercase{#3}\normalsize#1}
}

\newcommand{\Section}[1]{%
  \subsection*{\centering\normalfont\large\textsc{#1}}
}

\newcommand{\Subsection}[1]{%
  \pagebreak[1]\textit{#1}\pagebreak[0]
}

\newcommand{\Paragraph}[1]{\paragraph*{\indent\normalfont\itshape#1}}

\newcommand{\TranslatorsNote}[1]{%
  \small\settowidth{\TmpLen}
  {\textsc{Note}.---The translation of these lectures into English}
  \parbox[c]{\TmpLen}{%
    \hangindent2em#1%
  }
}

\renewenvironment{itemize}{%
  \begin{list}{}{\setlength{\topsep}{4pt plus 8pt}%
      \setlength{\itemsep}{0pt plus 2pt}%
      \setlength{\parsep}{4pt plus 2pt}%
      \setlength{\leftmargin}{4em}}}{\end{list}}

\newenvironment{CenterPage}{%
  \thispagestyle{empty}%
  \null\vfill%
  \begin{center}
  }{%
  \end{center}
  \vfill%
}

\newcounter{figno}
\newcommand{\Figure}[2][0.75\textwidth]{%
\begin{figure}[hbt!]
  \refstepcounter{figno}
  \centering
  \includegraphics[width=#1]{./images/#2.pdf}
  \caption{\textsc{Fig}.~\thefigno.}
  \label{fig:\thefigno}
\end{figure}
}

\newcommand{\First}[1]{\textsc{\large #1}}

% For corrections.
\newcommand{\TNote}[1]{}
\newcommand{\Change}[2]{#2}
\newcommand{\Add}[1]{\Change{}{#1}}

\newcommand{\PageSep}[1]{\ignorespaces}
\setlength{\emergencystretch}{1em}

\newlength{\TmpLen}

\DeclareInputText{176}{\ifmmode{{}^\circ}\else\textdegree\fi}
\DeclareInputText{183}{\ifmmode\cdot\else\textperiodcentered\fi}

\newcommand{\Tag}[1]{%
  \phantomsection
  \label{eqn:#1}
  \tag*{#1}
}

% and links
\newcommand{\Eqref}[1]{\hyperref[eqn:#1]{#1}}
\newcommand{\Figref}[1]{\hyperref[fig:#1]{Fig.~#1}}

% Miscellaneous math
\renewcommand{\Bar}[2][10pt]{%
  \settowidth{\TmpLen}{$#2$}%
  \addtolength{\TmpLen}{-1pt}%
  \overline{\makebox[\TmpLen][l]{\rule{0pt}{#1}$#2$}}\kern1pt
}
\newcommand{\dd}{\partial}
\newcommand{\Neg}[1][{\,}]{\phantom{#1-#1}}
\newcommand{\Oint}{\oint}%[** TN: Original uses \int_{0}]
\newcommand{\Subdots}{\,\mbox{\footnotesize$\cdots$}}

% Symbols for Christoffel symbol of first/second kind
\newcommand{\Chr}[3]{\vphantom{\big|}\smash{\genfrac{}{}{0pt}{1}{#1#2}{#3}}}

\newcommand{\Vector}[1]{\mathbf{#1}}
\newcommand{\e}{\Vector{e}}
\newcommand{\h}{\Vector{h}}
\newcommand{\veci}{\Vector{i}}
\renewcommand{\k}{\Vector{k}}
\newcommand{\q}{\Vector{q}}
\newcommand{\s}{\Vector{s}}
\renewcommand{\v}{\Vector{v}}

\newcommand{\K}{\Vector{K}}
\renewcommand{\P}{\Vector{P}}

%[** TN: \textgoth better matches the original than \mathfrak]
\newcommand{\Tensor}[1]{\textgoth{#1}}
\DeclareMathOperator{\tA}{\Tensor{A}}
\DeclareMathOperator{\tF}{\Tensor{F}}
\DeclareMathOperator{\tJ}{\Tensor{J}}
\DeclareMathOperator{\tT}{\Tensor{T}}

\DeclareMathOperator{\grad}{grad}
\DeclareMathOperator{\rot}{rot}

%%%%%%%%%%%%%%%%%%%%%%%% START OF DOCUMENT %%%%%%%%%%%%%%%%%%%%%%%%%%
\begin{document}
\pagenumbering{Alph}
\pagestyle{empty}
\BookMark{-1}{Front Matter}
%%%% PG BOILERPLATE %%%%
\BookMark{0}{PG Boilerplate}
\begin{center}
\begin{minipage}{\textwidth}
\small
\begin{PGtext}
The Project Gutenberg EBook of The Meaning of Relativity, by Albert Einstein

This eBook is for the use of anyone anywhere at no cost and with
almost no restrictions whatsoever.  You may copy it, give it away or
re-use it under the terms of the Project Gutenberg License included
with this eBook or online at www.gutenberg.net


Title: The Meaning of Relativity
       Four lectures delivered at Princeton University, May, 1921

Author: Albert Einstein

Translator: Edwin Plimpton Adams

Release Date: May 29, 2011 [EBook #36276]

Language: English

Character set encoding: ISO-8859-1

*** START OF THIS PROJECT GUTENBERG EBOOK THE MEANING OF RELATIVITY ***
\end{PGtext}
\end{minipage}
\end{center}
\newpage
%%%% Credits and transcriber's note %%%%
\begin{center}
\begin{minipage}{\textwidth}
\begin{PGtext}
Produced by Andrew D. Hwang. (This ebook was produced using
OCR text generously provided by Northeastern University's
Snell Library through the Internet Archive.)
\end{PGtext}
\end{minipage}
\end{center}
\vfill

\begin{minipage}{0.85\textwidth}
\small
\BookMark{0}{Transcriber's Note}
\subsection*{\centering\normalfont\scshape%
\normalsize\MakeLowercase{\TransNote}}%

\raggedright
\TransNoteText
\end{minipage}
%%%%%%%%%%%%%%%%%%%%%%%%%%% FRONT MATTER %%%%%%%%%%%%%%%%%%%%%%%%%%
\frontmatter
\pagestyle{empty}
\PageSep{i}
\null\vfill
\noindent\textbf{\large THE MEANING OF RELATIVITY}
\vfill
\cleardoublepage
\PageSep{ii}
%[Blank page]
\PageSep{iii}
% Title page
\begin{center}
\textbf{\huge THE MEANING OF \\[8pt]
RELATIVITY}
\bigskip

\normalsize
FOUR LECTURES DELIVERED AT \\
PRINCETON UNIVERSITY, MAY, 1921
\bigskip

{\footnotesize BY} \\[8pt]
\textbf{\huge ALBERT EINSTEIN}
\vfill\vfill

{\footnotesize WITH FOUR DIAGRAMS}
\vfill\vfill\vfill

\normalsize
PRINCETON \\
PRINCETON UNIVERSITY PRESS \\
1923
\end{center}
\newpage
\PageSep{iv}
\begin{center}
\null
\vfill\vfill
\itshape\footnotesize
Copyright 1922 \\
Princeton University Press \\
Published 1922
\vfill
\upshape
PRINTED IN GREAT BRITAIN \\
AT THE ABERDEEN UNIVERSITY PRESS \\
ABERDEEN
\vfill

\TranslatorsNote{%
  \textsc{Note}.---The translation of these lectures into English
  was made by \textsc{Edwin Plimpton Adams}, Professor
  of Physics in Princeton University}
\cleardoublepage
\end{center}
\PageSep{v}
\TableofContents
\iffalse %%%% Table of contents text %%%%
CONTENTS

LECTURE I                                   PAGE
Space and Time in Pre-Relativity Physics       1

LECTURE II
The Theory of Special Relativity              26

LECTURE III
The General Theory of Relativity              61

LECTURE IV
The General Theory of Relativity (continued)  87

Index 121
\fi %%%% Table of contents text %%%%
\newpage
\PageSep{1}


\mainmatter
\BookMark{-1}{Main Matter}
%[** TN: Text printed by \Lecture macro: THE MEANING OF RELATIVITY]
\Lecture{I}{Space and Time in Pre-Relativity
Physics}{Pre-Relativity Physics}

\First{The} theory of relativity is intimately connected with
the theory of space and time. I shall therefore begin
with a brief investigation of the origin of our ideas of space
and time, although in doing so I know that I introduce a
controversial subject. The object of all science, whether
natural science or psychology, is to co-ordinate our experiences
and to bring them into a logical system. How are
our customary ideas of space and time related to the
character of our experiences?

The experiences of an individual appear to us arranged
in a series of events; in this series the single events which
we remember appear to be ordered according to the criterion
of ``earlier'' and ``later,'' which cannot be analysed further.
There exists, therefore, for the individual, an I-time, or
subjective time. This in itself is not measurable. I can,
indeed, associate numbers with the events, in such a way
that a greater number is associated with the later event
than with an earlier one; but the nature of this association
may be quite arbitrary. This association I can define by
means of a clock by comparing the order of events furnished
\PageSep{2}
by the clock with the order of the given series of events.
We understand by a clock something which provides a
series of events which can be counted, and which has other
properties of which we shall speak later.

By the aid of speech different individuals can, to a certain
extent, compare their experiences. In this way it is shown
that certain sense perceptions of different individuals
correspond to each other, while for other sense perceptions
no such correspondence can be established. We are accustomed
to regard as real those sense perceptions which
are common to different individuals, and which therefore
are, in a measure, impersonal. The natural sciences, and
in particular, the most fundamental of them, physics, deal
with such sense perceptions. The conception of physical
bodies, in particular of rigid bodies, is a relatively constant
complex of such sense perceptions. A clock is also a body,
or a system, in the same sense, with the additional property
that the series of events which it counts is formed of
elements all of which can be regarded as equal.

The only justification for our concepts and system of
concepts is that they serve to represent the complex of
our experiences; beyond this they have no legitimacy. I
am convinced that the philosophers have had a harmful
effect upon the progress of scientific thinking in removing
certain fundamental concepts from the domain of empiricism,
where they are under our control, to the intangible
heights of the \textit{a~priori}. For even if it should appear that
the universe of ideas cannot be deduced from experience
by logical means, but is, in a sense, a creation of the human
mind, without which no science is possible, nevertheless
\PageSep{3}
this universe of ideas is just as little independent of the
nature of our experiences as clothes are of the form of
the human body. This is particularly true of our concepts
of time and space, which physicists have been
\index{Space@{Space, Concept of}}%
obliged by the facts to bring down from the Olympus of
the \textit{a~priori} in order to adjust them and put them in a
serviceable condition.

We now come to our concepts and judgments of space.
It is essential here also to pay strict attention to the
relation of experience to our concepts. It seems to me
that Poincar� clearly recognized the truth in the account
he gave in his book, ``La Science et l'Hypothese.''
Among all the changes which we can perceive in a rigid
body those are marked by their simplicity which can be
made reversibly by an arbitrary motion of the body;
Poincar� calls these, changes in position. By means of
simple changes in position we can bring two bodies into
contact. The theorems of congruence, fundamental in
\index{Congruence, theorems of}%
\index{Theorems of congruence}%
geometry, have to do with the laws that govern such
changes in position. For the concept of space the following
\index{Concept@{Concept of space}}%
seems essential. We can form new bodies by bringing
bodies $B$,~$C$,~\dots\ up to body~$A$; we say that we \emph{continue}
body~$A$. We can continue body~$A$ in such a way that
it comes into contact with any other body,~$X$. The
ensemble of all continuations of body~$A$ we can designate
as the ``space of the body~$A$.'' Then it is true that all
bodies are in the ``space of the (arbitrarily chosen) body~$A$.''
In this sense we cannot speak of space in the
abstract, but only of the ``space belonging to a body~$A$.''
The earth's crust plays such a dominant r�le in our daily
\PageSep{4}
life in judging the relative positions of bodies that it has
led to an abstract conception of space which certainly
cannot be defended. In order to free ourselves from this
fatal error we shall speak only of ``bodies of reference,''
or ``space of reference.'' It was only through the theory
\index{Reference, space of}%
\index{Spaces of reference}%
of general relativity that refinement of these concepts
became necessary, as we shall see later.

I shall not go into detail concerning those properties
of the space of reference which lead to our conceiving
points as elements of space, and space as a continuum.
Nor shall I attempt to analyse further the properties of
space which justify the conception of continuous series
of points, or lines. If these concepts are assumed, together
with their relation to the solid bodies of experience, then
it is easy to say what we mean by the three-dimensionality
of space; to each point three numbers, $x_{1}$,~$x_{2}$,~$x_{3}$ (co-ordinates),
may be associated, in such a way that this
association is uniquely reciprocal, and that $x_{1}$,~$x_{2}$\Change{,}{} and~$x_{3}$
vary continuously when the point describes a continuous
series of points (a line).

It is assumed in pre-relativity physics that the laws of
the orientation of ideal rigid bodies are consistent with
Euclidean geometry. What this means may be expressed
\index{Euclidean geometry}%
\index{Geometry, Euclidean}%
as follows: Two points marked on a rigid body form
an \emph{interval}. Such an interval can be oriented at rest,
relatively to our space of reference, in a multiplicity of
ways. If, now, the points of this space can be referred
to co-ordinates $x_{1}$,~$x_{2}$,~$x_{3}$, in such a way that the differences
of the co-ordinates, $\Delta x_{1}$,~$\Delta x_{2}$,~$\Delta x_{3}$, of the two ends of the
interval furnish the same sum of squares,
\[
s^{2} = {\Delta x_{1}}^{2} + {\Delta x_{2}}^{2} + {\Delta x_{3}}^{2}\Add{,}
\Tag{(1)}
\]
\PageSep{5}
for every orientation of the interval, then the space of
reference is called Euclidean, and the co-ordinates
Cartesian.\footnote
  {This relation must hold for an arbitrary choice of the origin and of the
  direction (ratios $\Delta x_{1} : \Delta x_{2} : \Delta x_{3}$) of the interval.}
It is sufficient, indeed, to make this assumption
in the limit for an infinitely small interval. Involved
in this assumption there are some which are rather less
special, to which we must call attention on account of
their fundamental significance. In the first place, it is
assumed that one can move an ideal rigid body in an
arbitrary manner. In the second place, it is assumed
that the behaviour of ideal rigid bodies towards orientation
is independent of the material of the bodies and their
changes of position, in the sense that if two intervals can
once be brought into coincidence, they can always and
everywhere be brought into coincidence. Both of these
assumptions, which are of fundamental importance for
geometry and especially for physical measurements,
naturally arise from experience; in the theory of general
relativity their validity needs to be assumed only for
bodies and spaces of reference which are infinitely small
compared to astronomical dimensions.

The quantity~$s$ we call the length of the interval. In
order that this may be uniquely determined it is necessary
to fix arbitrarily the length of a definite interval; for
example, we can put it equal to~$1$ (unit of length). Then
the lengths of all other intervals may be determined. If
we make the~$x_{\nu}$ linearly dependent upon a parameter~$\lambda$,
\[
x_{\nu} = a_{\nu} + \lambda b_{\nu},
\]
\PageSep{6}
we obtain a line which has all the properties of the straight
lines of the Euclidean geometry. In particular, it easily
follows that by laying off $n$~times the interval~$s$ upon a
straight line, an interval of length~$n�s$ is obtained. A
length, therefore, means the result of a measurement
carried out along a straight line by means of a unit
measuring rod. It has a significance which is as independent
of the system of co-ordinates as that of a straight
line, as will appear in the sequel.

We come now to a train of thought which plays an
analogous r�le in the theories of special and general
relativity. We ask the question: besides the Cartesian
co-ordinates which we have used are there other equivalent
co-ordinates? An interval has a physical meaning which
is independent of the choice of co-ordinates; and so has
the spherical surface which we obtain as the locus of the
end points of all equal intervals that we lay off from an
arbitrary point of our space of reference. If~$x_{\nu}$ as well as~${x'}_{\nu}$
($\nu$~from $1$~to~$3$) are Cartesian co-ordinates of our space
of reference, then the spherical surface will be expressed
in our two systems of co-ordinates by the equations
\begin{align*}
\sum {\Delta x_{\nu}}^{2} &= \text{const.}
\Tag{(2)} \\
\sum {\Delta {x'}_{\nu}}^{2} &= \text{const.}
\Tag{(2a)}
\end{align*}
How must the~${x'}_{\nu}$ be expressed in terms of the~$x_{\nu}$ in order
that equations \Eqref{(2)}~and~\Eqref{(2a)} may be equivalent to each
other? Regarding the~${x'}_{\nu}$ expressed as functions of the~$x_{\nu}$,
we can write, by Taylor's theorem, for small values of
the~${\Delta x}_{\nu}$,
\PageSep{7}
\[
{\Delta x'}_{\nu}
  = \sum_{\alpha} \frac{\dd {x'}_{\nu}}{\dd x_{\alpha}}\, \Delta x_{\alpha}
  + \frac{1}{2} \sum_{\alpha\Add{,}\beta}
    \frac{\dd^{2} {x'}_{\nu}}{\dd x_{\alpha}\, \dd x_{\beta}}\,
    \Delta x_{\alpha}\, \Delta x_{\beta}\ldots\Add{.}
\]
If we substitute~\Eqref{(2a)} in this equation and compare with~\Eqref{(1)},
we see that the~${x'}_{\nu}$ must be linear functions of the~$x_{\nu}$.
If we therefore put
\begin{align*}
{x'}_{\nu} &= a_{\nu} + \sum_{\alpha} b_{\nu\alpha} x_{\alpha}\Add{,}
\Tag{(3)} \\
\intertext{or}
{\Delta x'}_{\nu} &= \sum_{\alpha} b_{\nu\alpha}\, \Delta x_{\alpha}\Add{,}
\Tag{(3a)}
\end{align*}
then the equivalence of equations \Eqref{(2)}~and~\Eqref{(2a)} is expressed
in the form
\[
\sum {{\Delta x'}_{\nu}}^{2} = \lambda \sum {\Delta x_{\nu}}^{2}\quad
\text{($\lambda$~independent of~$\Delta x_{\nu}$).}
\Tag{(2b)}
\]
It therefore follows that $\lambda$~must be a constant. If we put
$\lambda = 1$, \Eqref{(2b)}~and~\Eqref{(3a)} furnish the conditions
\[
\sum_{\nu} b_{\nu\alpha} b_{\nu\beta} = \delta_{\alpha\beta}\Add{,}
\Tag{(4)}
\]
in which $\delta_{\alpha\beta} = 1$, or $\delta_{\alpha\beta} = 0$, according as $\alpha = \beta$ or
$\alpha \neq \beta$. The conditions~\Eqref{(4)} are called the conditions of orthogonality,
\index{Conditions of orthogonality}%
\index{Linear orthogonal transformation}%
\index{Orthogonality, conditions of}%
\index{Orthogonal transformations, linear}%
\index{Transformation@{Transformation, Galilean}!Linear orthogonal}%
and the transformations \Eqref{(3)},~\Eqref{(4)}, linear orthogonal
transformations. If we stipulate that $s^{2} = \sum {\Delta x_{\nu}}^{2}$ shall be
equal to the square of the length in every system of
co-ordinates, and if we always measure with the same unit
scale, then $\lambda$~must be equal to~$1$. Therefore the linear
orthogonal transformations are the only ones by means of
which we can pass from one Cartesian system of co-ordinates
in our space of reference to another. We see
\PageSep{8}
that in applying such transformations the equations of
a straight line become equations of a straight line.
Reversing equations~\Eqref{(3a)} by multiplying both sides by~$b_{\nu\beta}$
and summing for all the~$\nu$'s, we obtain
\[
\Change{\sum}{\sum_{\nu}} b_{\nu\beta}\, {\Delta x'}_{\nu}
  = \sum_{\nu, \alpha} b_{\nu\alpha} b_{\nu\beta}\, \Delta x_{\alpha}
  = \sum_{\alpha} \delta_{\alpha\beta}\, \Delta x_{\alpha}
  = \Delta x_{\beta}\Add{.}
\Tag{(5)}
\]
The same coefficients,~$b$, also determine the inverse
substitution of~$\Delta x_{\nu}$. Geometrically, $b_{\nu\alpha}$~is the cosine of the
angle between the ${x'}_{\nu}$~axis and the $x_{\alpha}$~axis.

To sum up, we can say that in the Euclidean geometry
\index{Coordinates@{Co-ordinates, preferred systems of}}%
\index{Preferred systems of co-ordinates}%
\index{Systems of co-ordinates, preferred}%
there are (in a given space of reference) preferred systems
of co-ordinates, the Cartesian systems, which transform
into each other by linear orthogonal transformations.
The distance~$s$ between two points of our space of
reference, measured by a measuring rod, is expressed in
such co-ordinates in a particularly simple manner. The
whole of geometry may be founded upon this conception
of distance. In the present treatment, geometry is
related to actual things (rigid bodies), and its theorems
are statements concerning the behaviour of these things,
which may prove to be true or false.

One is ordinarily accustomed to study geometry
divorced from any relation between its concepts and
experience. There are advantages in isolating that
which is purely logical and independent of what is, in
principle, incomplete empiricism. This is satisfactory to
the pure mathematician. He is satisfied if he can deduce
his theorems from axioms correctly, that is, without
errors of logic. The question as to whether Euclidean
\PageSep{9}
geometry is true or not does not concern him. But for
our purpose it is necessary to associate the fundamental
concepts of geometry with natural objects; without such
an association geometry is worthless for the physicist.
The physicist is concerned with the question as to
whether the theorems of geometry are true or not. That
Euclidean geometry, from this point of view, affirms
something more than the mere deductions derived
logically from definitions may be seen from the following
simple consideration.

Between $n$~points of space there are $\dfrac{n(n - 1)}{2}$ distances,~$s_{\mu\nu}$;
between these and the $3n$~co-ordinates we have the
relations
\[
{s_{\mu\nu}}^{2}
  = \bigl(x_{1(\mu)} - x_{1(\nu)}\bigr)^{2}
  + \bigl(x_{2(\mu)} - x_{2(\nu)}\bigr)^{2}
  + \dots\Add{.}
\]

From these $\dfrac{n(n - 1)}{2}$~equations the $3n$~co-ordinates
may be eliminated, and from this elimination at least
$\dfrac{n(n - 1)}{2} - 3n$ equations in the~$s_{\mu\nu}$ will result.\footnote
  {In reality there are $\dfrac{n(n - 1)}{2} - 3n + 6$ equations.}
Since
the~$s_{\mu\nu}$ are measurable quantities, and by definition are
independent of each other, these relations between the~$s_{\mu\nu}$
are not necessary \textit{a~priori}.

From the foregoing it is evident that the equations of
transformation \Eqref{(3)},~\Eqref{(4)} have a fundamental significance in
Euclidean geometry, in that they govern the transformation
from one Cartesian system of co-ordinates to another.
The Cartesian systems of co-ordinates are characterized
\PageSep{10}
by the property that in them the measurable distance
between two points,~$s$, is expressed by the equation
\[
s^{2} = \sum {\Delta x_{\nu}}^{2}.
\]

If $K_{(x_{\nu})}$~and~$K'_{(x_{\nu})}$ are two Cartesian systems of co-ordinates,
then
\[
\sum {\Delta x_{\nu}}^{2} =  \sum {{\Delta x'}_{\nu}}^{2}.
\]

The right-hand side is identically equal to the left-hand
side on account of the equations of the linear orthogonal
transformation, and the right-hand side differs from the
left-hand side only in that the~$x_{\nu}$ are replaced by the~${x'}_{\nu}$.
This is expressed by the statement that~$\sum {\Delta x_{\nu}}^{2}$ is an
invariant with respect to linear orthogonal transformations.
\index{Invariant|etseq}%
It is evident that in the Euclidean geometry only
such, and all such, quantities have an objective significance,
independent of the particular choice of the Cartesian
co-ordinates, as can be expressed by an invariant with
respect to linear orthogonal transformations. This is the
reason that the theory of invariants, which has to do with
the laws that govern the form of invariants, is so important
for analytical geometry.

As a second example of a geometrical invariant, consider
a volume. This is expressed by
\[
V = \iiint dx_{1}\, dx_{2}\, dx_{3}.
\]
By means of Jacobi's theorem we may write
\[
\iiint {dx'}_{1}\, {dx'}_{2}\, {dx'}_{3}
  = \iiint \frac{\dd({x'}_{1}, {x'}_{2}, {x'}_{3})}{\dd(x_{1}, x_{2}, x_{3})}\,
    dx_{1}\, dx_{2}\, dx_{3}
\]
\PageSep{11}
where the integrand in the last integral is the functional
determinant of the~${x'}_{\nu}$ with respect to the~$x_{\nu}$, and this by~\Eqref{(3)}
is equal to the determinant~$|b_{\mu\nu}|$ of the coefficients
of substitution,~$b_{\nu\alpha}$. If we form the determinant of the~$\delta_{\mu\alpha}$
from equation~\Eqref{(4)}, we obtain, by means of the theorem
of multiplication of determinants,
\[
1 = |\delta_{\alpha\beta}|
  = \left| \sum_{\nu} b_{\nu\alpha} b_{\nu\beta}\right|
  = |b_{\mu\nu}|^{2};\quad
|b_{\mu\nu}| = �1\Add{.}
\Tag{(6)}
\]
If we limit ourselves to those transformations which have
the determinant~$+1$,\footnote
  {There are thus two kinds of Cartesian systems which are designated
  as ``right-handed'' and ``left-handed'' systems. The difference between
  these is familiar to every physicist and engineer. It is interesting to note
  that these two kinds of systems cannot be defined geometrically, but only
  the contrast between them.}
and only these arise from continuous
variations of the systems of co-ordinates, then $V$~is
an invariant.

Invariants, however, are not the only forms by means
of which we can give expression to the independence of
the particular choice of the Cartesian co-ordinates. Vectors
and tensors are other forms of expression. Let us express
the fact that the point with the current co-ordinates~$x_{\nu}$ lies
upon a straight line. We have
\[
x_{\nu} - A_{\nu} = \lambda B_{\nu}\quad
  \text{($\nu$~from $1$~to~$3$).}
\]
Without limiting the generality we can put
\[
\sum {B_{\nu}}^{2} = 1.
\]

If we multiply the equations by~$b_{\beta\nu}$ (compare \Eqref{(3a)} and~\Eqref{(5)})
and sum for all the~$\nu$'s, we get
\[
{x'}_{\beta} - {A'}_{\beta} = \lambda {B'}_{\beta}\Add{,}
\]
\PageSep{12}
where we have written
\[
{B'}_{\beta} = \sum_{\nu} b_{\beta\nu} B_{\nu}; \quad
{A'}_{\beta} = \sum_{\nu} b_{\beta\nu} A_{\nu}.
\]

These are the equations of straight lines with respect
to a second Cartesian system of co-ordinates~$K'$. They
have the same form as the equations with respect to the
original system of co-ordinates. It is therefore evident
that straight lines have a significance which is independent
of the system of co-ordinates. Formally, this depends
upon the fact that the quantities $(x_{\nu} - A_{\nu}) - \lambda B_{\nu}$ are
transformed as the components of an interval,~$\Delta x_{\nu}$. The
ensemble of three quantities, defined for every system of
Cartesian co-ordinates, and which transform as the components
of an interval, is called a vector. If the three
components of a vector vanish for one system of Cartesian
co-ordinates, they vanish for all systems, because the equations
of transformation are homogeneous. We can thus
get the meaning of the concept of a vector without referring
to a geometrical representation. This behaviour of the
equations of a straight line can be expressed by saying
that the equation of a straight line is co-variant with respect
\index{Covariant@{Co-variant}|etseq}%
to linear orthogonal transformations.

We shall now show briefly that there are geometrical
\index{Tensor|etseq}%
entities which lead to the concept of tensors. Let $P_{0}$ be
the centre of a surface of the second degree, $P$~any point
on the surface, and $\xi_{\nu}$~the projections of the interval~$P_{0}P$
upon the co-ordinate axes. Then the equation of the
surface is
\[
\sum a_{\mu\nu} \xi_{\mu} \xi_{\nu} = 1.
\]
\PageSep{13}
In this, and in analogous cases, we shall omit the sign of
summation, and understand that the summation is to be
carried out for those indices that appear twice. We thus
write the equation of the surface
\[
a_{\mu\nu} \xi_{\mu} \xi_{\nu} = 1.
\]
The quantities~$a_{\mu\nu}$ determine the surface completely, for
a given position of the centre, with respect to the chosen
system of Cartesian co-ordinates. From the known law
of transformation for the~$\xi_{\nu}$ \Eqref{(3a)} for linear orthogonal
transformations, we easily find the law of transformation
%[** TN: Footnote mark comes before the colon in the original]
for the~$a_{\mu\nu}$:\footnote
  {The equation ${a'}_{\sigma\tau} {\xi'}_{\sigma} {\xi'}_{\tau} = 1$ may, by~\Eqref{(5)}, be replaced by
  ${a'}_{\sigma\tau} b_{\mu\sigma} b_{\nu\tau} \xi_{\Change{\sigma}{\mu}} \xi_{\Change{\tau}{\nu}} = 1$,
  from which the result stated immediately follows.}
\[
{a'}_{\sigma\tau} = b_{\sigma\mu} b_{\tau\nu} a_{\mu\nu}.
\]
This transformation is homogeneous and of the first degree
in the~$a_{\mu\nu}$. On account of this transformation, the~$a_{\mu\nu}$
are called components of a tensor of the second rank (the
latter on account of the double index). If all the components,~$a_{\mu\nu}$,
of a tensor with respect to any system of
Cartesian co-ordinates vanish, they vanish with respect to
every other Cartesian system. The form and the position
of the surface of the second degree is described by this
tensor~$(a)$.

Analytic tensors of higher rank (number of indices)
may be defined. It is possible and advantageous to
regard vectors as tensors of rank~$1$, and invariants (scalars)
as tensors of rank~$0$. In this respect, the problem of the
theory of invariants may be so formulated: according to
what laws may new tensors be formed from given tensors?
\PageSep{14}
\index{Operations on tensors|etseq}%
\index{Tensor!operations|etseq}%
We shall consider these laws now, in order to be able to
apply them later. We shall deal first only with the
properties of tensors with respect to the transformation
from one Cartesian system to another in the same space
of reference, by means of linear orthogonal transformations.
As the laws are wholly independent of the number
of dimensions, we shall leave this number,~$n$, indefinite at
first.

\Paragraph{Definition.} If a figure is defined with respect to every
system of Cartesian co-ordinates in a space of reference of
$n$~dimensions by the $n^{\alpha}$~numbers $A_{\mu\nu\rho\Subdots}$ ($\alpha = \text{number
of indices}$), then these numbers are the components of a
tensor of rank~$\alpha$ if the transformation law is
\index{Rank of tensor}%
\index{Tensor!Rank of}%
\[
{A'}_{\mu'\nu'\rho'\Subdots}
  = b_{\mu'\mu} b_{\nu'\nu} b_{\rho'\rho} \dots A_{\mu\nu\rho\Subdots}\Add{.}
\Tag{(7)}
\]

\Paragraph{Remark.} From this definition it follows that
\[
A_{\mu\nu\rho\Subdots} = B_{\mu} C_{\nu} D_{\rho}\dots
\Tag{(8)}
\]
is an invariant, provided that $(B)$,~$(C)$,~$(D)$~\dots\ are
vectors. Conversely, the tensor character of~$(A)$ may be
inferred, if it is known that the expression~\Eqref{(8)} leads to an
invariant for an arbitrary choice of the vectors $(B)$,~$(C)$,~etc.

\Paragraph{Addition and Subtraction.} By addition and subtraction
\index{Addition@{Addition and subtraction of tensors}}%
\index{Tensor!Addition and subtraction of}%
of the corresponding components of tensors of the same
rank, a tensor of equal rank results:
\[
A_{\mu\nu\rho\Subdots} � B_{\mu\nu\rho\Subdots} = C_{\mu\nu\rho\Subdots}.
\Tag{(9)}
\]
The proof follows from the definition of a tensor given
above.

\Paragraph{Multiplication.} From a tensor of rank~$\alpha$ and a tensor
\index{Multiplication of tensors}%
\index{Tensor!Multiplication of}%
\PageSep{15}
of rank~$\beta$ we may obtain a tensor of rank~$\alpha + \beta$ by
multiplying all the components of the first tensor by all
the components of the second tensor:
\[
T_{\mu\nu\rho\Subdots\alpha\beta\Subdots}
= A_{\mu\nu\rho\Subdots} B_{\alpha\beta\gamma\Subdots}\Add{.}
\Tag{(10)}
\]

\Paragraph{Contraction.} A tensor of rank~$\alpha - 2$ may be obtained
\index{Contraction of tensors}%
\index{Tensor!Contraction of}%
from one of rank~$\alpha$ by putting two definite indices equal
to each other and then summing for this single index:
\[
T_{\rho\Subdots}
  = A_{\mu\mu\rho\Subdots}
  ( = \sum_{\mu} A_{\mu\mu\rho\Subdots})\Add{.}
\Tag{(11)}
\]
The proof is
\begin{align*}
{A'}_{\mu\mu\rho\Subdots}
  = b_{\mu\alpha} b_{\mu\beta} b_{\rho\gamma}\dots A_{\alpha\beta\gamma\Subdots}
  = \delta_{\alpha\beta} & b_{\rho\gamma}\dots A_{\alpha\beta\gamma\Subdots} \\
  {} ={} & b_{\rho\gamma}\dots A_{\alpha\alpha\gamma\Subdots}\Add{.}
\end{align*}

In addition to these elementary rules of operation
there is also the formation of tensors by differentiation
(``erweiterung''):
\[
T_{\mu\nu\rho\Subdots\alpha}
  = \frac{\dd A_{\mu\nu\rho\Subdots}}{\dd x_{\alpha}}\Add{.}
\Tag{(12)}
\]

New tensors, in respect to linear orthogonal transformations,
may be formed from tensors according to these rules
of operation.

\Paragraph{Symmetrical Properties of Tensors.} Tensors are called
\index{Skew-symmetrical tensor}%
\index{Symmetrical tensor}%
\index{Tensor!Symmetrical and Skew-symmetrical}%
symmetrical or skew-symmetrical in respect to two of
their indices, $\mu$~and~$\nu$, if both the components which result
from interchanging the indices $\mu$~and~$\nu$ are equal to each
other or equal with opposite signs.
\begin{alignat*}{2}
&\text{Condition for symmetry:} &
  A_{\mu\nu\rho} &= A_{\Change{\mu\nu}{\nu\mu}\rho}. \\
&\text{Condition for skew-symmetry:}\quad &
  A_{\mu\nu\rho} &= -A_{\nu\mu\rho}. \\
\end{alignat*}

\Paragraph{Theorem.} The character of symmetry or skew-symmetry
exists independently of the choice of co-ordinates, and in
\PageSep{16}
this lies its importance. The proof follows from the
equation defining tensors.

\Subsection{Special Tensors.}

I\@. The quantities~$\delta_{\rho\sigma}$ \Eqref{(4)} are tensor components (fundamental
tensor).

\Paragraph{Proof.} If in the right-hand side of the equation of
transformation ${A'}_{\mu\nu} = b_{\mu\alpha} b_{\nu\beta} A_{\alpha\beta}$, we substitute for~$A_{\alpha\beta}$ the
quantities~$\delta_{\alpha\beta}$ (which are equal to $1$~or~$0$ according as
$\alpha = \beta$ or $\alpha \Change{ }{\neq} \beta$), we get
\[
{A'}_{\mu\nu} = b_{\mu\alpha} b_{\nu\alpha} = \delta_{\mu\nu}.
\]
The justification for the last sign of equality becomes
evident if one applies~\Eqref{(4)} to the inverse substitution~\Eqref{(5)}.

II\@. There is a tensor $(\delta_{\mu\nu\rho\Subdots})$ skew-symmetrical with
respect to all pairs of indices, whose rank is equal to the
number of dimensions,~$n$, and whose components are
equal to $+1$~or~$-1$ according as $\mu\,\nu\,\rho\dots$~is an even
or odd permutation of~$1\,2\,3\dots$\Add{.}

The proof follows with the aid of the theorem proved
above $|b_{\rho\sigma}| = 1$.

These few simple theorems form the apparatus from
the theory of invariants for building the equations of pre-relativity
physics and the theory of special relativity.

We have seen that in pre-relativity physics, in order to
specify relations in space, a body of reference, or a space
of reference, is required, and, in addition, a Cartesian
system of co-ordinates. We can fuse both these concepts
into a single one by thinking of a Cartesian system of
co-ordinates as a cubical frame-work formed of rods each
of unit length. The co-ordinates of the lattice points of
\PageSep{17}
this frame are integral numbers. It follows from the
fundamental relation
\[
s^{2} = {\Delta x_{1}}^{2} + {\Delta x_{2}}^{2} + {\Delta x_{3}}^{2}
\Change{}{\Tag{(13)}}
\]
that the members of such a space-lattice are all of unit
length. To specify relations in time, we require in
addition a standard clock placed at the origin of our
Cartesian system of co-ordinates or frame of reference.
If an event takes place anywhere we can assign to it three
co-ordinates,~$x_{\nu}$, and a time~$t$, as soon as we have specified
the time of the clock at the origin which is simultaneous
with the event. We therefore give an objective significance
to the statement of the simultaneity of distant
\index{Simultaneity}%
events, while previously we have been concerned only
with the simultaneity of two experiences of an individual.
The time so specified is at all events independent of the
position of the system of co-ordinates in our space of
reference, and is therefore an invariant with respect to
the transformation~\Eqref{(3)}.

It is postulated that the system of equations expressing
the laws of pre-relativity physics is co-variant with respect
to the transformation~\Eqref{(3)}, as are the relations of Euclidean
geometry. The isotropy and homogeneity of space is
\index{Homogeneity of space}%
\index{Isotropy of space}%
\index{Space@{Space, Concept of}!Homogeneity of}%
\index{Space@{Space, Concept of}!Isotropy of}%
expressed in this way.\footnote
  {The laws of physics could be expressed, even in case there were a
  unique direction in space, in such a way as to be co-variant with respect to
  the transformation~\Eqref{(3)}; but such an expression would in this case be unsuitable.
  If there were a unique direction in space it would simplify the
  description of natural phenomena to orient the system of co-ordinates in a
  definite way in this direction. But if, on the other hand, there is no unique
  direction in space it is not logical to formulate the laws of nature in such
  a way as to conceal the equivalence of systems of co-ordinates that are
  oriented differently. We shall meet with this point of view again in the
  theories of special and general relativity.}
We shall now consider some of
\PageSep{18}
the more important equations of physics from this point
of view.

The equations of motion of a material particle are
\[
m \frac{d^{2} x_{\nu}}{dt^{2}} = X_{\nu}\Add{;}
\Tag{(14)}
\]
$(dx_{\nu})$~is a vector; $dt$,~and therefore also~$\dfrac{1}{dt}$, an invariant;
thus $\left(\dfrac{dx_{\nu}}{dt}\right)$~is a vector; in the same way it may be shown
that $\left(\dfrac{d^{2} x_{\nu}}{dt^{2}}\right)$~is a vector. In general, the operation of differentiation
with respect to time does not alter the tensor
character. Since $m$~is an invariant (tensor of rank~$0$),
$\left(m\dfrac{d^{2} x_{\nu}}{dt^{2}}\right)$~is a vector, or tensor of rank~$1$ (by the theorem
of the multiplication of tensors). If the force~$(X_{\nu})$ has
a vector character, the same holds for the difference
$\left(m\dfrac{d^{2} x_{\nu}}{dt^{2}} - X_{\nu}\right)$. These equations of motion are therefore
valid in every other system of Cartesian co-ordinates in
the space of reference. In the case where the forces are
conservative we can easily recognize the vector character
of~$(X_{\nu})$. For a potential energy,~$\Phi$, exists, which depends
only upon the mutual distances of the particles, and is
therefore an invariant. The vector character of the force,
$X_{\nu} = -\dfrac{\dd \Phi}{\dd x_{\nu}}$, is then a consequence of our general theorem
about the derivative of a tensor of rank~$0$.
\PageSep{19}

Multiplying by the velocity, a tensor of rank~$1$, we
obtain the tensor equation
\[
\left(m\frac{d^{2} x_{\nu}}{dt^{2}} - X_{\nu}\right) \frac{dx_{\nu}}{dt} = 0.
\]
By contraction and multiplication by the scalar~$dt$ we
obtain the equation of kinetic energy
\[
d\left(\frac{mq^{2}}{2}\right) = X_{\nu}\, dx_{\nu}.
\]

If $\xi_{\nu}$~denotes the difference of the co-ordinates of
the material particle and a point fixed in space, then
the~$\xi_{\nu}$ have the character of vectors. We evidently
have $\dfrac{d^{2} x_{\nu}}{dt^{2}} = \dfrac{d^{2} \xi_{\nu}}{dt^{2}}$, so that the equations of motion of the
particle may be written
\[
m\frac{d^{2} \xi_{\nu}}{dt^{2}} - X_{\nu} = 0.
\]

Multiplying this equation by~$\xi_{\mu}$ we obtain a tensor
equation
\[
\left(m\frac{d^{2} \xi_{\nu}}{dt^{2}} - X_{\nu}\right) \xi_{\mu} = 0.
\]

Contracting the tensor on the left and taking the time
average we obtain the virial theorem, which we shall
not consider further. By interchanging the indices and
subsequent subtraction, we obtain, after a simple transformation,
the theorem of moments,
\[
\frac{d}{dt} \left[
  m \left(\xi_{\mu} \frac{d\xi_{\nu}}{dt} - \xi_{\nu} \frac{d\xi_{\mu}}{dt}\right)
\right] = \xi_{\mu} X_{\nu} - \xi_{\nu} X_{\mu}\Add{.}
\Tag{(15)}
\]

It is evident in this way that the moment of a vector
\PageSep{20}
is not a vector but a tensor. On account of their skew-symmetrical
character there are not nine, but only three
independent equations of this system. The possibility of
replacing skew-symmetrical tensors of the second rank in
space of three dimensions by vectors depends upon the
formation of the vector
\[
A_{\mu} = \frac{1}{2} A_{\sigma\tau} \delta_{\sigma\tau\mu}.
\]

If we multiply the skew-symmetrical tensor of rank~$2$
by the special skew-symmetrical tensor~$\delta$ introduced
above, and contract twice, a vector results whose components
are numerically equal to those of the tensor. These
are the so-called axial vectors which transform differently,
from a right-handed system to a left-handed system,
from the~$\Delta x_{\nu}$. There is a gain in picturesqueness in
regarding a skew-symmetrical tensor of rank~$2$ as a vector
in space of three dimensions, but it does not represent
the exact nature of the corresponding quantity so well as
considering it a tensor.

We consider next the equations of motion of a continuous
medium. Let $\rho$~be the density, $u_{\nu}$~the velocity
components considered as functions of the co-ordinates and
the time, $X_{\nu}$~the volume forces per unit of mass, and $p_{\nu\sigma}$~the
stresses upon a surface perpendicular to the $\sigma$-axis
in the direction of increasing~$x_{\nu}$. Then the equations of
motion are, by Newton's law,
\[
\rho \frac{du_{\nu}}{dt}
  = -\frac{\dd p_{\nu\sigma}}{\dd x_{\sigma}} + \rho X_{\nu}\Add{,}
\]
in which $\dfrac{du_{\nu}}{dt}$~is the acceleration of the particle which at
\PageSep{21}
time~$t$ has the co-ordinates~$x_{\nu}$. If we express this
acceleration by partial differential coefficients, we obtain,
after dividing by~$\rho$,
\[
\frac{\dd u_{\nu}}{dt} + \frac{\dd u_{\nu}}{dx_{\sigma}} u_{\sigma}
  = -\frac{1}{\rho}\, \frac{\dd p_{\nu\sigma}}{\dd x_{\sigma}} + X_{\nu}\Add{.}
\Tag{(16)}
\]

We must show that this equation holds independently
of the special choice of the Cartesian system of co-ordinates.
$(u_{\nu})$ is a vector, and therefore $\dfrac{\dd u_{\nu}}{\dd t}$~is also a vector. $\dfrac{\dd u_{\nu}}{\dd x_{\sigma}}$~is
a tensor of rank~$2$, $\dfrac{\dd u_{\nu}}{\dd x_{\sigma}} u_{\tau}$~is a tensor of rank~$3$. The second
term on the left results from contraction in the indices
$\sigma$,~$\tau$. The vector character of the second term on the right
is obvious. In order that the first term on the right may
also be a vector it is necessary for~$p_{\nu\sigma}$ to be a tensor.
Then by differentiation and contraction $\dfrac{\dd p_{\nu\sigma}}{\dd x_{\sigma}}$~results, and
is therefore a vector, as it also is after multiplication by
the reciprocal scalar~$\dfrac{1}{\rho}$. That $p_{\nu\sigma}$~is a tensor, and therefore
transforms according to the equation
\[
{p'}_{\mu\nu} = b_{\mu\alpha} b_{\nu\beta} p_{\alpha\beta},
\]
is proved in mechanics by integrating this equation over
an infinitely small tetrahedron. It is also proved there,
by application of the theorem of moments to an infinitely
small parallelopipedon, that $p_{\nu\sigma} = p_{\sigma\nu}$, and hence that the
tensor of the stress is a symmetrical tensor. From what
has been said it follows that, with the aid of the rules
\PageSep{22}
given above, the equation is co-variant with respect to
orthogonal transformations in space (rotational transformations);
and the rules according to which the
quantities in the equation must be transformed in order
that the equation may be co-variant also become evident.

The co-variance of the equation of continuity,
\index{Covariance@{Co-variance of equation of continuity}}%
\index{Equation of continuity, co-variance of}%
\[
\frac{\dd\rho}{\dd t} + \frac{\dd(\rho u_{\nu})}{\dd x_{\nu}} = 0\Add{,}
\Tag{(17)}
\]
requires, from the foregoing, no particular discussion.

We shall also test for co-variance the equations which
express the dependence of the stress components upon
the properties of the matter, and set up these equations
for the case of a compressible viscous fluid with the aid
\index{Compressible viscous fluid}%
\index{Viscous compressible fluid}%
of the conditions of co-variance. If we neglect the viscosity,
the pressure,~$p$, will be a scalar, and will depend
only upon the density and the temperature of the fluid.
The contribution to the stress tensor is then evidently
\index{Stress tensor}%
\[
p \delta_{\mu\nu}
\]
in which $\delta_{\mu\nu}$~is the special symmetrical tensor. This term
will also be present in the case of a viscous fluid. But in
this case there will also be pressure terms, which depend
upon the space derivatives of the~$u_{\nu}$. We shall assume
that this dependence is a linear one. Since these terms
must be symmetrical tensors, the only ones which enter
will be
\[
\alpha\left(\frac{\dd u_{\mu}}{\dd x_{\nu}}
          + \frac{\dd u_{\nu}}{\dd x_{\mu}}\right)
  + \beta\delta_{\mu\nu} \frac{\dd u_{\alpha}}{\dd x_{\alpha}}
\]
(for $\dfrac{\dd u_{\alpha}}{\dd x_{\alpha}}$~is a scalar). For physical reasons (no slipping)
\PageSep{23}
it is assumed that for symmetrical dilatations in all
directions, i.e.~when
\[
\frac{\dd u_{1}}{\dd x_{1}} =
\frac{\dd u_{2}}{\dd x_{2}} =
\frac{\dd u_{3}}{\dd x_{3}};\quad
\frac{\dd u_{1}}{\dd x_{2}}, \text{ etc.,} = 0,
\]
there are no frictional forces present, from which it
follows that $\beta = -\dfrac{2}{3}\alpha$. If only $\dfrac{\dd u_{1}}{\dd x_{\Change{2}{3}}}$~is different from
zero, let $p_{31} = -\Change{\eta}{\alpha} \dfrac{\dd u_{1}}{\dd x_{3}}$, by which $\alpha$~is determined. We
then obtain for the complete stress tensor,
\[
p_{\mu\nu} = p \delta_{\mu\nu} - \Change{\eta}{\alpha} \biggl[
  \biggl(\frac{\dd u_{\mu}}{\dd x_{\nu}}
      + \frac{\dd u_{\nu}}{\dd x_{\mu}}\biggr)
  - \frac{2}{3} \biggl(\frac{\dd u_{1}}{\dd x_{1}}
                    + \frac{\dd u_{2}}{\dd x_{2}}
                    + \frac{\dd u_{3}}{\dd x_{3}}\biggr)\delta_{\mu\nu}
\biggr]\Add{.}
\Tag{(18)}
\]

The heuristic value of the theory of invariants, which
arises from the isotropy of space (equivalence of all
directions), becomes evident from this example.

We consider, finally, Maxwell's equations in the form
\index{Maxwell's equations}%
which are the foundation of the electron theory of Lorentz.
\begin{align*}
\left.
  \begin{aligned}
    &\begin{alignedat}{3}
      \frac{\dd h_{3}}{\dd x_{2}} &- \frac{\dd h_{2}}{\dd x_{3}}
      &&= \frac{1}{c}\, \frac{\dd e_{1}}{\dd t}
      &&+ \frac{1}{c}\, i_{1}\Add{,} \\
%
      \frac{\dd h_{1}}{\dd x_{3}} &- \frac{\dd h_{3}}{\dd x_{1}}
      &&= \frac{1}{c}\, \frac{\dd e_{2}}{\dd t}
      &&+ \frac{1}{c}\, i_{2}\Add{,} \\
%
% [** TN: Adding code for third equation (elided in original)]
      \frac{\dd h_{2}}{\dd x_{1}} &- \frac{\dd h_{1}}{\dd x_{2}}
      &&= \frac{1}{c}\, \frac{\dd e_{3}}{\dd t}
      &&+ \frac{1}{c}\, i_{3}\Add{,}
    \end{alignedat} \\
    & \frac{\dd e_{1}}{\dd x_{1}}
    + \frac{\dd e_{2}}{\dd x_{2}}
    + \frac{\dd e_{3}}{\dd x_{3}} = \rho\Add{;}
  \end{aligned}
\right\}
\Tag{(19)} \\[8pt]
\left.
  \begin{aligned}
    &\begin{alignedat}{2}
      \frac{\dd e_{3}}{\dd x_{2}} &- \frac{\dd e_{2}}{\dd x_{3}}
      &&= -\frac{1}{c}\, \frac{\dd h_{1}}{\dd t}\Add{,}\qquad \\
%
      \frac{\dd e_{1}}{\dd x_{3}} &- \frac{\dd e_{3}}{\dd x_{1}}
      &&= -\frac{1}{c}\, \frac{\dd h_{2}}{\dd t}\Add{,} \\
%
% [** TN: Adding code for third equation (elided in original)]
      \frac{\dd e_{2}}{\dd x_{1}} &- \frac{\dd e_{1}}{\dd x_{2}}
      &&= -\frac{1}{c}\, \frac{\dd h_{3}}{\dd t}\Add{,}
    \end{alignedat} \\
    & \frac{\dd h_{1}}{\dd x_{1}}
    + \frac{\dd h_{2}}{\dd x_{2}}
    + \frac{\dd h_{3}}{\dd x_{3}} = 0\Add{.}
  \end{aligned}
\right\}
\Tag{(20)}
\end{align*}
\PageSep{24}

$\veci$~is a vector, because the current density is defined as
the density of electricity multiplied by the vector velocity
of the electricity. According to the first three equations
it is evident that $\e$~is also to be regarded as a vector.
Then $\h$~cannot be regarded as a vector.\footnote
  {These considerations will make the reader familiar with tensor operations
  without the special difficulties of the four-dimensional treatment;
  corresponding considerations in the theory of special relativity (Minkowski's
  interpretation of the field) will then offer fewer difficulties.}
The equations
may, however, easily be interpreted if $\h$~is regarded as a
\Change{}{skew-}symmetrical tensor of the second rank. In this sense, we
write $h_{23}$,~$h_{31}$,~$h_{12}$, in place of $h_{1}$,~$h_{2}$,~$h_{3}$ respectively. Paying
attention to the skew-symmetry of~$h_{\mu\nu}$, the first three
equations of \Eqref{(19)}~and~\Eqref{(20)} may be written in the form
\begin{gather*}
\frac{\dd h_{\mu\nu}}{\dd x_{\nu}}
  = \frac{1}{c}\, \frac{\dd e_{\mu}}{\dd t} + \frac{1}{c} i_{\mu}\Add{,}
\Tag{(19a)} \\
\frac{\dd e_{\mu}}{\dd x_{\nu}} - \frac{\dd e_{\nu}}{\dd x_{\mu}}
  = +\frac{1}{c}\, \frac{\dd h_{\mu\nu}}{\dd t}\Add{.}
\Tag{(20a)}
\end{gather*}
In contrast to~$\e$, $\h$~appears as a quantity which has the
same type of symmetry as an angular velocity. The
divergence equations then take the form
\begin{gather*}
\frac{\dd e_{\nu}}{\dd x_{\nu}} = \rho\Add{,}
\Tag{(19b)} \\
\frac{\dd h_{\mu\nu}}{\dd x_{\rho}} +
\frac{\dd h_{\nu\rho}}{\dd x_{\mu}} +
\frac{\dd h_{\rho\mu}}{\dd x_{\nu}} = 0\Add{.}
\Tag{(20b)}
\end{gather*}
The last equation is a skew-symmetrical tensor equation
of the third rank (the skew-symmetry of the left-hand
side with respect to every pair of indices may easily be
\PageSep{25}
proved, if attention is paid to the skew-symmetry of~$h_{\mu\nu}$).
This notation is more natural than the usual one, because,
in contrast to the latter, it is applicable to Cartesian left-handed
systems as well as to right-handed systems without
change of sign.
\PageSep{26}


\Lecture{II}{The Theory of Special Relativity}{Special Relativity}

\First{The} previous considerations concerning the configuration
of rigid bodies have been founded, irrespective
of the assumption as to the validity of the Euclidean
geometry, upon the hypothesis that all directions in space,
or all configurations of Cartesian systems of co-ordinates,
are physically equivalent. We may express this as the
\index{Equivalent spaces of reference}%
``principle of relativity with respect to direction,'' and it
has been shown how equations (laws of nature) may be
found, in accord with this principle, by the aid of the
calculus of tensors. We now inquire whether there is a
relativity with respect to the state of motion of the space
of reference; in other words, whether there are spaces of
reference in motion relatively to each other which are
physically equivalent. From the standpoint of mechanics
it appears that equivalent spaces of reference do exist.
\index{Spaces of reference!equivalence of}%
For experiments upon the earth tell us nothing of the
fact that we are moving about the sun with a velocity of
approximately $30$~kilometres a second. On the other
hand, this physical equivalence does not seem to hold for
spaces of reference in arbitrary motion; for mechanical
effects do not seem to be subject to the same laws in a
jolting railway train as in one moving with uniform
\PageSep{27}
velocity; the rotation of the earth must be considered in
writing down the equations of motion relatively to the
earth. It appears, therefore, as if there were Cartesian
systems of co-ordinates, the so-called inertial systems, with
reference to which the laws of mechanics (more generally
the laws of physics) are expressed in the simplest form.
We may infer the validity of the following theorem: If
$K$~is an inertial system, then every other system~$K'$ which
moves uniformly and without rotation relatively to~$K$, is
also an inertial system; the laws of nature are in concordance
for all inertial systems. This statement we shall
call the ``principle of special relativity.'' We shall draw
certain conclusions from this principle of ``relativity of
translation'' just as we have already done for relativity of
direction.

In order to be able to do this, we must first solve the
following problem. If we are given the Cartesian co-ordinates,~$x_{\nu}$,
and the time,~$t$, of an event relatively to one
inertial system,~$K$, how can we calculate the co-ordinates,~${x'}_{\nu}$,
and the time,~$t'$, of the same event relatively to an
inertial system~$K'$ which moves with uniform translation
relatively to~$K$? In the pre-relativity physics
\index{Prerelativity@{Pre-relativity physics, hypotheses of}}%
this problem was solved by making unconsciously two
hypotheses:---

1.~The time is absolute; the time of an event,~$t'$,
relatively to~$K'$ is the same as the time relatively to~$K$.
If instantaneous signals could be sent to a distance, and
if one knew that the state of motion of a clock had no
influence on its rate, then this assumption would be
physically established. For then clocks, similar to one
\PageSep{28}
another, and regulated alike, could be distributed over
the systems $K$~and~$K'$, at rest relatively to them, and
their indications would be independent of the state of
motion of the systems; the time of an event would then
be given by the clock in its immediate neighbourhood.

2.~Length is absolute; if an interval, at rest relatively
to~$K$, has a length~$s$, then it has the same length~$s$
relatively to a system~$K'$ which is in motion relatively
to~$K$.

If the axes of $K$~and~$K'$ are parallel to each other, a
simple calculation based on these two assumptions, gives
the equations of transformation
\[
\left.
\begin{aligned}
{x'}_{\nu} &= x_{\nu} - a_{\nu} - b_{\nu}t\Add{,} \\
t' &= t - b\Add{.}
\end{aligned}
\right\}
\Tag{(21)}
\]

This transformation is known as the ``Galilean Transformation.''
\index{Galilean@{Galilean regions}!transformation}%
\index{Transformation@{Transformation, Galilean}}%
Differentiating twice by the time, we get
\[
\frac{d^{2} {x'}_{\nu}}{dt^{2}} = \frac{d^{2} x_{\nu}}{dt^{2}}.
\]
Further, it follows that for two simultaneous events,
\[
{{x'}_{\nu}}^{(1)} - {{x'}_{\nu}}^{(2)} = {x_{\nu}}^{(1)} - {x_{\nu}}^{(2)}.
\]
The invariance of the distance between the two points
results from squaring and adding. From this easily
follows the co-variance of Newton's equations of motion
with respect to the Galilean transformation~\Eqref{(21)}. Hence
it follows that classical mechanics is in accord with the
principle of special relativity if the two hypotheses
respecting scales and clocks are made.

But this attempt to found relativity of translation upon
the Galilean transformation fails when applied to electromagnetic
\PageSep{29}
phenomena. The Maxwell-Lorentz electromagnetic
equations are not co-variant with respect to the
Galilean transformation. In particular, we note, by~\Eqref{(21)},
that a ray of light which referred to~$K$ has a velocity~$c$,
has a different velocity referred to~$K'$, depending upon
its direction. The space of reference of~$K$ is therefore
distinguished, with respect to its physical properties, from
all spaces of reference which are in motion relatively to it
(quiescent �ther). But all experiments have shown that
electro\Change{-}{}magnetic\TNote{** TN: Only hyphenated instance} and optical phenomena, relatively to the
earth as the body of reference, are not influenced by the
translational velocity of the earth. The most important
of these experiments are those of Michelson and Morley,
\index{Michelson and Morley}%
which I shall assume are known. The validity of the
principle of special relativity can therefore hardly be
doubted.

On the other hand, the Maxwell-Lorentz equations
have proved their validity in the treatment of optical
problems in moving bodies. No other theory has
satisfactorily explained the facts of aberration, the
propagation of light in moving bodies (Fizeau), and
\index{Fizeau}%
phenomena observed in double stars (De~Sitter). The
\index{Sitter}%
consequence of the Maxwell-Lorentz equations that in a
vacuum light is propagated with the velocity~$c$, at least
with respect to a definite inertial system~$K$, must therefore
be regarded as proved. According to the principle
of special relativity, we must also assume the truth of
this principle for every other inertial system.

Before we draw any conclusions from these two
principles we must first review the physical significance
\PageSep{30}
\index{Concept@{Concept of space}!time}%
\index{Time-concept}%
of the concepts ``time'' and ``velocity.'' It follows from
what has gone before, that co-ordinates with respect to
an inertial system are physically defined by means of
measurements and constructions with the aid of rigid
bodies. In order to measure time, we have supposed a
clock,~$U$, present somewhere, at rest relatively to~$K$. But
we cannot fix the time, by means of this clock, of an event
whose distance from the clock is not negligible; for there
are no ``instantaneous signals'' that we can use in order
to compare the time of the event with that of the clock.
In order to complete the definition of time we may
employ the principle of the constancy of the velocity of
light in a vacuum. Let us suppose that we place similar
clocks at points of the system~$K$, at rest relatively to it,
and regulated according to the following scheme. A ray
of light is sent out from one of the clocks,~$U_{m}$, at the
instant when it indicates the time~$t_{m}$, and travels through
a vacuum a distance~$r_{mn}$, to the clock~$U_{n}$; at the instant
when this ray meets the clock~$U_{n}$ the latter is set to
indicate the time $t_{n} = t_{m} + \dfrac{r_{mn}}{c}$.\footnote
  {Strictly speaking, it would be more correct to define simultaneity first,
\index{Simultaneity}%
  somewhat as follows: two events taking place at the points $A$~and~$B$ of
  the system~$K$ are simultaneous if they appear at the same instant when
  observed from the middle point,~$M$, of the interval~$AB$. Time is then
  defined as the ensemble of the indications of similar clocks, at rest
  relatively to~$K$, which register the same simultaneously.}
The principle of the
constancy of the velocity of light then states that this
adjustment of the clocks will not lead to contradictions.
With clocks so adjusted, we can assign the time to events
which take place near any one of them. It is essential to
\PageSep{31}
note that this definition of time relates only to the inertial
system~$K$, since we have used a system of clocks at rest
relatively to~$K$. The assumption which was made in the
pre-relativity physics of the absolute character of time
(i.e.~the independence of time of the choice of the inertial
system) does not follow at all from this definition.

The theory of relativity is often criticized for giving,
\index{Criticisms of theory of relativity}%
\index{Theory of relativity, criticisms of}%
without justification, a central theoretical r�le to the
propagation of light, in that it founds the concept of time
upon the law of propagation of light. The situation,
however, is somewhat as follows. In order to give
physical significance to the concept of time, processes of
some kind are required which enable relations to be
established between different places. It is immaterial
what kind of processes one chooses for such a definition
of time. It is advantageous, however, for the theory,
to choose only those processes concerning which we know
something certain. This holds for the propagation of
light \textit{in vacuo} in a higher degree than for any other process
which could be considered, thanks to the investigations
of Maxwell and H.~A.~Lorentz.

From all of these considerations, space and time data
have a physically real, and not a mere fictitious, significance;
in particular this holds for all the relations in
which co-ordinates and time enter, e.g.~the relations~\Eqref{(21)}.
There is, therefore, sense in asking whether those
equations are true or not, as well as in asking what the
true equations of transformation are by which we pass
from one inertial system~$K$ to another,~$K'$, moving
relatively to it. It may be shown that this is uniquely
\PageSep{32}
settled by means of the principle of the constancy of the
velocity of light and the principle of special relativity.

To this end we think of space and time physically
defined with respect to two inertial systems, $K$~and~$K'$, in
the way that has been shown. Further, let a ray of light
pass from one point~$P_{1}$ to another point~$P_{2}$ of~$K$ through
a vacuum. If $r$~is the measured distance between the two
points, then the propagation of light must satisfy the
equation
\[
r = c�\Delta t.
\]

If we square this equation, and express~$r^{2}$ by the
differences of the co-ordinates,~$\Delta x_{\nu}$, in place of this equation
we can write
\[
\sum (\Delta x_{\nu})^{2} - c^{2}\, \Delta t^{2} = 0\Add{.}
\Tag{(22)}
\]
This equation formulates the principle of the constancy
of the velocity of light relatively to~$K$. It must hold
whatever may be the motion of the source which emits
the ray of light.

The same propagation of light may also be considered
relatively to~$K'$, in which case also the principle of the
constancy of the velocity of light must be satisfied.
Therefore, with respect to~$K'$, we have the equation
\[
\sum ({\Delta x'}_{\nu})^{2} - c^{2}\, \Delta t'^{2} = 0\Add{.}
\Tag{(22a)}
\]

Equations \Eqref{(22a)}~and~\Eqref{(22)} must be mutually consistent
with each other with respect to the transformation which
transforms from~$K$ to~$K'$. A transformation which effects
this we shall call a ``Lorentz transformation.''
\index{Lorentz@{Lorentz electromotive force}!transformation}%

Before considering these transformations in detail we
\PageSep{33}
shall make a few general remarks about space and time.
In the pre-relativity physics space and time were separate
entities. Specifications of time were independent of
the choice of the space of reference. The Newtonian
mechanics was relative with respect to the space of
reference, so that, e.g.~the statement that two non-simultaneous
events happened at the same place had no objective
meaning (that is, independent of the space of reference).
But this relativity had no r�le in building up the theory.
One spoke of points of space, as of instants of time, as if
they were absolute realities. It was not observed that
the true element of the space-time specification was the
event, specified by the four numbers $x_{1}$,~$x_{2}$,~$x_{3}$,~$t$. The
conception of something happening was always that of a
four-dimensional continuum; but the recognition of this
\index{Continuum, four-dimensional}%
\index{Four-dimensional continuum}%
was obscured by the absolute character of the pre-relativity
time. Upon giving up the hypothesis of the absolute
character of time, particularly that of simultaneity, the
four-dimensionality of the time-space concept was immediately
\index{Time-space concept}%
recognized. It is neither the point in space,
nor the instant in time, at which something happens that
has physical reality, but only the event itself. There is
no absolute (independent of the space of reference) relation
in space, and no absolute relation in time between two
events, but there is an absolute (independent of the space
of reference) relation in space and time, as will appear in
the sequel. The circumstance that there is no objective
rational division of the four-dimensional continuum into
a three-dimensional space and a one-dimensional time
continuum indicates that the laws of nature will assume
\PageSep{34}
a form which is logically most satisfactory when expressed
as laws in the four-dimensional space-time continuum.
Upon this depends the great advance in method which
the theory of relativity owes to Minkowski. Considered
\index{Minkowski}%
from this standpoint, we must regard $x_{1}$,~$x_{2}$,~$x_{3}$,~$t$ as the
four co-ordinates of an event in the four-dimensional continuum.
We have far less success in picturing to ourselves
relations in this four-dimensional continuum than in the
three-dimensional Euclidean continuum; but it must be
emphasized that even in the Euclidean three-dimensional
geometry its concepts and relations are only of an abstract
nature in our minds, and are not at all identical with the
images we form visually and through our sense of touch.
The non-divisibility of the four-dimensional continuum
of events does not at all, however, involve the equivalence
of the space co-ordinates with the time co-ordinate. On
the contrary, we must remember that the time co-ordinate
is defined physically wholly differently from the space
co-ordinates. The relations \Eqref{(22)}~and~\Eqref{(22a)} which when
equated define the Lorentz transformation show, further,
a difference in the r�le of the time co-ordinate from that
of the space co-ordinates; for the term~$\Delta t^{2}$ has the opposite
sign to the space terms, ${\Delta x_{1}}^{2}$,~${\Delta x_{2}}^{2}$,~${\Delta x_{3}}^{2}$.

Before we analyse further the conditions which define
the Lorentz transformation, we shall introduce the light-time,
\index{Light time@{Light-time}}%
$l = ct$, in place of the time,~$t$, in order that the
constant~$c$ shall not enter explicitly into the formulas to
be developed later. Then the Lorentz transformation is
defined in such a way that, first, it makes the equation
\[
{\Delta x_{1}}^{2} + {\Delta x_{2}}^{2} + {\Delta x_{3}}^{2} - \Delta l^{2} = 0
\Tag{(22b)}
\]
\PageSep{35}
a co-variant equation, that is, an equation which is satisfied
with respect to every inertial system if it is satisfied in
the inertial system to which we refer the two given events
(emission and reception of the ray of light). Finally,
with Minkowski, we introduce in place of the real time
co-ordinate $l = ct$, the imaginary time co-ordinate
\[
x_{4} = il = ict\quad (\sqrt{-1} = i).
\]
Then the equation defining the propagation of light,
which must be co-variant with respect to the Lorentz
transformation, becomes
\[
\sum_{(4)} {\Delta x_{\nu}}^{2}
  = {\Delta x_{1}}^{2} + {\Delta x_{2}}^{2} + {\Delta x_{3}}^{2} + {\Delta x_{4}}^{2}
  = 0\Add{.}
\Tag{(22c)}
\]
This condition is always satisfied\footnote
  {That this specialization lies in the nature of the case will be evident
  later.}
if we satisfy the more
general condition that
\[
s^{2} = {\Delta x_{1}}^{2} + {\Delta x_{2}}^{2} + {\Delta x_{3}}^{2} + {\Delta x_{4}}^{2}
\Tag{(23)}
\]
shall be an invariant with respect to the transformation.
This condition is satisfied only by linear transformations,
that is, transformations of the type
\[
{x'}_{\mu} = a_{\mu} + b_{\mu\alpha} x_{\alpha}
\Tag{(24)}
\]
in which the summation over the~$\alpha$ is to be extended
from $\alpha = 1$ to $\alpha = 4$. A glance at equations \Eqref{(23)}~and~\Eqref{(24)}
shows that the Lorentz transformation so defined is
identical with the translational and rotational transformations
of the Euclidean geometry, if we disregard the
number of dimensions and the relations of reality. We
\PageSep{36}
can also conclude that the coefficients~$b_{\mu\alpha}$ must satisfy the
conditions
\[
b_{\mu\alpha}b_{\nu\alpha} = \delta_{\mu\nu} = b_{\alpha\mu}b_{\alpha\nu}\Add{.}
\Tag{(25)}
\]
Since the ratios of the~$x_{\nu}$ are real, it follows that all the~$a_{\mu}$
and the~$b_{\mu\alpha}$ are real, except $a_{4}$, $b_{41}$,~$b_{42}$,~$b_{43}$, $b_{14}$,~$b_{24}$\Change{,}{} and~$b_{34}$,
which are purely imaginary.

\Paragraph{Special Lorentz Transformation.} We obtain the
\index{Special Lorentz transformation}%
simplest transformations of the type of \Eqref{(24)}~and~\Eqref{(25)} if
only two of the co-ordinates are to be transformed, and if
all the~$a_{\mu}$, which determine the new origin, vanish. We
obtain then for the indices $1$~and~$2$, on account of the
three independent conditions which the relations~\Eqref{(25)}
furnish,
\[
\left.
\begin{alignedat}{2}
{x'}_{1} &= x_{1} \cos\phi &&- x_{2} \sin\phi\Add{,} \\
{x'}_{2} &= x_{1} \sin\phi &&+ x_{2} \cos\phi\Add{,} \\
{x'}_{3} &= x_{3}\Add{,} && \\
{x'}_{4} &= x_{4}\Add{.} &&
\end{alignedat}
\right\}
\Tag{(26)}
\]

This is a simple rotation in space of the (space)
co-ordinate system about $x_{3}$-axis. We see that the
rotational transformation in space (without the time
transformation) which we studied before is contained in
the Lorentz transformation as a special case. For the
indices $1$~and~$4$ we obtain, in an analogous manner,
\[
\left.
\begin{alignedat}{2}
{x'}_{1} &= x_{1} \cos\psi &&- x_{4} \sin\psi\Add{,} \\
{x'}_{4} &= x_{1} \sin\psi &&+ x_{4} \cos\psi\Add{,} \\
{x'}_{2} &= x_{2}\Add{,} && \\
{x'}_{3} &= x_{3}\Add{.} &&
\end{alignedat}
\right\}
\Tag{(26a)}
\]

On account of the relations of reality $\psi$~must be taken
as imaginary. To interpret these equations physically,
we introduce the real light-time~$l$ and the velocity~$v$ of~$K'$
\PageSep{37}
relatively to~$K$, instead of the imaginary angle~$\psi$. We
have, first,
\begin{alignat*}{4}
{x'}_{1} &= &&x_{1} \cos\psi &{}-{}& i&&l \sin \psi\Add{,} \\
l'     &= -i&&x_{1} \sin\psi &{}+{}&  &&l \cos\psi.
\end{alignat*}
Since for the origin of~$K'$ i.e., for $x_{1} = 0$, we must have
$x_{1} = vl$, it follows from the first of these equations that
\[
v = i\tan\psi\Add{,}
\Tag{(27)}
\]
and also
\[
\left.
\begin{aligned}
\sin\psi &= \frac{-iv}{\sqrt{1 - v^{2}}}\Add{,} \\
\cos\psi &= \frac{1}{\sqrt{1 - v^{2}}}\Add{,}
\end{aligned}
\right\}
\Tag{(28)}
\]
so that we obtain
\[
\left.
\begin{aligned}
{x'}_{1} &= \frac{x_{1} - vl}{\sqrt{1 - v^{2}}}\Add{,} \\
l' &= \frac{l - vx_{1}}{\sqrt{1 - v^{2}}}\Add{,} \\
{x'}_{2} &= x_{2}\Add{,} \\
{x'}_{3} &= x_{3}\Add{.}
\end{aligned}
\right\}
\Tag{(29)}
\]

These equations form the well-known special Lorentz
transformation, which in the general theory represents a
rotation, through an imaginary angle, of the four-dimensional
system of co-ordinates. If we introduce the ordinary
time~$t$, in place of the light-time~$l$, then in~\Eqref{(29)} we must
replace $l$~by~$ct$ and $v$~by~$\dfrac{v}{c}$.

We must now fill in a gap. From the principle of the
constancy of the velocity of light it follows that the
equation
\[
\sum {\Delta x_{\nu}}^{2} = 0
\]
\PageSep{38}
has a significance which is independent of the choice of
the inertial system; but the invariance of the quantity
$\sum {\Delta x_{\nu}}^{2}$ does not at all follow from this. This quantity
might be transformed with a factor. This depends upon
the fact that the right-hand side of~\Eqref{(29)} might be multiplied
by a factor~$\lambda$, independent of~$v$. But the principle
of relativity does not permit this factor to be different from~$1$,
as we shall now show. Let us assume that we have
a rigid circular cylinder moving in the direction of its
axis. If its radius, measured at rest with a unit measuring
rod is equal to~$R_{0}$, its radius~$R$ in motion, might be
different from~$R_{0}$, since the theory of relativity does not
make the assumption that the shape of bodies with respect
to a space of reference is independent of their motion
relatively to this space of reference. But all directions
in space must be equivalent to each other. $R$~may therefore
depend upon the magnitude~$q$ of the velocity, but
not upon its direction; $R$~must therefore be an even
function of~$q$. If the cylinder is at rest relatively to~$K'$
the equation of its lateral surface is
\[
%[** TN: Inconsistently upright R here, below in the original]
x'^{2} + y'^{2} = {R_{0}}^{2}.
\]
If we write the last two equations of~\Eqref{(29)} more generally
\begin{align*}
{x'}_{2} = \lambda x_{2}\Add{,} \\
{x'}_{3} = \lambda x_{3}\Add{,}
\end{align*}
then the lateral surface of the cylinder referred to~$K$
satisfies the equation
\[
x^{2} + y^{2} = \frac{{R_{0}}^{2}}{\lambda^{2}}.
\]
\PageSep{39}
The factor~$\lambda$ therefore measures the lateral contraction of
the cylinder, and can thus, from the above, be only an
even function of~$v$.

If we introduce a third system of co-ordinates,~$K''$,
which moves relatively to~$K'$ with velocity~$v$ in the direction
of the negative $x$-axis of~$K$, we obtain, by applying
\Eqref{(29)} twice,
\begin{align*}
{x''}_{1} &= \lambda(v) \lambda(-v) x_{1}\Add{,} \\
%[** TN: Added code for next two equations, elided in the original]
{x''}_{2} &= \lambda(v) \lambda(-v) x_{2}\Add{,} \\
{x''}_{3} &= \lambda(v) \lambda(-v) x_{3}\Add{,} \\
l'' &= \lambda(v) \lambda(-v) l.
\end{align*}
Now, since $\lambda(v)$~must be equal to~$\lambda(-v)$, and since we
assume that we use the same measuring rods in all the
systems, it follows that the transformation of~$K''$ to~$K$
must be the identical transformation (since the possibility
$\lambda = -1$ does not need to be considered). It is essential
for these considerations to assume that the behaviour of
the measuring rods does not depend upon the history of
their previous motion.

\Paragraph{Moving Measuring Rods and Clocks.} At the definite $K$-time,
\index{Clocks, moving}%
\index{Moving measuring rods and clocks}%
\index{Rods (measuring) and clocks in motion}%
$l = 0$, the position of the points given by the integers
${x'}_{1} = n$, is with respect to~$K$, given by $x_{1} = n\sqrt{1 - v^{2}}$;
this follows from the first of equations~\Eqref{(29)} and expresses
the Lorentz contraction. A clock at rest at the origin
$x_{1} = 0$ of~$K$, whose beats are characterized by $l = n$, will,
when observed from~$K'$, have beats characterized by
\[
l' = \frac{n}{\sqrt{1 - v^{2}}};
\]
this follows from the second of equations~\Eqref{(29)} and shows
\PageSep{40}
that the clock goes slower than if it were at rest relatively
to~$K'$. These two consequences, which hold, \textit{mutatis
mutandis}, for every system of reference, form the physical
content, free from convention, of the Lorentz transformation.

\Paragraph{Addition Theorem for Velocities.} If we combine two
\index{Addition@{Addition and subtraction of tensors}!theorem of velocities}%
\index{Theorem for addition of velocities}%
\index{Velocities, addition theorem of}%
special Lorentz transformations with the relative velocities
$v_{1}$~and~$v_{2}$, then the velocity of the single Lorentz transformation
which takes the place of the two separate ones
is, according to~\Eqref{(27)}, given by
\[
v_{12} = i \tan (\psi_{1} + \psi_{2})
  = i \frac{\tan\psi_{1} + \tan\psi_{2}}{1 - \tan\psi_{1} \tan\psi_{2}}
  = \frac{v_{1} + v_{2}}{1 + v_{1}v_{2}}.
\Tag{(30)}
\]

\Paragraph{General Statements about the Lorentz Transformation
and its Theory of Invariants.} The whole theory of
invariants of the special theory of relativity depends upon
the invariant~$s^{2}$~\Eqref{(23)}. Formally, it has the same r�le in
the four-dimensional space-time continuum as the invariant
${\Delta x_{1}}^{2} + {\Delta x_{2}}^{2} + {\Delta x_{3}}^{2}$ in the Euclidean geometry
and in the pre-relativity physics. The latter quantity is
not an invariant with respect to all the Lorentz transformations;
the quantity~$s^{2}$ of equation~\Eqref{(23)} assumes the
r�le of this invariant. With respect to an arbitrary
inertial system, $s^{2}$~may be determined by measurements;
with a given unit of measure it is a completely determinate
quantity, associated with an arbitrary pair of events.

The invariant~$s^{2}$ differs, disregarding the number of
dimensions, from the corresponding invariant of the
Euclidean geometry in the following points. In the
Euclidean geometry $s^{2}$~is necessarily positive; it vanishes
\PageSep{41}
only when the two points concerned come together. On
the other hand, from the vanishing of
\[
s^{2} = \sum_{(4)} {\Delta x_{\nu}}^{2}
  = {\Delta x_{1}}^{2} + {\Delta x_{2}}^{2} + {\Delta x_{3}}^{2} - {\Delta t}^{2}
\]
it cannot be concluded that the two space-time points
%[Illustration: Fig. 1.]
\Figure{041}
fall together; the vanishing of this quantity~$s^{2}$, is the
invariant condition that the two space-time points can be
connected by a light signal \textit{in vacuo}. If $P$~is a point
\PageSep{42}
(event) represented in the four-dimensional space of the
$x_{1}$,~$x_{2}$,~$x_{3}$,~$l$, then all the ``points'' which can be connected
to~$P$ by means of a light signal lie upon the cone $s^{2} = 0$
(compare \Figref{1}, in which the dimension~$x_{3}$ is suppressed).
The ``upper'' half of the cone may contain the ``points''
to which light signals can be sent from~$P$; then the
``lower'' half of the cone will contain the ``points'' from
which light signals can be sent to~$P$. The points~$P'$
enclosed by the conical surface furnish, with~$P$, a negative~$s^{2}$;
$PP'$,~as well as~$P'P$ is then, according to Minkowski,
of the nature of a time. Such intervals represent elements
of possible paths of motion, the velocity being less than
that of light.\footnote
  {That material velocities exceeding that of light are not possible,
  follows from the appearance of the radical $\sqrt{1 - v^{2}}$ in the special Lorentz
  transformation~\Eqref{(29)}.}
In this case the $l$-axis may be drawn in
the direction of~$PP'$ by suitably choosing the state of
motion of the inertial system. If $P'$~lies outside of the
``light-cone'' then $PP'$~is of the nature of a space; in
\index{Light cone@{Light-cone}}%
this case, by properly choosing the inertial system, $\Delta l$~can
be made to vanish.

By the introduction of the imaginary time variable,
$x_{4} = il$, Minkowski has made the theory of invariants for
the four-dimensional continuum of physical phenomena
fully analogous to the theory of invariants for the three-dimensional
continuum of Euclidean space. The theory
of four-dimensional tensors of special relativity differs from
the theory of tensors in three-dimensional space, therefore,
only in the number of dimensions and the relations of
reality.
\PageSep{43}

A physical entity which is specified by four quantities,~$A_{\nu}$,
\index{Four-vector}%
in an arbitrary inertial system of the $x_{1}$,~$x_{2}$,~$x_{3}$,~$x_{4}$, is
called a $4$-vector, with the components~$A_{\nu}$, if the $A_{\nu}$
correspond in their relations of reality and the properties
of transformation to the~$\Delta x_{\nu}$; it may be of the nature of
a space or of a time. The sixteen quantities~$A_{\mu\nu}$ then
form the components of a tensor of the second rank, if
they transform according to the scheme
\[
{A'}_{\mu\nu} = b_{\mu\alpha} b_{\nu\beta} A_{\alpha\beta}.
\]
It follows from this that the~$A_{\mu\nu}$ behave, with respect to
their properties of transformation and their properties
of reality, as the products of components,~$U_{\mu}V_{\nu}$, of two
$4$-vectors, $(U)$~and~$(V)$. All the components are real
except those which contain the index~$4$ once, those being
purely imaginary. Tensors of the third and higher ranks
may be defined in an analogous way. The operations
of addition, subtraction, multiplication, contraction and
differentiation for these tensors are wholly analogous to
the corresponding operations for tensors in three-dimensional
space.

Before we apply the tensor theory to the four-dimensional
space-time continuum, we shall examine more
particularly the skew-symmetrical tensors. The tensor
of the second rank has, in general, $16 = 4�4$ components.
In the case of skew-symmetry the components with two
equal indices vanish, and the components with unequal
indices are equal and opposite in pairs. There exist,
therefore, only six independent components, as is the
case in the electromagnetic field. In fact, it will be shown
\PageSep{44}
when we consider Maxwell's equations that these may
be looked upon as tensor equations, provided we regard
the electromagnetic field as a skew-symmetrical tensor.
Further, it is clear that the skew-symmetrical tensor of
the third rank (skew-symmetrical in all pairs of indices)
has only four independent components, since there are
only four combinations of three different indices.

We now turn to Maxwell's equations \Eqref{(19a)}, \Eqref{(19b)}, \Eqref{(20a)},
\Eqref{(20b)}, and introduce the notation:\footnote
  {In order to avoid confusion from now on we shall use the three-dimensional
  space indices, $x$,~$y$,~$z$ instead of $1$,~$2$,~$3$, and we shall reserve
  the numeral indices $1$,~$2$, $3$,~$4$ for the four-dimensional space-time continuum.}
\begin{gather*}
\left.
\begin{alignedat}{6}
\phi_{23}\qquad  & \phi_{31}\qquad && \phi_{12}\quad &&
\Neg\phi_{14}\quad && \Neg\phi_{24}\quad && \Neg\phi_{34} \\
\h_{23}  \qquad  & \h_{31}  \qquad && \h_{12}  \quad &&
-\Change{\veci}{i}\e_{x}\quad && -\Change{\veci}{i}\e_{y}\quad && -\Change{\veci}{i}\e_{z}
\end{alignedat}
\right\}
\Tag{(30a)}\displaybreak[1] \\
%
\left.
\begin{alignedat}{4}
& J_{1}\qquad && J_{2}\qquad && J_{3}\qquad && J_{4} \\
&\frac{1}{c}\veci_{x}\qquad &&
\frac{1}{c}\veci_{y}\qquad &&
\frac{1}{c}\veci_{z}\qquad && \Change{\veci}{i}\rho
\end{alignedat}
\right\}
\Tag{(31)}
\end{gather*}
with the convention that $\phi_{\mu\nu}$~shall be equal to~$-\phi_{\nu\mu}$.
Then Maxwell's equations may be combined into the
forms
\begin{gather*}
\frac{\dd \phi_{\mu\nu}}{\dd x_{\nu}} = J_{\mu}\Add{,}
\Tag{(32)} \\
\frac{\dd \phi_{\mu\nu}}{\dd x_{\sigma}} +
\frac{\dd \phi_{\nu\sigma}}{\dd x_{\mu}} +
\frac{\dd \phi_{\sigma\mu}}{\dd x_{\nu}} = 0\Add{,}
\Tag{(33)}
\end{gather*}
as one can easily verify by substituting from \Eqref{(30a)}~and~\Eqref{(31)}.
Equations \Eqref{(32)}~and~\Eqref{(33)} have a tensor character,
and are therefore co-variant with respect to Lorentz
transformations, if the~$\phi_{\mu\nu}$ and the~$J_{\nu}$ have a tensor
character, which we assume. Consequently, the laws for
\PageSep{45}
transforming these quantities from one to another allowable
(inertial) system of co-ordinates are uniquely
determined. The progress in method which electrodynamics
owes to the theory of special relativity lies
principally in this, that the number of independent
hypotheses is diminished. If we consider, for example,
equations~\Eqref{(19a)} only from the standpoint of relativity of
direction, as we have done above, we see that they have
three logically independent terms. The way in which
the electric intensity enters these equations appears to
be wholly independent of the way in which the magnetic
intensity enters them; it would not be surprising if instead
of~$\dfrac{\dd \e_{\mu}}{\dd l}$, we had, say,~$\dfrac{\dd^{2} \e_{\mu}}{\dd l^{2}}$, or if this term were absent. On
the other hand, only two independent terms appear in
equation~\Eqref{(32)}. The electromagnetic field appears as a
formal unit; the way in which the electric field enters
this equation is determined by the way in which the
magnetic field enters it. Besides the electromagnetic
field, only the electric current density appears as an
independent entity. This advance in method arises from
the fact that the electric and magnetic fields draw their
separate existences from the relativity of motion. A
field which appears to be purely an electric field, judged
from one system, has also magnetic field components
when judged from another inertial system. When applied
to an electromagnetic field, the general law of transformation
furnishes, for the special case of the special Lorentz
transformation, the equations
\PageSep{46}
\[
\left.
\begin{alignedat}{2}
{\e'}_{x} &= \e_{x}\qquad & {\h'}_{x} &= \h_{x}\Add{,} \\
{\e'}_{y} &= \frac{\e_{y} - v\h_{z}}{\sqrt{1 - v^{2}}}\qquad &
{\h'}_{y} &= \frac{\h_{y} + v\e_{z}}{\sqrt{1 - v^{2}}}\Add{,} \\
{\e'}_{z} &= \frac{\e_{z} + v\h_{y}}{\sqrt{1 - v^{2}}}\qquad &
{\h'}_{z} &= \frac{\h_{z} - v\e_{y}}{\sqrt{1 - v^{2}}}\Add{.}
\end{alignedat}
\right\}
\Tag{(34)}
\]

If there exists with respect to~$K$ only a magnetic field,~$\h$,
but no electric field,~$\e$, then with respect to~$K'$ there
exists an electric field~$\e'$ as well, which would act upon
an electric particle at rest relatively to~$K'$. An observer
at rest relatively to~$K$ would designate this force as the
Biot-Savart force, or the Lorentz electromotive force. It
\index{Biot-Savart force}%
\index{Lorentz@{Lorentz electromotive force}}%
therefore appears as if this electromotive force had become
fused with the electric field intensity into a single entity.

In order to view this relation formally, let us consider
the expression for the force acting upon unit volume of
electricity,
\[
\k = \rho\e + [\veci, \h]\Add{,}
\Tag{(35)}
\]
in which $\veci$~is the vector velocity of electricity, with the
velocity of light as the unit. If we introduce $J_{\mu}$~and~$\phi_{\mu\Change{}{\nu}}$
according to \Eqref{(30a)}~and~\Eqref{(31)}, we obtain for the first
component the expression
\[
\phi_{12} J_{2} + \phi_{13} J_{3} + \phi_{14} J_{4}.
\]
Observing that $\phi_{11}$~vanishes on account of the skew-symmetry
of the tensor~$(\phi)$, the components of~$\Change{k}{\k}$ are given
by the first three components of the four-dimensional
vector
\[
K_{\mu} = \phi_{\mu\nu} J_{\nu}\Add{,}
\Tag{(36)}
\]
and the fourth component is given by
\[
K_{4} = \phi_{41} J_{1} + \phi_{42} J_{2} + \phi_{43} J_{3}
  = \Change{\veci}{i}(\e_{x} \veci_{x} + \e_{y} \veci_{y} + \e_{z} \veci_{z})
  = \Change{\veci}{i}\lambda\Add{.}
\Tag{(37)}
\]
\PageSep{47}
There is, therefore, a four-dimensional vector of force per
unit volume, whose first three components, $\Change{k}{K}_{1}$,~$\Change{k}{K}_{2}$,~$\Change{k}{K}_{3}$, are
the ponderomotive force components per unit volume, and
whose fourth component is the rate of working of the field
per unit volume, multiplied by~$\sqrt{-1}$.
%[Illustration: Fig. 2.]
\Figure{047}

A comparison of \Eqref{(36)}~and~\Eqref{(35)} shows that the theory
of relativity formally unites the ponderomotive force of
the electric field,~$\rho\e$, and the Biot-Savart or Lorentz
force~$[\veci, \h]$.
\PageSep{48}

\Paragraph{Mass and Energy.} An important conclusion can be
\index{Energy@{Energy and mass}}%
\index{Mass and Energy}%
drawn from the existence and significance of the $4$-vector~$K_{\mu}$.
Let us imagine a body upon which the electromagnetic
field acts for a time. In the symbolic figure
(\Figref{2}) $Ox_{1}$~designates the $x_{1}$-axis, and is at the same
time a substitute for the three space axes $Ox_{1}$,~$Ox_{2}$,~$Ox_{3}$;
$Ol$~designates the real time axis. In this diagram a body
of finite extent is represented, at a definite time~$l$, by the
interval~$AB$; the whole space-time existence of the body
is represented by a strip whose boundary is everywhere
inclined less than~$45�$ to the $l$-axis. Between the time
sections, $l = l_{1}$ and $l = l_{2}$, but not extending to them,
a portion of the strip is shaded. This represents the
portion of the space-time manifold in which the electromagnetic
field acts upon the body, or upon the electric
charges contained in it, the action upon them being
transmitted to the body. We shall now consider the
changes which take place in the momentum and energy
of the body as a result of this action.

We shall assume that the principles of momentum
and energy are valid for the body. The change in
momentum, $\Delta I_{x}$,~$\Delta I_{y}$,~$\Delta I_{z}$, and the change in energy,~$\Delta E$,
are then given by the expressions
\begin{alignat*}{2}
%[** TN: \int_{l_{0}}^{l_{1}} in the original]
\Delta I_{x} &= \int_{l_{1}}^{l_{2}}dl \int \k_{x}\, dx\, dy\, dz
  &&= \frac{1}{\Change{\veci}{i}}\int K_{1}\, dx_{1}\, dx_{2}\, dx_{3}\, dx_{4}\Add{,} \\
%
%[** TN: Added code for next two equations, elided in the original]
\Delta I_{y} &= \int_{l_{1}}^{l_{2}}dl \int \k_{y}\, dx\, dy\, dz
  &&= \frac{1}{\Change{\veci}{i}}\int K_{2}\, dx_{1}\, dx_{2}\, dx_{3}\, dx_{4}\Add{,} \\
%
\Delta I_{z} &= \int_{l_{1}}^{l_{2}}dl \int \k_{z}\, dx\, dy\, dz
  &&= \frac{1}{\Change{\veci}{i}}\int K_{3}\, dx_{1}\, dx_{2}\, dx_{3}\, dx_{4}\Add{,} \\
%
\Delta E &= \int_{l_{1}}^{l_{2}}dl \int \lambda\, dx\, dy\, dz
  &&= \frac{1}{\Change{\veci}{i}}\int
      \frac{1}{\Change{\veci}{i}}K_{4}\, dx_{1}\, dx_{2}\, dx_{3}\, dx_{4}\Add{.}
\end{alignat*}
\PageSep{49}
Since the four-dimensional element of volume is an
invariant, and $(K_{1}, K_{2}, K_{3}, K_{4})$ forms a $4$-vector, the four-dimensional
integral extended over the shaded portion
transforms as a $4$-vector, as does also the integral between
the limits $l_{1}$~and~$l_{2}$, because the portion of the region which
is not shaded contributes nothing to the integral. It
follows, therefore, that $\Delta I_{x}$,~$\Delta I_{y}$,~$\Delta I_{z}$,~$i\Delta E$ form a $4$-vector.
Since the quantities themselves transform in the same
way as their increments, it follows that the aggregate of
the four quantities
\[
I_{x},\ I_{y},\ I_{z},\ \Change{\veci}{i} E
\]
has itself the properties of a vector; these quantities are
referred to an instantaneous condition of the body (e.g.~at
the time $l = l_{1}$).

This $4$-vector may also be expressed in terms of the
mass~$m$, and the velocity of the body, considered as a
material particle. To form this expression, we note first,
that
\[
-ds^{2} = d\tau^{2}
  = - ({dx_{1}}^{2} + {dx_{2}}^{2} + {dx_{3}}^{2}) - {dx_{4}}^{2}
  = dl^{2}(1 - q^{2})
\Tag{(38)}
\]
is an invariant which refers to an infinitely short portion
of the four-dimensional line which represents the motion
of the material particle. The physical significance of the
invariant~$d\tau$ may easily be given. If the time axis is
chosen in such a way that it has the direction of the line
differential which we are considering, or, in other words,
if we reduce the material particle to rest, we shall then
have $d\tau = dl$; this will therefore be measured by the
light-seconds clock which is at the same place, and at
rest relatively to the material particle. We therefore call
\PageSep{50}
$\tau$~the proper time of the material particle. As opposed
to~$dl$, $d\tau$~is therefore an invariant, and is practically
equivalent to~$dl$ for motions whose velocity is small
compared to that of light. Hence we see that
\[
u_{\sigma} = \frac{dx_{\sigma}}{d\tau}
\Tag{(39)}
\]
has, just as the~$dx_{\nu}$, the character of a vector; we shall
designate~$(u_{\sigma})$ as the four-dimensional vector (in brief,
$4$-vector) of velocity. Its components satisfy, by~\Eqref{(38)},
the condition
\[
\sum {u_{\sigma}}^{2} = -1\Add{.}
\Tag{(40)}
\]
We see that this $4$-vector, whose components in the
ordinary notation are
\[
\frac{\q_{x}}{\sqrt{1 - q^{2}}},\quad
\frac{\q_{y}}{\sqrt{1 - q^{2}}},\quad
\frac{\q_{z}}{\sqrt{1 - q^{2}}},\quad
\frac{\Change{\veci}{i}}{\sqrt{1 - q^{2}}}
\Tag{(41)}
\]
is the only $4$-vector which can be formed from the velocity
components of the material particle which are defined in
three dimensions by
\[
\q_{x} = \frac{dx}{dl},\quad
\q_{y} = \frac{dy}{dl},\quad
\q_{z} = \frac{dz}{dl}.
\]
We therefore see that
\[
\left(m \frac{dx_{\mu}}{d\tau}\right)
\Tag{(42)}
\]
must be that $4$-vector which is to be equated to the
$4$-vector of momentum and energy whose existence we
have proved above. By equating the components, we
obtain, in three-dimensional notation,
\PageSep{51}
\[
\left.
\begin{aligned}
I_{x} &= \frac{m\q_{x}}{\sqrt{1 - q^{2}}}\Add{,} \\
%[** TN: Added code for next two equations, elided in the original]
I_{y} &= \frac{m\q_{y}}{\sqrt{1 - q^{2}}}\Add{,} \\
I_{z} &= \frac{m\q_{z}}{\sqrt{1 - q^{2}}}\Add{,} \\
E &= \frac{m}{\sqrt{1 - q^{2}}}\Add{.}
\end{aligned}
\right\}
\Tag{(43)}
\]

We recognize, in fact, that these components of
momentum agree with those of classical mechanics for
velocities which are small compared to that of light. For
large velocities the momentum increases more rapidly
than linearly with the velocity, so as to become infinite
on approaching the velocity of light.

If we apply the last of equations~\Eqref{(43)} to a material
particle at rest ($q = 0$), we see that the energy,~$E_{0}$, of a
\index{Energy@{Energy and mass}}%
body at rest is equal to its mass. Had we chosen the
second as our unit of time, we would have obtained
\[
E_{0} = mc^{2}\Add{.}
\Tag{(44)}
\]
Mass and energy are therefore essentially alike; they are
\index{Equivalence of mass and energy}%
\index{Mass and Energy}%
only different expressions for the same thing. The mass
of a body is not a constant; it varies with changes in its
energy.\footnote
  {The emission of energy in radioactive processes is evidently connected
  with the fact that the atomic weights are not integers. Attempts have
  been made to draw conclusions from this concerning the structure and
  stability of the atomic nuclei.}
We see from the last of equations~\Eqref{(43)} that $E$
becomes infinite when $q$~approaches~$1$, the velocity of
light. If we develop~$E$ in powers of~$q^{2}$, we obtain,
\[
E = m + \frac{m}{2}q^{2} + \frac{3}{8}mq^{4} +\dots\Add{.}
\Tag{(45)}
\]
\PageSep{52}
The second term of this expansion corresponds to the
kinetic energy of the material particle in classical
mechanics.

\Paragraph{Equations of Motion of Material Particles.} From~\Eqref{(43)}
\index{Equations of motion of \Change{materia}{material} particle}%
\index{Motion of particle, equations of}%
we obtain, by differentiating by the time~$l$, and using
the principle of momentum, in the notation of three-dimensional
vectors,
\[
\K = \frac{d}{dl}\left(\frac{m\q}{\sqrt{1 - q^{2}}}\right)\Add{.}
\Tag{(46)}
\]

This equation, which was previously employed by
H.~A. Lorentz for the motion of electrons, has been
proved to be true, with great accuracy, by experiments
with $\beta$-rays.

\Paragraph{Energy Tensor of the Electromagnetic Field.} Before the
\index{Energy@{Energy and mass}!tensor@{tensor of electromagnetic field}}%
development of the theory of relativity it was known
that the principles of energy and momentum could
be expressed in a differential form for the electromagnetic
field. The four-dimensional formulation of
these principles leads to an important conception, that of
the energy tensor, which is important for the further
development of the theory of relativity.

If in the expression for the $4$-vector of force per unit
volume,
\[
K_{\mu} = \phi_{\mu\nu} J_{\nu}\Add{,}
\]
using the field equations~\Eqref{(32)}, we express~$J_{\nu}$ in terms of
the field intensities,~$\phi_{\mu\nu}$, we obtain, after some transformations
and repeated application of the field equations
\Eqref{(32)}~and~\Eqref{(33)}, the expression
\[
K_{\mu} = -\frac{\dd T_{\mu\nu}}{\dd x_{\nu}}\Add{,}
\Tag{(47)}
\]
\PageSep{53}
where we have written\footnote
  {To be summed for the indices $\alpha$~and~$\beta$.}
\[
T_{\mu\nu}
  = -\tfrac{1}{4}{\phi_{\alpha\beta}}^{2} \delta_{\mu\nu}
    + \phi_{\mu\alpha} \phi_{\nu\alpha}\Add{.}
\Tag{(48)}
\]

The physical meaning of equation~\Eqref{(47)} becomes evident
if in place of this equation we write, using a new
notation,
\[
\left.
\begin{alignedat}{4}
\k_{x} &= -\frac{\dd p_{xx}}{\dd x}
  &&- \frac{\dd p_{xy}}{\dd y}
  &&- \frac{\dd p_{xz}}{\dd z}
  &&- \frac{\dd (\Change{\veci}{i} b_{x})}{\dd (\Change{\veci}{i} l)}\Add{,} \\
%
%[** TN: Added code for next two equations]
\k_{y} &= -\frac{\dd p_{yx}}{\dd x}
  &&- \frac{\dd p_{yy}}{\dd y}
  &&- \frac{\dd p_{yz}}{\dd z}
  &&- \frac{\dd (\Change{\veci}{i} b_{y})}{\dd (\Change{\veci}{i} l)}\Add{,} \\
%
\k_{z} &= -\frac{\dd p_{zx}}{\dd x}
  &&- \frac{\dd p_{zy}}{\dd y}
  &&- \frac{\dd p_{zz}}{\dd z}
  &&- \frac{\dd (\Change{\veci}{i} b_{z})}{\dd (\Change{\veci}{i} l)}\Add{,} \\
%
\Change{\veci}{i}\lambda
  &= -\frac{\dd (\Change{\veci}{i} \s_{x})}{\dd x}
  &&- \frac{\dd (\Change{\veci}{i} \s_{y})}{\dd y}
  &&- \frac{\dd (\Change{\veci}{i} \s_{z})}{\dd z}
  &&- \frac{\dd (-\eta)}{\dd (\Change{\veci}{i} l)}\Add{;}
\end{alignedat}
\right\}
\Tag{(47a)}
\]
or, on eliminating the imaginary,
\[
\left.
\begin{alignedat}{4}
\k_{x} &= -\frac{\dd p_{xx}}{\Change{\dd_{x}}{\dd x}}
  &&- \frac{\dd p_{xy}}{\dd y}
  &&- \frac{\dd p_{xz}}{\dd z}
  &&- \frac{\dd b_{x}}{\dd l}\Add{,} \\
%
%[** TN: Added code for next two equations]
\k_{y} &= -\frac{\dd p_{yx}}{\dd x}
  &&- \frac{\dd p_{yy}}{\dd y}
  &&- \frac{\dd p_{yz}}{\dd z}
  &&- \frac{\dd b_{y}}{\dd l}\Add{,} \\
%
\k_{z} &= -\frac{\dd p_{zx}}{\dd x}
  &&- \frac{\dd p_{zy}}{\dd y}
  &&- \frac{\dd p_{zz}}{\dd z}
  &&- \frac{\dd b_{z}}{\dd l}\Add{,} \\
%
\lambda
  &= -\frac{\dd \s_{x}}{\dd x}
  &&- \frac{\dd \s_{y}}{\dd y}
  &&- \frac{\dd \s_{z}}{\dd z}
  &&- \frac{\dd \eta}{\dd l}\Add{.}
\end{alignedat}
\right\}
\Tag{(47b)}
\]

When expressed in the latter form, we see that the
first three equations state the principle of momentum;
$p_{xx}$\Add{,}\ldots\Add{,}~$p_{zx}$ are the Maxwell stresses in the electromagnetic
field, and $(b_{x}, b_{y}, b_{z})$ is the vector momentum
per unit volume of the field. The last of equations~\Eqref{(47b)}
expresses the energy principle; $\s$~is the vector flow of
energy, and $\eta$~the energy per unit volume of the field.
In fact, we get from~\Eqref{(48)} by introducing the well-known
expressions for the components of the field intensity from
electrodynamics,
\PageSep{54}
\[
\left.
\begin{aligned}
&\begin{alignedat}{4}
p_{xx} = &{} - \h_{x} \h_{x}
  &&+ \tfrac{1}{2}({\h_{x}}^{2} &&+ {\h_{y}}^{2} &&+ {\h_{z}}^{2}) \\
        &{} - \e_{x} \e_{\Change{y}{x}}
  &&+ \tfrac{1}{2}({\e_{x}}^{2} &&+ {\e_{y}}^{2} &&+ {\e_{z}}^{2})\Add{,}
\end{alignedat} \\
%
&\qquad\qquad\qquad\qquad\qquad
\begin{alignedat}{3}
p_{xy} = &{} - \h_{x} \h_{y}\quad
       && p_{xz} = && {} - \h_{x} \h_{z} \\
        &{} - \e_{x} \e_{y}\Add{,}\quad
       &&  && {} - \e_{x} \e_{z}\Add{,}
\end{alignedat} \\
%
%**** Hard-coded width
& \makebox[3.5in][c]{$\vdots$} \\
&b_{x} = \s_{x} = \e_{y}\h_{z} - \e_{z}\h_{y}\Add{,} \\
%[** TN: Added code for next two equations, elided in the original]
&b_{y} = \s_{y} = \e_{z}\h_{x} - \e_{x}\h_{z}\Add{,} \\
&b_{z} = \s_{z} = \e_{x}\h_{y} - \e_{y}\h_{x}\Add{,} \\
%
&\eta = +\tfrac{1}{2}({\e_{x}}^{2} + {\e_{y}}^{2} + {\e_{z}}^{2}
  + {\h_{x}}^{2} + {\h_{y}}^{2} + {\h_{z}}^{2})\Add{.}
\end{aligned}
\right\}
\Tag{(48a)}
\]
We conclude from~\Eqref{(48)} that the energy tensor of the
electromagnetic field is symmetrical; with this is connected
the fact that the momentum per unit volume and
the flow of energy are equal to each other (relation
between energy and inertia).

We therefore conclude from these considerations that
the energy per unit volume has the character of a tensor.
This has been proved directly only for an electromagnetic
field, although we may claim universal validity for it.
Maxwell's equations determine the electromagnetic field
when the distribution of electric charges and currents is
known. But we do not know the laws which govern
the currents and charges. We do know, indeed, that
electricity consists of elementary particles (electrons,
positive nuclei), but from a theoretical point of view we
cannot comprehend this. We do not know the energy
factors which determine the distribution of electricity in
particles of definite size and charge, and all attempts to
complete the theory in this direction have failed. If then
we can build upon Maxwell's equations in general, the
\PageSep{55}
energy tensor of the electromagnetic field is known only
outside the charged particles.\footnote
  {It has been attempted to remedy this lack of knowledge by considering
  the charged particles as proper singularities. But in my opinion this means
  giving up a real understanding of the structure of matter. It seems to me
  much better to give in to our present inability rather than to be satisfied
  by a solution that is only apparent.}
In these regions, outside
of charged particles, the only regions in which we can
believe that we have the complete expression for the
energy tensor, we have, by~\Eqref{(47)},
\[
\frac{\dd T_{\mu\nu}}{\dd x_{\nu}} = 0\Add{.}
\Tag{(47c)}
\]

\Paragraph{General Expressions for the Conservation Principles.} We
\index{Conservation principles}%
\index{Principles of conservation}%
can hardly avoid making the assumption that in all other
cases, also, the space distribution of energy is given by a
symmetrical tensor,~$T_{\mu\Change{}{\nu}}$, and that this complete energy
tensor everywhere satisfies the relation~\Eqref{(47c)}. At any
rate we shall see that by means of this assumption we
obtain the correct expression for the integral energy
principle.

Let us consider a spatially bounded, closed system,
which, four-dimensionally, we may represent as a strip,
outside of which the~$T_{\mu\nu}$ vanish. Integrate equation~\Eqref{(47c)}
over a space section. Since the integrals of
$\dfrac{\dd T_{\mu1}}{\dd x_{1}}$, $\dfrac{\dd T_{\mu2}}{\dd x_{2}}$ and $\dfrac{\dd T_{\mu3}}{\dd x_{3}}$ vanish because the~$T_{\mu\nu}$ vanish at the
limits of integration, we obtain
\[
\frac{\dd}{\dd l}\left\{\int T_{\mu4}\, dx_{1}\, dx_{2}\, dx_{3} \right\}
  = 0\Add{.}
\Tag{(49)}
\]
Inside the parentheses are the expressions for the
\PageSep{56}
momentum of the whole system, multiplied by~$i$, together
with the negative energy of the system, so that \Eqref{(49)}
expresses the conservation principles in their integral
form. That this gives the right conception of energy and
%[Illustration: Fig. 3.]
\Figure{056}
the conservation principles will be seen from the following
considerations.

\Section{Phenomenological Representation of the
Energy Tensor of Matter.}
\index{Energy@{Energy and mass}!tensor@{tensor of electromagnetic field}!of matter}%

\Paragraph{Hydrodynamical Equations.} We know that matter is
\index{Hydrodynamical equations}%
built up of electrically charged particles, but we do not
\PageSep{57}
know the laws which govern the constitution of these
particles. In treating mechanical problems, we are therefore
obliged to make use of an inexact description of
matter, which corresponds to that of classical mechanics.
The density~$\sigma$, of a material substance and the hydrodynamical
pressures are the fundamental concepts upon
which such a description is based.

Let~$\sigma_{0}$ be the density of matter at a place, estimated
with reference to a system of co-ordinates moving with
the matter. Then~$\sigma_{0}$, the density at rest, is an invariant.
If we think of the matter in arbitrary motion and neglect
the pressures (particles of dust \textit{in vacuo}, neglecting the
size of the particles and the temperature), then the energy
tensor will depend only upon the velocity components,
$u_{\nu}$~and~$\sigma_{0}$. We secure the tensor character of~$T_{\mu\nu}$ by
putting
\[
T_{\mu\nu} = \sigma_{0} u_{\mu} u_{\nu}\Add{,}
\Tag{(50)}
\]
in which the~$u_{\mu}$, in the three-dimensional representation,
are given by~\Eqref{(41)}. In fact, it follows from~\Eqref{(50)} that for
$q = 0$, $T_{44} = -\sigma_{0}$ (equal to the negative energy per unit
volume), as it should, according to the theorem of the
equivalence of mass and energy, and according to the
physical interpretation of the energy tensor given above.
If an external force (four-dimensional vector,~$K_{\mu}$) acts
upon the matter, by the principles of momentum and
energy the equation
\[
K_{\mu} = \frac{\dd T_{\mu\nu}}{\dd x_{\nu}}
\]
\PageSep{58}
must hold. We shall now show that this equation leads
to the same law of motion of a material particle as that
already obtained. Let us imagine the matter to be of
infinitely small extent in space, that is, a four-dimensional
thread; then by integration over the whole thread with
respect to the space co-ordinates $x_{1}$,~$x_{2}$,~$x_{3}$, we obtain
\begin{align*}
%[** TN: Moved equals sign to the second line]
\int K_{1}\, dx_{1}\, dx_{2}\, dx_{3}
  &= \int \frac{\dd T_{14}}{\dd x_{4}}\, dx_{1}\, dx_{2}\, dx_{3} \\
  &= -\Change{\veci}{i} \frac{d}{dl}\left\{
    \int \sigma_{0}\frac{dx_{1}}{d\tau}\, \frac{dx_{4}}{d\tau}\, dx_{1}\, dx_{2}\, dx_{3}
  \right\}.
\end{align*}

Now $\int dx_{1}\, dx_{2}\, dx_{3}\, dx_{4}$ is an invariant, as is, therefore, also
$\int \sigma_{0}\, dx_{1}\, dx_{2}\, dx_{3}\, dx_{4}$. We shall calculate this integral, first
with respect to the inertial system which we have chosen,
and second, with respect to a system relatively to which
the matter has the velocity zero. The integration is to
be extended over a filament of the thread for which $\sigma_{0}$
may be regarded as constant over the whole section. If
the space volumes of the filament referred to the two
systems are $dV$~and~$dV_{0}$ respectively, then we have
\[
\int \sigma_{0}\, dV\, dl = \int \sigma_{0}\, dV_{0}\, d\tau
\]
and therefore also
\[
\int \sigma_{0}\, dV = \int \sigma_{0}\, dV_{0}\, \frac{d\tau}{dl}
  = \int dm\, \Change{\veci}{i}\, \frac{d\tau}{dx_{4}}.
\]

If we substitute the right-hand side for the left-hand
side in the former integral, and put $\dfrac{dx_{1}}{d\tau}$ outside the sign
\PageSep{59}
of integration, we obtain,
\[
\K_{x} = \frac{d}{dl}\left(m \frac{dx_{1}}{d\tau}\right) \\
  = \frac{d}{dl}\left(\frac{m\q_{x}}{\sqrt{1 - q^{2}}}\right).
\]
We see, therefore, that the generalized conception of the
energy tensor is in agreement with our former result.

\Paragraph{The Eulerian Equations for Perfect Fluids.} In order
to get nearer to the behaviour of real matter we must add
to the energy tensor a term which corresponds to the
pressures. The simplest case is that of a perfect fluid in
which the pressure is determined by a scalar~$p$. Since
the tangential stresses $p_{xy}$,~etc., vanish in this case, the
contribution to the energy tensor must be of the form~$p\delta_{\nu\mu}$.
We must therefore put
\[
T_{\mu\nu} = \sigma u_{\mu} u_{\nu} + p\delta_{\mu\nu}\Add{.}
\Tag{(51)}
\]
At rest, the density of the matter, or the energy per unit
volume, is in this case, not~$\sigma$ but~$\sigma - p$. For
\[
-T_{44} = -\sigma \frac{dx_{4}}{d\tau}\, \frac{dx_{4}}{d\tau} - p\delta_{44}
  = \sigma - p.
\]
In the absence of any force, we have
\[
\frac{\dd T_{\mu\nu}}{\dd x_{\nu}}
  = \sigma u_{\nu} \frac{\dd u_{\mu}}{\dd x_{\nu}}
  + u_{\mu} \frac{\dd (\sigma u_{\nu})}{\dd x_{\nu}}
  + \frac{\dd p}{\dd x_{\mu}} = 0.
\]
If we multiply this equation by~$u_{\mu}$ $\left(= \dfrac{dx_{\mu}}{d\tau}\right)$ and sum for
the $\mu$'s~we obtain, using~\Eqref{(40)},
\[
-\frac{\dd (\sigma u_{\nu})}{\dd x_{\nu}} + \frac{dp}{d\tau} = 0\Add{,}
\Tag{(52)}
\]
\PageSep{60}
where we have put $\dfrac{\dd p}{\dd x_{\mu}}\, \dfrac{dx_{\mu}}{d\tau} = \dfrac{dp}{d\tau}$. This is the equation of
continuity, which differs from that of classical mechanics
by the term~$\dfrac{dp}{d\tau}$, which, practically, is vanishingly small.
Observing~\Eqref{(52)}, the conservation principles take the form
\[
\sigma \frac{du_{\mu}}{d\tau}
  + u_{\mu} \frac{dp}{d\tau}
  + \frac{\dd p}{\dd x_{\mu}} = 0\Add{.}
\Tag{(53)}
\]
The equations for the first three indices evidently correspond
to the Eulerian equations. That the equations
\Eqref{(52)}~and~\Eqref{(53)} correspond, to a first approximation, to the
hydrodynamical equations of classical mechanics, is a
further confirmation of the generalized energy principle.
The density of matter and of energy has the character of
a symmetrical tensor.
\PageSep{61}


\Lecture{III}{The General Theory of Relativity}{The General Theory}

\First{All} of the previous considerations have been based
upon the assumption that all inertial systems are
equivalent for the description of physical phenomena, but
that they are preferred, for the formulation of the laws
of nature, to spaces of reference in a different state of
motion. We can think of no cause for this preference
for definite states of motion to all others, according to
our previous considerations, either in the perceptible
bodies or in the concept of motion; on the contrary, it
must be regarded as an independent property of the
space-time continuum. The principle of inertia, in
particular, seems to compel us to ascribe physically
objective properties to the space-time continuum. Just
as it was necessary from the Newtonian standpoint to
make both the statements, \textit{tempus est absolutum}, \textit{spatium
est absolutum}, so from the standpoint of the special theory
of relativity we must say, \textit{continuum spatii et temporis est
absolutum}. In this latter statement \textit{absolutum} means not
only ``physically real,'' but also ``independent in its
physical properties, having a physical effect, but not itself
influenced by physical conditions.''

As long as the principle of inertia is regarded as the
\PageSep{62}
keystone of physics, this standpoint is certainly the only
one which is justified. But there are two serious criticisms
of the ordinary conception. In the first place, it is contrary
to the mode of thinking in science to conceive of a thing
(the space-time continuum) which acts itself, but which
cannot be acted upon. This is the reason why E.~Mach
\index{Mach}%
was led to make the attempt to eliminate space as an
active cause in the system of mechanics. According to
him, a material particle does not move in unaccelerated
motion relatively to space, but relatively to the centre of
all the other masses in the universe; in this way the
series of causes of mechanical phenomena was closed, in
contrast to the mechanics of Newton and Galileo. In
order to develop this idea within the limits of the modern
theory of action through a medium, the properties of
the space-time continuum which determine inertia must
be regarded as field properties of space, analogous to
the electromagnetic field. The concepts of classical
mechanics afford no way of expressing this. For this
reason Mach's attempt at a solution failed for the time
being. We shall come back to this point of view later.
In the second place, classical mechanics indicates a
limitation which directly demands an extension of the
principle of relativity to spaces of reference which are not
in uniform motion relatively to each other. The ratio of
the masses of two bodies is defined in mechanics in two
ways which differ from each other fundamentally; in the
first place, as the reciprocal ratio of the accelerations
which the same motional force imparts to them (inert
mass), and in the second place, as the ratio of the forces
\PageSep{63}
which act upon them in the same gravitational field
(gravitational mass). The equality of these two masses,
\index{Gravitational mass}%
\index{Inert and gravitational mass, equality of}%
\index{Mass and Energy!equality of gravitational and inert}%
\index{Mass and Energy!gravitational}%
so differently defined, is a fact which is confirmed by
experiments of very high accuracy (experiments of E�tv�s),
and classical mechanics offers no explanation for this
equality. It is, however, clear that science is fully justified
in assigning such a numerical equality only after this
numerical equality is reduced to an equality of the real
nature of the two concepts.

That this object may actually be attained by an extension
of the principle of relativity, follows from the following
consideration. A little reflection will show that the
theorem of the equality of the inert and the gravitational
mass is equivalent to the theorem that the acceleration
imparted to a body by a gravitational field is independent
of the nature of the body. For Newton's equation of
motion in a gravitational field, written out in full, is
\begin{multline*}
(\text{Inert mass})�(\text{Acceleration}) = (\text{Intensity of the} \\
  \text{gravitational field})�(\text{Gravitational mass}).
\end{multline*}
It is only when there is numerical equality between the
inert and gravitational mass that the acceleration is independent
of the nature of the body. Let now $K$~be an
inertial system. Masses which are sufficiently far from
each other and from other bodies are then, with respect
to~$K$, free from acceleration. We shall also refer these
masses to a system of co-ordinates~$K'$, uniformly accelerated
with respect to~$K$. Relatively to~$K'$ all the masses
have equal and parallel accelerations; with respect to~$K'$
they behave just as if a gravitational field were present and
\PageSep{64}
$K'$~were unaccelerated. Overlooking for the present the
question as to the ``cause'' of such a gravitational field,
which will occupy us later, there is nothing to prevent
our conceiving this gravitational field as real, that is, the
conception that $K'$~is ``at rest'' and a gravitational field
is present we may consider as equivalent to the conception
that only $K$~is an ``allowable'' system of co-ordinates
and no gravitational field is present. The assumption of
the complete physical equivalence of the systems of co-ordinates,
$K$~and~$K'$, we call the ``principle of equivalence;''
\index{Principle@{Principle of equivalence}}%
this principle is evidently intimately connected
with the theorem of the equality between the inert and
the gravitational mass, and signifies an extension of the
principle of relativity to co-ordinate systems which are
in non-uniform motion relatively to each other. In fact,
through this conception we arrive at the unity of the
nature of inertia and gravitation. For according to our
way of looking at it, the same masses may appear to be
either under the action of inertia alone (with respect to~$K$)
or under the combined action of inertia and gravitation
(with respect to~$K'$). The possibility of explaining
the numerical equality of inertia and gravitation by the
unity of their nature gives to the general theory of
relativity, according to my conviction, such a superiority
over the conceptions of classical mechanics, that all the
difficulties encountered in development must be considered
as small in comparison.

What justifies us in dispensing with the preference
for inertial systems over all other co-ordinate systems, a
preference that seems so securely established by experiment
\PageSep{65}
based upon the principle of inertia? The weakness
\index{Criticism of principle of inertia}%
\index{Principle@{Principle of equivalence}!inertia, criticism of}%
of the principle of inertia lies in this, that it involves an
argument in a circle: a mass moves without acceleration
if it is sufficiently far from other bodies; we know that
it is sufficiently far from other bodies only by the fact
that it moves without acceleration. Are there, in general,
any inertial systems for very extended portions of the
space-time continuum, or, indeed, for the whole universe?
We may look upon the principle of inertia as established,
to a high degree of approximation, for the space of our
planetary system, provided that we neglect the perturbations
due to the sun and planets. Stated more exactly,
there are finite regions, where, with respect to a suitably
chosen space of reference, material particles move freely
without acceleration, and in which the laws of the special
theory of relativity, which have been developed above,
hold with remarkable accuracy. Such regions we shall
call ``Galilean regions.'' We shall proceed from the
\index{Galilean@{Galilean regions}}%
consideration of such regions as a special case of known
properties.

The principle of equivalence demands that in dealing
with Galilean regions we may equally well make use of
non-inertial systems, that is, such co-ordinate systems as,
relatively to inertial systems, are not free from acceleration
and rotation. If, further, we are going to do away
completely with the difficult question as to the objective
reason for the preference of certain systems of co-ordinates,
then we must allow the use of arbitrarily moving systems
of co-ordinates. As soon as we make this attempt seriously
\PageSep{66}
\index{Rotation}%
we come into conflict with that physical interpretation of
space and time to which we were led by the special theory
of relativity. For let $K'$ be a system of co-ordinates whose
$z'$-axis coincides with the $z$-axis of~$K$, and which rotates
about the latter axis with constant angular velocity. Are
the configurations of rigid bodies, at rest relatively to~$K'$,
in accordance with the laws of Euclidean geometry?
Since $K'$~is not an inertial system, we do not know
directly the laws of configuration of rigid bodies with
respect to~$K'$, nor the laws of nature, in general. But
we do know these laws with respect to the inertial system~$K$,
and we can therefore estimate them with respect to~$K'$.
Imagine a circle drawn about the origin in the \Change{$x'y'$}{$x'$-$y'$}~plane
of~$K'$, and a diameter of this circle. Imagine, further, that
we have given a large number of rigid rods, all equal to
each other. We suppose these laid in series along the
periphery and the diameter of the circle, at rest relatively
to~$K'$. If $U$~is the number of these rods along the periphery,
$D$~the number along the diameter, then, if $K'$~does
not rotate relatively to~$K$, we shall have
\[
\frac{U}{D} = \pi.
\]
But if $K'$~rotates we get a different result. Suppose
that at a definite time~$t$\Change{,}{} of~$K$ we determine the ends of
all the rods. With respect to~$K$ all the rods upon the
periphery experience the Lorentz contraction, but the
rods upon the diameter do not experience this contraction
\PageSep{67}
(along their lengths!).\footnote
  {These considerations assume that the behaviour of rods and clocks
  depends only upon velocities, and not upon accelerations, or, at least, that
  the influence of acceleration does not counteract that of velocity.}
It therefore follows that
\[
\frac{U}{D} > \pi.
\]

It therefore follows that the laws of configuration of
rigid bodies with respect to~$K'$ do not agree with the
laws of configuration of rigid bodies that are in accordance
with Euclidean geometry. If, further, we place two
similar clocks (rotating with~$K'$), one upon the periphery,
and the other at the centre of the circle, then, judged
from~$K$, the clock on the periphery will go slower than
the clock at the centre. The same thing must take place,
judged from~$K'$, if we define time with respect to~$K'$ in
a not wholly unnatural way, that is, in such a way that
the laws with respect to~$K'$ depend explicitly upon the
time. Space and time, therefore, cannot be defined
with respect to~$K'$ as they were in the special theory of
relativity with respect to inertial systems. But, according
to the principle of equivalence, $K'$~is also to be considered
as a system at rest, with respect to which there
is a gravitational field (field of centrifugal force, and
\index{Centrifugal force}%
force of Coriolis). We therefore arrive at the result:
the gravitational field influences and even determines the
metrical laws of the space-time continuum. If the laws
of configuration of ideal rigid bodies are to be expressed
geometrically, then in the presence of a gravitational
field the geometry is not Euclidean.
\PageSep{68}

The case that we have been considering is analogous
to that which is presented in the two-dimensional treatment
of surfaces. It is impossible in the latter case
also, to introduce co-ordinates on a surface (e.g.~the
surface of an ellipsoid) which have a simple metrical
significance, while on a plane the Cartesian co-ordinates,
$x_{1}$,~$x_{2}$, signify directly lengths measured by a unit
measuring rod. Gauss overcame this difficulty, in his
\index{Gauss}%
theory of surfaces, by introducing curvilinear co-ordinates
\index{Curvilinear co-ordinates}%
which, apart from satisfying conditions of continuity,
were wholly arbitrary, and afterwards these co-ordinates
were related to the metrical properties of the surface.
In an analogous way we shall introduce in the general
theory of relativity arbitrary co-ordinates, $x_{1}$,~$x_{2}$,~$x_{3}$,~$x_{4}$,
which shall number uniquely the space-time points, so
that neighbouring events are associated with neighbouring
values of the co-ordinates; otherwise, the choice of
co-ordinates is arbitrary. We shall be true to the
principle of relativity in its broadest sense if we give
such a form to the laws that they are valid in every
such four-dimensional system of co-ordinates, that is, if
the equations expressing the laws are co-variant with
respect to arbitrary transformations.

The most important point of contact between Gauss's
theory of surfaces and the general theory of relativity
lies in the metrical properties upon which the concepts
of both theories, in the main, are based. In the case
of the theory of surfaces, Gauss's argument is as follows.
Plane geometry may be based upon the concept of the
distance~$ds$, between two indefinitely near points. The
\PageSep{69}
concept of this distance is physically significant because
the distance can be measured directly by means of a
rigid measuring rod. By a suitable choice of Cartesian
co-ordinates this distance may be expressed by the
formula $ds^{2} = {dx_{1}}^{2} + {dx_{2}}^{2}$. We may base upon this
quantity the concepts of the straight line as the geodesic
($\delta\! \int\!\! ds = 0$), the interval, the circle, and the angle, upon
which the Euclidean plane geometry is built. A
geometry may be developed upon another continuously
curved surface, if we observe that an infinitesimally
small portion of the surface may be regarded as plane,
to within relatively infinitesimal quantities. There are
Cartesian co-ordinates, $X_{1}$,~$X_{2}$, upon such a small
portion of the surface, and the distance between two
points, measured by a measuring rod, is given by
\[
ds^{2} = {dX_{1}}^{2} + {dX_{2}}^{2}.
\]
If we introduce arbitrary curvilinear co-ordinates, $x_{1}$,~$x_{2}$,
on the surface, then $dX_{1}$,~$dX_{2}$, may be expressed linearly
in terms of $dx_{1}$,~$dx_{2}$. Then everywhere upon the surface
we have
\[
ds^{2} = g_{11}\, {dx_{1}}^{2} + 2g_{12}\, dx_{1}\, dx_{2}
     + g_{22}\, {dx_{2}}^{2}\Add{,}
\]
where $g_{11}$,~$g_{12}$,~$g_{22}$ are determined by the nature of the
surface and the choice of co-ordinates; if these quantities
are known, then it is also known how networks of rigid
rods may be laid upon the surface. In other words, the
geometry of surfaces may be based upon this expression
for~$ds^{2}$ exactly as plane geometry is based upon the
corresponding expression.

There are analogous relations in the four-dimensional
\PageSep{70}
space-time continuum of physics. In the immediate
neighbourhood of an observer, falling freely in a gravitational
field, there exists no gravitational field. We
can therefore always regard an infinitesimally small
region of the space-time continuum as Galilean. For
such an infinitely small region there will be an inertial
system (with the space co-ordinates, $X_{1}$,~$X_{2}$,~$X_{3}$, and the
time co-ordinate~$X_{4}$) relatively to which we are to regard
the laws of the special theory of relativity as valid. The
quantity which is directly measurable by our unit
measuring rods and clocks,
\[
{dX_{1}}^{2} + {dX_{2}}^{2} + {dX_{3}}^{2} - {dX_{4}}^{2}\Add{,}
\]
or its negative,
\[
ds^{2} = -{dX_{1}}^{2} - {dX_{2}}^{2} - {dX_{3}}^{2} + {dX_{4}}^{2}\Add{,}
\Tag{(54)}
\]
is therefore a uniquely determinate invariant for two
neighbouring events (points in the four-dimensional
continuum), provided that we use measuring rods that
are equal to each other when brought together and
superimposed, and clocks whose rates are the same
when they are brought together. In this the physical
assumption is essential that the relative lengths of two
measuring rods and the relative rates of two clocks are
independent, in principle, of their previous history. But
this assumption is certainly warranted by experience;
if it did not hold there could be no sharp spectral lines;
for the single atoms of the same element certainly do
not have the same history, and it would be absurd to
suppose any relative difference in the structure of the
\PageSep{71}
single atoms due to their previous history if the mass
and frequencies of the single atoms of the same element
were always the same.

Space-time regions of finite extent are, in general,
not Galilean, so that a gravitational field cannot be done
away with by any choice of co-ordinates in a finite
region. There is, therefore, no choice of co-ordinates
for which the metrical relations of the special theory of
relativity hold in a finite region. But the invariant~$ds$
always exists for two neighbouring points (events) of
the continuum. This invariant~$ds$ may be expressed in
arbitrary co-ordinates. If one observes that the local~$dX_{\nu}$
may be expressed linearly in terms of the co-ordinate
differentials~$dx_{\nu}$, $ds^{2}$~may be expressed in the
form
\[
ds^{2} = g_{\mu\nu}\, dx_{\mu}\, dx_{\nu}\Add{.}
\Tag{(55)}
\]

The functions~$g_{\mu\nu}$ describe, with respect to the arbitrarily
chosen system of co-ordinates, the metrical relations
of the space-time continuum and also the
gravitational field. As in the special theory of relativity,
we have to discriminate between time-like and space-like
line elements in the four-dimensional continuum;
owing to the change of sign introduced, time-like
line elements have a real, space-like line elements an
imaginary~$ds$. The time-like~$ds$ can be measured directly
by a suitably chosen clock.

According to what has been said, it is evident that
the formulation of the general theory of relativity
assumes a generalization of the theory of invariants and
the theory of tensors; the question is raised as to the
\PageSep{72}
\index{Tensor|etseq}%
form of the equations which are co-variant with respect
\index{Covariant@{Co-variant}!vector}%
to arbitrary point transformations. The generalized
calculus of tensors was developed by mathematicians
long before the theory of relativity. Riemann first
\index{Riemann}%
extended Gauss's train of thought to continua of any
number of dimensions; with prophetic vision he saw
the physical meaning of this generalization of Euclid's
geometry. Then followed the development of the theory
in the form of the calculus of tensors, particularly by
Ricci and Levi-Civita. This is the place for a brief
presentation of the most important mathematical concepts
and operations of this calculus of tensors.

We designate four quantities, which are defined as
functions of the~$x_{\nu}$ with respect to every system of co-ordinates,
as components,~$A^{\nu}$, of a contra-variant vector,
\index{Contravariant@{Contra-variant vectors}}%
\index{Vector@{Vector, co-variant}!contra-variant}%
if they transform in a change of co-ordinates as the co-ordinate
differentials~$dx_{\nu}$. We therefore have
\[
%[** TN: Place prime directly on the tensor name throughout, for consistency]
\Change{A^{\mu'}}{{A'}^{\mu}} = \frac{{\dd x'}_{\mu}}{\dd x_{\nu}} A^{\nu}\Add{.}
\Tag{(56)}
\]
Besides these contra-variant vectors, there are also co-variant
vectors. If $B_{\nu}$~are the components of a co-variant
vector, these vectors are transformed according to the
rule
\[
{B'}_{\mu} = \frac{\dd x_{\nu}}{{\dd \Change{x}{x'}}_{\mu}} B_{\nu}\Add{.}
\Tag{(57)}
\]
The definition of a co-variant vector is chosen in such a
\index{Vector@{Vector, co-variant}}%
way that a co-variant vector and a contra-variant vector
together form a scalar according to the scheme,
\[
\phi = B_{\nu} A^{\nu}\quad \text{(summed over the~$\nu$).}
\]
\PageSep{73}
Accordingly,
\[
{B'}_{\mu} \Change{A^{\mu'}}{{A'}^{\mu}}
  = \frac{\dd x_{\alpha}}{{\dd x'}_{\mu}}
    \frac{{\dd x'}_{\mu}}{\dd x_{\beta}} B_{\alpha} A^{\beta}
  = B_{\alpha} A^{\alpha}\Add{.}
\]
In particular, the derivatives~$\dfrac{\dd \phi}{\dd x_{\alpha}}$ of a scalar~$\phi$, are components
of a co-variant vector, which, with the co-ordinate
differentials, form the scalar~$\dfrac{\dd \phi}{\dd x_{\alpha}}\, dx_{\alpha}$; we see from this
example how natural is the definition of the co-variant
vectors.

There are here, also, tensors of any rank, which may
have co-variant or contra-variant character with respect
to each index; as with vectors, the character is designated
by the position of the index. For example, $\Change{{A_{\mu}}^{\nu}}{A_{\mu}^{\nu}}$~denotes
a tensor of the second rank, which is co-variant
with respect to the index~$\mu$, and contra-variant with respect
to the index~$\nu$. The tensor character indicates
that the equation of transformation is
\[
\Change{A_{\mu}^{\nu'}}{{A'}_{\mu}^{\nu}}
  = \frac{\dd x_{\alpha}}{{\dd x'}_{\mu}}
    \frac{{\dd x'}_{\nu}}{\dd x_{\beta}} A_{\alpha}^{\beta}\Add{.}
\Tag{(58)}
\]

Tensors may be formed by the addition and subtraction
of tensors of equal rank and like character, as in the
theory of invariants of orthogonal linear substitutions, for
example,
\[
A_{\mu}^{\nu} + B_{\mu}^{\nu} = C_{\mu}^{\nu}\Add{.}
\Tag{(59)}
\]
The proof of the tensor character of~$C_{\mu}^{\nu}$ depends upon~\Eqref{(58)}.

Tensors may be formed by multiplication, keeping the
character of the indices, just as in the theory of invariants
of linear orthogonal transformations, for example,
\[
A_{\mu}^{\nu} B_{\sigma\tau} = C_{\mu\sigma\tau}^{\nu}\Add{.}
\Tag{(60)}
\]
\PageSep{74}
The proof follows directly from the rule of transformation.

Tensors may be formed by contraction with respect to
two indices of different character, for example,
\[
A_{\mu\sigma\tau}^{\mu} = B_{\sigma\tau}\Add{.}
\Tag{(61)}
\]
The tensor character of~$A_{\mu\sigma\tau}^{\mu}$ determines the tensor
character of~$B_{\sigma\tau}$. Proof---
\[
\Change{A_{\mu\sigma\tau}^{\mu'}}{{A'}_{\mu\sigma\tau}^{\mu}}
= \frac{\dd x_{\alpha}}{{\dd x'}_{\mu}} \frac{{\dd x'}_{\mu}}{\dd x_{\beta}}
  \frac{\dd x_{s}}{{\dd x'}_{\sigma}} \frac{\dd x_{t}}{{\dd x'}_{\tau}}
  \Change{}{A_{\alpha st}^{\beta}}
= \frac{\dd x_{s}}{{\dd x'}_{\sigma}} \frac{\dd x_{t}}{{\dd x'}_{\tau}}
  A_{\alpha st}^{\alpha}.
\]

The properties of symmetry and skew-symmetry of a
tensor with respect to two indices of like character have
the same significance as in the theory of invariants.

With this, everything essential has been said with
regard to the algebraic properties of tensors.

\Paragraph{The Fundamental Tensor.} It follows from the invariance
\index{Fundamental tensor}%
\index{Tensor!Fundamental}%
of~$ds^{2}$ for an arbitrary choice of the~$dx_{\nu}$, in connexion
with the condition of symmetry consistent with~\Eqref{(55)}, that
the~$g_{\mu\nu}$ are components of a symmetrical co-variant tensor
(Fundamental Tensor). Let us form the determinant,~$g$,
of the~$g_{\mu\nu}$, and also the minors, divided by~$g$, corresponding
to the single~$g_{\mu\nu}$. These minors, divided by~$g$,
will be denoted by~$g^{\mu\nu}$, and their co-variant character
is not yet known. Then we have
\[
g_{\mu\alpha} g^{\mu\beta} = \delta_{\alpha}^{\beta}
%[** TN: No brace in original]
  = \begin{cases}
    1 & \text{if $\alpha = \beta$\Add{,}} \\
    0 & \text{if $\alpha \neq \beta$\Add{.}}
  \end{cases}
\Tag{(62)}
\]

If we form the infinitely small quantities (co-variant
vectors)
\[
d\xi_{\mu} = g_{\mu\alpha}\, dx_{\alpha}\Add{,}
\Tag{(63)}
\]
\PageSep{75}
multiply by~$g^{\mu\beta}$ and sum over the~$\mu$, we obtain, by the
use of~\Eqref{(62)},
\[
dx_{\beta} = g^{\beta\mu}\, d\xi_{\mu}\Add{.}
\Tag{(64)}
\]
Since the ratios of the~$d\xi_{\mu}$ are arbitrary, and the~$dx_{\beta}$ as
well as the~$dx_{\mu}$ are components of vectors, it follows that
\index{Contravariant@{Contra-variant vectors}!tensors}%
the~$g^{\mu\nu}$ are the components of a contra-variant tensor\footnote
  {If we multiply~\Eqref{(64)} by~$\dfrac{{\dd x'}_{\alpha}}{\dd x_{\beta}}$, sum over the~$\beta$, and replace the~$d\xi^{\mu}$ by a
  transformation to the accented system, we obtain
  \[
  {dx'}_{\alpha}
    = \frac{{\dd x'}_{\sigma}}{\dd x_{\mu}}\,
      \frac{{\dd x'}_{\alpha}}{\dd x_{\beta}}\, g^{\mu\beta}\, {d\xi'}_{\sigma}.
  \]
  The statement made above follows from this, since, by~\Eqref{(64)}, we must also
  have ${dx'}_{\alpha} = g^{\sigma\alpha'}\, {d\xi'}_{\alpha}$, and both equations must hold for every choice of the~${d\xi'}_{\sigma}$.}
(contra-variant fundamental tensor). The tensor character
of~$\delta_{\alpha}^{\beta}$ (mixed fundamental tensor) accordingly follows,
by~\Eqref{(62)}. By means of the fundamental tensor, instead
of tensors with co-variant index character, we can
introduce tensors with contra-variant index character,
and conversely. For example,
\begin{align*}
A^{\mu} &= g^{\mu\alpha} A_{\alpha}\Add{,} \\
A_{\mu} &= g_{\mu\alpha} A^{\alpha}\Add{,} \\
T_{\mu}^{\sigma} &= g^{\sigma\nu} T_{\mu\nu}.
\end{align*}

\Paragraph{Volume Invariants.} The volume element
\[
\int dx_{1}\, dx_{2}\, dx_{3}\, dx_{4} = dx
\]
is not an invariant. For by Jacobi's theorem,
\[
dx' = \left| \frac{{dx'}_{\mu}}{dx_{\nu}}\right| dx\Add{.}
\Tag{(65)}
\]
\PageSep{76}
But we can complement~$dx$ so that it becomes an invariant.
If we form the determinant of the quantities
\[
{g'}_{\mu\nu}
  = \frac{\dd x_{\alpha}}{{\dd x'}_{\mu}}\,
    \frac{\dd x_{\beta}}{{\dd x'}_{\nu}}\, g_{\alpha\beta}\Add{,}
\]
we obtain, by a double application of the theorem of
multiplication of determinants,
\[
g' = |{g'}_{\mu\nu}|
  = \left|\frac{\dd x_{\nu}}{{\dd x'}_{\mu}}\right|^{2}�|g_{\mu\nu}|
  = \left|\frac{{\dd x'}_{\mu}}{\dd x_{\nu}}\right|^{-2} g.
\Change{}{\Tag{(66)}}
\]
We therefore get the invariant,
\[
\sqrt{g'}\, dx' = \sqrt{g\vphantom{g'}}\, dx.
\]

\Paragraph{Formation of Tensors by Differentiation.} Although
\index{Differentiation of tensors}%
\index{Tensors, formation by differentiation}%
the algebraic operations of tensor formation have proved
to be as simple as in the special case of invariance with
respect to linear orthogonal transformations, nevertheless
in the general case, the invariant differential operations
are, unfortunately, considerably more complicated. The
reason for this is as follows. If $A^{\mu}$~is a contra-variant
vector, the coefficients of its transformation,~$\dfrac{{\dd x'}_{\mu}}{\dd x_{\nu}}$, are independent
of position only if the transformation is a linear
one. For then the vector components, $A^{\mu} + \dfrac{\dd A^{\mu}}{\dd x_{\alpha}}\, dx_{\alpha}$, at
a neighbouring point transform in the same way as the~$A^{\mu}$,
from which follows the vector character of the vector
differentials, and the tensor character of~$\dfrac{\dd A^{\mu}}{\dd x_{\alpha}}$. But if the
$\dfrac{{\dd x'}_{\mu}}{\dd x_{\nu}}$ are variable this is no longer true.
\PageSep{77}
\index{Hypotheses of pre-relativity physics}%

That there are, nevertheless, in the general case, invariant
differential operations for tensors, is recognized
most satisfactorily in the following way, introduced by
Levi-Civita and Weyl. Let $(A^{\mu})$ be a contra-variant vector
\index{Levi-Civita}%
\index{Weyl}%
whose components are given with respect to the co-ordinate
system of the~$x_{\nu}$. Let $P_{1}$~and~$P_{2}$ be two infinitesimally
near points of the continuum. For the
infinitesimal region surrounding the point~$P_{1}$, there is,
according to our way of considering the matter, a co-ordinate
system of the~$X_{\nu}$ (with imaginary $X_{\Change{\nu}{4}}$-\Change{co-ordinates}{co-ordinate})
for which the continuum is Euclidean. Let
$A_{(1)}^{\mu}$~be the co-ordinates of the vector at the point~$P_{1}$.
Imagine a vector drawn at the point~$P_{2}$, using the local
system of the~$X_{\nu}$, with the same co-ordinates (parallel
vector through~$P_{2}$), then this parallel vector is uniquely
determined by the vector at~$P_{1}$ and the displacement.
We designate this operation, whose uniqueness will appear
in the sequel, the parallel displacement of the vector~$A_{\mu}$
from~$P_{1}$ to the infinitesimally near point~$P_{2}$\Change{}{.} If we form
the vector difference of the vector~$(A^{\mu})$ at the point~$P_{2}$
and the vector obtained by parallel displacement from~$P_{1}$
to~$P_{2}$, we get a vector which may be regarded as the
differential of the vector~$(A^{\mu})$ for the given displacement~$(dx_{\nu})$.

This vector displacement can naturally also be considered
with respect to the co-ordinate system of the~$x_{\nu}$.
If $A^{\nu}$~are the co-ordinates of the vector at~$P_{1}$, $A^{\nu} + \delta A^{\nu}$
the co-ordinates of the vector displaced to~$P_{2}$ along the
interval~$(dx_{\nu})$, then the~$\delta A^{\nu}$ do not vanish in this case.
We know of these quantities, which do not have a vector
\PageSep{78}
character, that they must depend linearly and homogeneously
upon the~$dx_{\nu}$ and the~$A^{\nu}$. We therefore put
\[
\delta A^{\nu} = -\Gamma_{\alpha\beta}^{\nu} A^{\alpha}\, dx_{\beta}\Add{.}
\Tag{(67)}
\]

In addition, we can state that the~$\Gamma_{\alpha\beta}^{\nu}$ must be symmetrical
with respect to the indices $\alpha$~and~$\beta$. For we
can assume from a representation by the aid of a Euclidean
system of local co-ordinates that the same parallelogram
will be described by the displacement of an element~$d^{(1)}x_{\nu}$
along a second element~$d^{(2)}x_{\nu}$ as by a displacement
of~$d^{(2)}x_{\nu}$ along~$d^{(1)}x_{\nu}$. We must therefore have
\begin{multline*}
d^{(2)}x_{\nu} + (d^{(1)}x_{\nu}
  - \Gamma_{\alpha\beta}^{\nu}\, d^{(1)}x_{\alpha}\, d^{(2)}x_{\beta}) \\
  = d^{(1)}x_{\nu} + (d^{(2)}x_{\nu}
  - \Gamma_{\alpha\beta}^{\nu}\, d^{(2)}x_{\alpha}\, d^{(1)}x_{\beta}).
\end{multline*}
The statement made above follows from this, after interchanging
the indices of summation, $\alpha$~and~$\beta$, on the
right-hand side.

Since the quantities~$g_{\mu\nu}$ determine all the metrical
properties of the continuum, they must also determine
the~$\Gamma_{\alpha\beta}^{\nu}$. If we consider the invariant of the vector~$A^{\nu}$,
that is, the square of its magnitude,
\[
g_{\mu\nu} A^{\mu} A^{\nu}\Add{,}
\]
which is an invariant, this cannot change in a parallel
displacement. We therefore have
\[
0 = \delta(g_{\mu\nu} A^{\mu} A^{\nu})
  = \frac{\dd g_{\mu\nu}}{\dd x_{\alpha}} A^{\mu} A^{\nu}\, dx_{\alpha}
  + g_{\mu\nu} A^{\mu} \delta A^{\nu} + g_{\mu\nu} A^{\nu} \delta A^{\mu}
\]
or, by~\Eqref{(67)},
\[
\left(\frac{\dd g_{\mu\nu}}{\dd x_{\alpha}}
  - g_{\mu\beta} \Gamma_{\nu\alpha}^{\beta}
  - g_{\nu\beta} \Gamma_{\mu\alpha}^{\beta}\right) A^{\mu} A^{\nu}\, dx_{\alpha}
= 0.
\]
\PageSep{79}

Owing to the symmetry of the expression in the
brackets with respect to the indices $\mu$~and~$\nu$, this equation
can be valid for an arbitrary choice of the vectors $(A^{\mu})$
and~$dx_{\nu}$ only when the expression in the brackets vanishes
for all combinations of the indices. By a cyclic interchange
of the indices $\mu$,~$\nu$,~$\alpha$, we obtain thus altogether
three equations, from which we obtain, on taking into
account the symmetrical property of the~$\Gamma_{\mu\nu}^{\alpha}$,
\[
\left[\Chr{\mu}{\nu}{\alpha}\right]
  = g_{\alpha\beta} \Gamma_{\mu\nu}^{\beta}\Add{,}
\Tag{(68)}
\]
in which, following Christoffel, the abbreviation has been
used,
\[
\left[\Chr{\mu}{\nu}{\alpha}\right]
  = \tfrac{1}{2}\left(
    \frac{\dd g_{\mu\alpha}}{\dd x_{\nu}}
  + \frac{\dd g_{\nu\alpha}}{\dd x_{\mu}}
  - \frac{\dd g_{\mu\nu}}{\dd x_{\alpha}}
\right)\Add{.}
\Tag{(69)}
\]

If we multiply~\Eqref{(68)} by~$g^{\alpha\sigma}$ and sum over the~$\alpha$, we
obtain
\[
\Gamma_{\mu\nu}^{\alpha}
  = \tfrac{1}{2} g^{\sigma\alpha}\left(
    \frac{\dd g_{\mu\alpha}}{\dd x_{\nu}}
  + \frac{\dd g_{\nu\alpha}}{\dd x_{\mu}}
  - \frac{\dd g_{\mu\nu}}{\dd x_{\alpha}}
\right) = \left\{\Chr{\mu}{\nu}{\sigma}\right\}\Add{,}
\Tag{(70)}
\]
in which $\left\{\Chr{\mu}{\nu}{\sigma}\right\}$ is the Christoffel symbol of the second
kind. Thus the quantities~$\Gamma$ are deduced from the~$g_{\mu\nu}$.
Equations \Eqref{(67)}~and~\Eqref{(70)} are the foundation for the
following discussion.

\Paragraph{Co-variant Differentiation of Tensors.} If $(A^{\mu} + \delta A^{\mu})$ is
\index{Differentiation of tensors}%
the vector resulting from an infinitesimal parallel displacement
from~$P_{1}$ to~$P_{2}$, and $(A^{\mu} + dA^{\mu})$~the vector~$A^{\mu}$ at the
point~$P_{2}$, then the difference of these two,
\[
dA^{\mu} - \delta A^{\mu} = \left(
  \frac{\Change{\delta}{\dd} A^{\mu}}{\Change{\delta}{\dd} x_{\sigma}}
  + \Gamma_{\sigma\alpha}^{\mu} A^{\alpha}\right) dx_{\sigma}\Add{,}
\]
\PageSep{80}
is also a vector. Since this is the case for an arbitrary
choice of the~$dx_{\sigma}$, it follows that
\[
%[** TN: Covariant derivative index not subscripted in original]
{A^{\mu}}_{;\, \sigma}
  = \frac{\dd A^{\mu}}{\dd x_{\sigma}} + \Gamma_{\sigma\alpha}^{\mu} A^{\alpha}
\Tag{(71)}
\]
is a tensor, which we designate as the co-variant derivative
of the tensor of the first rank (vector). Contracting this
tensor, we obtain the divergence of the contra-variant
tensor~$A^{\mu}$. In this we must observe that according to~\Eqref{(70)},
\[
\Gamma_{\mu\sigma}^{\sigma}
  = \tfrac{1}{2} g^{\sigma\alpha} \frac{\dd g_{\sigma\alpha}}{\dd x_{\mu}}
  = \frac{1}{\sqrt{g}}\, \frac{\dd \sqrt{g}}{\dd x_{\mu}}\Add{.}
\Tag{(72)}
\]
If we put, further,
\[
A^{\mu} \sqrt{g} = \tA^{\mu}\Add{,}
\Tag{(73)}
\]
a quantity designated by Weyl as the contra-variant tensor
density\footnote
  {This expression is justified, in that $A^{\mu}\sqrt{g}\, dx = \tA^{\mu}\, dx$ has a tensor
  character. Every tensor, when multiplied by~$\sqrt{g}$, changes into a tensor
  density. We employ capital Gothic letters for tensor densities.}
of the first rank, it follows that,
\[
\tA = \frac{\dd \tA^{\mu}}{\dd x_{\mu}}
\Tag{(74)}
\]
is a scalar density.

We get the law of parallel displacement for the
co-variant vector~$B_{\mu}$ by stipulating that the parallel
displacement shall be effected in such a way that the
scalar
\[
\phi = A^{\mu} B_{\mu}
\]
remains unchanged, and that therefore
\[
A^{\mu}\, \delta B_{\mu} + B_{\mu}\, \delta A^{\mu}
\]
\PageSep{81}
vanishes for every value assigned to~$(A^{\mu})$. We therefore
get
\[
\delta B_{\mu} = \Gamma_{\mu\sigma}^{\alpha} A_{\alpha}\, dx_{\sigma}\Add{.}
\Tag{(75)}
\]

From this we arrive at the co-variant derivative of the
co-variant vector by the same process as that which led
to~\Eqref{(71)},
\[
%[** TN: Covariant derivative index not subscripted in original]
B_{\mu;\, \sigma}
  = \frac{\dd B_{\mu}}{\dd x_{\sigma}} - \Gamma_{\mu\sigma}^{\alpha} B_{\alpha}\Add{.}
\Tag{(76)}
\]
By interchanging the indices $\mu$~and~$\beta$, and subtracting,
we get the skew-symmetrical tensor,
\[
\phi_{\mu\sigma}
  = \frac{\dd B_{\mu}}{\dd x_{\sigma}} - \frac{\dd B_{\sigma}}{\dd x_{\mu}}\Add{.}
\Tag{(77)}
\]

For the co-variant differentiation of tensors of the
second and higher ranks we may use the process by
which \Eqref{(75)}~was deduced. Let, for example, $(A_{\sigma\tau})$~be a
co-variant tensor of the second rank. Then $A_{\sigma\tau} E^{\sigma} F^{\tau}$~is
a scalar, if $E$~and~$F$ are vectors. This expression must
not be changed by the $\delta$-displacement; expressing this
by a formula, we get, using~\Eqref{(67)},~$\delta A_{\sigma\tau}$, whence we get the
desired co-variant derivative,
\[
A_{\sigma\tau;\, \rho}
  = \frac{\dd A_{\sigma\tau}}{\dd x_{\rho}}
  - \Gamma_{\sigma\rho}^{\alpha} A_{\alpha\tau}
  - \Gamma_{\tau\rho}^{\alpha} A_{\sigma\alpha}\Add{.}
\Tag{(78)}
\]

In order that the general law of co-variant differentiation
of tensors may be clearly seen, we shall write
down two co-variant derivatives deduced in an analogous
way:
\begin{align*}
A_{\sigma;\, \rho}^{\tau}
  &= \frac{\dd A_{\sigma}^{\tau}}{\dd x_{\rho}}
  - \Gamma_{\sigma\rho}^{\alpha} A_{\alpha}^{\tau}
  + \Gamma_{\alpha\rho}^{\tau} A_{\sigma}^{\alpha}\Add{,}
\Tag{(79)} \\
%
{A^{\sigma\tau}}_{;\, \rho}
  &= \frac{\dd A^{\sigma\tau}}{\dd x_{\rho}}
  + \Gamma_{\alpha\rho}^{\sigma} A^{\alpha\tau}
  + \Gamma_{\alpha\rho}^{\tau} A^{\sigma\alpha}\Add{.}
\Tag{(80)}
\end{align*}
\PageSep{82}
The general law of formation now becomes evident.
From these formul� we shall deduce some others which
are of interest for the physical applications of the theory.

In case $A_{\sigma\tau}$~is skew-symmetrical, we obtain the tensor
\[
A_{\sigma\tau\rho}
  = \frac{\dd A_{\sigma\tau}}{\dd x_{\rho}}
  + \frac{\dd A_{\tau\rho}}{\dd x_{\sigma}}
  + \frac{\dd A_{\rho\sigma}}{\dd x_{\tau}}\Add{,}
\Tag{(81)}
\]
which is skew-symmetrical in all pairs of indices, by cyclic
interchange and addition.

If, in~\Eqref{(78)}, we replace~$A_{\sigma\tau}$ by the fundamental tensor,~$g_{\sigma\tau}$,
then the right-hand side vanishes identically; an
analogous statement holds for~\Eqref{(80)} with respect to~$g^{\sigma\tau}$;
that is, the co-variant derivatives of the fundamental
tensor vanish. That this must be so we see directly in
the local system of co-ordinates.

In case $A^{\sigma\tau}$~is skew-symmetrical, we obtain from~\Eqref{(80)},
by contraction with respect to $\tau$~and~$\rho$,
\[
\tA^{\sigma} = \frac{\dd \tA^{\sigma\tau}}{\dd x_{\tau}}\Add{.}
\Tag{(82)}
\]

In the general case, from \Eqref{(79)}~and~\Eqref{(80)}, by contraction
with respect to $\tau$~and~$\rho$, we obtain the equations,
\begin{align*}
\tA_{\sigma} &= \frac{\dd \tA_{\sigma}^{\alpha}}{\dd x_{\alpha}}
  - \Gamma_{\sigma\beta}^{\alpha} \tA_{\alpha}^{\beta}\Add{,}
\Tag{(83)} \\
%
\tA^{\sigma} &= \frac{\dd \tA^{\sigma\alpha}}{\dd x_{\alpha}}
  + \Gamma_{\alpha\beta}^{\sigma} \tA^{\alpha\beta}\Add{.}
\Tag{(84)}
\end{align*}

\Paragraph{The Riemann Tensor.} If we have given a curve extending
\index{Riemann!tensor}%
from the point~$P$ to the point~$G$ of the continuum,
then a vector~$A^{\mu}$, given at~$P$, may, by a parallel displacement,
be moved along the curve to~$G$. If the continuum
\PageSep{83}
is Euclidean (more generally, if by a suitable choice of
co-ordinates the~$g_{\mu\nu}$ are constants) then the vector obtained
at~$G$ as a result of this displacement does not depend
upon the choice of the curve joining $P$~and~$G$. But
otherwise, the result depends upon the path of the displacement.
In this case, therefore, a vector suffers a
change,~$\Delta A^{\mu}$ (in its direction, not its magnitude), when it
is carried from a point~$P$ of a closed curve, along the
%[Illustration: Fig. 4.]
\Figure[0.66\textwidth]{083}
curve, and back to~$P$. We shall now calculate this vector
change:
\[
\Delta A^{\mu} = \Oint \delta A^{\mu}.
\]
As in Stokes' theorem for the line integral of a vector
around a closed curve, this problem may be reduced to
the integration around a closed curve with infinitely small
linear dimensions; we shall limit ourselves to this case.
\PageSep{84}

We have, first, by~\Eqref{(67)},
\[
\Delta A^{\mu} = -\Oint \Gamma_{\alpha\beta}^{\mu} A^{\alpha}\, dx_{\beta}.
\]

In this, $\Gamma_{\alpha\beta}^{\mu}$~is the value of this quantity at the variable
point~$G$ of the path of integration. If we put
\[
\xi^{\mu} = (x_{\mu})_{G} - (x_{\mu})_{P}
\]
and denote the value of~$\Gamma_{\alpha\beta}^{\mu}$ at~$P$ by~$\Bar{\Gamma_{\alpha\beta}^{\mu}}$, then we have,
with sufficient accuracy,
\[
\Gamma_{\alpha\beta}^{\mu}
  = \Bar{\Gamma_{\alpha\beta}^{\mu}}
  + \frac{\Bar{\dd \Gamma_{\alpha\beta}^{\mu}}}{\dd x_{\nu}}\, \xi^{\nu}.
\]

Let, further, $A^{\alpha}$~be the value obtained from~$\Bar{A^{\alpha}}$ by a
parallel displacement along the curve from~$P$ to~$G$. It
may now easily be proved by means of~\Eqref{(67)} that $A^{\mu} - \Bar{A^{\mu}}$
is infinitely small of the first order, while, for a curve of
infinitely small dimensions of the first order, $\Delta A^{\mu}$~is
infinitely small of the second order. Therefore there is
an error of only the second order if we put
\[
A^{\alpha} = \Bar{A^{\alpha}}
  - \Bar{\Gamma_{\sigma\tau}^{\alpha}}\;
    \Bar{A^{\Change{\alpha}{\sigma}}}\;
    \Bar{\xi^{\Change{\beta}{\tau}}}.
\]

If we introduce these values of $\Gamma_{\alpha\beta}^{\mu}$~and~$A^{\alpha}$ into the
integral, we obtain, neglecting all quantities of a higher
order of small quantities than the second,
\[
\Delta A^{\mu}
  = - \left(\frac{\dd \Gamma_{\sigma\beta}^{\mu}}{\dd x_{\alpha}}
      - \Gamma_{\rho\beta}^{\mu} \Gamma_{\sigma\alpha}^{\rho}\right)
    A^{\sigma} \Oint \xi^{\alpha}\, d\xi^{\beta}\Add{.}
\Tag{(85)}
\]
The quantity removed from under the sign of integration
\PageSep{85}
refers to the point~$P$. Subtracting~$\frac{1}{2} d(\xi^{\alpha} \xi^{\beta})$ from the
integrand, we obtain
\[
\tfrac{1}{2} \Oint (\xi^{\alpha}\, d\xi^{\beta} - \xi^{\beta}\, d\xi^{\alpha}).
\]
This skew-symmetrical tensor of the second rank,~$\Change{f_{\alpha\beta}}{f^{\alpha\beta}}$,
characterizes the surface element bounded by the curve
in magnitude and position. If the expression in the
brackets in~\Eqref{(85)} were skew-symmetrical with respect to
the indices $\alpha$~and~$\beta$, we could conclude its tensor character
from~\Eqref{(85)}. We can accomplish this by interchanging
the summation indices $\alpha$~and~$\beta$ in~\Eqref{(85)} and adding the
resulting equation to~\Eqref{(85)}. We obtain
\[
2\Delta A^{\mu} = -R_{\sigma\alpha\beta}^{\mu} A^{\sigma} f^{\alpha\beta}\Add{,}
\Tag{(86)}
\]
in which
\[
R_{\sigma\alpha\beta}^{\mu}
  = - \frac{\dd \Gamma_{\sigma\alpha}^{\mu}}{\dd x_{\beta}}
    + \frac{\dd \Gamma_{\sigma\beta}^{\mu}}{\dd x_{\alpha}}
    + \Gamma_{\rho\alpha}^{\mu} \Gamma_{\sigma\beta}^{\rho}
    - \Gamma_{\rho\beta}^{\mu} \Gamma_{\sigma\alpha}^{\rho}\Add{.}
\Tag{(87)}
\]

The tensor character of~$R_{\sigma\alpha\beta}^{\mu}$ follows from~\Eqref{(86)}; this is
the Riemann curvature tensor of the fourth rank, whose
\index{Riemann!tensor}%
properties of symmetry we do not need to go into. Its
vanishing is a sufficient condition (disregarding the reality
of the chosen co-ordinates) that the continuum is
Euclidean.

By contraction of the Riemann tensor with respect to
the indices $\mu$,~$\beta$, we obtain the symmetrical tensor of the
second rank,
\[
R_{\mu\nu}
  = - \frac{\dd \Gamma_{\mu\nu}^{\alpha}}{\dd x_{\alpha}}
    + \Gamma_{\mu\beta}^{\alpha} \Gamma_{\nu\alpha}^{\beta}
    + \frac{\dd \Gamma_{\mu\alpha}^{\alpha}}{\dd x_{\nu}}
    - \Gamma_{\mu\nu}^{\alpha}\Gamma_{\alpha\beta}^{\beta}\Add{.}
\Tag{(88)}
\]
The last two terms vanish if the system of co-ordinates
\PageSep{86}
is so chosen that $g = \text{constant}$. From~$R_{\mu\nu}$ we can form
the scalar,
\[
R = g^{\mu\nu} R_{\mu\nu}\Add{.}
\Tag{(89)}
\]

\Paragraph{Straightest \(Geodetic\) Lines.} A line may be constructed
\index{Geodetic lines}%
\index{Straightest lines}%
in such a way that its successive elements arise from each
other by parallel displacements. This is the natural
generalization of the straight line of the Euclidean
geometry. For such a line, we have
\[
\delta \left(\frac{dx_{\mu}}{ds}\right)
  = -\Gamma_{\alpha\beta}^{\mu} \frac{dx_{\alpha}}{ds}\, dx_{\beta}.
\]
The left-hand side is to be replaced by~$\dfrac{d^{2} x_{\mu}}{ds^{2}}$,\footnote
  {The direction vector at a neighbouring point of the curve results, by a
  parallel displacement along the line element~$(dx_{\beta})$, from the direction vector
  of each point considered.}
so that we
have
\[
\frac{d^{2} x_{\mu}}{ds^{2}}
  + \Gamma_{\alpha\beta}^{\mu} \frac{dx_{\alpha}}{ds}\, \frac{dx_{\beta}}{ds}
  = 0\Add{.}
\Tag{(90)}
\]
We get the same line if we find the line which gives a
stationary value to the integral
\[
\int ds\quad\text{or}\quad
\int \sqrt{g_{\mu\nu}\, dx_{\mu}\, dx_{\nu}}
\]
between two points (geodetic line).
\PageSep{87}


\Lecture[ (\textit{continued})]{IV}{The General Theory of Relativity}
{The General Theory}

\First{We} are now in possession of the mathematical
apparatus which is necessary to formulate the
laws of the general theory of relativity. No attempt
will be made in this presentation at systematic completeness,
but single results and possibilities will be developed
progressively from what is known and from the
results obtained. Such a presentation is most suited
to the present provisional state of our knowledge.

A material particle upon which no force acts moves,
according to the principle of inertia, uniformly in a
straight line. In the four-dimensional continuum of the
special theory of relativity (with real time co-ordinate)
this is a real straight line. The natural, that is, the
simplest, generalization of the straight line which is
plausible in the system of concepts of Riemann's general
theory of invariants is that of the straightest, or geodetic,
line. We shall accordingly have to assume, in the sense
of the principle of equivalence, that the motion of a
material particle, under the action only of inertia and
gravitation, is described by the equation,
\[
\frac{d^{2} x_{\mu}}{ds^{2}}
  + \Gamma_{\alpha\beta}^{\mu} \frac{dx_{\alpha}}{ds}\, \frac{dx_{\beta}}{ds}
  = 0\Add{.}
\tag*{(90)} %[** TN: Repeated from previous lecture]
\]
\PageSep{88}
In fact, this equation reduces to that of a straight line
if all the components,~$\Gamma_{\alpha\beta}^{\mu}$, of the gravitational field
vanish.

How are these equations connected with Newton's
equations of motion? According to the special theory
of relativity, the~$g_{\mu\nu}$ as well as the~$g^{\mu\nu}$, have the values,
with respect to an inertial system (with real time co-ordinate
and suitable choice of the sign of~$ds^{2}$),
\[
\left.
\begin{array}{*{4}{>{\quad}r}}
-1 & 0 & 0 & 0 \\
0 & -1 & 0 & 0 \\
0 & 0 & -1 & 0 \\
0 & 0 & 0  & \Neg[]1
\end{array}
\right\}\Add{.}
\Tag{(91)}
\]
The equations of motion then become
\[
\frac{d^{2} x_{\mu}}{ds^{2}} = 0.
\]
We shall call this the ``first approximation'' to the $g_{\mu\nu}$-field.
In considering approximations it is often useful,
as in the special theory of relativity, to use an imaginary
$x_{4}$-co-ordinate, as then the~$g_{\mu\nu}$, to the first approximation,
assume the values
\[
\left.
\begin{array}{*{4}{>{\quad}r}}
-1 & 0 & 0 & 0 \\
0 & -1 & 0 & 0 \\
0 & 0 & -1 & 0 \\
0 & 0 & 0 & -1
\end{array}
\right\}\Add{.}
\Tag{(91a)}
\]
These values may be collected in the relation
\[
g_{\mu\nu} = -\delta_{\mu\nu}.
\]
To the second approximation we must then put
\[
g_{\mu\nu} = -\delta_{\mu\nu} + \gamma_{\mu\nu}\Add{,}
\Tag{(92)}
\]
\PageSep{89}
where the~$\gamma_{\mu\nu}$ are to be regarded as small of the first
order.

Both terms of our equation of motion are then small
of the first order. If we neglect terms which, relatively
to these, are small of the first order, we have to put
\begin{gather*}
ds^{2} = \Change{}{-}{dx_{\nu}}^{2} = dl^{2} (1 - q^{2})\Add{,}
\Tag{(93)} \\
%
\Gamma_{\alpha\beta}^{\mu}
  = -\delta_{\mu\sigma} \left[\Chr{\alpha}{\beta}{\sigma}\right]
  = -\left[\Chr{\alpha}{\beta}{\mu}\right]
  = \frac{1}{2} \left(
    \frac{\dd \gamma_{\alpha\beta}}{\dd x_{\mu}}
  - \frac{\dd \gamma_{\alpha\mu}}{\dd x_{\beta}}
  - \frac{\dd \gamma_{\beta\mu}}{\dd x_{\alpha}}\right)\Add{.}
\Tag{(94)}
\end{gather*}
We shall now introduce an approximation of a second
kind. Let the velocity of the material particles be very
small compared to that of light. Then $ds$~will be the
same as the time differential,~$dl$. Further, $\dfrac{dx_{1}}{ds}$, $\dfrac{dx_{2}}{ds}$, $\dfrac{dx_{3}}{ds}$
will vanish compared to~$\dfrac{dx_{4}}{ds}$. We shall assume, in addition,
that the gravitational field varies so little with the
time that the derivatives of the~$\gamma_{\mu\nu}$ by~$x_{4}$ may be
neglected. Then the equation of motion (for $\mu = 1, 2, 3$)
reduces to
\[
\frac{d^{2} x_{\mu}}{dl^{2}}
  = \frac{\dd}{\dd x_{\mu}} \left(\frac{\gamma_{44}}{2}\right)\Add{.}
\Tag{(90a)}
\]
This equation is identical with Newton's equation of
motion for a material particle in a gravitational field, if
we identify~$\left(\dfrac{\gamma_{44}}{2}\right)$ with the potential of the gravitational
field; whether or not this is allowable, naturally depends
upon the field equations of gravitation, that is, it depends
upon whether or not this quantity satisfies, to a
first approximation, the same laws of the field as the
\PageSep{90}
gravitational potential in Newton's theory. A glance
at \Eqref{(90)}~and~\Eqref{(90a)} shows that the~$\Gamma_{\alpha\beta}^{\mu}$ actually do play
the r�le of the intensity of the gravitational field.
These quantities do not have a tensor character.

Equations~\Eqref{(90)} express the influence of inertia and
gravitation upon the material particle. The unity of
inertia and gravitation is formally expressed by the fact
that the whole left-hand side of~\Eqref{(90)} has the character
of a tensor (with respect to any transformation of co-ordinates),
but the two terms taken separately do not
have tensor character, so that, in analogy with Newton's
equations, the first term would be regarded as the expression
for inertia, and the second as the expression
for the gravitational force.

We must next attempt to find the laws of the gravitational
field. For this purpose, Poisson's equation,
\index{Poisson's equation}%
\[
\Delta\phi = 4\pi K\rho
\]
of the Newtonian theory must serve as a model. This
equation has its foundation in the idea that the gravitational
field arises from the density~$\rho$ of ponderable
matter. It must also be so in the general theory of
relativity. But our investigations of the special theory
of relativity have shown that in place of the scalar
density of matter we have the tensor of energy per unit
volume. In the latter is included not only the tensor
of the energy of ponderable matter, but also that of the
electromagnetic energy. We have seen, indeed, that
in a more complete analysis the energy tensor can be
regarded only as a provisional means of representing
\PageSep{91}
matter. In reality, matter consists of electrically charged
particles, and is to be regarded itself as a part, in fact,
the principal part, of the electromagnetic field. It is
only the circumstance that we have not sufficient knowledge
of the electromagnetic field of concentrated charges
that compels us, provisionally, to leave undetermined
in presenting the theory, the true form of this tensor.
From this point of view our problem now is to introduce
a tensor,~$T_{\mu\nu}$, of the second rank, whose structure we do
not know provisionally, and which includes in itself the
energy density of the electromagnetic field and of ponderable
matter; we shall denote this in the following as
the ``energy tensor of matter.''

According to our previous results, the principles of
momentum and energy are expressed by the statement
that the divergence of this tensor vanishes~\Eqref{(47c)}. In
the general theory of relativity, we shall have to assume
as valid the corresponding general co-variant equation.
If $(T_{\mu\nu})$~denotes the co-variant energy tensor of matter,
$\tT_{\sigma}^{\nu}$~the corresponding mixed tensor density, then, in
accordance with~\Eqref{(83)}, we must require that
\[
0 = \frac{\dd \tT_{\sigma}^{\alpha}}{\dd x_{\alpha}}
  - \Gamma_{\sigma\beta}^{\alpha} \tT_{\alpha}^{\beta}
\Tag{(95)}
\]
be satisfied. It must be remembered that besides the
energy density of the matter there must also be given
an energy density of the gravitational field, so that there
can be no talk of principles of conservation of energy
and momentum for matter alone. This is expressed
mathematically by the presence of the second term in~\Eqref{(95)},
\PageSep{92}
which makes it impossible to conclude the existence
of an integral equation of the form of~\Eqref{(49)}. The gravitational
field transfers energy and momentum to the
``matter,'' in that it exerts forces upon it and gives it
energy; this is expressed by the second term in~\Eqref{(95)}.

If there is an analogue of Poisson's equation in the
general theory of relativity, then this equation must be
a tensor equation for the tensor~$g_{\mu\nu}$ of the gravitational
potential; the energy tensor of matter must appear on
the right-hand side of this equation. On the left-hand
side of the equation there must be a differential tensor
in the~$g_{\mu\nu}$. We have to find this differential tensor.
It is completely determined by the following three
conditions:---

1.~It may contain no differential coefficients of the~$g_{\mu\nu}$
higher than the second.

2.~It must be linear and homogeneous in these second
differential coefficients.

3.~Its divergence must vanish identically.

The first two of these conditions are naturally taken
from Poisson's equation. Since it may be proved
mathematically that all such differential tensors can be
formed algebraically (i.e.~without differentiation) from
Riemann's tensor, our tensor must be of the form
\[
R_{\mu\nu} + ag_{\mu\nu} R\Add{,}
\]
in which $R_{\mu\nu}$~and~$R$ are defined by \Eqref{(88)}~and~\Eqref{(89)} respectively.
Further, it may be proved that the third condition
requires~$a$ to have the value~$-\frac{1}{2}$. For the law
\PageSep{93}
of the gravitational field we therefore get the equation
\[
R_{\mu\nu} - \tfrac{1}{2}g_{\mu\nu} R = - \kappa T_{\mu\nu}\Add{.}
\Tag{(96)}
\]
Equation~\Eqref{(95)} is a consequence of this equation. $\kappa$~denotes
a constant, which is connected with the Newtonian
gravitation constant.

In the following I shall indicate the features of the
theory which are interesting from the point of view of
physics, using as little as possible of the rather involved
mathematical method. It must first be shown that the
divergence of the left-hand side actually vanishes. The
energy principle for matter may be expressed, by~\Eqref{(83)},
\[
0 = \frac{\dd \tT_{\sigma}^{\alpha}}{\dd x_{\alpha}}
  - \Gamma_{\sigma\beta}^{\alpha} \tT_{\alpha}^{\beta}\Add{,}
\Tag{(97)}
\]
in which
\[
\tT_{\sigma}^{\alpha} = T_{\sigma\tau} g^{\tau\alpha} \sqrt{-g}.
\]
The analogous operation, applied to the left-hand side
of~\Eqref{(96)}, will lead to an identity.

In the region surrounding each world-point there are
systems of co-ordinates for which, choosing the $x_{\Change{\mu}{4}}$-co-ordinate
imaginary, at the given point,
\[
g_{\mu\nu} = g^{\mu\nu}
  = -\delta_{\mu\nu}
%[** TN: Moved "="s outside brace]
  = \begin{cases}
    -1 & \text{if $\mu = \nu$\Add{,}} \\
    \Neg[]0 & \text{if $\mu \neq \nu$,}
  \end{cases}
\]
and for which the first derivatives of the~$g_{\mu\nu}$ and the~$g^{\mu\nu}$
vanish. We shall verify the vanishing of the divergence
of the left-hand side at this point. At this point
the components~$\Gamma_{\sigma\beta}^{\alpha}$ vanish, so that we have to prove
the vanishing only of
\[
\frac{\dd}{\dd x_{\sigma}} \left[
  \sqrt{-g}\, g^{\nu\sigma} (R_{\mu\nu} - \tfrac{1}{2}g_{\mu\nu} R)
\right].
\]
\PageSep{94}
Introducing \Eqref{(88)}~and~\Eqref{(70)} into this expression, we see
that the only terms that remain are those in which third
derivatives of the~$g^{\mu\nu}$ enter. Since the~$g_{\mu\nu}$ are to be
replaced by~$-\delta_{\mu\nu}$, we obtain, finally, only a few terms
which may easily be seen to cancel each other. Since
the quantity that we have formed has a tensor character,
its vanishing is proved for every other system of co-ordinates
also, and naturally for every other four-dimensional
point. The energy principle of matter~\Eqref{(97)} is thus a
mathematical consequence of the field equations~\Eqref{(96)}.

In order to learn whether the equations~\Eqref{(96)} are
consistent with experience, we must, above all else, find
out whether they lead to the Newtonian theory as a
first approximation. For this purpose we must introduce
various approximations into these equations. We
already know that Euclidean geometry and the law of the
constancy of the velocity of light are valid, to a certain
approximation, in regions of a great extent, as in the
planetary system. If, as in the special theory of relativity,
we take the fourth co-ordinate imaginary, this
means that we must put
\[
g_{\mu\nu} = -\delta_{\mu\nu} + \gamma_{\mu\nu}\Add{,}
\Tag{(98)}
\]
in which the~$\gamma_{\mu\nu}$ are so small compared to~$1$ that we
can neglect the higher powers of the~$\gamma_{\mu\nu}$ and their
derivatives. If we do this, we learn nothing about the
structure of the gravitational field, or of metrical space of
cosmical dimensions, but we do learn about the influence
of neighbouring masses upon physical phenomena.

Before carrying through this approximation we shall
\PageSep{95}
transform~\Eqref{(96)}. We multiply~\Eqref{(96)} by~$g^{\mu\nu}$, summed over
the $\mu$~and~$\nu$; observing the relation which follows from
the definition of the~$g^{\mu\nu}$,
\[
g_{\mu\nu} g^{\mu\nu} = 4\Add{,}
\]
we obtain the equation
\[
R = \kappa g^{\mu\nu} T_{\mu\nu} = \kappa T.
\]
If we put this value of~$R$ in~\Eqref{(96)} we obtain
\[
R_{\mu\nu} = -\kappa (T_{\mu\nu} - \tfrac{1}{2}g_{\mu\nu} T)
  = -\kappa T_{\mu\nu}^{*}\Add{.}
\Tag{(96a)}
\]
When the approximation which has been mentioned is
carried out, we obtain for the left-hand side,
\[
-\tfrac{1}{2}\left(
    \frac{\dd^{2} \gamma_{\mu\nu}}{{\dd x_{\alpha}}^{2}}
  + \frac{\dd^{2} \gamma_{\alpha\alpha}}{\dd x_{\mu}\, \dd x_{\nu}}
  - \frac{\dd^{2} \gamma_{\mu\alpha}}{\dd x_{\nu}\, \dd x_{\alpha}}
  - \frac{\dd^{2} \gamma_{\nu\alpha}}{\dd x_{\mu}\, \dd x_{\alpha}}
\right)
\]
or
\[
-\tfrac{1}{2}\frac{\dd^{2} \gamma_{\mu\nu}}{{\dd x_{\alpha}}^{2}}
  + \tfrac{1}{2} \frac{\dd}{\dd x_{\nu}}\left(
    \frac{{\dd \gamma'}_{\mu\alpha}}{\dd x_{\alpha}}
  \right)
  + \tfrac{1}{2} \frac{\dd}{\dd x_{\mu}}\left(
    \frac{{\dd \gamma'}_{\nu\alpha}}{\dd x_{\alpha}}
  \right)\Add{,}
\]
in which has been put
\[
{\gamma'}_{\mu\nu}
  = \gamma_{\mu\nu} - \tfrac{1}{2}\gamma_{\sigma\sigma}\delta_{\mu\nu}\Add{.}
\Tag{(99)}
\]

We must now note that equation~\Eqref{(96)} is valid for any
system of co-ordinates. We have already specialized the
system of co-ordinates in that we have chosen it so that
within the region considered the $g_{\mu\nu}$~differ infinitely little
from the constant values~$-\delta_{\mu\nu}$. But this condition
remains satisfied in any infinitesimal change of co-ordinates,
so that there are still four conditions to which
the~$\gamma_{\mu\nu}$ may be subjected, provided these conditions do
not conflict with the conditions for the order of magnitude
\PageSep{96}
of the~$\gamma_{\mu\nu}$. We shall now assume that the system of co-ordinates
is so chosen that the four relations---
\[
0 = \frac{{\dd\gamma'}_{\mu\nu}}{\dd x_{\nu}}
  = \frac{\dd\gamma_{\mu\nu}}{\dd x_{\nu}}
  - \tfrac{1}{2} \frac{\dd\gamma_{\sigma\sigma}}{\dd x_{\mu}}
\Tag{(100)}
\]
are satisfied. Then \Eqref{(96a)}~takes the form
\[
\frac{\dd^{2} \gamma_{\mu\nu}}{{\dd x_{\alpha}}^{2}} = 2\kappa T_{\mu\nu}^{*}\Add{.}
\Tag{(96b)}
\]
These equations may be solved by the method, familiar
in electrodynamics, of retarded potentials; we get, in an
easily understood notation,
\[
\gamma_{\mu\nu} = -\frac{\kappa}{2\pi} \int
  \frac{T_{\mu\nu}^{*}(x_{0}, y_{0}, z_{0}, t - r)}{r}\, dV_{0}\Add{.}
\Tag{(101)}
\]

In order to see in what sense this theory contains the
Newtonian theory, we must consider in greater detail
the energy tensor of matter. Considered phenomenologically,
this energy tensor is composed of that of the
electromagnetic field and of matter in the narrower sense.
If we consider the different parts of this energy tensor
with respect to their order of magnitude, it follows
from the results of the special theory of relativity that
the contribution of the electromagnetic field practically
vanishes in comparison to that of ponderable matter. In
our system of units, the energy of one gram of matter is
equal to~$1$, compared to which the energy of the electric
fields may be ignored, and also the energy of deformation
of matter, and even the chemical energy. We get an
approximation that is fully sufficient for our purpose if
\PageSep{97}
we put
\[
\left.
\begin{aligned}
T^{\mu\nu} &= \sigma \frac{dx_{\mu}}{ds}\, \frac{dx_{\nu}}{ds}\Add{,} \\
ds^{2} &= g_{\mu\nu}\, dx_{\mu}\, dx_{\nu}\Add{.}
\end{aligned}
\right\}
\Tag{(102)}
\]
In this, $\sigma$~is the density at rest, that is, the density of the
ponderable matter, in the ordinary sense, measured with
the aid of a unit measuring rod, and referred to a Galilean
system of co-ordinates moving with the matter.

We observe, further, that in the co-ordinates we have
chosen, we shall make only a relatively small error if we
replace the~$g_{\mu\nu}$ by~$-\delta_{\mu\nu}$, so that we put
\[
ds^{2} = -\sum {dx_{\mu}}^{2}\Add{.}
\Tag{(102a)}
\]

The previous developments are valid however rapidly
the masses which generate the field may move relatively
to our chosen system of quasi-Galilean co-ordinates. But
in astronomy we have to do with masses whose velocities,
relatively to the co-ordinate system employed, are always
small compared to the velocity of light, that is, small
compared to~$1$, with our choice of the unit of time.
We therefore get an approximation which is sufficient
for nearly all practical purposes if in~\Eqref{(101)} we replace
the retarded potential by the ordinary (non-retarded)
potential, and if, for the masses which generate the field,
we put
\[
\frac{dx_{1}}{ds} = \frac{dx_{2}}{ds} = \frac{dx_{3}}{ds} = 0,\quad
\frac{dx_{4}}{ds} = \frac{\sqrt{-1}\, dl}{dl} = \sqrt{-1}\Add{.}
\Tag{(\Change{103a}{103})}
\]
\PageSep{98}
Then we get for $T^{\mu\nu}$~and~$T_{\mu\nu}$ the values
\[
\left.
\begin{array}{*{3}{>{\qquad}r}>{\quad}r}
0 & 0 & 0 & 0 \\
0 & 0 & 0 & 0 \\
0 & 0 & 0 & 0 \\
0 & 0 & 0  & -\sigma
\end{array}
\right\}\Add{.}
\Tag{(104)}
\]
For $T$ we get the value~$\sigma$, and, finally, for~$T_{\mu\nu}^{*}$ the
values,
\[
\left.
\begin{array}{*{3}{>{\qquad}c}>{\quad}r}
\dfrac{\sigma}{2} & 0 & 0 & 0 \\
0 & \dfrac{\sigma}{2} & 0 & 0 \\
0 & 0 & \dfrac{\sigma}{2} & 0 \\
0 & 0 & 0  & -\dfrac{\sigma}{2}
\end{array}
\right\}\Add{.}
\Tag{(104a)}
\]

We thus get, from~\Eqref{(101)},
\[
\left.
\begin{aligned}
\gamma_{11} = \gamma_{22} = \gamma_{33}
  &= -\frac{\kappa}{4\pi} \int \frac{\sigma\, dV_{0}}{r}\Add{,} \\
\gamma_{44} &= +\frac{\kappa}{4\pi} \int \frac{\sigma\, dV_{0}}{r}\Add{,}
\end{aligned}
\right\}
\Tag{(101a)}
\]
while all the other~$\gamma_{\mu\nu}$ vanish. The least of these equations,
in connexion with equation~\Eqref{(90a)}, contains Newton's
theory of gravitation. If we replace~$l$ by~$ct$ we
get
\[
\frac{d^{2} x_{\mu}}{dt^{2}}
  = \frac{\kappa c^{2}}{8\pi}\, \frac{\dd}{\dd x_{\mu}} \left\{
    \int \frac{\sigma\, dV_{0}}{r}
  \right\}\Add{.}
\Tag{(90b)}
\]
We see that the Newtonian gravitation constant~$K$, is
\index{Gravitation constant}%
\index{Newtonian gravitation constant}%
connected with the constant~$\kappa$ that enters into our field
equations by the relation
\[
K = \frac{\kappa c^{2}}{8\pi}\Add{.}
\Tag{(105)}
\]
\PageSep{99}
From the known numerical value of~$K$, it therefore
follows that
\[
\kappa = \frac{8\pi K}{c^{2}}
  = \frac{8\pi�6.67�10^{-8}}{9�10^{20}}
  = 1.86�10^{-27}\Add{.}
\Tag{(105a)}
\]
From \Eqref{(101)} we see that even in the first approximation
the structure of the gravitational field differs fundamentally
from that which is consistent with the Newtonian theory;
this difference lies in the fact that the gravitational
potential has the character of a tensor and not a scalar.
This was not recognized in the past because only the
component~$g_{44}$, to a first approximation, enters the equations
of motion of material particles.

In order now to be able to judge the behaviour of
measuring rods and clocks from our results, we must
observe the following. According to the principle of
equivalence, the metrical relations of the Euclidean
geometry are valid relatively to a Cartesian system of
reference of infinitely small dimensions, and in a suitable
state of motion (freely falling, and without rotation).
We can make the same statement for local systems of
co-ordinates which, relatively to these, have small accelerations,
and therefore for such systems of co-ordinates
as are at rest relatively to the one we have selected. For
such a local system, we have, for two neighbouring point
events,
\[
ds^{2} = - {dX_{1}}^{2} - {dX_{2}}^{2} - {dX_{3}}^{2} + dT^{2}
  = - dS^{2} + dT^{2}\Add{,}
\]
where $dS$~is measured directly by a measuring rod and
$dT$~by a clock at rest relatively to the system; these are
\PageSep{100}
the naturally measured lengths and times. Since~$ds^{2}$, on
the other hand, is known in terms of the co-ordinates~$x_{\nu}$
employed in finite regions, in the form
\[
ds^{2} = g_{\mu\nu}\, dx_{\mu}\, dx_{\nu}\Add{,}
\]
we have the possibility of getting the relation between
naturally measured lengths and times, on the one hand,
and the corresponding differences of co-ordinates, on the
other hand. As the division into space and time is in
agreement with respect to the two systems of co-ordinates,
so when we equate the two expressions for~$ds^{2}$ we get
two relations. If, by~\Eqref{(101a)}, we put
\begin{multline*}
ds^{2} = -\left(1 + \frac{\kappa}{4\pi} \int \frac{\sigma\, dV_{0}}{r}\right)
  ({dx_{1}}^{2} + {dx_{2}}^{2} + {dx_{3}}^{2}) \\
  + \left(1 - \frac{\kappa}{4\pi} \int \frac{\sigma\, dV_{0}}{r}\right) dl^{2}\Add{,}
\end{multline*}
we obtain, to a sufficiently close approximation,
\[
\left.
\begin{aligned}
  &\sqrt{{dX_{1}}^{2} + {dX_{2}}^{2} + {dX_{3}}^{2}} \\
  &\qquad
  \begin{aligned}
  &= \left(1 + \frac{\kappa}{8\pi} \int \frac{\sigma\, dV_{0}}{r}\right)
     \sqrt{{dx_{1}}^{2} + {dx_{2}}^{2} + {dx_{3}}^{2}}\Add{,} \\
dT &= \left(1 - \frac{\kappa}{8\pi} \int \frac{\sigma\, dV_{0}}{r}\right) dl.
\end{aligned}
\end{aligned}
\right\}
\Tag{(106)}
\]

The unit measuring rod has therefore the length,
\[
1 - \frac{\kappa}{8\pi} \int \frac{\sigma\, dV_{0}}{r}
\]
in respect to the system of co-ordinates we have selected.
The particular system of co-ordinates we have selected
\PageSep{101}
insures that this length shall depend only upon
the place, and not upon the direction. If we had
chosen a different system of co-ordinates this would not
be so. But however we may choose a system of co-ordinates,
the laws of configuration of rigid rods do not
agree with those of Euclidean geometry; in other words,
we cannot choose any system of co-ordinates so that the
co-ordinate differences, $\Delta x_{1}$,~$\Delta x_{2}$,~$\Delta x_{3}$, corresponding to the
ends of a unit measuring rod, oriented in any way, shall
always satisfy the relation ${\Delta x_{1}}^{2} + {\Delta x_{2}}^{2} + {\Delta x_{3}}^{2} = 1$. In
this sense space is not Euclidean, but ``curved.'' It
follows from the second of the relations above that the
interval between two beats of the unit clock ($dT = 1$)
corresponds to the ``time''
\[
1 + \frac{\kappa}{8\pi} \int \frac{\sigma\, dV_{0}}{r}
\]
in the unit used in our system of co-ordinates. The rate
of a clock is accordingly slower the greater is the mass of
the ponderable matter in its neighbourhood. We therefore
\index{Displacement of spectral lines}%
\index{Spectral lines, displacement of}%
conclude that spectral lines which are produced on
the sun's surface will be displaced towards the red,
compared to the corresponding lines produced on the
earth, by about $2�10^{-6}$~of their wave-lengths. At first,
this important consequence of the theory appeared to
conflict with experiment; but results obtained during the
past year seem to make the existence of this effect more
probable, and it can hardly be doubted that this consequence
of the theory will be confirmed within the next
year.
\PageSep{102}

Another important consequence of the theory, which
\index{Light ray, path of}%
\index{Path of light ray}%
\index{Ray of light, path of}%
can be tested experimentally, has to do with the path of
rays of light. In the general theory of relativity also
the velocity of light is everywhere the same, relatively to
a local inertial system. This velocity is unity in our
natural measure of time. The law of the propagation of
light in general co-ordinates is therefore, according to the
general theory of relativity, characterized, by the equation
\[
ds^{2} = 0.
\]
To within the approximation which we are using, and in
the system of co-ordinates which we have selected, the
velocity of light is characterized, according to~\Eqref{(106)}, by
the equation
\[
\biggl(1 + \frac{\kappa}{4\pi} \int \frac{\sigma\, dV_{0}}{r}\biggr)
  ({dx_{1}}^{2} + {dx_{2}}^{2} + {dx_{3}}^{2})
  = \biggl(1 - \frac{\kappa}{4\pi} \int \frac{\sigma\, dV_{0}}{r}\biggr) dl^{2}.
\]
The velocity of light~$L$, is therefore expressed in our
co-ordinates by
\[
\frac{\sqrt{{dx_{1}}^{2} + {dx_{2}}^{2} + {dx_{3}}^{2}}}{dl}
  = 1 - \frac{\kappa}{4\pi} \int \frac{\sigma\, dV_{0}}{r}\Add{.}
\Tag{(107)}
\]
We can therefore draw the conclusion from this, that a
ray of light passing near a large mass is deflected. If
we imagine the sun, of mass~$M$ concentrated at the
origin of our system of co-ordinates, then a ray of light,
travelling parallel to the $x_{3}$-axis, in the $x_{1}$-$x_{3}$~plane,
at a distance~$\Delta$ from the origin, will be deflected, in all,
by an amount
\PageSep{103}
\[
\alpha = \int_{-\infty}^{+\infty} \frac{1}{L}\, \frac{\dd L}{\dd x_{1}}\, dx_{3}
\]
towards the sun. On performing the integration we get
\[
\alpha = \frac{\kappa M}{2\pi\Delta}\Add{.}
\Tag{(108)}
\]

The existence of this deflection, which amounts to~$1.7''$
for $\Delta$~equal to the radius of the sun, was confirmed,
with remarkable accuracy, by the English Solar Eclipse
\index{Solar Eclipse expedition (1919)}%
Expedition in~1919, and most careful preparations have
been made to get more exact observational data at the
solar eclipse in~1922. It should be noted that this
result, also, of the theory is not influenced by our
arbitrary choice of a system of co-ordinates.

This is the place to speak of the third consequence of
the theory which can be tested by observation, namely,
that which concerns the motion of the perihelion
of the planet Mercury. The secular changes in the
\index{Mercury, perihelion of}%
\index{Perihelion of Mercury}%
planetary orbits are known with such accuracy that the
approximation we have been using is no longer sufficient
for a comparison of theory and observation. It is necessary
to go back to the general field equations~\Eqref{(96)}. To
solve this problem I made use of the method of successive
approximations. Since then, however, the problem
of the central symmetrical statical gravitational field has
been completely solved by Schwarzschild and others;
the derivation given by H.~Weyl in his book, ``Raum-Zeit-Materie,''
\index{Weyl}%
is particularly elegant. The calculation
can be simplified somewhat if we do not go back directly
\PageSep{104}
to the equation~\Eqref{(96)}, but base it upon a principle of
variation that is equivalent to this equation. I shall
indicate the procedure only in so far as is necessary for
understanding the method.

In the case of a statical field, $ds^{2}$~must have the form
\[
\left.
\begin{aligned}
ds^{2} &= -d\sigma^{2} + f^{2}\, {dx_{4}}^{2}\Add{,} \\
d\sigma^{2}
   &= \sum_{\text{$1$--$3$}} \gamma_{\alpha\beta}\, dx_{\alpha}\, dx_{\beta}\Add{,}
\end{aligned}
\right\}
\Tag{(109)}
\]
where the summation on the right-hand side of the last
equation is to be extended over the space variables only\Change{,}{.}
The central symmetry of the field requires the~$\gamma_{\mu\nu}$ to be
of the form,
\[
\gamma_{\alpha\beta}
  = \mu \delta_{\alpha\beta} + \lambda x_{\alpha} x_{\beta}\Add{;}
\Tag{(110)}
\]
$f^{2}$,~$\mu$ and~$\lambda$ are functions of $r = \sqrt{{x_{1}}^{2} + {x_{2}}^{2} + {x_{3}}^{2}}$ only.
One of these three functions can be chosen arbitrarily,
because our system of co-ordinates is, \textit{a~priori}, completely
arbitrary; for by a substitution
\begin{align*}
{x'}_{4} &= x_{4}\Add{,} \\
{x'}_{\alpha} &= F(r) x_{\alpha}\Add{,}
\end{align*}
we can always insure that one of these three functions
shall be an assigned function of~$r'$. In place of~\Eqref{(110)} we
can therefore put, without limiting the generality,
\[
\gamma_{\alpha\beta}
  = \delta_{\alpha\beta} + \lambda x_{\alpha} x_{\beta}\Add{.}
\Tag{(110a)}
\]

In this way the~$g_{\mu\nu}$ are expressed in terms of the two
quantities $\lambda$~and~$f$. These are to be determined as functions
of~$r$, by introducing them into equation~\Eqref{(96)}, after
\PageSep{105}
first calculating the~$\Change{{\Gamma_{\sigma}}^{\mu\nu}}{\Gamma_{\mu\nu}^{\sigma}}$ from \Eqref{(\Change{107}{109})}~and~\Eqref{(\Change{108a}{110a})}. We
have
\[
\left.
\begin{aligned}
\Gamma_{\alpha\beta}^{\sigma}
  &= \tfrac{1}{2} \frac{x_{\sigma}}{r}
     � \frac{\lambda' x_{\alpha} x_{\beta} + 2\lambda r\, \delta_{\alpha\beta}}
            {1 + \lambda r^{2}}
  \text{ (for $\alpha, \beta, \sigma = 1, 2, 3$)}\Add{,} \\
%
\Gamma_{44}^{4}
  &= \Gamma_{4\beta}^{\alpha} = \Gamma_{\alpha\beta}^{4} = 0\quad
  \text{(for $\alpha, \beta = 1, 2, 3$)}\Add{,} \\
%
\Gamma_{4\alpha}^{4} &= \tfrac{1}{2} f^{-2}\, \frac{\dd f^{2}}{\dd x_{\alpha}},\quad
\Gamma_{44}^{\alpha}
  = -\tfrac{1}{2} \Change{f^{-2}}{g^{\alpha\beta}}\,
     \frac{\dd f^{2}}{\dd \Change{x_{\alpha}}{x_{\beta}}}\Add{.}
\end{aligned}
\right\}
\Tag{(\Change{108b}{110b})}
\]

With the help of these results, the field equations
furnish Schwarzschild's solution:
\[
ds^{2} = \left(1 - \frac{A}{r}\right) dl^{2} - \left[
  \frac{dr^{2}}
       {1 - \dfrac{A}{r}} + r^{2} (\sin^{2}\theta\, d\phi^{2} + d\theta^{2})
\right]\Add{,}
\Tag{(\Change{109}{109a})}
\]
in which we have put
\[
\left.
\begin{aligned}
x_{4} &= l\Add{,} \\
x_{1} &= r \sin\theta \sin\phi\Add{,} \\
x_{2} &= r \sin\theta \cos\phi\Add{,} \\
x_{3} &= r \cos\theta \Add{,} \\
A &= \frac{\kappa M}{4\pi}\Add{.}
\end{aligned}
\right\}
\Tag{(\Change{109a}{109b})}
\]

$M$~denotes the sun's mass, centrally symmetrically
placed about the origin of co-ordinates; the solution~\Eqref{(109)}
is valid only outside of this mass, where all the~$T_{\mu\nu}$ vanish.
If the motion of the planet takes place in the $x_{1}$-$x_{2}$~plane
then we must replace~\Eqref{(\Change{109}{109a})} by
\[
ds^{2} = \left(1 - \frac{A}{r}\right) dl^{2}
  - \frac{dr^{2}}{1 - \dfrac{A}{r}} - r^{2}\, d\phi^{2}\Add{.}
\Tag{(\Change{109b}{109c})}
\]
\PageSep{106}

The calculation of the planetary motion depends upon
equation~\Eqref{(90)}. From the first of equations \Eqref{(\Change{108b}{110b})}~and~\Eqref{(90)}
we get, for the indices $1, 2, 3$,
\[
\frac{d}{ds}
  \left(x_{\alpha} \frac{dx_{\beta}}{ds} - x_{\beta} \frac{dx_{\alpha}}{ds}\right) = 0\Add{,}
\]
or, if we integrate, and express the result in polar co-ordinates,
\[
r^{2} \frac{d\phi}{ds} = \text{constant}.
\Tag{(111)}
\]

From~\Eqref{(90)}, for $\mu = 4$, we get
\[
0 = \frac{d^{2} l}{ds^{2}}
  + \frac{1}{f^{2}}\, \frac{df^{2}}{dx_{\alpha}}\, \frac{dx_{\alpha}}{ds}
    \Change{}{\, \frac{dl}{ds}}
  = \frac{d^{2} l}{ds^{2}} + \frac{1}{f^{2}}\, \frac{df^{2}}{ds}
    \Change{}{\, \frac{dl}{ds}}.
\]
From this, after multiplication by~$f^{2}$ and integration, we
have
\[
f^{2} \frac{dl}{ds} = \text{constant}.
\Tag{(112)}
\]

In \Eqref{(\Change{109b}{109c})},~\Eqref{(111)} and~\Eqref{(112)} we have three equations
between the four variables $s$,~$r$,~$l$ and~$\phi$, from which the
motion of the planet may be calculated in the same way
as in classical mechanics. The most important result we
get from this is a secular rotation of the elliptic orbit of
the planet in the same sense as the revolution of the
planet, amounting in radians per revolution to
\[
\frac{24 \pi^{3} a^{2}}{(1 - e^{2}) c^{2} T^{2}}\Add{,}
\Tag{(113)}
\]
\PageSep{107}
where
\begin{align*}
a &= \text{the semi-major axis of the planetary orbit in
centimetres.} \\
e &= \text{the numerical eccentricity.} \\
c &= 3�10^{+10}, \text{ the velocity of light \textit{in vacuo}.} \\
T &= \text{the period of revolution in seconds.}
\end{align*}
This expression furnishes the explanation of the motion
of the perihelion of the planet Mercury, which has been
\index{Mercury, perihelion of}%
\index{Perihelion of Mercury}%
known for a hundred years (since Leverrier), and for
which theoretical astronomy has hitherto been unable
satisfactorily to account.

There is no difficulty in expressing Maxwell's theory
of the electromagnetic field in terms of the general theory
of relativity; this is done by application of the tensor
formation \Eqref{(81)},~\Eqref{(82)} and~\Eqref{(77)}. Let $\phi_{\mu}$~be a tensor of the
first rank, to be denoted as an electromagnetic $4$-potential;
then an electromagnetic field tensor may be defined by
the relations,
\[
\phi_{\mu\nu}
  = \frac{\dd \phi_{\mu}}{\dd x_{\nu}} - \frac{\dd \phi_{\nu}}{\dd x_{\mu}}\Add{.}
\Tag{(114)}
\]
The second of Maxwell's systems of equations is then
defined by the tensor equation, resulting from this,
\[
\frac{\dd \phi_{\mu\nu}}{\dd x_{\rho}} +
\frac{\dd \phi_{\nu\rho}}{\dd x_{\mu}} +
\frac{\dd \phi_{\rho\mu}}{\dd x_{\nu}} = 0\Add{,}
\Tag{(114a)}
\]
and the first of Maxwell's systems of equations is defined
by the tensor-density relation
\[
\frac{\dd \tF^{\mu\nu}}{\dd x_{\nu}} = \tJ^{\mu}\Add{,}
\Tag{(115)}
\]
\PageSep{108}
in which
\begin{align*}
\tF^{\mu\nu} &= \sqrt{-g}\, g^{\mu\nu} g^{\nu\tau} \phi_{\sigma\tau}\Add{,} \\
\tJ^{\mu} &= \sqrt{-g}\, \rho \frac{dx_{\nu}}{ds}.
\end{align*}
If we introduce the energy tensor of the electromagnetic
field into the right-hand side of~\Eqref{(96)}, we obtain~\Eqref{(115)},
for the special case $\tJ^{\mu} = 0$, as a consequence of~\Eqref{(96)} by
taking the divergence. This inclusion of the theory of
electricity in the scheme of the general theory of relativity
has been considered arbitrary and unsatisfactory by
many theoreticians. Nor can we in this way conceive of
the equilibrium of the electricity which constitutes the
elementary electrically charged particles. A theory in
which the gravitational field and the electromagnetic field
enter as an essential entity would be much preferable.
H.~Weyl, and recently Th.~Kaluza, have discovered some
\index{Kaluza}%
\index{Weyl}%
ingenious theorems along this direction; but concerning
them, I am convinced that they do not bring us nearer to
the true solution of the fundamental problem. I shall
not go into this further, but shall give a brief discussion
of the so-called cosmological problem, for without this,
the considerations regarding the general theory of relativity
would, in a certain sense, remain unsatisfactory.

Our previous considerations, based upon the field
equations~\Eqref{(96)}, had for a foundation the conception that
space on the whole is Galilean-Euclidean, and that this
character is disturbed only by masses embedded in it.
This conception was certainly justified as long as we were
dealing with spaces of the order of magnitude of those
\PageSep{109}
that astronomy has to do with. But whether portions of
the universe, however large they may be, are quasi-Euclidean,
is a wholly different question. We can make
this clear by using an example from the theory of surfaces
which we have employed many times. If a portion of a
surface is observed by the eye to be practically plane, it
does not at all follow that the whole surface has the form
of a plane; the surface might just as well be a sphere, for
example, of sufficiently large radius. The question as to
whether the universe as a whole is non-Euclidean was
much discussed from the geometrical point of view before
the development of the theory of relativity. But with the
theory of relativity, this problem has entered upon a
new stage, for according to this theory the geometrical
properties of bodies are not independent, but depend
upon the distribution of masses.

If the universe were quasi-Euclidean, then Mach was
\index{Mach}%
wholly wrong in his thought that inertia, as well as
gravitation, depends upon a kind of mutual action between
bodies. For in this case, with a suitably selected system
of co-ordinates, the~$g_{\mu\nu}$ would be constant at infinity, as
they are in the special theory of relativity, while within
finite regions the~$g_{\mu\nu}$ would differ from these constant
values by small amounts only, with a suitable choice of
co-ordinates, as a result of the influence of the masses in
finite regions. The physical properties of space would
not then be wholly independent, that is, uninfluenced by
matter, but in the main they would be, and only in
small measure, conditioned by matter. Such a dualistic
conception is even in itself not satisfactory; there are,
\PageSep{110}
however, some important physical arguments against it,
which we shall consider.

The hypothesis that the universe is infinite and
\index{Finiteness of universe}%
\index{Universe@{Universe, Finiteness of}}%
Euclidean at infinity, is, from the relativistic point of
view, a complicated hypothesis. In the language of the
general theory of relativity it demands that the Riemann
\index{Riemann!tensor}%
tensor of the fourth rank~$R_{iklm}$ shall vanish at infinity,
which furnishes twenty independent conditions, while only
ten curvature components~$R_{\mu\nu}$, enter into the laws of the
gravitational field. It is certainly unsatisfactory to
postulate such a far-reaching limitation without any
physical basis for it.

But in the second place, the theory of relativity makes
it appear probable that Mach was on the right road in
\index{Mach}%
his thought that inertia depends upon a mutual action of
matter. For we shall show in the following that, according
to our equations, inert masses do act upon each other
in the sense of the relativity of inertia, even if only very
feebly. What is to be expected along the line of Mach's
thought?
\begin{itemize}
\item[1.] The inertia of a body must increase when ponderable
  masses are piled up in its neighbourhood.

\item[2.] A body must experience an accelerating force when
  neighbouring masses are accelerated, and, in fact,
  the force must be in the same direction as the
  acceleration.

\item[3.] A rotating hollow body must generate inside of
  itself a ``Coriolis field,'' which deflects moving
  bodies in the sense of the rotation, and a radial
  centrifugal field as well.
\end{itemize}
\PageSep{111}

We shall now show that these three effects, which are
to be expected in accordance with Mach's ideas, are
\index{Mach}%
actually present according to our theory, although their
magnitude is so small that confirmation of them by
laboratory experiments is not to be thought of. For this
purpose we shall go back to the equations of motion of
a material particle~\Eqref{(90)}, and carry the approximations
somewhat further than was done in equation~\Eqref{(90a)}.

First, we consider~$\gamma_{44}$ as small of the first order. The
square of the velocity of masses moving under the influence
of the gravitational force is of the same order, according
to the energy equation. It is therefore logical to regard
the velocities of the material particles we are considering,
as well as the velocities of the masses which generate the
field, as small, of the order~$\frac{1}{2}$. We shall now carry out the
approximation in the equations that arise from the field
equations~\Eqref{(101)} and the equations of motion~\Eqref{(90)} so far
as to consider terms, in the second member of~\Eqref{(90)}, that
are linear in those velocities. Further, we shall not put
$ds$~and~$dl$ equal to each other, but, corresponding to the
higher approximation, we shall put
\[
ds = \sqrt{g_{44}}\, dl
  = \left(1 - \frac{\gamma_{44}}{2}\right) dl.
\]
From~\Eqref{(90)} we obtain, at first,
\[
\frac{d}{dl}\left[\left(1 + \frac{\gamma_{44}}{2}\right) \frac{dx_{\mu}}{dl}\right]
  = -\Gamma_{\alpha\beta}^{\mu} \frac{dx_{\alpha}}{dl}\, \frac{dx_{\beta}}{dl}
    \left(1 + \frac{\gamma_{44}}{2}\right).
\Tag{(116)}
\]

From~\Eqref{(101)} we get, to the approximation sought for,
\PageSep{112}
\[
\left.
\begin{aligned}
-\gamma_{11} = -\gamma_{22} = -\gamma_{33}
  &= \gamma_{44} = \frac{\kappa}{4\pi} \int \frac{\sigma\, dV_{0}}{r}\Add{,} \\
%
\gamma_{4\alpha}
  &= -\frac{\Change{\veci}{i} \kappa}{2}
      \int \frac{\sigma \dfrac{dx_{\alpha}}{ds}\, dV_{0}}{r}\Add{,} \\
%
\gamma_{\alpha\beta} &= 0\Add{,}
\end{aligned}
\right\}
\Tag{(117)}
\]
in which, in~\Eqref{(117)}, $\alpha$~and~$\beta$ denote the space indices only.

On the right-hand side of~\Eqref{(116)} we can replace
$1 + \dfrac{\gamma_{44}}{2}$ by~$1$ and $-\Change{\Gamma_{\mu}^{\alpha\beta}}{\Gamma_{\alpha\beta}^{\mu}}$ by~$\left[\Chr{\alpha}{\beta}{\mu}\right]$. It is easy to see, in
addition, that to this degree of approximation we must
put
\begin{align*}
\left[\Chr{4}{4}{\mu}\right]
  &= -\tfrac{1}{2} \frac{\dd \gamma_{44}}{\dd x_{\mu}}
                 + \frac{\dd \gamma_{4\mu}}{\dd x_{4}}\Add{,} \\
%
\left[\Chr{\alpha}{4}{\mu}\right]
  &= \tfrac{1}{2} \left(
    \frac{\dd \gamma_{4\mu}}{\dd x_{\alpha}} - \frac{\dd \gamma_{4\alpha}}{\dd x_{\mu}}
  \right)\Add{,} \\
\left[\Chr{\alpha}{\beta}{\mu}\right] &= 0\Add{,}
\end{align*}
in which $\alpha$,~$\beta$ and~$\mu$ denote space indices. We therefore
obtain from~\Eqref{(116)}, in the usual vector notation,
\[
\left.
\begin{aligned}
\frac{d}{dl}\bigl[(1 + \Bar[7pt]{\sigma}) \v\bigr]
  &= \grad \Bar[7pt]{\sigma} + \frac{\dd \tA}{\dd l} + [\rot \tA, \v]\Add{,} \\
\Bar[7pt]{\sigma} &= \frac{\kappa}{8\pi} \int \frac{\sigma\, dV_{0}}{r}\Add{,} \\
\tA &= \frac{\kappa}{2\Change{}{\pi}}
       \int \frac{\sigma \dfrac{dx_{\alpha}}{dl}\, dV_{0}}{r}\Add{.}
\end{aligned}
\right\}
\Tag{(118)}
\]

The equations of motion,~\Eqref{(118)}, show now, in fact, that
\PageSep{113}
\begin{itemize}
\item[1.] The inert mass is proportional to $1 + \Bar[7pt]{\sigma}$, and
  therefore increases when ponderable masses
  approach the test body.

\item[2.] There is an inductive action of accelerated masses,
\index{Accelerated masses, inductive action of}%
\index{Inductive action of accelerated masses}%
  of the same sign, upon the test body. This is
  the term~$\dfrac{\dd \tA}{\dd l}$.

\item[3.] A material particle, moving perpendicularly to the
  axis of rotation inside a rotating hollow body,
  is deflected in the sense of the rotation (Coriolis
  field). The centrifugal action, mentioned above,
  inside a rotating hollow body, also follows from
  the theory, as has been shown by Thirring.\footnotemark
\index{Thirring}%
\end{itemize}
\footnotetext{That the centrifugal action must be inseparably connected with the
  existence of the Coriolis field may be recognized, even without calculation,
  in the special case of a co-ordinate system rotating uniformly relatively to
  an inertial system; our general co-variant equations naturally must apply
  to such a case.}

Although all of these effects are inaccessible to experiment,
because $\kappa$~is so small, nevertheless they certainly
exist according to the general theory of relativity. We
must see in them a strong support for Mach's ideas as to
\index{Mach}%
the relativity of all inertial actions. If we think these
ideas consistently through to the end we must expect the
\emph{whole} inertia, that is, the \emph{whole} $g_{\mu\nu}$-field, to be determined
by the matter of the universe, and not mainly by the
boundary conditions at infinity.

For a satisfactory conception of the $g_{\mu\nu}$-field of cosmical
dimensions, the fact seems to be of significance that the
relative velocity of the stars is small compared to the
velocity of light. It follows from this that, with a suitable
\PageSep{114}
choice of co-ordinates, $g_{44}$~is nearly constant in the
universe, at least, in that part of the universe in which
there is matter. The assumption appears natural, moreover,
that there are stars in all parts of the universe, so
that we may well assume that the inconstancy of~$g_{44}$
depends only upon the circumstance that matter is not
distributed continuously, but is concentrated in single
celestial bodies and systems of bodies. If we are willing
to ignore these more local non-uniformities of the density
of matter and of the $g_{\mu\nu}$-field, in order to learn something
of the geometrical properties of the universe as a whole,
it appears natural to substitute for the actual distribution
of masses a continuous distribution, and furthermore to
assign to this distribution a uniform density~$\sigma$. In this
imagined universe all points with space directions will
be geometrically equivalent; with respect to its space
extension it will have a constant curvature, and will be
cylindrical with respect to its $x_{4}$-co-ordinate. The possibility
seems to be particularly satisfying that the universe
is spatially bounded and thus, in accordance with our
assumption of the constancy of~$\sigma$, is of constant curvature,
being either spherical or elliptical; for then the boundary
conditions at infinity which are so inconvenient from the
standpoint of the general theory of relativity, may be
replaced by the much more natural conditions for a closed
surface.

According to what has been said, we are to put
\[
ds^{2} = {dx_{4}}^{2} - \gamma_{\mu\nu}\, dx_{\mu}\, dx_{\nu}\Add{,}
\Tag{(119)}
\]
in which the indices $\mu$~and~$\nu$ run from~$1$ to~$3$ only. The~$\gamma_{\mu\nu}$
\PageSep{115}
will be such functions of $x_{1}$,~$x_{2}$,~$x_{3}$ as correspond
to a three-dimensional continuum of constant positive
curvature. We must now investigate whether such an
assumption can satisfy the field equations of gravitation.

In order to be able to investigate this, we must first
find what differential conditions the three-dimensional
manifold of constant curvature satisfies. A spherical
manifold of three dimensions, embedded in a Euclidean
continuum of four dimensions,\footnote
  {The aid of a fourth space dimension has naturally no significance
  except that of a mathematical artifice.}
is given by the equations
\begin{align*}
{x_{1}}^{2} + {x_{2}}^{2} + {x_{3}}^{2} + {x_{4}}^{2} &= a^{2}\Add{,} \\
{dx_{1}}^{2} + {dx_{2}}^{2} + {dx_{3}}^{2} + {dx_{4}}^{2} &= ds^{2}.
\end{align*}
By eliminating~$x_{4}$, we get
\[
ds^{2} = {dx_{1}}^{2} + {dx_{2}}^{2} + {dx_{3}}^{2}
  + \frac{(x_{1}\, dx_{1} + x_{2}\, dx_{2} + x_{3}\, dx_{3})^{2}}
         {a^{2} - {x_{1}}^{2} - {x_{2}}^{2} - {x_{3}}^{2}}.
\]

As far as terms of the third and higher degrees in the~$x_{\nu}$,
we can put, in the neighbourhood of the origin of
co-ordinates,
\[
ds^{2} = \left(\delta_{\mu\nu} + \frac{x_{\mu} x_{\nu}}{a^{2}}\right)
  dx_{\mu}\, dx_{\nu}.
\]

Inside the brackets are the~$g_{\mu\nu}$ of the manifold in the
neighbourhood of the origin. Since the first derivatives
of the~$g_{\mu\nu}$, and therefore also the~$\Gamma_{\mu\nu}^{\sigma}$, vanish at the
origin, the calculation of the~$R_{\mu\nu}$ for this manifold, by~\Eqref{(88)},
is very simple at the origin. We have
\[
R_{\mu\nu} = \Change{}{-}\frac{2}{a^{2}} \delta_{\mu\nu}
  = \frac{2}{a^{2}} g_{\mu\nu}.
\]
\PageSep{116}

Since the relation $R_{\mu\nu} = \dfrac{2}{a^{2}} g_{\mu\nu}$ is universally co-variant,
and since all points of the manifold are geometrically
equivalent, this relation holds for every system of co-ordinates,
and everywhere in the manifold. In order to
avoid confusion with the four-dimensional continuum,
we shall, in the following, designate quantities that refer
to the three-dimensional continuum by Greek letters,
and put
\[
\P_{\mu\nu} = -\frac{2}{a^{2}} \gamma_{\mu\nu}\Add{.}
\Tag{(120)}
\]

We now proceed to apply the field equations~\Eqref{(96)} to
our special case. From~\Eqref{(119)} we get for the four-dimensional
manifold,
\[
\left.
\begin{aligned}
R_{\mu\nu} &= \text{$\P_{\mu\nu}$ for the indices~$1$ to~$3$}\Add{,} \\
R_{14} &= R_{24} = R_{34} = R_{44} = 0\Add{.}
\end{aligned}
\right\}
\Tag{(121)}
\]

For the right-hand side of~\Eqref{(96)} we have to consider
the energy tensor for matter distributed like a cloud of
dust. According to what has gone before we must
therefore put
\[
T^{\mu\nu} = \sigma \frac{dx_{\mu}}{ds}\, \frac{dx_{\nu}}{ds}
\]
specialized for the case of rest. But in addition, we
shall add a pressure term that may be physically established
as follows. Matter consists of electrically charged
particles. On the basis of Maxwell's theory these
cannot be conceived of as electromagnetic fields free
from singularities. In order to be consistent with the
\PageSep{117}
facts, it is necessary to introduce energy terms, not contained
in Maxwell's theory, so that the single electric
particles may hold together in spite of the mutual repulsions
between their elements, charged with electricity
of one sign. For the sake of consistency with this fact,
Poincar� has assumed a pressure to exist inside these
particles which balances the electrostatic repulsion. It
cannot, however, be asserted that this pressure vanishes
outside the particles. We shall be consistent with this
circumstance if, in our phenomenological presentation,
we add a pressure term. This must not, however, be
confused with a hydrodynamical pressure, as it serves
only for the energetic presentation of the dynamical
relations inside matter. In this sense we put
\[
T_{\mu\nu}
  = g_{\mu\sigma} g_{\nu\beta} \frac{dx_{\alpha}}{ds}\, \frac{dx_{\beta}}{ds}
  - g_{\mu\nu} p\Add{.}
\Tag{(122)}
\]

In our special case we have, therefore, to put
\begin{align*}
T_{\mu\nu}
  &= \gamma_{\mu\nu} p \text{ (for $\mu$~and~$\nu$ from~$1$ to~$3$)}\Add{,} \\
T_{44} &= \sigma - p\Add{,} \\
T &= -\gamma^{\mu\nu} \gamma_{\mu\nu} p + \sigma - p
  = \sigma - 4p.
\end{align*}
Observing that the field equation~\Eqref{(96)} may be written
in the form
\[
R_{\mu\nu} = -\kappa(T_{\mu\nu} - \tfrac{1}{2} g_{\mu\nu} T)\Add{,}
\]
we get from~\Eqref{(96)} the equations,
\begin{align*}
+\frac{2}{a^{2}} \gamma_{\mu\nu}
  &=  \kappa \left(\frac{\sigma}{2} - p\right) \gamma_{\mu\nu}\Add{,} \\
0 &= -\kappa \left(\frac{\sigma}{2} + p\right).
\end{align*}
\PageSep{118}
From this follows
\[
\left.
\begin{aligned}
p &= -\frac{\sigma}{2}\Add{,} \\
a &= \sqrt{\frac{2}{\kappa\sigma}}\Add{.}
\end{aligned}
\right\}
\Tag{(123)}
\]

If the universe is quasi-Euclidean, and its radius of
\index{Radius of Universe}%
\index{Universe@{Universe, Finiteness of}!Radius of}%
curvature therefore infinite, then $\sigma$~would vanish. But
it is improbable that the mean density of matter in the
universe is actually zero; this is our third argument
against the assumption that the universe is quasi-Euclidean.
Nor does it seem possible that our hypothetical
pressure can vanish; the physical nature of this
pressure can be appreciated only after we have a better
theoretical knowledge of the electromagnetic field.
According to the second of equations~\Eqref{(123)} the radius,~$a$,
of the universe is determined in terms of the total
mass,~$M$, of matter, by the equation
\[
a = \frac{M\kappa}{4\pi^{2}}\Add{.}
\Tag{(124)}
\]
The complete dependence of the geometrical upon the
physical properties becomes clearly apparent by means
of this equation.

Thus we may present the following arguments against
the conception of a space-infinite, and for the conception
of a space-bounded, universe:---

1.~From the standpoint of the theory of relativity,
the condition for a closed surface is very much simpler
than the corresponding boundary condition at infinity
of the quasi-Euclidean structure of the universe.
\PageSep{119}

2.~The idea that Mach expressed, that inertia depends
\index{Mach}%
upon the mutual action of bodies, is contained, to a
first approximation, in the equations of the theory of
relativity; it follows from these equations that inertia
depends, at least in part, upon mutual actions between
masses. As it is an unsatisfactory assumption to make
that inertia depends in part upon mutual actions, and
in part upon an independent property of space, Mach's
idea gains in probability. But this idea of Mach's
corresponds only to a finite universe, bounded in space,
and not to a quasi-Euclidean, infinite universe. From
the standpoint of epistemology it is more satisfying to
have the mechanical properties of space completely determined
by matter, and this is the case only in a space-bounded
universe.

3.~An infinite universe is possible only if the mean
density of matter in the universe vanishes. Although
such an assumption is logically possible, it is less probable
than the assumption that there is a finite mean
density of matter in the universe.
\PageSep{120}
%[Blank page]
\PageSep{121}


\PrintIndex
\iffalse %%%% Index text %%%%
%INDEX
% A

Accelerated masses, inductive action of 113

Addition@{Addition and subtraction of tensors}#addition 14
  theorem of velocities 40
% B

Biot-Savart force 46
% C

Centrifugal force 67

Clocks, moving#clocks 39

Compressible viscous fluid 22

Concept@{Concept of space}#concept 3
  time 30

Conditions of orthogonality 7

Congruence, theorems of#congruence 3

Conservation principles 55

Continuum, four-dimensional#continuum 33

Contraction of tensors 15

Contravariant@{Contra-variant vectors}#contra-variant 72
  tensors 75

Coordinates@{Co-ordinates, preferred systems of}#co-ordinates 8

Covariance@{Co-variance of equation of continuity}#co-variance 22

Covariant@{Co-variant}#co-variant 12 %** et seq.
  vector 72

Criticism of principle of inertia#criticism 65

Criticisms of theory of relativity#criticisms 31

Curvilinear co-ordinates 68
% D

Differentiation of tensors 76, 79

Displacement of spectral lines 101
% E

Energy@{Energy and mass}#energy 48, 51
  tensor@{tensor of electromagnetic field}#tensor 52
    of matter 56

Equation of continuity, co-variance of#co-variance 22

Equations of motion of \Change{materia}{material} particle#equations of motion 52

Equivalence of mass and energy#equivalence 51

Equivalent spaces of reference#equivalent 26

Euclidean geometry 4
% F

Finiteness of universe 110

Fizeau 29

Four-dimensional continuum 33

Four-vector 43

Fundamental tensor 74
% G

Galilean@{Galilean regions}#Galilean 65
  transformation 28

Gauss 68

Geodetic lines 86

Geometry, Euclidean 4

Gravitational mass 63

Gravitation constant 98
% H

Homogeneity of space 17

Hydrodynamical equations 56

Hypotheses of pre-relativity physics 77
%\PageSep{122}
% I

Inductive action of accelerated masses 113

Inert and gravitational mass, equality of#gravitational mass 63

Invariant 10 %** et seq.

Isotropy of space 17
% K

Kaluza 108
% L

Levi-Civita 77

Light cone@{Light-cone}#light-cone 42

Light ray, path of 102

Light time@{Light-time}#light-time 34

Linear orthogonal transformation 7

Lorentz@{Lorentz electromotive force}#Lorentz 46
  transformation 32
% M

Mach 62, 109, 110, 111, 113, 119

Mass and Energy 48, 51
  equality of gravitational and inert#gravitational 63
  gravitational 63

Maxwell's equations 23

Mercury, perihelion of#Mercury 103, 107

Michelson and Morley 29

Minkowski 34

Motion of particle, equations of#equations of motion 52

Moving measuring rods and clocks#clocks 39

Multiplication of tensors#multiplication 14
% N

Newtonian gravitation constant#Newtonian 98
% O

Operations on tensors 14 %** et seq.

Orthogonality, conditions of#orthogonality 7

Orthogonal transformations, linear 7
% P

Path of light ray 102

Perihelion of Mercury 103, 107

Poisson's equation 90

Preferred systems of co-ordinates 8

Prerelativity@{Pre-relativity physics, hypotheses of}#pre-relativity 27

Principle@{Principle of equivalence}#principle of 64
  inertia, criticism of 65

Principles of conservation 55
% R

Radius of Universe#radius 118

Rank of tensor#rank 14

Ray of light, path of 102

Reference, space of 4

Riemann 72
  tensor 82, 85, 110

Rods (measuring) and clocks in motion#clocks 39

Rotation 66
% S

Simultaneity 17, 30

Sitter 29

Skew-symmetrical tensor 15

Solar Eclipse expedition (1919)#eclipse 103

Space@{Space, Concept of}#space 3 %[** TN: "concept of" in the original]
  Homogeneity of 17 %[** TN: "homogeneity of" in the original]
  Isotropy of 17

Spaces of reference 4
  equivalence of 26

Special Lorentz transformation 36

Spectral lines, displacement of 101

Straightest lines 86

Stress tensor 22

Symmetrical tensor 15

Systems of co-ordinates, preferred#preferred 8

% T

%Tensor, 12 et seq., 72 et seq.
Tensor 12, 72
  Addition and subtraction of 14
  Contraction of 15
  Fundamental 74
  Multiplication of 14
%\PageSep{123}
%Tensor,
  operations 14 %** et seq.
  Rank of 14
  Symmetrical and Skew-symmetrical 15

Tensors, formation by differentiation 76

Theorem for addition of velocities 40

Theorems of congruence 3

Theory of relativity, criticisms of#criticisms 31

Thirring 113

Time-concept 30

Time-space concept 33

Transformation@{Transformation, Galilean}#transformation 28
  Linear orthogonal 7
% U

Universe@{Universe, Finiteness of}#universe 110 %[** TN: "finiteness of" in the original]
  Radius of 118 %[** TN: "radius of" in the original]
% V

Vector@{Vector, co-variant}#vector 72
  contra-variant 72

Velocities, addition theorem of 40

Viscous compressible fluid 22
% W

Weyl 77, 103, 108
\fi %%%% End of index text %%%%
\PageSep{124}
\cleardoublepage
\newpage
\begin{CenterPage}
\scriptsize
PRINTED IN GREAT BRITAIN AT THE UNIVERSITY PRESS, ABERDEEN
\end{CenterPage}

%%%%%%%%%%%%%%%%%%%%%%%%% GUTENBERG LICENSE %%%%%%%%%%%%%%%%%%%%%%%%%%

\PGLicenseInit
\begin{PGtext}
End of Project Gutenberg's The Meaning of Relativity, by Albert Einstein

*** END OF THIS PROJECT GUTENBERG EBOOK THE MEANING OF RELATIVITY ***

***** This file should be named 36276-pdf.pdf or 36276-pdf.zip *****
This and all associated files of various formats will be found in:
        http://www.gutenberg.org/3/6/2/7/36276/

Produced by Andrew D. Hwang. (This ebook was produced using
OCR text generously provided by Northeastern University's
Snell Library through the Internet Archive.)


Updated editions will replace the previous one--the old editions
will be renamed.

Creating the works from public domain print editions means that no
one owns a United States copyright in these works, so the Foundation
(and you!) can copy and distribute it in the United States without
permission and without paying copyright royalties.  Special rules,
set forth in the General Terms of Use part of this license, apply to
copying and distributing Project Gutenberg-tm electronic works to
protect the PROJECT GUTENBERG-tm concept and trademark.  Project
Gutenberg is a registered trademark, and may not be used if you
charge for the eBooks, unless you receive specific permission.  If you
do not charge anything for copies of this eBook, complying with the
rules is very easy.  You may use this eBook for nearly any purpose
such as creation of derivative works, reports, performances and
research.  They may be modified and printed and given away--you may do
practically ANYTHING with public domain eBooks.  Redistribution is
subject to the trademark license, especially commercial
redistribution.



*** START: FULL LICENSE ***

THE FULL PROJECT GUTENBERG LICENSE
PLEASE READ THIS BEFORE YOU DISTRIBUTE OR USE THIS WORK

To protect the Project Gutenberg-tm mission of promoting the free
distribution of electronic works, by using or distributing this work
(or any other work associated in any way with the phrase "Project
Gutenberg"), you agree to comply with all the terms of the Full Project
Gutenberg-tm License (available with this file or online at
http://gutenberg.net/license).


Section 1.  General Terms of Use and Redistributing Project Gutenberg-tm
electronic works

1.A.  By reading or using any part of this Project Gutenberg-tm
electronic work, you indicate that you have read, understand, agree to
and accept all the terms of this license and intellectual property
(trademark/copyright) agreement.  If you do not agree to abide by all
the terms of this agreement, you must cease using and return or destroy
all copies of Project Gutenberg-tm electronic works in your possession.
If you paid a fee for obtaining a copy of or access to a Project
Gutenberg-tm electronic work and you do not agree to be bound by the
terms of this agreement, you may obtain a refund from the person or
entity to whom you paid the fee as set forth in paragraph 1.E.8.

1.B.  "Project Gutenberg" is a registered trademark.  It may only be
used on or associated in any way with an electronic work by people who
agree to be bound by the terms of this agreement.  There are a few
things that you can do with most Project Gutenberg-tm electronic works
even without complying with the full terms of this agreement.  See
paragraph 1.C below.  There are a lot of things you can do with Project
Gutenberg-tm electronic works if you follow the terms of this agreement
and help preserve free future access to Project Gutenberg-tm electronic
works.  See paragraph 1.E below.

1.C.  The Project Gutenberg Literary Archive Foundation ("the Foundation"
or PGLAF), owns a compilation copyright in the collection of Project
Gutenberg-tm electronic works.  Nearly all the individual works in the
collection are in the public domain in the United States.  If an
individual work is in the public domain in the United States and you are
located in the United States, we do not claim a right to prevent you from
copying, distributing, performing, displaying or creating derivative
works based on the work as long as all references to Project Gutenberg
are removed.  Of course, we hope that you will support the Project
Gutenberg-tm mission of promoting free access to electronic works by
freely sharing Project Gutenberg-tm works in compliance with the terms of
this agreement for keeping the Project Gutenberg-tm name associated with
the work.  You can easily comply with the terms of this agreement by
keeping this work in the same format with its attached full Project
Gutenberg-tm License when you share it without charge with others.

1.D.  The copyright laws of the place where you are located also govern
what you can do with this work.  Copyright laws in most countries are in
a constant state of change.  If you are outside the United States, check
the laws of your country in addition to the terms of this agreement
before downloading, copying, displaying, performing, distributing or
creating derivative works based on this work or any other Project
Gutenberg-tm work.  The Foundation makes no representations concerning
the copyright status of any work in any country outside the United
States.

1.E.  Unless you have removed all references to Project Gutenberg:

1.E.1.  The following sentence, with active links to, or other immediate
access to, the full Project Gutenberg-tm License must appear prominently
whenever any copy of a Project Gutenberg-tm work (any work on which the
phrase "Project Gutenberg" appears, or with which the phrase "Project
Gutenberg" is associated) is accessed, displayed, performed, viewed,
copied or distributed:

This eBook is for the use of anyone anywhere at no cost and with
almost no restrictions whatsoever.  You may copy it, give it away or
re-use it under the terms of the Project Gutenberg License included
with this eBook or online at www.gutenberg.net

1.E.2.  If an individual Project Gutenberg-tm electronic work is derived
from the public domain (does not contain a notice indicating that it is
posted with permission of the copyright holder), the work can be copied
and distributed to anyone in the United States without paying any fees
or charges.  If you are redistributing or providing access to a work
with the phrase "Project Gutenberg" associated with or appearing on the
work, you must comply either with the requirements of paragraphs 1.E.1
through 1.E.7 or obtain permission for the use of the work and the
Project Gutenberg-tm trademark as set forth in paragraphs 1.E.8 or
1.E.9.

1.E.3.  If an individual Project Gutenberg-tm electronic work is posted
with the permission of the copyright holder, your use and distribution
must comply with both paragraphs 1.E.1 through 1.E.7 and any additional
terms imposed by the copyright holder.  Additional terms will be linked
to the Project Gutenberg-tm License for all works posted with the
permission of the copyright holder found at the beginning of this work.

1.E.4.  Do not unlink or detach or remove the full Project Gutenberg-tm
License terms from this work, or any files containing a part of this
work or any other work associated with Project Gutenberg-tm.

1.E.5.  Do not copy, display, perform, distribute or redistribute this
electronic work, or any part of this electronic work, without
prominently displaying the sentence set forth in paragraph 1.E.1 with
active links or immediate access to the full terms of the Project
Gutenberg-tm License.

1.E.6.  You may convert to and distribute this work in any binary,
compressed, marked up, nonproprietary or proprietary form, including any
word processing or hypertext form.  However, if you provide access to or
distribute copies of a Project Gutenberg-tm work in a format other than
"Plain Vanilla ASCII" or other format used in the official version
posted on the official Project Gutenberg-tm web site (www.gutenberg.net),
you must, at no additional cost, fee or expense to the user, provide a
copy, a means of exporting a copy, or a means of obtaining a copy upon
request, of the work in its original "Plain Vanilla ASCII" or other
form.  Any alternate format must include the full Project Gutenberg-tm
License as specified in paragraph 1.E.1.

1.E.7.  Do not charge a fee for access to, viewing, displaying,
performing, copying or distributing any Project Gutenberg-tm works
unless you comply with paragraph 1.E.8 or 1.E.9.

1.E.8.  You may charge a reasonable fee for copies of or providing
access to or distributing Project Gutenberg-tm electronic works provided
that

- You pay a royalty fee of 20% of the gross profits you derive from
     the use of Project Gutenberg-tm works calculated using the method
     you already use to calculate your applicable taxes.  The fee is
     owed to the owner of the Project Gutenberg-tm trademark, but he
     has agreed to donate royalties under this paragraph to the
     Project Gutenberg Literary Archive Foundation.  Royalty payments
     must be paid within 60 days following each date on which you
     prepare (or are legally required to prepare) your periodic tax
     returns.  Royalty payments should be clearly marked as such and
     sent to the Project Gutenberg Literary Archive Foundation at the
     address specified in Section 4, "Information about donations to
     the Project Gutenberg Literary Archive Foundation."

- You provide a full refund of any money paid by a user who notifies
     you in writing (or by e-mail) within 30 days of receipt that s/he
     does not agree to the terms of the full Project Gutenberg-tm
     License.  You must require such a user to return or
     destroy all copies of the works possessed in a physical medium
     and discontinue all use of and all access to other copies of
     Project Gutenberg-tm works.

- You provide, in accordance with paragraph 1.F.3, a full refund of any
     money paid for a work or a replacement copy, if a defect in the
     electronic work is discovered and reported to you within 90 days
     of receipt of the work.

- You comply with all other terms of this agreement for free
     distribution of Project Gutenberg-tm works.

1.E.9.  If you wish to charge a fee or distribute a Project Gutenberg-tm
electronic work or group of works on different terms than are set
forth in this agreement, you must obtain permission in writing from
both the Project Gutenberg Literary Archive Foundation and Michael
Hart, the owner of the Project Gutenberg-tm trademark.  Contact the
Foundation as set forth in Section 3 below.

1.F.

1.F.1.  Project Gutenberg volunteers and employees expend considerable
effort to identify, do copyright research on, transcribe and proofread
public domain works in creating the Project Gutenberg-tm
collection.  Despite these efforts, Project Gutenberg-tm electronic
works, and the medium on which they may be stored, may contain
"Defects," such as, but not limited to, incomplete, inaccurate or
corrupt data, transcription errors, a copyright or other intellectual
property infringement, a defective or damaged disk or other medium, a
computer virus, or computer codes that damage or cannot be read by
your equipment.

1.F.2.  LIMITED WARRANTY, DISCLAIMER OF DAMAGES - Except for the "Right
of Replacement or Refund" described in paragraph 1.F.3, the Project
Gutenberg Literary Archive Foundation, the owner of the Project
Gutenberg-tm trademark, and any other party distributing a Project
Gutenberg-tm electronic work under this agreement, disclaim all
liability to you for damages, costs and expenses, including legal
fees.  YOU AGREE THAT YOU HAVE NO REMEDIES FOR NEGLIGENCE, STRICT
LIABILITY, BREACH OF WARRANTY OR BREACH OF CONTRACT EXCEPT THOSE
PROVIDED IN PARAGRAPH 1.F.3.  YOU AGREE THAT THE FOUNDATION, THE
TRADEMARK OWNER, AND ANY DISTRIBUTOR UNDER THIS AGREEMENT WILL NOT BE
LIABLE TO YOU FOR ACTUAL, DIRECT, INDIRECT, CONSEQUENTIAL, PUNITIVE OR
INCIDENTAL DAMAGES EVEN IF YOU GIVE NOTICE OF THE POSSIBILITY OF SUCH
DAMAGE.

1.F.3.  LIMITED RIGHT OF REPLACEMENT OR REFUND - If you discover a
defect in this electronic work within 90 days of receiving it, you can
receive a refund of the money (if any) you paid for it by sending a
written explanation to the person you received the work from.  If you
received the work on a physical medium, you must return the medium with
your written explanation.  The person or entity that provided you with
the defective work may elect to provide a replacement copy in lieu of a
refund.  If you received the work electronically, the person or entity
providing it to you may choose to give you a second opportunity to
receive the work electronically in lieu of a refund.  If the second copy
is also defective, you may demand a refund in writing without further
opportunities to fix the problem.

1.F.4.  Except for the limited right of replacement or refund set forth
in paragraph 1.F.3, this work is provided to you 'AS-IS' WITH NO OTHER
WARRANTIES OF ANY KIND, EXPRESS OR IMPLIED, INCLUDING BUT NOT LIMITED TO
WARRANTIES OF MERCHANTIBILITY OR FITNESS FOR ANY PURPOSE.

1.F.5.  Some states do not allow disclaimers of certain implied
warranties or the exclusion or limitation of certain types of damages.
If any disclaimer or limitation set forth in this agreement violates the
law of the state applicable to this agreement, the agreement shall be
interpreted to make the maximum disclaimer or limitation permitted by
the applicable state law.  The invalidity or unenforceability of any
provision of this agreement shall not void the remaining provisions.

1.F.6.  INDEMNITY - You agree to indemnify and hold the Foundation, the
trademark owner, any agent or employee of the Foundation, anyone
providing copies of Project Gutenberg-tm electronic works in accordance
with this agreement, and any volunteers associated with the production,
promotion and distribution of Project Gutenberg-tm electronic works,
harmless from all liability, costs and expenses, including legal fees,
that arise directly or indirectly from any of the following which you do
or cause to occur: (a) distribution of this or any Project Gutenberg-tm
work, (b) alteration, modification, or additions or deletions to any
Project Gutenberg-tm work, and (c) any Defect you cause.


Section  2.  Information about the Mission of Project Gutenberg-tm

Project Gutenberg-tm is synonymous with the free distribution of
electronic works in formats readable by the widest variety of computers
including obsolete, old, middle-aged and new computers.  It exists
because of the efforts of hundreds of volunteers and donations from
people in all walks of life.

Volunteers and financial support to provide volunteers with the
assistance they need are critical to reaching Project Gutenberg-tm's
goals and ensuring that the Project Gutenberg-tm collection will
remain freely available for generations to come.  In 2001, the Project
Gutenberg Literary Archive Foundation was created to provide a secure
and permanent future for Project Gutenberg-tm and future generations.
To learn more about the Project Gutenberg Literary Archive Foundation
and how your efforts and donations can help, see Sections 3 and 4
and the Foundation web page at http://www.pglaf.org.


Section 3.  Information about the Project Gutenberg Literary Archive
Foundation

The Project Gutenberg Literary Archive Foundation is a non profit
501(c)(3) educational corporation organized under the laws of the
state of Mississippi and granted tax exempt status by the Internal
Revenue Service.  The Foundation's EIN or federal tax identification
number is 64-6221541.  Its 501(c)(3) letter is posted at
http://pglaf.org/fundraising.  Contributions to the Project Gutenberg
Literary Archive Foundation are tax deductible to the full extent
permitted by U.S. federal laws and your state's laws.

The Foundation's principal office is located at 4557 Melan Dr. S.
Fairbanks, AK, 99712., but its volunteers and employees are scattered
throughout numerous locations.  Its business office is located at
809 North 1500 West, Salt Lake City, UT 84116, (801) 596-1887, email
business@pglaf.org.  Email contact links and up to date contact
information can be found at the Foundation's web site and official
page at http://pglaf.org

For additional contact information:
     Dr. Gregory B. Newby
     Chief Executive and Director
     gbnewby@pglaf.org


Section 4.  Information about Donations to the Project Gutenberg
Literary Archive Foundation

Project Gutenberg-tm depends upon and cannot survive without wide
spread public support and donations to carry out its mission of
increasing the number of public domain and licensed works that can be
freely distributed in machine readable form accessible by the widest
array of equipment including outdated equipment.  Many small donations
($1 to $5,000) are particularly important to maintaining tax exempt
status with the IRS.

The Foundation is committed to complying with the laws regulating
charities and charitable donations in all 50 states of the United
States.  Compliance requirements are not uniform and it takes a
considerable effort, much paperwork and many fees to meet and keep up
with these requirements.  We do not solicit donations in locations
where we have not received written confirmation of compliance.  To
SEND DONATIONS or determine the status of compliance for any
particular state visit http://pglaf.org

While we cannot and do not solicit contributions from states where we
have not met the solicitation requirements, we know of no prohibition
against accepting unsolicited donations from donors in such states who
approach us with offers to donate.

International donations are gratefully accepted, but we cannot make
any statements concerning tax treatment of donations received from
outside the United States.  U.S. laws alone swamp our small staff.

Please check the Project Gutenberg Web pages for current donation
methods and addresses.  Donations are accepted in a number of other
ways including including checks, online payments and credit card
donations.  To donate, please visit: http://pglaf.org/donate


Section 5.  General Information About Project Gutenberg-tm electronic
works.

Professor Michael S. Hart is the originator of the Project Gutenberg-tm
concept of a library of electronic works that could be freely shared
with anyone.  For thirty years, he produced and distributed Project
Gutenberg-tm eBooks with only a loose network of volunteer support.


Project Gutenberg-tm eBooks are often created from several printed
editions, all of which are confirmed as Public Domain in the U.S.
unless a copyright notice is included.  Thus, we do not necessarily
keep eBooks in compliance with any particular paper edition.


Most people start at our Web site which has the main PG search facility:

     http://www.gutenberg.net

This Web site includes information about Project Gutenberg-tm,
including how to make donations to the Project Gutenberg Literary
Archive Foundation, how to help produce our new eBooks, and how to
subscribe to our email newsletter to hear about new eBooks.
\end{PGtext}

% %%%%%%%%%%%%%%%%%%%%%%%%%%%%%%%%%%%%%%%%%%%%%%%%%%%%%%%%%%%%%%%%%%%%%%% %
%                                                                         %
% End of Project Gutenberg's The Meaning of Relativity, by Albert Einstein%
%                                                                         %
% *** END OF THIS PROJECT GUTENBERG EBOOK THE MEANING OF RELATIVITY ***   %
%                                                                         %
% ***** This file should be named 36276-t.tex or 36276-t.zip *****        %
% This and all associated files of various formats will be found in:      %
%         http://www.gutenberg.org/3/6/2/7/36276/                         %
%                                                                         %
% %%%%%%%%%%%%%%%%%%%%%%%%%%%%%%%%%%%%%%%%%%%%%%%%%%%%%%%%%%%%%%%%%%%%%%% %

\end{document}
###
@ControlwordReplace = (
  ['\\PGLicenseInit', ''],
  ['\\PrintIndex', ''],
  ['\\TableofContents', '']
  );

@ControlwordArguments = (
  ['\\BookMark', 1, 0, '', '', 1, 0, '', ''],
  ['\\item', 0, 1, '', ' '],
  ['\\Lecture', 0, 0, '', '', 1, 1, '', '. ', 1, 1, '', '', 1, 0, '', ''],
  ['\\Section', 1, 1, '', ''],
  ['\\Subsection', 1, 1, '', ''],
  ['\\Paragraph', 1, 1, '', ''],
  ['\\TranslatorsNote', 1, 1, '', ''],
  ['\\Change', 1, 0, '', '', 1, 1, '', ''],
  ['\\TNote', 1, 0, '', ''],
  ['\\Add', 1, 1, '', ''],
  ['\\PageSep', 1, 0, '', ''],
  ['\\Figure', 0, 0, '', '', 1, 0, '', ''],
  ['\\Figref', 1, 1, 'Fig. ', ''],
  ['\\Eqref', 1, 1, '', ''],
  ['\\First', 1, 1, '', '']
  );
###
This is pdfTeXk, Version 3.141592-1.40.3 (Web2C 7.5.6) (format=pdflatex 2010.5.6)  30 MAY 2011 09:21
entering extended mode
 %&-line parsing enabled.
**36276-t.tex
(./36276-t.tex
LaTeX2e <2005/12/01>
Babel <v3.8h> and hyphenation patterns for english, usenglishmax, dumylang, noh
yphenation, arabic, farsi, croatian, ukrainian, russian, bulgarian, czech, slov
ak, danish, dutch, finnish, basque, french, german, ngerman, ibycus, greek, mon
ogreek, ancientgreek, hungarian, italian, latin, mongolian, norsk, icelandic, i
nterlingua, turkish, coptic, romanian, welsh, serbian, slovenian, estonian, esp
eranto, uppersorbian, indonesian, polish, portuguese, spanish, catalan, galicia
n, swedish, ukenglish, pinyin, loaded.
(/usr/share/texmf-texlive/tex/latex/base/book.cls
Document Class: book 2005/09/16 v1.4f Standard LaTeX document class
(/usr/share/texmf-texlive/tex/latex/base/bk12.clo
File: bk12.clo 2005/09/16 v1.4f Standard LaTeX file (size option)
)
\c@part=\count79
\c@chapter=\count80
\c@section=\count81
\c@subsection=\count82
\c@subsubsection=\count83
\c@paragraph=\count84
\c@subparagraph=\count85
\c@figure=\count86
\c@table=\count87
\abovecaptionskip=\skip41
\belowcaptionskip=\skip42
\bibindent=\dimen102
) (/usr/share/texmf-texlive/tex/latex/base/inputenc.sty
Package: inputenc 2006/05/05 v1.1b Input encoding file
\inpenc@prehook=\toks14
\inpenc@posthook=\toks15
(/usr/share/texmf-texlive/tex/latex/base/latin1.def
File: latin1.def 2006/05/05 v1.1b Input encoding file
)) (/usr/share/texmf-texlive/tex/latex/base/ifthen.sty
Package: ifthen 2001/05/26 v1.1c Standard LaTeX ifthen package (DPC)
) (/usr/share/texmf-texlive/tex/latex/amsmath/amsmath.sty
Package: amsmath 2000/07/18 v2.13 AMS math features
\@mathmargin=\skip43
For additional information on amsmath, use the `?' option.
(/usr/share/texmf-texlive/tex/latex/amsmath/amstext.sty
Package: amstext 2000/06/29 v2.01
(/usr/share/texmf-texlive/tex/latex/amsmath/amsgen.sty
File: amsgen.sty 1999/11/30 v2.0
\@emptytoks=\toks16
\ex@=\dimen103
)) (/usr/share/texmf-texlive/tex/latex/amsmath/amsbsy.sty
Package: amsbsy 1999/11/29 v1.2d
\pmbraise@=\dimen104
) (/usr/share/texmf-texlive/tex/latex/amsmath/amsopn.sty
Package: amsopn 1999/12/14 v2.01 operator names
)
\inf@bad=\count88
LaTeX Info: Redefining \frac on input line 211.
\uproot@=\count89
\leftroot@=\count90
LaTeX Info: Redefining \overline on input line 307.
\classnum@=\count91
\DOTSCASE@=\count92
LaTeX Info: Redefining \ldots on input line 379.
LaTeX Info: Redefining \dots on input line 382.
LaTeX Info: Redefining \cdots on input line 467.
\Mathstrutbox@=\box26
\strutbox@=\box27
\big@size=\dimen105
LaTeX Font Info:    Redeclaring font encoding OML on input line 567.
LaTeX Font Info:    Redeclaring font encoding OMS on input line 568.
\macc@depth=\count93
\c@MaxMatrixCols=\count94
\dotsspace@=\muskip10
\c@parentequation=\count95
\dspbrk@lvl=\count96
\tag@help=\toks17
\row@=\count97
\column@=\count98
\maxfields@=\count99
\andhelp@=\toks18
\eqnshift@=\dimen106
\alignsep@=\dimen107
\tagshift@=\dimen108
\tagwidth@=\dimen109
\totwidth@=\dimen110
\lineht@=\dimen111
\@envbody=\toks19
\multlinegap=\skip44
\multlinetaggap=\skip45
\mathdisplay@stack=\toks20
LaTeX Info: Redefining \[ on input line 2666.
LaTeX Info: Redefining \] on input line 2667.
) (/usr/share/texmf-texlive/tex/latex/amsfonts/amssymb.sty
Package: amssymb 2002/01/22 v2.2d
(/usr/share/texmf-texlive/tex/latex/amsfonts/amsfonts.sty
Package: amsfonts 2001/10/25 v2.2f
\symAMSa=\mathgroup4
\symAMSb=\mathgroup5
LaTeX Font Info:    Overwriting math alphabet `\mathfrak' in version `bold'
(Font)                  U/euf/m/n --> U/euf/b/n on input line 132.
)) (/usr/share/texmf-texlive/tex/latex/base/alltt.sty
Package: alltt 1997/06/16 v2.0g defines alltt environment
) (/usr/share/texmf-texlive/tex/latex/tools/array.sty
Package: array 2005/08/23 v2.4b Tabular extension package (FMi)
\col@sep=\dimen112
\extrarowheight=\dimen113
\NC@list=\toks21
\extratabsurround=\skip46
\backup@length=\skip47
) (/usr/share/texmf-texlive/tex/latex/footmisc/footmisc.sty
Package: footmisc 2005/03/17 v5.3d a miscellany of footnote facilities
\FN@temptoken=\toks22
\footnotemargin=\dimen114
\c@pp@next@reset=\count100
\c@@fnserial=\count101
Package footmisc Info: Declaring symbol style bringhurst on input line 817.
Package footmisc Info: Declaring symbol style chicago on input line 818.
Package footmisc Info: Declaring symbol style wiley on input line 819.
Package footmisc Info: Declaring symbol style lamport-robust on input line 823.

Package footmisc Info: Declaring symbol style lamport* on input line 831.
Package footmisc Info: Declaring symbol style lamport*-robust on input line 840
.
) (/usr/share/texmf-texlive/tex/latex/bigfoot/perpage.sty
Package: perpage 2006/07/15 1.12 Reset/sort counters per page
\c@abspage=\count102
) (/usr/share/texmf-texlive/tex/latex/tools/multicol.sty
Package: multicol 2006/05/18 v1.6g multicolumn formatting (FMi)
\c@tracingmulticols=\count103
\mult@box=\box28
\multicol@leftmargin=\dimen115
\c@unbalance=\count104
\c@collectmore=\count105
\doublecol@number=\count106
\multicoltolerance=\count107
\multicolpretolerance=\count108
\full@width=\dimen116
\page@free=\dimen117
\premulticols=\dimen118
\postmulticols=\dimen119
\multicolsep=\skip48
\multicolbaselineskip=\skip49
\partial@page=\box29
\last@line=\box30
\mult@rightbox=\box31
\mult@grightbox=\box32
\mult@gfirstbox=\box33
\mult@firstbox=\box34
\@tempa=\box35
\@tempa=\box36
\@tempa=\box37
\@tempa=\box38
\@tempa=\box39
\@tempa=\box40
\@tempa=\box41
\@tempa=\box42
\@tempa=\box43
\@tempa=\box44
\@tempa=\box45
\@tempa=\box46
\@tempa=\box47
\@tempa=\box48
\@tempa=\box49
\@tempa=\box50
\@tempa=\box51
\c@columnbadness=\count109
\c@finalcolumnbadness=\count110
\last@try=\dimen120
\multicolovershoot=\dimen121
\multicolundershoot=\dimen122
\mult@nat@firstbox=\box52
\colbreak@box=\box53
) (/usr/share/texmf-texlive/tex/latex/base/makeidx.sty
Package: makeidx 2000/03/29 v1.0m Standard LaTeX package
) (/usr/share/texmf-texlive/tex/latex/caption/caption.sty
Package: caption 2007/01/07 v3.0k Customising captions (AR)
(/usr/share/texmf-texlive/tex/latex/caption/caption3.sty
Package: caption3 2007/01/07 v3.0k caption3 kernel (AR)
(/usr/share/texmf-texlive/tex/latex/graphics/keyval.sty
Package: keyval 1999/03/16 v1.13 key=value parser (DPC)
\KV@toks@=\toks23
)
\captionmargin=\dimen123
\captionmarginx=\dimen124
\captionwidth=\dimen125
\captionindent=\dimen126
\captionparindent=\dimen127
\captionhangindent=\dimen128
)) (/usr/share/texmf-texlive/tex/latex/graphics/graphicx.sty
Package: graphicx 1999/02/16 v1.0f Enhanced LaTeX Graphics (DPC,SPQR)
(/usr/share/texmf-texlive/tex/latex/graphics/graphics.sty
Package: graphics 2006/02/20 v1.0o Standard LaTeX Graphics (DPC,SPQR)
(/usr/share/texmf-texlive/tex/latex/graphics/trig.sty
Package: trig 1999/03/16 v1.09 sin cos tan (DPC)
) (/etc/texmf/tex/latex/config/graphics.cfg
File: graphics.cfg 2007/01/18 v1.5 graphics configuration of teTeX/TeXLive
)
Package graphics Info: Driver file: pdftex.def on input line 90.
(/usr/share/texmf-texlive/tex/latex/pdftex-def/pdftex.def
File: pdftex.def 2007/01/08 v0.04d Graphics/color for pdfTeX
\Gread@gobject=\count111
))
\Gin@req@height=\dimen129
\Gin@req@width=\dimen130
) (/usr/share/texmf-texlive/tex/latex/tools/calc.sty
Package: calc 2005/08/06 v4.2 Infix arithmetic (KKT,FJ)
\calc@Acount=\count112
\calc@Bcount=\count113
\calc@Adimen=\dimen131
\calc@Bdimen=\dimen132
\calc@Askip=\skip50
\calc@Bskip=\skip51
LaTeX Info: Redefining \setlength on input line 75.
LaTeX Info: Redefining \addtolength on input line 76.
\calc@Ccount=\count114
\calc@Cskip=\skip52
) (/usr/share/texmf-texlive/tex/latex/yfonts/yfonts.sty
Package: yfonts 2003/01/08 v1.3 (WaS)
) (/usr/share/texmf-texlive/tex/latex/fancyhdr/fancyhdr.sty
\fancy@headwidth=\skip53
\f@ncyO@elh=\skip54
\f@ncyO@erh=\skip55
\f@ncyO@olh=\skip56
\f@ncyO@orh=\skip57
\f@ncyO@elf=\skip58
\f@ncyO@erf=\skip59
\f@ncyO@olf=\skip60
\f@ncyO@orf=\skip61
) (/usr/share/texmf-texlive/tex/latex/geometry/geometry.sty
Package: geometry 2002/07/08 v3.2 Page Geometry
\Gm@cnth=\count115
\Gm@cntv=\count116
\c@Gm@tempcnt=\count117
\Gm@bindingoffset=\dimen133
\Gm@wd@mp=\dimen134
\Gm@odd@mp=\dimen135
\Gm@even@mp=\dimen136
\Gm@dimlist=\toks24
(/usr/share/texmf-texlive/tex/xelatex/xetexconfig/geometry.cfg)) (/usr/share/te
xmf-texlive/tex/latex/hyperref/hyperref.sty
Package: hyperref 2007/02/07 v6.75r Hypertext links for LaTeX
\@linkdim=\dimen137
\Hy@linkcounter=\count118
\Hy@pagecounter=\count119
(/usr/share/texmf-texlive/tex/latex/hyperref/pd1enc.def
File: pd1enc.def 2007/02/07 v6.75r Hyperref: PDFDocEncoding definition (HO)
) (/etc/texmf/tex/latex/config/hyperref.cfg
File: hyperref.cfg 2002/06/06 v1.2 hyperref configuration of TeXLive
) (/usr/share/texmf-texlive/tex/latex/oberdiek/kvoptions.sty
Package: kvoptions 2006/08/22 v2.4 Connects package keyval with LaTeX options (
HO)
)
Package hyperref Info: Option `hyperfootnotes' set `false' on input line 2238.
Package hyperref Info: Option `bookmarks' set `true' on input line 2238.
Package hyperref Info: Option `linktocpage' set `false' on input line 2238.
Package hyperref Info: Option `pdfdisplaydoctitle' set `true' on input line 223
8.
Package hyperref Info: Option `pdfpagelabels' set `true' on input line 2238.
Package hyperref Info: Option `bookmarksopen' set `true' on input line 2238.
Package hyperref Info: Option `colorlinks' set `true' on input line 2238.
Package hyperref Info: Hyper figures OFF on input line 2288.
Package hyperref Info: Link nesting OFF on input line 2293.
Package hyperref Info: Hyper index ON on input line 2296.
Package hyperref Info: Plain pages OFF on input line 2303.
Package hyperref Info: Backreferencing OFF on input line 2308.
Implicit mode ON; LaTeX internals redefined
Package hyperref Info: Bookmarks ON on input line 2444.
(/usr/share/texmf-texlive/tex/latex/ltxmisc/url.sty
\Urlmuskip=\muskip11
Package: url 2005/06/27  ver 3.2  Verb mode for urls, etc.
)
LaTeX Info: Redefining \url on input line 2599.
\Fld@menulength=\count120
\Field@Width=\dimen138
\Fld@charsize=\dimen139
\Choice@toks=\toks25
\Field@toks=\toks26
Package hyperref Info: Hyper figures OFF on input line 3102.
Package hyperref Info: Link nesting OFF on input line 3107.
Package hyperref Info: Hyper index ON on input line 3110.
Package hyperref Info: backreferencing OFF on input line 3117.
Package hyperref Info: Link coloring ON on input line 3120.
\Hy@abspage=\count121
\c@Item=\count122
)
*hyperref using driver hpdftex*
(/usr/share/texmf-texlive/tex/latex/hyperref/hpdftex.def
File: hpdftex.def 2007/02/07 v6.75r Hyperref driver for pdfTeX
\Fld@listcount=\count123
)
\c@pp@a@footnote=\count124
\@indexfile=\write3
\openout3 = `36276-t.idx'.

Writing index file 36276-t.idx
\openout2 = `relativity.ist'.

\c@figno=\count125
\TmpLen=\skip62
(./36276-t.aux)
\openout1 = `36276-t.aux'.

LaTeX Font Info:    Checking defaults for OML/cmm/m/it on input line 491.
LaTeX Font Info:    ... okay on input line 491.
LaTeX Font Info:    Checking defaults for T1/cmr/m/n on input line 491.
LaTeX Font Info:    ... okay on input line 491.
LaTeX Font Info:    Checking defaults for OT1/cmr/m/n on input line 491.
LaTeX Font Info:    ... okay on input line 491.
LaTeX Font Info:    Checking defaults for OMS/cmsy/m/n on input line 491.
LaTeX Font Info:    ... okay on input line 491.
LaTeX Font Info:    Checking defaults for OMX/cmex/m/n on input line 491.
LaTeX Font Info:    ... okay on input line 491.
LaTeX Font Info:    Checking defaults for U/cmr/m/n on input line 491.
LaTeX Font Info:    ... okay on input line 491.
LaTeX Font Info:    Checking defaults for LY/yfrak/m/n on input line 491.
LaTeX Font Info:    ... okay on input line 491.
LaTeX Font Info:    Checking defaults for LYG/ygoth/m/n on input line 491.
LaTeX Font Info:    ... okay on input line 491.
LaTeX Font Info:    Checking defaults for PD1/pdf/m/n on input line 491.
LaTeX Font Info:    ... okay on input line 491.
(/usr/share/texmf-texlive/tex/latex/ragged2e/ragged2e.sty
Package: ragged2e 2003/03/25 v2.04 ragged2e Package (MS)
(/usr/share/texmf-texlive/tex/latex/everysel/everysel.sty
Package: everysel 1999/06/08 v1.03 EverySelectfont Package (MS)
LaTeX Info: Redefining \selectfont on input line 125.
)
\CenteringLeftskip=\skip63
\RaggedLeftLeftskip=\skip64
\RaggedRightLeftskip=\skip65
\CenteringRightskip=\skip66
\RaggedLeftRightskip=\skip67
\RaggedRightRightskip=\skip68
\CenteringParfillskip=\skip69
\RaggedLeftParfillskip=\skip70
\RaggedRightParfillskip=\skip71
\JustifyingParfillskip=\skip72
\CenteringParindent=\skip73
\RaggedLeftParindent=\skip74
\RaggedRightParindent=\skip75
\JustifyingParindent=\skip76
)
Package caption Info: hyperref package v6.74m (or newer) detected on input line
 491.
(/usr/share/texmf/tex/context/base/supp-pdf.tex
[Loading MPS to PDF converter (version 2006.09.02).]
\scratchcounter=\count126
\scratchdimen=\dimen140
\scratchbox=\box54
\nofMPsegments=\count127
\nofMParguments=\count128
\everyMPshowfont=\toks27
\MPscratchCnt=\count129
\MPscratchDim=\dimen141
\MPnumerator=\count130
\everyMPtoPDFconversion=\toks28
)
-------------------- Geometry parameters
paper: class default
landscape: --
twocolumn: --
twoside: true
asymmetric: --
h-parts: 9.03374pt, 325.215pt, 9.03375pt
v-parts: 4.15848pt, 495.49379pt, 6.23773pt
hmarginratio: 1:1
vmarginratio: 2:3
lines: --
heightrounded: --
bindingoffset: 0.0pt
truedimen: --
includehead: true
includefoot: true
includemp: --
driver: pdftex
-------------------- Page layout dimensions and switches
\paperwidth  343.28249pt
\paperheight 505.89pt
\textwidth  325.215pt
\textheight 433.62pt
\oddsidemargin  -63.23625pt
\evensidemargin -63.23624pt
\topmargin  -68.11151pt
\headheight 12.0pt
\headsep    19.8738pt
\footskip   30.0pt
\marginparwidth 98.0pt
\marginparsep   7.0pt
\columnsep  10.0pt
\skip\footins  10.8pt plus 4.0pt minus 2.0pt
\hoffset 0.0pt
\voffset 0.0pt
\mag 1000
\@twosidetrue \@mparswitchtrue 
(1in=72.27pt, 1cm=28.45pt)
-----------------------
(/usr/share/texmf-texlive/tex/latex/graphics/color.sty
Package: color 2005/11/14 v1.0j Standard LaTeX Color (DPC)
(/etc/texmf/tex/latex/config/color.cfg
File: color.cfg 2007/01/18 v1.5 color configuration of teTeX/TeXLive
)
Package color Info: Driver file: pdftex.def on input line 130.
)
Package hyperref Info: Link coloring ON on input line 491.
(/usr/share/texmf-texlive/tex/latex/hyperref/nameref.sty
Package: nameref 2006/12/27 v2.28 Cross-referencing by name of section
(/usr/share/texmf-texlive/tex/latex/oberdiek/refcount.sty
Package: refcount 2006/02/20 v3.0 Data extraction from references (HO)
)
\c@section@level=\count131
)
LaTeX Info: Redefining \ref on input line 491.
LaTeX Info: Redefining \pageref on input line 491.
(./36276-t.out) (./36276-t.out)
\@outlinefile=\write4
\openout4 = `36276-t.out'.

LaTeX Font Info:    Try loading font information for U+msa on input line 524.
(/usr/share/texmf-texlive/tex/latex/amsfonts/umsa.fd
File: umsa.fd 2002/01/19 v2.2g AMS font definitions
)
LaTeX Font Info:    Try loading font information for U+msb on input line 524.
(/usr/share/texmf-texlive/tex/latex/amsfonts/umsb.fd
File: umsb.fd 2002/01/19 v2.2g AMS font definitions
) [1

{/var/lib/texmf/fonts/map/pdftex/updmap/pdftex.map}] [2] [1

] [2

] [3] (./36276-t.toc)
\tf@toc=\write5
\openout5 = `36276-t.toc'.

[4




] [1

] [2] [3] [4] [5] [6] [7] [8] [9] [10] [11]
Underfull \hbox (badness 2452) in paragraph at lines 1094--1094
[][]\OT1/cmr/m/n/10 The equa-tion $[][][][][][] = 1$ may, by [][](5)[][], be re
-placed by
 []

[12] [13] [14] [15] [16] [17] [18] [19] [20] [21] [22] [23] [24] [25


] [26] [27] [28] [29] [30] [31] [32] [33] [34] [35] [36] [37] [38] <./images/04
1.pdf, id=432, 289.08pt x 305.14pt>
File: ./images/041.pdf Graphic file (type pdf)
<use ./images/041.pdf> [39] [40 <./images/041.pdf>] [41] [42] [43] [44] <./imag
es/047.pdf, id=498, 319.1925pt x 307.1475pt>
File: ./images/047.pdf Graphic file (type pdf)
<use ./images/047.pdf> [45] [46 <./images/047.pdf>] [47] [48] [49] [50] [51] [5
2] [53] <./images/056.pdf, id=601, 318.18875pt x 300.12125pt>
File: ./images/056.pdf Graphic file (type pdf)
<use ./images/056.pdf> [54] [55 <./images/056.pdf>] [56] [57] [58] [59


] [60] [61] [62] [63] [64] [65] [66] [67] [68] [69] [70] [71] [72] [73] [74] [7
5] [76] [77] [78] <./images/083.pdf, id=818, 290.08376pt x 213.79875pt>
File: ./images/083.pdf Graphic file (type pdf)
<use ./images/083.pdf> [79] [80 <./images/083.pdf>] [81] [82] [83] [84


] [85] [86] [87] [88] [89] [90] [91] [92] [93] [94] [95] [96] [97] [98] [99] [1
00] [101] [102] [103] [104] [105] [106] [107] [108] [109] [110] [111] [112] [11
3] [114] [115] (./36276-t.ind [116



] [117] [118] [119]) [120



] [1

] [2] [3] [4] [5] [6] [7] [8] (./36276-t.aux)

 *File List*
    book.cls    2005/09/16 v1.4f Standard LaTeX document class
    bk12.clo    2005/09/16 v1.4f Standard LaTeX file (size option)
inputenc.sty    2006/05/05 v1.1b Input encoding file
  latin1.def    2006/05/05 v1.1b Input encoding file
  ifthen.sty    2001/05/26 v1.1c Standard LaTeX ifthen package (DPC)
 amsmath.sty    2000/07/18 v2.13 AMS math features
 amstext.sty    2000/06/29 v2.01
  amsgen.sty    1999/11/30 v2.0
  amsbsy.sty    1999/11/29 v1.2d
  amsopn.sty    1999/12/14 v2.01 operator names
 amssymb.sty    2002/01/22 v2.2d
amsfonts.sty    2001/10/25 v2.2f
   alltt.sty    1997/06/16 v2.0g defines alltt environment
   array.sty    2005/08/23 v2.4b Tabular extension package (FMi)
footmisc.sty    2005/03/17 v5.3d a miscellany of footnote facilities
 perpage.sty    2006/07/15 1.12 Reset/sort counters per page
multicol.sty    2006/05/18 v1.6g multicolumn formatting (FMi)
 makeidx.sty    2000/03/29 v1.0m Standard LaTeX package
 caption.sty    2007/01/07 v3.0k Customising captions (AR)
caption3.sty    2007/01/07 v3.0k caption3 kernel (AR)
  keyval.sty    1999/03/16 v1.13 key=value parser (DPC)
graphicx.sty    1999/02/16 v1.0f Enhanced LaTeX Graphics (DPC,SPQR)
graphics.sty    2006/02/20 v1.0o Standard LaTeX Graphics (DPC,SPQR)
    trig.sty    1999/03/16 v1.09 sin cos tan (DPC)
graphics.cfg    2007/01/18 v1.5 graphics configuration of teTeX/TeXLive
  pdftex.def    2007/01/08 v0.04d Graphics/color for pdfTeX
    calc.sty    2005/08/06 v4.2 Infix arithmetic (KKT,FJ)
  yfonts.sty    2003/01/08 v1.3 (WaS)
fancyhdr.sty    
geometry.sty    2002/07/08 v3.2 Page Geometry
geometry.cfg
hyperref.sty    2007/02/07 v6.75r Hypertext links for LaTeX
  pd1enc.def    2007/02/07 v6.75r Hyperref: PDFDocEncoding definition (HO)
hyperref.cfg    2002/06/06 v1.2 hyperref configuration of TeXLive
kvoptions.sty    2006/08/22 v2.4 Connects package keyval with LaTeX options (HO
)
     url.sty    2005/06/27  ver 3.2  Verb mode for urls, etc.
 hpdftex.def    2007/02/07 v6.75r Hyperref driver for pdfTeX
ragged2e.sty    2003/03/25 v2.04 ragged2e Package (MS)
everysel.sty    1999/06/08 v1.03 EverySelectfont Package (MS)
supp-pdf.tex
   color.sty    2005/11/14 v1.0j Standard LaTeX Color (DPC)
   color.cfg    2007/01/18 v1.5 color configuration of teTeX/TeXLive
 nameref.sty    2006/12/27 v2.28 Cross-referencing by name of section
refcount.sty    2006/02/20 v3.0 Data extraction from references (HO)
 36276-t.out
 36276-t.out
    umsa.fd    2002/01/19 v2.2g AMS font definitions
    umsb.fd    2002/01/19 v2.2g AMS font definitions
./images/041.pdf
./images/047.pdf
./images/056.pdf
./images/083.pdf
 36276-t.ind
 ***********


LaTeX Font Warning: Size substitutions with differences
(Font)              up to 5.0pt have occurred.

 ) 
Here is how much of TeX's memory you used:
 6041 strings out of 94074
 80668 string characters out of 1165154
 151688 words of memory out of 1500000
 8690 multiletter control sequences out of 10000+50000
 19264 words of font info for 72 fonts, out of 1200000 for 2000
 649 hyphenation exceptions out of 8191
 34i,22n,45p,258b,599s stack positions out of 5000i,500n,6000p,200000b,5000s
</usr/share/texmf-texlive/fonts/type1/bluesky/cm/cmbx12.pfb></usr/share/texmf
-texlive/fonts/type1/bluesky/cm/cmcsc10.pfb></usr/share/texmf-texlive/fonts/typ
e1/bluesky/cm/cmex10.pfb></usr/share/texmf-texlive/fonts/type1/bluesky/cm/cmmi1
0.pfb></usr/share/texmf-texlive/fonts/type1/bluesky/cm/cmmi12.pfb></usr/share/t
exmf-texlive/fonts/type1/bluesky/cm/cmmi6.pfb></usr/share/texmf-texlive/fonts/t
ype1/bluesky/cm/cmmi7.pfb></usr/share/texmf-texlive/fonts/type1/bluesky/cm/cmmi
8.pfb></usr/share/texmf-texlive/fonts/type1/bluesky/cm/cmr10.pfb></usr/share/te
xmf-texlive/fonts/type1/bluesky/cm/cmr12.pfb></usr/share/texmf-texlive/fonts/ty
pe1/bluesky/cm/cmr6.pfb></usr/share/texmf-texlive/fonts/type1/bluesky/cm/cmr7.p
fb></usr/share/texmf-texlive/fonts/type1/bluesky/cm/cmr8.pfb></usr/share/texmf-
texlive/fonts/type1/bluesky/cm/cmsy10.pfb></usr/share/texmf-texlive/fonts/type1
/bluesky/cm/cmsy5.pfb></usr/share/texmf-texlive/fonts/type1/bluesky/cm/cmsy6.pf
b></usr/share/texmf-texlive/fonts/type1/bluesky/cm/cmsy7.pfb></usr/share/texmf-
texlive/fonts/type1/bluesky/cm/cmsy8.pfb></usr/share/texmf-texlive/fonts/type1/
bluesky/cm/cmti10.pfb></usr/share/texmf-texlive/fonts/type1/bluesky/cm/cmti12.p
fb></usr/share/texmf-texlive/fonts/type1/bluesky/cm/cmtt10.pfb></usr/share/texm
f-texlive/fonts/type1/bluesky/cm/cmtt8.pfb></usr/share/texmf-texlive/fonts/type
1/public/gothic/ygoth.pfb>
Output written on 36276-t.pdf (134 pages, 630001 bytes).
PDF statistics:
 1563 PDF objects out of 1728 (max. 8388607)
 519 named destinations out of 1000 (max. 131072)
 133 words of extra memory for PDF output out of 10000 (max. 10000000)

